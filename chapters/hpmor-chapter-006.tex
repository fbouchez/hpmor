\chapter{L'illusion de la planification}

\lettrine{C}{\emph{ertains}} enfants auraient attendu d'avoir \emph{fini} leur première visite au Chemin de Traverse.

«~Sac d'élément 79~», dit Harry, et il retira sa main, vide, de la bourse en peau de Moke.

La plupart des enfants auraient au moins attendu d'avoir leur \emph{baguette magique}.

«~Sac d'\emph{okane}~», dit Harry. Le lourd sac d'or apparut dans sa main.

Harry sortit le sac, puis le plongea à nouveau dans la bourse. Il sortit sa main, la remit à l'intérieur, et dit~: «~sac de jetons d'échanges financiers~». Cette fois-ci sa main ressortit vide.

«~Rends-moi le sac que je viens juste de mettre.~» Une fois de plus le sac sortit de la bourse.

Harry James Potter-Evans-Verres avait enfin mis la main sur un objet magique. Pourquoi attendre~?

«~Professeure McGonagall, dit Harry à la sorcière déconcertée qui cheminait à ses côtés, pourriez-vous me donner deux mots, un qui signifie or, et un signifiant autre chose n'étant pas monétaire, le tout dans une langue que je connais pas~? Mais ne me dites pas qui est quoi.

--- \emph{Ahava} et \emph{zahav}, dit McGonagall. C'est de l'hébreu, et l'autre mot veut dire amour.

--- Merci, Professeure. Sac d'\emph{ahava}.~» Vide.

«~Sac de \emph{zahav}.~» Et il apparut dans sa main.

«~Zahav veut dire or~?~» s'enquit Harry, et McGonagall hocha la tête.

Harry contempla les données expérimentales qu'il avait recueillies.
C'était un essai des plus bruts et préliminaires, mais c'était suffisant pour confirmer au moins une conclusion~:

«~\emph{Aaaaaarrrgh ça n'a aucun sens~!}~»

La sorcière à ses côtés leva un auguste sourcil.

«~Des problèmes, M. Potter~?

--- Je viens de falsifier chacune des hypothèses que j'avais~!
Comment la bourse peut-elle savoir que “sac de 115 gallions” est valide, mais pas “sac de 90 plus 25 gallions”~? Elle peut \emph{compter}, mais pas \emph{additionner}~? Elle peut comprendre les noms, mais pas les syntagmes nominaux de même sens~?
La personne qui l'a fabriquée ne parlait probablement pas japonais et \emph{je} ne parle pas hébreux, donc la bourse n'utilise pas \emph{son} savoir ni \emph{mon} savoir…~» Harry agita une main avec impuissance. «~Les règles paraissent \emph{en gros} cohérentes, mais elles ne \emph{veulent rien dire}~! Et je ne vais même pas commencer à m'interroger sur la façon dont une \emph{bourse} peut être équipée d'une reconnaissance vocale et d'une compréhension du langage naturel, alors que les meilleurs programmeurs en intelligence artificielle n'arrivent toujours pas à faire cette prouesse sur les supercalculateurs les plus rapides, après trente-cinq ans de dur labeur.~»
Harry repris sa respiration, «~mais \emph{comment} est-ce \emph{possible}~?

--- Magie, dit McGonagall.

--- C'est juste un \emph{mot}~! Même après m'avoir dit ça, je ne peux pas faire de nouvelles prédictions. C'est exactement comme de dire “phlogiston” ou “élan vital” ou “émergence” ou “complexité”~!~»

La sorcière en robe noire rit à haute voix. «~Mais c'\emph{est} de la magie, M. Potter.~»

Harry s'affaissa un peu.

«~Avec tout le respect que je vous dois, professeure McGonagall, je ne suis pas tout à fait sûr que vous compreniez que ce j'essaie de faire ici.

--- Avec tout le respect que je vous dois, M. Potter, je suis tout à fait sûre de ne pas le comprendre. À moins que -- c'est juste une supposition -- vous ne soyez en train d'essayer de conquérir le monde~?

--- Non~! Je veux dire oui -- enfin, \emph{non}~!

--- Je pense que je devrais peut-être m'alarmer de votre difficulté à répondre à cette question.~»

Morose, Harry se remémora la conférence de Dartmouth sur l'intelligence artificielle de 1956. Elle avait été la première conférence jamais organisée sur ce sujet, celle qui avait créé l'expression «~intelligence artificielle~». Ils avaient identifié des problèmes clés, tels que faire en sorte que les ordinateurs comprennent le langage, apprennent, et s'améliorent par eux-mêmes, et avaient suggéré, avec un parfait sérieux, que des progrès significatifs pourraient être accomplis par une dizaine scientifiques travaillant ensemble durant deux mois.

\emph{Non. Relève la tête. Tu viens à peine de} commencer \emph{à essayer d'élucider les mystères de la magie. Tu ne peux pas} savoir \emph{si cela va être trop difficile pour le faire en deux mois.}

«~Et vous n'avez \emph{vraiment} jamais entendu parler d'autres sorciers posant ce genre de questions ou faisant ce genre d'expériences scientifiques~?~» demanda à nouveau Harry.
Cela lui semblait tellement \emph{évident}.

Mais après tout, il avait fallu attendre plus de deux cents ans \emph{après} l'invention de la méthode scientifique pour qu'un Moldu pense à étudier de façon systématique quelles phrases un \emph{humain de quatre ans} pouvait ou ne pouvait pas comprendre.
La psychologie du développement du langage aurait pu être découverte au dix-huitième siècle, du moins en principe, mais personne n'avait jamais pensé à regarder avant le vingtième. Donc vous ne pouviez pas vraiment blâmer le monde magique, bien plus petit, si personne n'avait encore étudié le sort de Récupération.

McGonagall pinca les lèvres puis haussa les épaules.
«~Je ne suis toujours pas certaine de ce que vous entendez par “expérience scientifique,” M. Potter.
Comme je l'ai dit, j'ai vu des élèves nés-Moldus essayer de faire fonctionner la science moldue à Poudlard, et de nouveaux sortilèges et nouvelles potions sont inventés chaque année.~»

Harry secoua la tête.
«~La technologie et la science ne sont pas du tout la même chose.
Essayer de faire quelque chose de plein de façons différentes n'a rien voir avec le fait d'expérimenter pour comprendre les règles.~»
Beaucoup de gens avaient tenté d'inventer des machines volantes en essayant plein de trucs-avec-des-ailes, mais seuls les frères Wright avaient construit une soufflerie pour mesurer la portance…
«~Hmm, combien d'enfants élevés par des Moldus entrent à Poudlard chaque année~?

--- Environ dix~?~»

Harry faillit trébucher sur ses propres pieds. «~\emph{Dix}~?~»

Le monde Moldu avait une population de plus de six milliards d'individus.
S'il n'y avait qu'une personne comme vous sur un million, alors cela en faisait sept à Londres, et mille de plus en Chine.
Il était inévitable que parmi la population Moldue apparaisse \emph{quelques} enfants de onze ans capables de résoudre des équations différentielles --
Harry savait qu'il n'était pas le seul.  Il avait rencontré d'autres prodiges lors de compétitions de mathématiques.
À vrai dire s'était fait battre à plates coutures par des concurrents qui passaient probablement littéralement \emph{toutes leurs journées} à s'entraîner sur des problèmes de mathématiques et qui n'avaient \emph{jamais} lu de livre de science-fiction et qui s'épuiseraient \emph{complètement} avant leur \emph{puberté} et ne feraient \emph{jamais} rien de leur vie future parce qu'ils avaient simplement appliqué des techniques \emph{connues} au lieu d'apprendre à penser de façon \emph{créative}. (Harry était du genre mauvais perdant.)

Mais… dans le monde des sorciers…

Dix enfants de Moldus par an qui arrêtent tous leur éducation Moldue à l'âge de onze ans~?
Et McGonagall, même si elle n'était peut-être pas objective, avait affirmé que Poudlard était la plus grande et la plus prestigieuse des écoles de magie du monde… et la formation s'arrêtait à l'âge de dix-sept ans.

McGonagall connaissait sans aucun doute dans les moindres détails la façon dont on pouvait se transformer en chat.
Mais elle semblait n'avoir littéralement jamais \emph{entendu} parler de la méthode scientifique.
Pour elle, c'était juste la magie des Moldus.
Et elle ne semblait même pas \emph{curieuse} de découvrir quels mystères se cachaient derrière la compréhension du langage naturel que possédait le sort de Récupération.

Il ne restait donc réellement que deux possibilités.

Possibilité numéro un~: La magie était si incroyablement opaque, alambiquée et impénétrable que même si les sorciers et sorcières avaient fait de leur mieux pour essayer de la comprendre, ils n'avaient fait que peu, voire aucun progrès et avaient fini par laisser tomber~; et Harry ne ferait pas mieux.

\emph{Ou alors…}

Harry fit craquer ses phalanges avec détermination, mais cela ne fit qu'un petit claquement discret, au lieu de rebondir sinistrement en écho sur les murs du Chemin de Traverse.

Possibilité numéro deux~: il allait conquérir le monde.

Un jour ou l'autre. Peut-être pas tout de suite.

Il \emph{pouvait} arriver que ce genre de chose prenne plus de deux mois.
La science Moldue ne s'était pas rendue sur la lune une semaine après Galilée.

Harry n'arrivait cependant pas à empêcher son sourire de s'étendre jusqu'aux oreilles, à tel point que ses joues commençaient à lui être douloureuses.

Harry avait toujours été anxieux à l'idée de finir comme un de ces enfants prodiges qui n'aboutissent jamais à rien et passent le reste de leur vie à se vanter d'à quel point ils étaient en avance à l'âge de dix ans.
Cela dit, la plupart des génies adultes n'aboutissent à rien non plus.
Il y a probablement mille personnes aussi intelligentes qu'Einstein pour chaque Einstein de l'Histoire.
Mais ces autres génies n'avaient pas mis la main sur la seule chose dont vous aviez absolument besoin pour atteindre la vraie grandeur: ils n'avaient jamais trouvé de problème important.

\emph{Tu m'appartiens désormais}, Harry pensa aux murs du Chemin de Traverse, à tous les magasins et leurs articles, à tous les commerçants et leurs clients~; et à toutes les terres et tous les habitants de l'Angleterre magique, et au monde magique plus vaste encore~; et à l'univers tout entier, bien moins compris par les scientifiques Moldus que ce qu'ils croyaient.
\emph{Moi, Harry James Potter-Evans-Verres, revendique ce territoire au nom de la Science}.

Les éclairs et le tonnerre échouèrent complètement à jaillir et gronder dans le ciel sans nuage.

«~Qu'est-ce qui vous fait sourire~? s'enquit McGonagall avec méfiance et lassitude.

--- Je me demande s'il existe un sort permettant de faire jaillir des éclairs en arrière-plan à chaque fois que je prends une résolution sinistre~», expliqua Harry.
Il était en train de soigneusement mémoriser les mots exacts de sa résolution sinistre afin que les futurs livres d'histoire ne se trompent pas.

«~J'ai le sentiment prononcé que je devrais faire quelque chose à ce sujet, soupira McGonagall.

--- Ignorez-le, ça partira. Oh, ça brille~!~» Harry mit ses pensées de conquête mondiale en attente et sautilla vers un magasin à la devanture ouverte. McGonagall suivit.

\later

Harry avait maintenant acheté ses ingrédients pour les potions, un chaudron, et, oh, quelques petites choses supplémentaires.
Des objets qu'il lui semblait utile de transporter dans son sac sans fond (alias la Super Bourse en Peau de Moke \abbrev{QX31} avec sort d'Extension Indétectable, sort de Récupération, et Ouverture Élargissante).
Des achats intelligents et judicieux.

Harry ne comprenait honnêtement pas pourquoi McGonagall avait l'air si \emph{méfiante}.

Il se trouvait actuellement dans une boutique suffisamment luxueuse pour exposer dans la rue principale et sinueuse du Chemin de Traverse.
La devanture du magasin était ouverte et la marchandise disposée sur des étals de bois inclinés, gardée par de légères lueurs grises et sous la surveillance d'une jeune vendeuse vêtue d’une version fortement raccourcie de la classique robe de sorcière, révélant ses coudes et genoux.

Harry examinait l'équivalent magique d'un kit de premier soins, le Kit de Soins d'Urgence Plus.
Il y avait deux garrots auto-serrants.
Une seringue de ce qui semblait être du feu liquide, et était supposée considérablement réduire la circulation sanguine dans la zone traitée tout en continuant à oxygéner le sang durant au moins trois minutes, si vous aviez besoin d'empêcher un poison de se répandre dans le reste du corps.
Un tissu blanc dont ont pouvait envelopper une partie du corps pour l'anesthésier temporairement.
Plus toute une quantité d'autres objets dont l'utilité échappait complètement à Harry, tels que le “Traitement contre l'Exposition aux Détraqueurs,” qui sentait et ressemblait à du chocolat ordinaire.
Ou un “anti-Morsmental,” qui ressemblait à un petit œuf tremblotant accompagné d'une notice expliquant comment l'enfoncer dans la narine de quelqu'un.

«~Une affaire pour cinq gallions, qu'en dites-vous~?~» dit Harry à McGonagall, et la jeune vendeuse, qui l'observait non loin, hocha la tête avec enthousiasme.

Harry s'était attendu à ce que la professeure fasse une remarque approbatrice sur sa prudence et son sens de la préparation.

Ce qu'il reçut à la place ne pouvait être décrit que d'une seule manière~: un regard-qui-tue.

«~Et \emph{pourquoi} donc, dit McGonagall d'une voix chargée de scepticisme, vous attendriez-vous à avoir \emph{besoin} d'un kit de soins, jeune homme~?~»
(Après le regrettable incident au magasin de potions, McGonagall essayait d'éviter de dire «~M. Potter~» lorsque quelqu'un se trouvait non loin.)

La bouche de Harry s'ouvrit puis se ferma.
«~Je ne m'\emph{attends} pas à en avoir besoin~! C'est juste au cas où~!

--- Juste au cas où \emph{quoi}~?~»

Harry écarquilla les yeux. «~Vous pensez que j'ai \emph{prévu} de faire quelque chose de dangereux, et que c'est pour \emph{ça} que je veux un kit médical~?~»

Harry reçut pour toute réponse un regard empli de sombre suspicion et d'incrédulité ironique.

«~Nom de Zeus~!~» dit Harry. (C'était une expression qu'il avait récupérée du savant fou Doc Brown de \emph{Retour vers le Futur}.) «~C'est aussi ce que vous pensiez quand j'ai acheté la potion chute-de-plume, la branchiflore, et la bouteille de pilules nutritives et de gélules d'eau~?

--- Oui.~»

Harry secoua la tête avec stupéfaction. «~Mais quel genre de plan pensez-vous donc que je suis en train de mettre en place~?

% Et quelle sorte de plan pensez-vous que j'ai \emph{mis} \emph{en route}~?

--- Je ne sais pas, dit sombrement McGonagall, mais ça se termine soit par une livraison d'une tonne d'argent à Gringotts, soit en domination mondiale.

--- “Domination mondiale” est une expression si laide. Je préfère l'appeler “optimisation mondiale.”~»

Cette blague hilarante échoua totalement à rassurer la sorcière et son regard du jugement dernier.

«~Waouh, dit Harry se rendant compte qu'elle était sérieuse. Vous le pensez vraiment. Vous pensez vraiment que je prévois de faire quelque chose de dangereux.

--- Oui.

--- Parce que c'est l'\emph{unique} raison pour laquelle qui que ce soit achèterait jamais un kit de premiers soins~?
Ne le prenez pas mal, professeure McGonagall, mais \emph{de quelle sorte d'enfants cinglés avez-vous l'habitude de vous occuper}~?

--- Des Gryffondor~», lâcha McGonagall. Le mot portait une charge d'amertume et de désespoir s'abattant comme une malédiction éternelle jetée sur tout enthousiasme et joie de vivre de la jeunesse.

«~Directrice adjointe professeure McGonagall~» dit Harry, les mains solennellement posées sur les hanches. «~Je n'irai pas à Gryffondor…~»

À cet instant, McGonagall glissa quelque chose comme quoi si cela s'avérait \emph{être} le cas, elle découvrirait comment faire pour tuer un chapeau, une remarque étrange que Harry laissa passer sans commentaire, bien que cela semble déclencher chez la vendeuse une soudaine quinte de toux.

«~ Je serai envoyé à Serdaigle. Et si vous pensez vraiment que je prévois de faire quelque chose de dangereux, alors, en toute honnêteté, vous ne me comprenez pas \emph{du tout}.
Je n'\emph{aime} pas le danger, cela fait \emph{peur}.  Je suis \emph{prudent}.  Je suis \emph{précautionneux}. Je me prépare à des \emph{contingences imprévues}.
Comme mes parents avaient l'habitude de me chanter~:
\emph{Be prepared! That's the Boy Scout's marching song! Be prepared! As through life you march along! Don't be nervous, don't be flustered, don't be scared - be prepared!"}
\footnote{\emph{Soyez prêts~! C'est la chanson de marche des Scouts, soyez prêts~! Comme on marche dans la vie~! Pas nerveux, pas énervé, pas effrayé, soyez prêt~!}~»}

(Les parents de Harry n'avaient de fait chanté que \emph{ces vers-là} de la chanson de Tom Lehrer, et Harry vivait dans l'ignorance bienheureuse du reste de cette chanson.)

La posture de McGonagall s'était légèrement adoucie -- essentiellement au moment où Harry avait dit qu'il serait placé à Serdaigle.

«~À quelle sorte de \emph{contingence} imaginez-vous que ce kit vous puisse vous préparer, \emph{jeune homme}~?

--- L'une de mes camarades de classe se fait mordre par un horrible monstre, et alors que je fouille frénétiquement dans ma bourse en peau de Moke à la recherche de quelque chose qui pourrait l'aider, elle me regarde tristement, et dans son dernier souffle me dit~: “\emph{Pourquoi n'étais-tu pas prêt~?}” Alors que ses yeux se ferment pour mourir, je sais que jamais elle ne me pardonnera…~»


Harry entendit la vendeuse laisser échapper un petit cri d'effroi~; il leva les yeux et vit qu'elle le braquait du regard, les lèvres fermement serrées.
Puis la jeune femme fit volte-face et s'enfuit vers le fond du magasin.

\emph{Qu'est-ce qu…?}

McGonagall se pencha pour prendre les mains de Harry dans les siennes, doucement mais fermement, et entraîna Harry hors de la rue principale du Chemin de Traverse, le menant dans une impasse bordée par deux magasins, pavée de briques sales et fermée par un mur de terre noire compactée.

De sa hauteur, la sorcière pointa sa baguette en direction de la rue principale et prononça~: «~\emph{Sourdinam}~». Un écran de silence s'abattit autour d'eux, bloquant tous les bruits de la rue.

\emph{Qu'est-ce que j'ai fait de mal…}

McGonagall se tourna pour regarder Harry.
Elle n'arborait pas le visage d'une adulte en colère envers un enfant désobéissant.
Son expression était lissée, contrôlée.

«~Vous devez vous rappeler, M. Potter, que ce pays était en guerre il n'y a pas dix ans de cela. Tout le monde ici a perdu quelqu'un, et parler d'amis mourants dans vos bras… n'est pas à prendre à la légère.

--- Je, je ne voulais pas…~»
La déduction s'abattit comme une pierre dans l'imagination exceptionnellement vive de Harry.
Il venait de parler de quelqu'un rendant son dernier souffle -- puis la vendeuse s'était enfuie -- et la guerre avait pris fin dix ans auparavant, donc cette fille devait avoir huit ou neuf ans tout au plus quand, quand…
«~Je suis désolé, je ne voulais pas…~»
Harry sentit sa gorge se serrer, il se détourna pour fuir le regard de McGonagall, mais il y avait un mur en terre en travers de son chemin et il n'avait pas encore sa baguette magique. «~Je suis désolé, je suis désolé, je suis \emph{désolé}~!~»

Un lourd soupir s'éleva derrière lui. «~Je sais que vous l'êtes, M. Potter.~»

Harry osa jeter un coup d'œil derrière lui.
McGonagall avait juste l'air triste à présent.
«~Je suis désolé~», dit à nouveau Harry, se sentant misérable.
«~Est-ce qu'une situation similaire vous…~» et Harry resserra ses lèvres et se plaqua la main sur la bouche par sécurité.

Le visage de la sorcière s'attrista un peu plus.
«~Vous devez apprendre à penser avant de parler, M. Potter, ou risquez de passer votre vie avec bien peu d'amis.
Ça a été le sort de nombreux Serdaigle, et j'espère que ce ne sera pas le vôtre.~»

Harry aurait juste voulu s'enfuir en courant; sortir une baguette magique et effacer toute l'histoire de la mémoire de McGonagall, être à nouveau avec elle devant le magasin, \emph{faire que cela n'ait pas eu lieu}.

«~Mais pour répondre à votre question, M. Potter, non, rien de \emph{tel} ne m'est jamais arrivé.
Indéniablement, j'ai vus des amis rendre leur dernier soupir, plus d'une fois.
Mais aucun d'entre eux ne m'a jamais maudite en mourant, et je n'ai jamais pensé qu'ils pourraient ne pas me pardonner.
Qu'est-ce qui vous a poussé à \emph{dire} une chose pareille, M. Potter~? Qu'est-ce qui vous a poussé à même y \emph{penser}~?

--- Je, je,  je, Harry avala sa salive. C'est juste que j'essaie toujours d'imaginer la pire chose qui puisse arriver~», et peut-être qu'il avait aussi voulu blaguer un peu, mais il aurait préféré se couper la langue plutôt que de l'avouer.

--- Quoi? Mais \emph{Pourquoi}~?

--- Pour que je puisse empêcher que ça ait lieu~!

--- M. Potter…~» la voix de McGonagall resta en suspens.  Puis elle soupira, et s'accroupit à côté de lui.
«~M. Potter, dit-elle, gentiment à présent, ce n'est pas votre responsabilité que de prendre soin des étudiants de Poudlard. C'est la mienne.
Je ne laisserai rien de mal arriver, ni à vous ni à qui que ce soit d'autre.
Poudlard est l'endroit le plus sûr pour les enfants sorciers de tout le monde magique, et madame Pomfresh a une infirmerie en parfait état de fonctionnement.
Vous n'aurez pas besoin d'un kit de soin, qui plus est un coûtant cinq gallions.

--- Mais \emph{si}~! éclata Harry. \emph{Aucun endroit} n'est parfaitement sûr~! Et si mes parents font une crise cardiaque ou ont un accident quand je rentrerai à Noël -- Madame Pomfresh ne sera pas là, j'aurai besoin d'avoir mon propre kit de soin…

--- Par la barbe de Merlin, \emph{qu'est-ce qu}…~» dit McGonagall. Elle se leva et regarda Harry, partagée entre l'irritation et la préoccupation.
«~Il ne faut pas penser à des choses aussi terribles, M. Potter~!~»

Lorsqu'il entendit cela, l'expression de Harry vira vers l'amertume.
«~Si, il le \emph{faut}~! Si vous n'y pensez pas, ce n'est pas simplement que vous allez vous retrouver blessé, vous finissez par blesser les autres~!~»

McGonagall ouvrit la bouche, puis la ferma. Elle se frotta l'arête du nez, l'air pensive.
«~M. Potter… si je vous proposais un moment d'écoute… y a-t-il quelque chose dont vous voudriez me parler~?

--- À propos de quoi~?

--- À propos de la raison pour laquelle vous êtes convaincu que vous devez toujours être sur vos gardes contre les terribles choses qui pourraient vous arriver.~»

Harry la fixa, perplexe. C'était un axiome qui allait de soi. «~Eh bien…~» dit-il lentement. Il essaya d'organiser ses pensées. Comment \emph{pouvait}-il expliquer cela à une sorcière-professeure, alors qu'elle n'avait même pas les bases~?

«~Les chercheurs Moldus ont découvert que les gens sont toujours très optimistes, par rapport à la réalité.
Par exemple, ils disent que quelque chose va prendre deux jours et ça en prend dix, ou ils disent que ça va prendre deux mois et ça prend plus de trente-cinq ans.
Dans une expérience, des chercheurs ont demandé à des étudiants les durées maximum avant lesquelles ils étaient sûrs à 50~\%, 75~\% et 99~\% qu'ils auraient terminé leurs devoirs, et seuls 13~\%, 19~\% et 45~\% des étudiants ont terminé dans les temps qu'ils avaient donnés.
Ils se sont rendu compte que c'était parce que lorsqu'on demande à un groupe de gens leur estimation dans le meilleur des cas, si tout se passe le mieux possible, et à un autre groupe leur estimation dans le cas moyen, si tout se passe comme d'habitude, on reçoit des réponses qui sont statistiquement impossible à distinguer.
Vous voyez, si vous demandez à quelqu'un ce à quoi il s'attend dans le cas \emph{normal}, il visualise ce qui semble être le plus probable à chaque étape du parcours -- que tout se déroule comme prévu, sans surprise.
Mais en vrai, puisque plus de la moitié des étudiants n'ont pas fini au moment où ils étaient pourtant certains à 99~\% que ce serait le cas, cela veut dire que la réalité est habituellement légèrement pire que le “pire des cas possibles.”
C'est ce qu'on appelle l'\emph{illusion de la planification}, et la meilleure façon de la corriger est de vous demander combien de temps vous avez mis pour faire certaines tâches la dernière fois que vous avez essayé les faire.
On appelle cela utiliser le point de vue extérieur au lieu du point de vue intérieur.
Mais quand vous faites quelque chose de nouveau, vous ne pouvez pas utiliser cette méthode, alors vous devez être vraiment, vraiment, vraiment pessimiste.
En gros, tellement pessimiste que la réalité finit par être \emph{meilleure} que ce à quoi vous vous attendiez environ aussi souvent qu'elle finit par être pire.
C'est en fait \emph{très difficile} d'être \emph{suffisamment} pessimiste pour se retrouver avec de bonnes chances de \emph{dépasser} la réalité.
Par exemple, je fais un gros effort pour être sinistre et imaginer qu'une de mes camarades se fait mordre, mais ce qui se passe réellement c'est que les Mangemorts survivants attaquent l'école toute entière pour s'en prendre à moi. Mais le bon côté des choses c'est que…

--- Arrêtez~», dit McGonagall.

Harry s'arrêta. Il était juste sur le point de faire remarquer qu'au moins ils savaient que le Seigneur des Ténèbres n'attaquerait pas puisqu'il était mort.

«~Je pense que je n'ai peut-être pas été assez claire, dit McGonagall, son accent écossais redoublant de précaution. Y a-t-il quoi que ce soit qui vous soit arrivé \emph{personnellement} et qui vous ait effrayé, M. Potter~?

--- Ce qui m'est arrivé ne constitue que des éléments anecdotiques, expliqua Harry. Cela n'a pas le même poids qu'un article de journal scientifique, répliqué, évalué par des pairs, sur un essai randomisé contrôlé, avec de nombreux participants, des effets de grande ampleur et statistiquement significatif.~»

McGonagall pinça l'arête de son nez, inhala puis exhala.
«~Je voudrais tout de même que vous m'en parliez, dit-elle.

--- Euh…~» dit Harry. Il prit une profonde inspiration.
«~Il y avait eu quelques vols dans mon quartier, et ma mère m'avait demandé de ramener une poêle qu'elle avait empruntée aux voisins deux pâtés de maisons plus loin, et j'ai dit que je ne voulais pas y aller parce que je risquais de me faire agresser, et elle a dit “Harry, ne dis pas des choses pareilles~!”
Comme si y penser allait \emph{faire} que ça ait lieu, et donc que si je n'en parlais pas, je serais en sécurité.
J'ai essayé de lui expliquer pourquoi cela ne me rassurait pas, et elle m'a quand même forcé à rapporter la poêle.
J'étais trop jeune pour savoir à quel point il était statistiquement improbable qu'un voleur me prenne pour cible, mais j'étais suffisamment âgé pour savoir que ne pas penser à quelque chose ne l'empêchait pas d'avoir lieu, donc j'avais vraiment peur.

--- Rien d'autre~?~» dit McGonagall après une pause, lorsqu'il fut apparent que Harry avait terminé. «~Il n'y a rien d'\emph{autre} qui vous soit arrivé~?

--- Je sais que ça n'a pas \emph{l'air} bien important, se défendit Harry.
Mais c'était un de ces moments critiques d'une vie, vous voyez~?
Je veux dire que je \emph{savais} que ne pas penser à quelque chose ne l'empêchait pas d'avoir lieu, je le \emph{savais}, mais je pouvais voir que maman, elle, le croyait vraiment.~»
Harry s'arrêta, luttant contre la colère qui recommençait à monter alors qu'il se remémorait ce moment.
«~Elle ne \emph{voulait pas écouter}. J'ai essayé de lui dire, je l'ai \emph{suppliée} de ne pas m'envoyer dehors, et elle \emph{en a rit}.
Tout ce que je disais, elle le considérait comme une sorte de blague…~»
Harry força sa colère noire à redescendre.
«~C'est à ce moment là que je me suis rendu compte que tous ceux qui étaient censés me protéger étaient en réalité fous, et qu'ils ne m'écouteraient pas, peu importe que je les supplie, et que je ne pourrai jamais vraiment compter sur eux pour quoi que ce soit d'importance.~»
Parfois les bonnes intentions ne suffisaient pas, parfois il fallait être sain d'esprit…

Il y eut un long silence.

Harry prit le temps de respirer profondément et de se calmer.
Il n'y avait aucune raison de se mettre en colère.
Il n'y avait \emph{aucune} raison de se mettre en colère.
\emph{Tous} les parents étaient comme ça, \emph{aucun} adulte n'était prêt à s'abaisser suffisamment pour se mettre au même niveau qu'un enfant et l'écouter, ses parents génétiques n'auraient pas été différents.
La santé mentale était une petite étincelle dans la nuit, une rare exception infinitésimale à la règle qui est la folie, il était donc futile de se mettre en colère.

Harry ne s'aimait pas quand il était en colère.

«~Merci d'avoir partagé cela, M. Potter~», dit McGonagall au bout d'un moment.
Elle avait une expression distraite (presque exactement la même que celle qui était apparue sur le propre visage de Harry pendant qu'il faisait ses expériences avec la bourse en peau de Moke, mais il aurait fallu que Harry se voie dans un miroir pour pouvoir s'en rendre compte).
«~Je vais avoir besoin d'y réfléchir.~» Elle se tourna vers l'entrée de l'impasse et leva sa baguette…

«~Euh, dit Harry, peut-on aller chercher le kit de soins maintenant~?~»

McGonagall s'interrompit, et se retourna pour le regarder fermement. «~Et si je dis non, que c'est trop cher et vous n'en aurez pas besoin, que se passe-t-il~?~»

Le visage de Harry se couvrit d'amertume.
«~Exactement ce que vous pensez, professeure McGonagall. \emph{Exactement} ce que vous pensez.
J'en conclus que vous êtes une autre de ces adultes fous à laquelle je ne peux pas parler, et je commence à élaborer un plan pour me procurer un kit de soins autrement.

--- Je suis votre gardienne pour cette sortie, commença McGonagall, une pointe de menace dans la voix. Je ne vous \emph{permettrai pas} de me faire du chantage.

--- Je comprends~», dit Harry. Il réussit à ne pas laisser transparaître sa rancœur dans sa voix, et ne dit aucune des autres choses qui lui venaient à l'esprit. McGonagall lui avait dit de penser avant de parler. Il ne s'en souviendrait probablement pas le lendemain, mais il pouvait au moins s'en souvenir pendant cinq minutes.

La baguette de McGonagall décrivit un petit cercle au bout de sa main, et l'on entendit à nouveaux les bruits du Chemin de Traverse. «~Très bien, jeune homme, dit-elle. Allons acheter ce kit de soins.~»

La mâchoire de Harry tomba de surprise. Puis il se dépêcha de suivre la sorcière, trébuchant presque dans sa précipitation.

\later

Le magasin était tel qu'ils l'avaient laissé, avec des objets identifiables et d'autres inindentifiables présentés sur les étals de bois inclinés, la lueur grise protectrice et la vendeuse de retour à sa position originale.
Elle les regarda alors qu'ils s'approchaient, la surprise visible sur son visage.

«~Je suis désolée~», dit-elle quand ils arrivèrent, et Harry dit presque au même moment~: «~Je vous demande pardon pour…~»

Ils s'interrompirent et se regardèrent, puis la vendeuse eut un petit rire.
«~Je ne voulais pas vous causer d'ennuis avec la professeure McGonagall~», dit-elle.
Sa voix baissa d'un ton, conspiratrice.
«~J'espère qu'elle n'a pas été \emph{trop} horrible avec vous.

--- \emph{Della~!} dit McGonagall scandalisée.

--- Sac d'or~», dit Harry à sa bourse, et il s'adressa de nouveau à la vendeuse pendant qu'il comptait cinq gallions.
«~Ne vous en faites pas, je comprends bien que si elle est horrible avec moi c'est seulement parce qu'elle m'aime.~»

Il donna les gallions à la vendeuse pendant que McGonagall bafouillait quelque chose sans importance.
«~Un Pack de Soins d'Urgence Plus, s'il vous plaît.~»

C'était en fait assez déstabilisant de voir l'Ouverture Élargissante avaler le kit médical qui avait tout de même la taille d'une mallette.
Harry ne pouvait pas s'empêcher de se demander ce qui se passerait s'il essayait de grimper dans la bourse lui-même, étant donné que seule la personne qui y avait mis quelque chose était censée pouvoir la récupérer.

Lorsque la bourse eut fini de… manger… son achat durement gagné, Harry jura avoir entendu un petit rot.
Cela \emph{devait} être un sort rajouté volontairement.
L'hypothèse alternative était trop horrifiante pour être envisagée… en fait Harry n'arrivait même pas à \emph{imaginer} une hypothèse alternative.
Il releva la tête vers McGonagall alors qu'ils recommençaient leur marche le long du Chemin de Traverse. «~Où allons-nous ensuite~?~»

McGonagall pointa du doigt un magasin qui semblait être fait de chair plutôt que de briques et couvert de fourrure plutôt que de peinture.
«~Les petits animaux de compagnie sont autorisés à Poudlard -- vous pourriez avoir une chouette pour envoyer des lettres, par exemple…

--- Est-ce que je peux payer quelque chose comme une noise pour \emph{louer} une chouette quand j'ai besoin d'envoyer du courrier~?

--- Oui, dit McGonagall.

--- Alors je pense catégoriquement \emph{non}.~»

McGonagall hocha la tête, comme si elle cochait mentalement une case.
«~Pourrais-je vous demander pourquoi~?

--- J'ai eu un jour un caillou de compagnie. Il est mort.

--- Vous ne pensez pas pouvoir prendre soin d'un animal domestique~?

--- Je \emph{pourrais}, dit Harry, mais je finirai immanquablement par en faire une obsession, me demandant à longueur de journée si j'ai bien pensé à la nourrir le matin même ou si elle meurt lentement de faim dans sa cage, ne sachant ni où est son maître ni pourquoi il n'y a pas de nourriture.

--- Pauvre chouette, dit McGonagall d'une voix douce. Abandonnée ainsi. Je me demande ce qu'elle ferait.

--- Eh bien, je suppose qu'elle commencerait à avoir vraiment faim et qu'elle essayerait de se libérer de sa cage ou de sa boîte à coups de bec, mais ça ne fonctionnerait probablement pas…~» Harry s'arrêta net.

McGonagall continua, toujours de cette voix douce~: «~Et que lui arriverait-t-il ensuite~?~»

«~Excusez-moi~», dit Harry. Il prit McGonagall par la main, doucement mais fermement, et l'entraîna vers une autre ruelle~;
après avoir esquivé tant d'admirateurs que s'en était presque devenu une routine imperceptible.
«~Lancez ce sort de silence s'il vous plaît.

--- \emph{Sourdinam}.~»

La voix de Harry tremblait. «~Cette chouette ne me symbolise \emph{pas}, mes parents ne m'ont \emph{jamais} enfermé dans un placard pour me laisser mourir de faim, je n'ai \emph{pas} de peurs d'abandon et \emph{je n'aime pas le fil de vos pensées, professeure McGonagall~!}~»

La sorcière le regarda gravement.

«~Et quelles seraient ces pensées, M. Potter~?

--- Vous pensez que j'ai été… Harry avait du mal à le dire, que j'ai été \emph{maltraité}~?

--- L'avez-vous été~?

--- \emph{Non~!} cria Harry. Non, jamais~! Vous pensez que je suis \emph{stupide}~?
Je \emph{connais} le concept d'abus infantile, je \emph{sais} ce que sont des attouchements inappropriés et si quoi que ce soit qui y ressemble m'arrivait j'appellerais la police~!
Et je le signalerais au directeur de l'école~!
Et je chercherais le numéro des services sociaux dans l'annuaire~!
Et j'en parlerais à grand-mère et à grand-père et à Mlle Figg~!
Mais mes parents n'ont \emph{jamais} fait quoi que ce soit de ce genre, jamais jamais \emph{jamais}~!
Comment \emph{osez}-vous suggérer une chose pareille~!~»

McGonagall le regarda fixement. «~Il est de mon devoir en tant que directrice adjointe d'investiguer tout signe de maltraitance possible chez les enfants qui sont sous ma garde.~»

La colère de Harry vrillait hors de contrôle, se transformant en pure fureur noire.

«~Ne vous avisez \emph{jamais} de souffler un mot de ces, de ces \emph{insinuations} à qui que ce soit~!
À \emph{personne}, vous m'entendez, McGonagall~?
Une accusation comme celle-ci peut briser des vies et détruire des familles alors même que les parents sont complètement innocents~!
J'ai lu des choses à ce sujet dans les journaux~!~»
La voix de Harry grimpait dans les aigus.
«~Le \emph{système} ne sait pas s'\emph{arrêter}, il ne croit pas les parents \emph{ni} les enfants lorsqu'ils disent qu'il ne s'est rien passé~!
\emph{Je vous interdit de menacer ma famille ainsi~!  Je ne vous laisserai pas détruire mon foyer~!}

--- Harry~», fit doucement McGonagall, et elle tendit sa main vers lui…

Harry recula d'un pas rapide, sa main se leva brusquement pour repousser celle de la sorcière.

McGonagall se figea, puis retira sa main, et fit un pas en arrière.

«~Harry, tout va bien, dit-elle. Je vous crois.

--- \emph{Vraiment…}~», siffla Harry. La fureur grondait toujours dans ses veines.
«~Ou vous attendez juste d'être seule pour pouvoir remplir un formulaire~?

--- Harry, j'ai vu votre maison. Je vous ai vu avec vos parents. Ils vous aiment. Vous les aimez.
Je vous crois lorsque vous dites que vos parents ne vous ont pas maltraité.
Mais il \emph{fallait} que je pose la question, car il y a quelque chose d'étrange à l'œuvre ici.~»

Harry la fixa froidement. «~Comme quoi~?~»

--- Harry, j'ai vu de nombreux enfants victimes d'abus depuis que je travaille à Poudlard, cela vous briserait le cœur de savoir combien.
Et quand vous êtes joyeux, vous ne vous comportez pas comme l'un de ces enfants, pas du \emph{tout}.
Vous souriez aux étrangers, vous serrez les gens dans vos bras, j'ai mis ma main sur votre épaule et vous n'avez pas bronché.
Mais parfois, juste parfois, vous dites ou faites quelque chose qui fait \emph{fortement} penser à…
quelqu'un qui aurait passé les onze premières années de sa vie enfermé dans une cave.
Pas dans la famille aimante que j'ai vue.~»
McGonagall inclina la tête, elle semblait de nouveau perplexe.

Harry absorba tout cela, analysant ces informations. Sa rage noire commença à s'estomper, car il réalisait qu'on l'écoutait avec respect, et que sa famille n'était pas en danger.

«~Et comment \emph{expliquez}-vous vos observations, professeure McGonagall~?

--- Je ne sais pas, dit-elle. Mais il est possible que quelque chose vous soit arrivé, quelque chose dont vous ne vous souvenez pas.

Harry sentit de nouveau la fureur monter. Cela ressemblait beaucoup trop aux histoires de familles brisées qu'il avait lues dans les journaux.

«~Les souvenirs refoulés ce n'est que de la \emph{pseudoscience}~!
Les gens ne répriment \emph{pas} leurs souvenirs traumatiques, ils ne s'en souviennent que \emph{trop bien} pour le restant de leurs vies~!

--- Non, M. Potter. Il existe un sortilège nommé Oubliettes.~»

Harry se figea sur place. «~Un sort qui efface la mémoire~?~»

McGonagall acquiesça. «~Mais pas tous les effets du souvenir, si vous voyez ce que je veux dire, M. Potter.~»

Un frisson lui parcourut la colonne vertébrale. \emph{Cette} hypothèse… n'était \emph{pas} simple à réfuter.

«~Mais mes parents ne pourraient pas faire ça~!

--- Non, effectivement, dit McGonagall. Il faudrait quelqu'un venu du monde magique. Il n'y a… aucun moyen d'en être certain, j'en ai peur.~»

Les compétences rationalistes de Harry se remirent en fonctionnement. «~Professeure McGonagall, à quel point êtes-vous certaine de vos observations, et quelles explications alternatives pourrait-il y avoir~?~»

McGonagall ouvrit ses mains comme pour montrer qu'elle ne tenait rien.
«~Certaine~? Je ne suis certaine de \emph{rien}, M. Potter.
De toute va vie je n'ai jamais rencontré personne qui vous ressemble.
Parfois vous ne paraissez tout simplement pas avoir onze ans ni même être vraiment \emph{humain}.~»

Les sourcils de Harry s'élevèrent vers le ciel…

«~Pardon~! corrigea rapidement McGonagall. Je suis vraiment désolée, M. Potter.
J'essayais juste d'expliquer mon point de vue, ce n'est pas du tout ce que je voulais dire.

--- Au contraire, professeure, dit Harry, souriant lentement.
Je prends cette remarque comme un très grand compliment.
Mais puis-je vous proposer une explication alternative~?

--- Allez-y, je vous en prie.

--- Les enfants ne sont pas censés être beaucoup plus intelligents que leurs parents, dit Harry.
Ou beaucoup plus sains d'esprit, peut-être --
mon père pourrait probablement se montrer plus intelligent que moi, vous savez, s'il \emph{essayait} vraiment, au lieu d'utiliser son cerveau d'adulte pour trouver de nouvelles raisons de ne pas changer sa façon de penser -- Harry s'interrompit.
Je suis trop intelligent, professeure.
Je ne sais pas de quoi parler avec des enfants normaux.
Les adultes ne me respectent pas suffisamment pour me parler.
Et franchement, même s'ils le faisaient, ils n'auraient pas l'air aussi intelligent que Richard Feynman, donc autant lire ce qu'a écrit Richard Feynman.
Je suis \emph{isolé}, professeure McGonagall. J'ai été isolé toute ma vie.
Peut-être que cela induit en partie les mêmes effets que d'être enfermé dans une cave.
Et je suis trop intelligent pour lever les yeux vers mes parents comme sont censés le faire les enfants.
Mes parents m'aiment, mais ils ne se sentent pas tenus de répondre à la raison, et parfois j'ai la sensation que ce sont eux les enfants --
des enfants qui \emph{n'écoutent pas}, et qui ont une autorité absolue sur toute mon existence.
J'essaie de ne pas être trop amer à ce sujet, mais j'essaie aussi d'être \emph{honnête} avec moi-même, et donc, oui, je suis amer.
Et j'ai aussi du mal à gérer ma colère, mais j'y travaille.
Voilà, c'est tout.

--- \emph{C'est tout~?}~»

Harry acquiesça avec ferveur. «~C'est tout.
Professeure McGonagall, j'imagine que, même en Angleterre magique, l'explication normale mérite toujours d'être \emph{prise en compte}, non~?~»

\later

C'était plus tard dans la journée, le soleil commençait à descendre dans le ciel d'été, et le flot de passants à faiblir dans les rues.
Certains magasins avaient déjà fermé~; Harry et McGonagall avaient acheté les manuels scolaires chez Fleury et Bott juste avant la fermeture.
Il y avait seulement eu une légère commotion quand Harry avait foncé droit vers le rayon “Arithmancie” et avait découvert que les livres de septième année ne contenaient rien de plus mathématiquement avancé que de la trigonométrie.

À cet instant, cependant, l'esprit de Harry ne pensait plus à la recherche et à ses rêves de découvertes originales et faciles à atteindre.

À cet instant, Harry et McGonagall sortaient de chez Ollivander, et Harry fixait sa baguette.
Il venait de l'agiter, ce qui avait produit des étincelles multicolores.
Cela n'aurait pas vraiment dû le choquer particulièrement après tout ce qu'il avait déjà vu, mais malgré tout…

\emph{Je peux faire de la magie.}

\emph{Moi. Comme dans “Moi, personnellement.” Je suis magique~; je suis un sorcier.}

Il avait \emph{senti} la magie affluer dans son bras, et réalisé à cet instant qu'il avait toujours eu ce sens, qu'il l'avait possédé toute sa vie, le sens qui n'était ni la vue ni le son ni l'odorat ni le goût ni le toucher, mais simplement la magie.
C'était comme avoir des yeux, mais de les avoir toujours gardés fermés, si bien que vous ne vous ne vous rendiez même pas compte que vous voyiez du noir~;
et un jour les yeux se sont ouvert, et vous avez découvert le monde.
Le choc s'était déversé en lui, venant effleurer son être, le réveillant, et avait disparu ensuite en quelques secondes~;
ne laissant que la certitude qu'il était maintenant un sorcier, l'avait toujours été,
et étrangement, qu'il l'avait toujours su.

Et puis…

% Discarded translation.
% «~\emph{Il est en effet très curieux que vous soyez destiné à cette baguette sachant que sa sœur, oui, sa sœur est responsable de votre cicatrice.}~»
% Actual translation taken from first H. Potter book
«~\emph{Il est très étrange que ce soit précisément cette baguette qui vous ait convenu, car sa sœur n'est autre que celle qui... qui vous a fait cette cicatrice au front.}~»

Cela ne \emph{pouvait} pas être une coïncidence.
Il y avait des \emph{milliers} de baguettes dans ce magasin.
Bon, d'accord, cela \emph{pouvait} être une coïncidence, il y avait six milliards de personnes sur Terre, des coïncidences à une chance contre mille, cela arrivait tous les jours.
Mais le théorème de Bayes nous disait que toute hypothèse raisonnable
qui rendait \emph{plus} probable d'une chance sur mille que Harry se retrouve avec la sœur de la baguette du Seigneur des Ténèbres aurait une longueur d'avance.

McGonagall avait simplement dit \emph{comme c'est étrange} et en était restée là, ce qui avait provoqué chez Harry un état de choc face à la flagrante, à l'écrasante \emph{incuriosité} des sorciers et sorcières.
Harry n'aurait pu, dans aucun monde \emph{imaginable}, simplement faire «~Hmm~» et sortir du magasin sans même \emph{essayer} de trouver une hypothèse à ce qui venait de se passer.

Sa main gauche vint toucher sa cicatrice.

Qu'est-ce qui… \emph{exactement}…

«~Vous êtes maintenant un sorcier à part entière, dit McGonagall. Félicitations.~»

Harry hocha la tête.

«~Et que pensez-vous du monde magique~?

--- C'est étrange, dit Harry. Je devrais être en train de penser à tout ce que j'ai vu de la magie…
tout ce que je sais maintenant être possible, et tout ce que je sais maintenant être un mensonge, ainsi que tout le travail qui me reste à accomplir avant de la comprendre.
Et pourtant je me rends compte que je suis distrait par de relatives trivialités telles que… Harry baissa la voix, …toute cette histoire de “Survivant.”~»
Il ne semblait y avoir personne aux alentours, mais autant ne pas tenter le diable.

McGonagall s'éclaircit la voix. «~Vraiment~? Vous m'en direz tant.~»

Harry hocha la tête. «~Oui. C'est juste… \emph{étrange}.
Se rendre compte que vous faites partie de cette grande histoire, la quête pour vaincre le formidable et terrible Seigneur des Ténèbres, et c'est déjà \emph{fini}.
Terminé. Complètement réglé. Comme si vous étiez Frodon Sacquet, que vous appreniez que vos parents vous avaient emmené à la Montagne du Destin, qu'ils vous avaient fait jeter l'Anneau unique quand vous aviez un an et que vous ne vous en souveniez même pas.~»

Le sourire de McGonagall s'était plus ou moins figé.

«~Vous savez, si j'étais quelqu'un d'autre, n'importe qui d'autre, je serais probablement anxieux à l'idée de vivre à la hauteur de ce départ.
\emph{Ça alors, Harry, qu'avez-vous fait depuis que vous avez vaincu le Seigneur des Ténèbres~? Votre propre librairie~? C'est super~! Dites, vous saviez que j'ai appelé mon enfant comme vous~?}
Mais j'ai bon espoir que cela ne pose pas de problème.
Harry soupira.
Tout de même… j'en suis presque à espérer qu'il reste \emph{quelques} quêtes secondaires à terminer, juste pour que je puisse dire que j'ai vraiment, vous savez, \emph{participé} d'une certaine façon.

--- Oh~? dit McGonagall d'un ton étrange. Qu'aviez-vous à l'esprit~?

--- Eh bien par exemple, vous avez mentionné que mes parents ont été trahis. Qui les a trahis~?

--- Sirius Black~, dit McGonagall, sifflant presque le nom. Il est à Azkaban. Prison des sorciers.

--- Quelle est la probabilité que Sirius Black s'échappe de prison et que je doive le traquer et le vaincre dans un duel spectaculaire, ou encore mieux, que je mette une large prime sur sa tête et aille me cacher en Australie pendant que j'attends le résultat~?~»

McGonagall cligna des yeux. Deux fois. «~Peu probable. Personne ne s'est jamais échappé d'Azkaban, et je doute qu'\emph{il} soit le premier.~»

Harry était un peu sceptique de ce “\emph{personne} ne s'est \emph{jamais} échappé d'Azkaban.”
Mais bon, peut-être que grâce à la magie vous pouviez avoir une prison de proche de 100~\% de perfection, et encore plus si vous aviez une baguette et les autres non.
La meilleure façon d'en sortir serait de commencer par ne jamais y mettre les pieds.

«~Très bien, dit Harry. On dirait que tout a été bouclé.~» Il soupira en se frottant les cheveux.
«~Ou peut-être que le Seigneur des Ténèbres n'est pas \emph{vraiment} mort cette nuit-là.  Pas complètement.
Son esprit subsiste, chuchotant aux gens dans des cauchemars qui déteignent sur le monde éveillé, cherchant un moyen de revenir sur les terres des vivants qu'il a juré de détruire, et maintenant, en accord avec l'ancienne prophétie, lui et moi sommes destinés à un duel à mort où le gagnant perdra et le perdant gagnera…~»

La tête de McGonagall pivota, ses yeux scannêrent les alentours, à la recherche de personnes ayant pu entendre.

«~Je \emph{plaisante}, professeure~», dit Harry un peu agacé. Pfff, pourquoi prenait-elle toujours tout au sérieux…

Une sensation de lent effondrement commença à poindre au creux de l'estomac de Harry.

McGonagall regarda Harry d'un air calme. D'un air très, \emph{très} calme. Puis ajouta un sourire. «~Bien sûr que vous plaisantez, M. Potter.~»

\emph{Eh merde.}

Si Harry avait eu besoin de formaliser l'inférence muette qui venait de jaillir dans son esprit, il en serait sorti quelque chose comme~:
«~Si j'estime la probabilité que ce que je viens d'observer chez la professeure est le résultat d'un contrôle minutieux d'elle-même, par rapport à la distribution de probabilité de toutes les choses qu'elle ferait \emph{naturellement} si je faisais une mauvaise blague, alors ce comportement est un élément de preuve significatif qu'elle cache quelque chose.~»

Mais ce que Harry pensa en réalité fut~: \emph{Eh merde}.

Harry pivota sa propre tête pour scanner la rue. Non, personne dans le coin.  «~Il n'est \emph{pas} mort, c'est ça~? soupira Harry.

--- M. Potter…

--- Le Seigneur des Ténèbres est vivant. \emph{Bien sûr} qu'il est vivant.
C'était un \emph{acte} d'\emph{optimisme} total de ma part que d'avoir même \emph{rêvé} qu'il put en être autrement.
J'ai \emph{dû} perdre la raison, je n'arrive même pas à \emph{imaginer} à quoi je \emph{pensais}.
Juste parce que \emph{quelqu'un} a dit que son corps avait été retrouvé \emph{calciné}, comment ai-je pu penser qu'il était \emph{mort}.
J'ai \emph{clairement} encore beaucoup à apprendre sur l'art du véritable \emph{pessimisme}.

--- M. Potter…

--- Dites-moi au moins qu'il n'y a pas véritablement de prophétie…~» McGonagall lui renvoyait toujours ce sourire radieux et figé. «~Oh, bon sang, mais dites moi que c'est une \emph{blague}.

--- M. Potter, vous ne devriez pas inventer de choses si inquiétantes.

--- C'est \emph{vraiment} \emph{ça} que vous voulez me dire~?
Vous imaginez ma réaction dans quelques temps, quand j'apprendrai que finalement, oui, il y avait des choses inquiétantes après tout.~»

Le sourire de McGonagall se flétrit.

Les épaules de Harry s'affaissèrent. «~J'ai un monde entier de magie à analyser. Je n'ai \emph{pas} de temps à consacrer à ça.~»

Puis les deux se turent, tandis qu'un homme vêtu d'amples habits oranges et flottants apparaissait dans la rue et les dépassait lentement.
McGonagall le suivit discrètement des yeux.
Harry se mordillait la lèvre inférieure. Fortement.
Quelqu'un l'observant de près aurait pu voir un léger point de sang apparaître.

Lorsque l'homme en orange fut suffisamment éloigné, Harry parla à nouveau, dans un murmure.
«~Allez-vous me dire la vérité à présent, professeure McGonagall~? Et n'essayez pas de prétendre qu'il n'y a rien, je ne suis pas stupide.

--- Vous avez \emph{onze ans}, M. Potter~! chuchota-t-elle sévèrement.

--- Et par conséquent sous-humain. Pardon… pendant un instant, j'avais \emph{oublié}.

--- Ce sont des affaires redoutables et importantes~! Ce sont des \emph{secrets}, M. Potter~!
C'est déjà une \emph{catastrophe} que vous en sachiez autant, alors que vous êtes un enfant~!
Vous ne devez le dire à \emph{personne}, vous comprenez~? Absolument personne~!~»

Et, comme cela arrivait parfois quand Harry se mettait \emph{suffisamment} en colère, son sang se refroidit au lieu de s'échauffer, et une clarté sinistre envahit son esprit, listant toutes les tactiques possibles et évaluant les conséquences avec un réalisme implacable.

\emph{Fais remarquer que tu as le droit de savoir~: Échec. Les enfants de onze ans n'ont aucun droit à savoir quoi que ce soit, aux yeux de McGonagall.}

\emph{Dis que tu ne seras plus son ami~: Échec. Elle n'accorde pas assez de valeur à ton amitié.}

\emph{Fais remarquer que tu seras en danger si tu ne sais pas~: Échec.
Des plans ont déjà été pensés, basés sur ton ignorance.
Le désagrément} certain \emph{de devoir changer les plans lui semblera bien moins digeste que la perspective} incertaine \emph{que tu finisses blessé.}

\emph{La justice et la raison échoueront. Tu dois soit trouver quelque chose que tu as et qu'elle veut, soit trouver quelque chose que tu peux faire et qu'elle craint…}

Ah.

«~Très bien, professeure, dit Harry d'une voix basse et glaciale, on dirait que j'ai quelque chose que vous désirez.
Vous pouvez, si vous le souhaitez, me dire la vérité, \emph{toute} la vérité, en échange de quoi je garderai vos secrets.
Ou vous pouvez essayer de me garder dans l'ignorance pour pouvoir m'utiliser comme un pion, auquel cas je ne vous devrai rien.~»

McGonagall s'arrêta net au milieu de la rue. Ses yeux flamboyèrent et sa voix se transforma en un sifflement.

«~Comment osez-vous~!

--- \emph{Comment osez-vous~!} chuchota-t-il en retour.

--- Vous me feriez \emph{chanter}~?~»

Les lèvres de Harry se pincèrent.
«~Je vous \emph{offre} une \emph{faveur}. Je vous \emph{donne} une chance de protéger \emph{votre} précieux secret.
Si vous refusez, j'aurais \emph{tous} les motifs du monde pour aller poser des questions ailleurs, non par rancune envers vous, mais parce que j'ai \emph{besoin de savoir}~!
Dépassez votre colère futile envers un \emph{enfant} qui, selon vous, se doit de vous obéir, et vous comprendrez que tout adulte sain d'esprit ferait de même~!
\emph{Regardez les choses de mon point de vue~! Comment vous sentiriez-vous si c'était VOUS~?}~»

Harry observa McGonagall, elle respirait fortement.
Il se rendit compte qu'il était temps d'adoucir la pression, de la laisser mijoter un moment.
«~Vous n'avez pas à décider tout de suite, continua Harry d'un ton plus normal.
Je comprendrai si vous avez besoin de plus de temps pour réfléchir à mon \emph{offre}… mais je vous préviens tout de suite~», la voix de Harry se fit plus froide.
«~N'essayez pas ce sortilège d'amnésie sur moi.
Il y a quelque temps, j'ai élaboré un signal, et je me le suis déjà envoyé à moi-même.
Si je trouve ce signal et que je ne me \emph{souviens} pas l'avoir envoyé…~»
La phrase qu'Harry avait laissée en suspens était lourde de sens.

Le visage de McGonagall peinait à suivre tous les changements d'expression de la sorcière.

«~Je… je ne pensais pas à vous Oublietter, M. Potter… mais pourquoi auriez-vous \emph{inventé} un signal si vous ne connaissiez pas l'existence de…

--- J'y ai pensé en lisant un livre de science-fiction Moldu, et je me suis dit, \emph{bon, juste au cas où}… Et non, je ne vous dirai pas quel est le signal, je ne suis pas stupide.

--- Je ne comptais pas vous le demander~», dit McGonagall. Elle semblait se replier sur elle-même, et parut soudain très vieille et très fatiguée.
«~La journée a été épuisante, M. Potter. Pouvons-nous aller chercher votre malle et vous raccompagner à la maison~?
Je vais vous faire confiance pour garder le silence sur cette affaire avant que j'aie eu le temps d'y réfléchir.
Gardez à l'esprit qu'il n'y a que deux autres personnes au monde qui sont au courant, le directeur Albus Dumbledore et le professeur Severus Rogue.~»

Tiens. De nouvelles informations~; c'était une offre de paix. Harry acquiesça, tourna la tête pour regarder devant lui et recommença à marcher, tandis que son sang revenait lentement à sa température habituelle.

«~Donc maintenant je dois trouver un moyen de tuer un mage noir immortel, dit Harry, qui soupira de frustration. J'aurais vraiment aimé que vous me disiez ça \emph{avant} qu'on commence à faire du shopping.~»

\later

Le magasin de malles était plus richement décoré que tout autre magasin que Harry avait jamais vu~;
les rideaux étaient luxueux et ornés de motifs délicats, le sol et les murs en bois teinté, et les malles occupaient les places d'honneur sur des estrades en ivoire poli.
Le vendeur avait une tenue d'une qualité seulement un cran en dessous de celle de Lucius Malfoy, et il s'adressait tant à Harry qu'à McGonagall avec une politesse raffinée et mielleuse.

Harry avait posé ses questions, et gravité vers une malle de bois dense, non poli mais chaud et solide, gravée avec le motif d'un dragon gardien dont les yeux se déplaçaient pour regarder toute personne s'approchant d'elle.
Une malle enchantée pour être légère, réduire de taille sur commande, faire pousser des petits tentacules munis de griffes de sa base et se tortiller pour suivre son maître.
Une malle avec deux tiroirs sur chacun de ses quatre côtés qui s'ouvraient pour révéler chacun un compartiment aussi profond que la malle entière.
Un couvercle équipé de quatre cadenas, chacun d'entre eux révélant un espace intérieur différent.
Et -- c'était la partie importante -- une poignée sur le fond permettant de faire coulisser un panneau et révéler un escalier, lequel descendait vers une petite pièce éclairée qui pourrait contenir, selon une estimation rapide de Harry, environ douze étagères.

S'ils faisaient des bagages comme ça, Harry ne comprenait pas pourquoi qui que ce soit s'embêtait à posséder une maison.

Cent huit gallions d'or. C'était le prix d'une bonne malle, assez peu utilisée.
À cinquante livres sterling le gallion, c'était suffisant pour acheter une voiture d'occasion.
Cela ferait plus cher que la somme de tout ce que Harry avait pu acheter de toute sa vie.

Quatre-vingt-dix-sept gallions. C'était ce qui restait dans le sac d'or que Harry avait été autorisé à retirer de chez Gringotts.

McGonagall avait l'air chagrinée.
Après une longue journée de shopping elle n'avait pas eu besoin de demander à Harry combien d'or il restait dans le sac après que le vendeur eut donné le prix, ce qui voulait dire que la professeure était capable de calculer mentalement sans crayon ni papier.
À nouveau, Harry se rappela à lui-même que \emph{scientifiquement illettré} n'était pas du tout la même chose que \emph{stupide}.

«~Je suis désolée, jeune homme, dit McGonagall. C'est entièrement de ma faute. Je vous proposerais bien de vous ramener à Gringotts, mais la banque est à présent fermée pour tout ce qui n'est pas service d'urgence.~»

Harry regarda McGonagall, se demandant…

«~Eh bien, soupira-t-elle, alors qu'elle pivotait sur son talon, autant partir, je suppose.~»

…elle \emph{n'avait pas} complètement disjoncté lorsqu'un enfant avait osé la défier.
Elle n'avait pas été contente, mais elle avait \emph{réfléchi} au lieu d'exploser de rage.
C'était peut-être simplement parce qu'il y avait un Seigneur des Ténèbres immortel à combattre -- elle avait besoin de la bonne volonté de Harry.
Mais la plupart des adultes n'auraient pas été capables de réfléchir aussi loin~;
ils n'auraient même pas pris le temps de penser aux \emph{conséquences futures}, si une personne de rang inférieur avait refusé de leur obéir…

«~Professeure~?~» commença Harry.

McGonagall se retourna pour le regarder.

Harry prit une profonde inspiration.
Il avait besoin d'être un peu en colère pour ce qu'il voulait maintenant essayer, autrement jamais il n'aurait le courage de le faire.
\emph{Elle ne m'a pas écouté}, se dit-il dans sa tête, \emph{j'aurais pris plus d'or, mais elle n'a pas voulu m'écouter}…
Il focalisa toute sa concentration sur McGonagall et sur la nécessité qu'il avait de maîtriser la direction de la conversation.

«~Professeure, vous pensiez que cent gallions seraient plus que suffisants pour une malle.
C'est pourquoi vous n'avez pas pris la peine de me prévenir avant qu'il n'en reste que quatre-vingt-dix-sept.
C'est précisément ce genre de choses que montre la recherche --
voilà ce qui se passe quand les gens pensent qu'ils se donnent une \emph{petite} marge de manœuvre.
Ils ne sont pas suffisamment pessimistes.
Si cela avait été ma décision, j'aurai pris \emph{deux cents} gallions, juste pour être tranquille.
Il y avait largement assez d'argent dans cette chambre forte, et j'aurai pu y remettre l'excédent plus tard.
Mais j'ai pensé que vous ne me laisseriez pas le faire.
J'ai pensé que vous vous fâcheriez contre moi si je ne faisais même que le demander.
Ai-je eu tort~?

--- Je suppose que je dois admettre que vous avez raison, dit McGonagall, cependant, jeune homme…

--- C'est exactement pour ce genre de raisons qu'il m'est difficile de faire confiance aux adultes, Harry empêcha sa voix de trembler,
car il se mettent en colère sitôt que vous \emph{essayez} de les raisonner.
Pour eux, c'est de la provocation, de l'insolence, et une remise en cause de leur statut tribal supérieur.
Si vous essayez de leur parler, ils se mettent en \emph{colère}.
Donc, si j'avais quelque chose de \emph{vraiment important} à faire, je n'arriverais pas à vous faire confiance.
Même si vous écoutiez avec une attention profonde ce que j'avais à dire -- parce que cela fait aussi partie du \emph{rôle} de quelqu'un qui joue à l'adulte attentionné -- jamais vous ne changeriez vos actions, vous ne vous comporteriez pas différemment en réaction à quelque chose que j'aurais dit.~»

Le vendeur les observait tous les deux avec une fascination non dissimulée.

«~Je peux comprendre votre point de vue, finit par dire McGonagall.
Si je vous parais parfois trop stricte, n'oubliez pas que j'ai été directrice de la maison Gryffondor pendant ce qui me semble être plusieurs milliers d'années.~»

Harry acquiesça et poursuivit. «~Donc -- supposons que j'aie un moyen d'obtenir plus de gallions de ma chambre forte \emph{sans} que nous retournions à Gringotts, mais que cela implique que je déroge à mon rôle d'enfant docile.
Pourrais-je vous mettre dans la confidence, même si vous deviez sortir de votre propre rôle de professeure McGonagall pour en profiter ?

--- \emph{Quoi ?} dit McGonagall.

--- En d'autres termes, si je pouvais faire en sorte que la journée d'aujourd'hui se soit déroulée différemment, faire que nous n'ayons \emph{pas} retiré trop peu d'argent, est-ce que ce serait acceptable alors même que cela impliquerait qu'un enfant ait été insolent envers un adulte rétrospectivement ?"

--- Je... suppose...~», dit McGonagall, l'air perplexe.

Harry sortit sa bourse en peau de Moke et dit «~Onze gallions provenant de ma chambre forte familiale~».

Harry tenait maintenant de l'or dans sa main.

La bouche de McGonagall resta un instant béante, puis sa mâchoire se referma brusquement et  ses yeux se rétrécirent, jetant à Harry un regard inquisiteur. «~\emph{Où avez-vous obtenu cet…}

--- De ma chambre forte, comme je viens de le dire.

--- \emph{Comment~?}

--- \emph{Magie.}

--- Ce n'est pas une réponse~! s'écria McGonagall, avant de s'arrêter en clignant des yeux.

--- Non, ce n'en est pas une, n'est-ce pas ?
Je \emph{devrais} prétendre que c'est parce que j'ai découvert expérimentalement les véritables secrets du fonctionnement de la bourse et qu'elle peut en fait récupérer des objets de n'importe où, et pas uniquement de son propre intérieur, si vous formulez la demande correctement.
Mais en réalité, c'est quand je suis tombé dans ce tas d'or tout à l'heure, j'ai fourré quelques gallions dans ma poche.
Quiconque comprenant le pessimisme sait que l'argent est une chose dont on peut avoir besoin rapidement et sans préavis.
Donc, êtes-vous maintenant en colère parce que j'ai défié votre autorité~?
Ou contente que notre journée se termine par le succès de cette important mission~?

Les yeux du vendeur étaient écarquillés comme des soucoupes.

McGonagall se tenait là, silencieuse.

«~La discipline \emph{doit} être respectée à Poudlard, dit-elle après qu'une minute entière se fut écoulée. Pour le bien de \emph{tous} les élèves. Et cela \emph{doit} inclure la courtoisie et l'obéissance de votre part envers \emph{tous} les professeurs.~»

--- Je comprends, professeure McGonagall.

--- Bien. Maintenant, achetons cette malle et rentrons à la maison.~»

Harry avait envie d'applaudir, ou vomir, ou s'évanouir, ou \emph{quelque chose}.
C'était la première fois que son raisonnement rigoureux avait jamais fonctionné sur \emph{qui que ce soit}.
Peut-être aussi parce que c'était la première fois qu'il possédait quelque chose de vraiment sérieux dont un adulte avait besoin, mais tout de même…

Minerva McGonagall, +1 point.

Harry s'inclina, et déposa le sac d'or et les onze gallions supplémentaires dans les mains de la sorcière.
«~Merci beaucoup, professeure. Pouvez-vous terminer l'achat pour moi~?
Il faut que j'aille aux toilettes.~»

Le vendeur, de nouveau onctueux, pointa du doigt en direction d'une porte munie d'une poignée en or.
Alors que Harry s'éloignait, il entendit le vendeur derrière lui demander de sa voix mielleuse~:
«~Puis-je m'informer de l'identité de cette personne, Madame McGonagall~? J'imagine qu'il est Serpentard -- troisième année, peut-être~? -- et d'une famille éminente, mais je n'ai pas reconnu…~»

Harry n'entendit pas la fin de la phrase, coupée par la porte des toilettes qui s'était refermée.
Il identifia le loquet, verrouilla la porte, attrapa la serviette magiquement auto-nettoyante et, les mains tremblantes, essuya son front humide.
Son corps entier était baigné d'une sueur qui avait imprégné ses vêtements Moldus, mais au moins cela ne se voyait pas à travers ses vêtements de sorcier.

\later

Le soleil commençait à se coucher et il était effectivement très tard au moment où il se retrouvèrent pour la deuxième fois de la journée dans l'arrière-cour du Chaudron Baveur, petite interface couverte de feuilles mortes entre le Chemin de Traverse de l'Angleterre magique et le monde entier Moldu.
(C'était une économie \emph{extrêmement} découplée…) 
Il était convenu qu'Harry appellerait son père depuis une cabine téléphonique une fois de l'autre côté.
Il était apparemment inutile de s'inquiéter que son bagage soit volé.
Sa malle avait le statut d'objet magique majeur, ce qui impliquait que la plupart des Moldus étaient incapables de la remarquer~;
voilà, entre autres, ce que vous pouviez obtenir dans le monde magique, si vous étiez prêt à payer le prix d'une voiture d'occasion.

«~C'est ici que nos chemins se séparent, pour un temps~», dit McGonagall. Elle secoua la tête avec émerveillement.
«~C'était le jour le plus étrange de ma vie depuis… bien des années.
Depuis le jour où j'ai appris qu'un enfant avait vaincu Vous-Savez-Qui.
Je me demande maintenant, rétrospectivement, si c'était le dernier jour sensé de ce monde.~»

Oh, comme si \emph{elle} avait à se plaindre de quoi que ce soit. \emph{Vous pensez que votre journée était surréelle~? Essayez la mienne pour voir}.

«~Vous m'avez grandement impressionné aujourd'hui, lui dit Harry. J'aurais dû penser à vous complimenter à voix haute, je vous attribuais des points dans ma tête et tout et tout.

--- Merci, M. Potter, dit McGonagall. Si vous aviez déjà été placé dans une Maison je vous aurais retiré tellement de points que vos petits-enfants en seraient encore à perdre la Coupe des Quatre Maisons.

--- Merci à \emph{vous}, professeure.~» C'était probablement encore un peu tôt pour l'appeler Minnie.

Cette femme était peut-être l'adulte le plus sain d'esprit que Harry ait jamais rencontré, en dépit de son manque de culture scientifique.
Harry envisageait même de lui offrir la position de numéro deux dans le groupe qu'il ne manquerait pas de former pour combattre le Seigneur des Ténèbres, mais il se gardait bien de le dire à voix haute.
\emph{Hmmm, quel serait un bon nom pour ce groupe…~? Les Mange-Mangemorts~?}

«~Je vous reverrai très bientôt, à la rentrée, dit McGonagall. Et, M. Potter, à propos de votre baguette…

--- Je sais ce que vous allez me demander~», dit Harry.
Il sortit sa précieuse baguette et, avec un profond pincement au cœur, la retourna dans sa main pour la présenter par le manche.
«~Prenez-la. Je n'avais pas prévu de faire quoi que ce soit, pas la moindre petite chose, mais je ne veux pas que vous ayez des cauchemars dans lesquels je fais exploser ma maison.~»

McGonagall secoua vivement la tête. «~Oh, non, M. Potter~! Il ne s'agit pas de ça.
Je voulais juste vous prévenir de ne pas \emph{utiliser} votre baguette chez vous, le ministère de la Magie peut détecter les mineurs qui utilisent la magie, et c'est interdit sans supervision.

--- Ah, dit Harry. Cela me semble être une règle très sensée. Je suis ravi de voir que le monde magique prend ce genre de choses au sérieux.~»

McGonagall le regarda intensément.

«~Vous le pensez vraiment.

--- Oui, dit Harry. J'ai compris. La magie c'est dangereux et les règles sont là pour une bonne raison. Il y a aussi certains sujets qui sont dangereux.
Je l'ai compris également. Souvenez-vous que je ne suis pas stupide.

--- Il y a bien peu de chances que je l'oublie jamais. Merci, Harry, cela m'aide à me sentir mieux concernant certaines choses que je vous ai confiées. Au revoir à présent.~»

Harry se retourna pour partir, rentrer dans le Chaudron Baveur et ressortir dans le monde Moldu.

Alors que sa main touchait la poignée de la porte, il entendit un dernier murmure derrière lui.

«~Hermione Granger.

--- Quoi~? dit Harry, la main toujours sur la porte.

--- Cherchez une fille de première année nommée Hermione Granger dans le train pour Poudlard.

--- Qui est-elle~?~»

Il n'y eut pas de réponse, et quand Harry se retourna, McGonagall était partie.

\latersection{Épilogue~:}

Le directeur Albus Dumbledore se pencha par-dessus son bureau. Ses yeux pétillants dévisagèrent Minerva. «~Alors, ma chère, comment avez-vous trouvé Harry~?~»

Minerva ouvrit la bouche. Puis elle ferma la bouche. Puis elle ouvrit à nouveau la bouche. Aucun mot ne sortit.

«~Je vois, dit Albus avec gravité. Merci pour votre rapport, Minerva. Vous pouvez disposer.~»

%  LocalWords:  ome zahav ahava Aaaaaaarrrgh QX31 ahemmed Sheesh
%  LocalWords:  Aw
