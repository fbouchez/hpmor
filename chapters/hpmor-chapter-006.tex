\chapter{L'illusion de la planification}

\lettrine{C}{\emph{ertains}} enfants auraient attendu \emph{après} leur premier voyage au Chemin de Traverse.

«~Sac d'élément 79~», dit Harry, et il retira sa main, vide, de la bourse en peau de Moke.

La plupart des enfants auraient au moins attendu d'avoir leur \emph{baguette magique}.

«~Sac d'\emph{okane}~», dit Harry. Le lourd sac d'or apparut dans sa main.

Harry sortit le sac, puis le plongea à nouveau dans la bourse. Il sortit sa main, la remit à l'intérieur, et dit~: «~sac de jetons d'échanges financiers.~» Cette fois-ci sa main ressortit vide.

«~Rends-moi le sac que je viens juste de mettre.~» Une fois de plus le sac sortit de la bourse.

Harry James Potter-Evans-Verres avait mis la main sur au moins un objet magique. Pourquoi attendre~?

«~Professeure McGonagall, dit Harry à la sorcière déconcertée qui cheminait à ses côtés, pourriez-vous me donner deux mots, un qui signifie or, et un autre signifiant autre chose n'étant pas de l'argent, le tout dans une langue que je connais pas~? Mais ne me dites pas qui est quoi.

--- \emph{Ahava} et \emph{zahav}, dit McGonagall. C'est de l'hébreu, et l'autre mot veut dire amour.

--- Merci, Professeure. Sac d'\emph{ahava}.~» Vide.

«~Sac de \emph{zahav}.~» Et il apparut dans sa main.

«~Zahav veut dire or~?~» s'enquit Harry, et McGonagall hocha la tête.

Harry contempla les données expérimentales qu'il avait recueillies.
C'était un essai des plus bruts et préliminaires, mais c'était suffisant pour confirmer au moins une conclusion~:

«~\emph{Aaaaaarrrgh ça n'a aucun sens~!}~»

La sorcière à ses côtés leva un auguste sourcil.

«~Des problèmes, M. Potter~?

--- Je viens de falsifier chacune des hypothèses que j'avais~!
Comment la bourse peut-elle savoir que “sac de 115 gallions” est valide, mais pas “sac de 90 plus 25 gallions”~? Elle peut \emph{compter}, mais pas \emph{additionner}~? Elle peut comprendre les noms, mais pas les syntagmes nominaux de même sens~?
La personne qui l'a fabriquée ne parlait probablement pas japonais et \emph{je} ne parle pas hébreux, donc la bourse n'utilise pas \emph{son} savoir ni \emph{mon} savoir…~» Harry agita une main avec impuissance. «~Les règles paraissent \emph{en gros} cohérentes, mais elles ne \emph{veulent rien dire}~! Et je ne vais même pas commencer à m'interroger sur la façon dont une \emph{bourse} peut être équipée d'une reconnaissance vocale et d'une compréhension du langage naturel, alors que les meilleurs programmeurs en intelligence artificielle n'arrivent toujours pas à faire cette prouesse sur les supercalculateurs les plus rapides, après trente-cinq ans de dur labeur.~»
Harry repris sa respiration, «~mais \emph{comment} est-ce \emph{possible}~?

--- Magie, dit McGonagall.

--- C'est juste un \emph{mot}~! Même après m'avoir dit ça, je ne peux pas faire de nouvelles prédictions. C'est exactement comme de dire “phlogiston” ou “élan vital” ou “émergence” ou “complexité”~!~»

La sorcière en robe noire rit à haute voix. «~Mais c'\emph{est} de la magie, M. Potter.~»

Harry s'affaissa un peu.

«~Avec tout le respect que je vous dois, professeure McGonagall, je ne suis pas tout à fait sûr que vous compreniez que ce j'essaie de faire ici.

--- Avec tout le respect que je vous dois, M. Potter, je suis tout à fait sûre de ne pas le comprendre. À moins que -- c'est juste une supposition -- vous ne soyez en train d'essayer de conquérir le monde~?

--- Non~! Je veux dire oui -- enfin, \emph{non}~!

--- Je pense que je devrais peut-être m'alarmer de votre difficulté à répondre à cette question.~»

Morose, Harry se remémora la conférence de Dartmouth sur l'intelligence artificielle de 1956. Elle avait été la première conférence jamais organisée sur ce sujet, celle qui avait créé l'expression «~intelligence artificielle~». Ils avaient identifié des problèmes clés, tels que faire en sorte que les ordinateurs comprennent le langage, apprennent, et s'améliorent par eux-mêmes, et avaient suggéré, avec un parfait sérieux, que des progrès significatifs pourraient être accomplis par une dizaine scientifiques travaillant ensemble durant deux mois.

\emph{Non. Relève la tête. Tu viens à peine de} commencer \emph{à essayer d'élucider les secrets de la magie. Tu ne peux pas} savoir \emph{si cela va être trop difficile pour le faire en deux mois.}

«~Et vous n'avez \emph{vraiment} jamais entendu parler d'autres sorciers posant ce genre de questions ou faisant ce genre d'expériences scientifiques~?~» demanda à nouveau Harry.
Cela lui semblait tellement \emph{évident}.

Mais après tout, il avait fallu attendre plus de deux cents ans \emph{après} l'invention de la méthode scientifique pour qu'un Moldu pense à étudier de façon systématique quelles phrases un \emph{humain de quatre ans} pouvait ou ne pouvait pas comprendre.
La psychologie du développement du langage aurait pu être découverte au dix-huitième siècle, du moins en principe, mais personne n'avait jamais pensé à regarder avant le vingtième. Donc vous ne pouviez pas vraiment blâmer le monde magique, bien plus petit, si personne n'avait encore étudié le sort de Récupération.

McGonagall pinca les lèvres puis haussa les épaules.
«~Je ne suis toujours pas certaine de ce que vous entendez par “expérience scientifique,” M. Potter.
Comme je l'ai dit, j'ai vu des élèves nés-Moldus essayer de faire fonctionner la science moldue à Poudlard, et de nouveaux sortilèges et nouvelles potions sont inventés chaque année.~»

Harry secoua la tête.
«~La technologie et la science ne sont pas du tout la même chose.
Essayer de faire quelque chose de plein de façons différentes n'a rien voir avec le fait d'expérimenter pour comprendre les règles.~»
Beaucoup de gens avaient tenté d'inventer des machines volantes en essayant plein de trucs-avec-des-ailes, mais seuls les frères Wright avaient construit une soufflerie pour mesurer la portance…
«~Hmm, combien d'enfants élevés par des Moldus entrent à Poudlard chaque année~?

--- Environ dix~?~»

Harry faillit trébucher sur ses propres pieds. «~\emph{Dix}~?~»

Le monde Moldu avait une population de plus de six milliards d'individus.
S'il n'y avait qu'une personne comme vous sur un million, alors cela en faisait sept à Londres, et mille de plus en Chine.
Il était inévitable que parmi la population Moldue apparaisse \emph{quelques} enfants de onze ans capables de résoudre des équations différentielles --
Harry savait qu'il n'était pas le seul.  Il avait rencontré d'autres prodiges lors de compétitions de mathématiques.
À vrai dire s'était fait battre à plates coutures par des concurrents qui passaient probablement littéralement \emph{toutes leurs journées} à s'entraîner sur des problèmes de mathématiques et qui n'avaient \emph{jamais} lu de livre de science-fiction et qui s'épuiseraient \emph{complètement} avant leur \emph{puberté} et ne feraient \emph{jamais} rien de leur vie future parce qu'ils avaient simplement appliqué des techniques \emph{connues} au lieu d'apprendre à penser de façon \emph{créative}. (Harry était du genre mauvais perdant.)

Mais… dans le monde des sorciers…

Dix enfants de Moldus par an qui arrêtent tous leur éducation Moldue à l'âge de onze ans~?
Et McGonagall, même si elle n'était peut-être pas objective, avait affirmé que Poudlard était la plus grande et la plus prestigieuse des écoles de magie du monde… et la formation s'arrêtait à l'âge de dix-sept ans.

McGonagall connaissait sans aucun doute dans les moindres détails la façon dont on pouvait se transformer en chat.
Mais elle semblait n'avoir littéralement jamais \emph{entendu} parler de la méthode scientifique.
Pour elle, c'était juste la magie des Moldus.
Et elle ne semblait même pas \emph{curieuse} de découvrir quels mystères se cachaient derrière la compréhension du langage naturel que possédait le sort de Récupération.

Il ne restait donc plus vraiment que deux possibilités.

Possibilité numéro un~: La magie était si incroyablement opaque, alambiquée et impénétrable que même si les sorciers et sorcières avaient fait de leur mieux pour essayer de la comprendre, ils n'avaient fait que peu, voire aucun progrès et avaient fini par laisser tomber~; et Harry ne ferait pas mieux.

\emph{Ou alors…}

Harry fit craquer ses phalanges avec détermination, mais cela ne fit qu'un petit claquement discret, au lieu de rebondir sinistrement en écho sur les murs du Chemin de Traverse.

Possibilité numéro deux~: il allait conquérir le monde.

Un jour ou l'autre. Peut-être pas tout de suite.

Il \emph{pouvait} arriver que ce genre de chose prenne plus de deux mois.
La science Moldue ne s'était pas rendue sur la lune une semaine après Galilée.

Harry n'arrivait cependant pas à empêcher son sourire de s'étendre jusqu'aux oreilles, à tel point que ses joues commençaient à lui être douloureuses.

Harry avait toujours été anxieux à l'idée de finir comme un de ces enfants prodiges qui n'aboutissent jamais à rien et passent le reste de leur vie à se vanter d'à quel point ils étaient en avance à l'âge de dix ans.
Cela dit, la plupart des génies adultes n'aboutissent à rien non plus.
Il y a probablement mille personnes aussi intelligentes qu'Einstein pour chaque Einstein de l'Histoire.
Mais ces autres génies n'avaient pas mis la main sur la seule chose dont vous aviez absolument besoin pour atteindre la vraie grandeur: ils n'avaient jamais trouvé de problème important.

\emph{Tu m'appartiens désormais}, Harry pensa aux murs du Chemin de Traverse, à tous les magasins et leurs articles, à tous les commerçants et leurs clients~; et à toutes les terres et tous les habitants de l'Angleterre magique, et au monde magique plus vaste encore~; et à l'univers tout entier, bien moins compris par les scientifiques Moldus que ce qu'ils croyaient.
\emph{Moi, Harry James Potter-Evans-Verres, revendique ce territoire au nom de la Science}.

Les éclairs et le tonnerre échouèrent complètement à jaillir et gronder dans le ciel sans nuage.

«~Qu'est-ce qui vous fait sourire~? s'enquit McGonagall avec méfiance et lassitude.

--- Je me demande s'il existe un sort permettant de faire jaillir des éclairs en arrière-plan à chaque fois que je prends une résolution sinistre~», expliqua Harry.
Il était en train de soigneusement mémoriser les mots exacts de sa résolution sinistre afin que les futurs livres d'histoire ne se trompent pas.

«~J'ai un sentiment prononcé que je devrais faire quelque chose à ce sujet, soupira McGonagall.

--- Ignorez-le, ça partira. Oh, ça brille~!~» Harry mit ses pensées de conquête mondiale en attente et sautilla vers un magasin à la devanture ouverte, et McGonagall suivit.

\later

Harry avait maintenant acheté ses ingrédients pour les potions, un chaudron, et, oh, quelques petites choses supplémentaires.
Des objets qu'il lui semblait utile de transporter dans son sac sans fond (alias la Super Bourse en Peau de Moke QX31 avec sort d'Extension Indétectable, sort de Récupération, et Ouverture Élargissante).
Des achats intelligents et judicieux.

Harry ne comprenait honnêtement pas pourquoi McGonagall avait l'air si \emph{méfiante}.

Harry se trouvait actuellement dans une boutique suffisamment luxueuse pour exposer dans la rue principale et sinueuse du Chemin de Traverse.
La devanture du magasin était ouverte et la marchandise disposée sur des étals de bois inclinés, sous la surveillance de légères lueurs grises et d'une jeune vendeuse vêtue d’une version fortement raccourcie de la classique robe de sorcière, révélant ses coudes et genoux.

Harry examinait l'équivalent magique d'un kit de premier soins, le Kit de Soins d'Urgence Plus.
Il y avait deux garrots auto-serrants.
Une seringue de ce qui semblait être du feu liquide, et était supposée considérablement réduire la circulation sanguine dans la zone traitée tout en continuant à oxygéner le sang pendant trois minutes maximum, si vous aviez besoin d'empêcher un poison de se répandre dans le reste du corps.
Un tissu blanc dont ont pouvait envelopper une partie du corps pour atténuer temporairement la douleur.
Plus toute une quantité d'autres objets dont l'utilité échappait complètement à Harry, tels que le “Traitement contre l'Exposition aux Détraqueurs,” qui sentait et ressemblait à du chocolat ordinaire.
Ou un “anti-Morsmental,” qui ressemblait à un petit œuf tremblotant accompagné d'une notice expliquant comment l'enfoncer dans la narine de quelqu'un.

«~Une affaire pour cinq gallions, qu'en dites-vous~?~» dit Harry à McGonagall, et la jeune vendeuse, qui l'observait non loin, hocha la tête avec enthousiasme.

Harry s'était attendu à ce que la professeure fasse une remarque approbatrice sur sa prudence et son sens de la préparation.

Ce qu'il reçut à la place ne pouvait être décrit que d'une seule manière~: un regard-qui-tue.

«~Et \emph{pourquoi} donc, dit McGonagall d'une voix chargée de scepticisme, vous attendriez-vous à avoir \emph{besoin} d'un kit de soins, jeune homme~?~»
(Après le regrettable incident au magasin de potions, McGonagall essayait d'éviter de dire «~M. Potter~» lorsque quelqu'un se trouvait non loin.)

La bouche de Harry s'ouvrit puis se ferma.
«~Je ne m'\emph{attends} pas à en avoir besoin~! C'est juste au cas où~!

--- Juste au cas où \emph{quoi}~?~»

Harry écarquilla les yeux. «~Vous pensez que j'ai \emph{prévu} de faire quelque chose de dangereux, et que c'est pour \emph{ça} que je veux un kit médical~?~»

Harry reçut pour toute réponse un regard empli de sombre suspicion et d'incrédulité ironique.

«~Nom de Zeus~!~» dit Harry. (C'était une expression qu'il avait récupérée du savant fou Doc Brown de \emph{Retour vers le Futur}.) «~C'est aussi ce que vous pensiez quand j'ai acheté la potion chute-de-plume, la branchiflore, et la bouteille de pilules nutritives et de gélules d'eau~?

--- Oui.~»

Harry secoua la tête avec stupéfaction. «~Mais quel genre de plan pensez-vous donc que je suis en train de mettre en place~?

% Et quelle sorte de plan pensez-vous que j'ai \emph{mis} \emph{en route}~?

--- Je ne sais pas, dit sombrement McGonagall, mais ça se termine soit par une livraison d'une tonne d'argent à Gringotts, soit en domination mondiale.

--- “Domination mondiale” est une expression si laide. Je préfère l'appeler “optimisation mondiale.”~»

Cette blague hilarante échoua totalement à rassurer la sorcière et son regard du jugement dernier.
% de l'apocalypse.
% Cela échoua à rassurer le professeur McGonagall, qui lui donnait toujours le Regard de la Mort.

«~Waouh, dit Harry se rendant compte qu'elle était sérieuse. Vous le pensez vraiment. Vous pensez vraiment que je prévois de faire quelque chose de dangereux.

--- Oui.

--- Parce que c'est l'\emph{unique} raison pour laquelle qui que ce soit achèterait jamais un kit de premiers soins~?
Ne le prenez pas mal, professeure McGonagall, mais \emph{de quelle sorte d'enfants cinglés avez-vous l'habitude de vous occuper}~?

--- Des Gryffondor~», lâcha McGonagall. Le mot portait une charge d'amertume et de désespoir s'abattant comme une malédiction éternelle jetée sur tout enthousiasme et joie de vivre de la jeunesse.

«~Directrice adjointe professeure McGonagall~» dit Harry, les mains solennellement posées sur les hanches. «~Je n'irai pas à Gryffondor…~»

À cet instant, McGonagall glissa quelque chose comme quoi si cela s'avérait \emph{être} le cas, elle découvrirait comment faire pour tuer un chapeau, une remarque étrange que Harry laissa passer sans commentaire, bien que cela semble déclencher chez la vendeuse une soudaine crise de toux.

«~ Je serai envoyé à Serdaigle. Et si vous pensez vraiment que je prévois de faire quelque chose de dangereux, alors, en toute honnêteté, vous ne me comprenez pas \emph{du tout}.
Je n'\emph{aime} pas le danger, cela fait \emph{peur}.  Je suis \emph{prudent}.  Je suis \emph{précautionneux}. Je me prépare à des \emph{contingences imprévues}.
Comme mes parents me chantonnaient~:
\emph{Be prepared! That's the Boy Scout's marching song! Be prepared! As through life you march along! Don't be nervous, don't be flustered, don't be scared - be prepared!"}
\footnote{\emph{Soyez prêts~! C'est la chanson de marche des Scouts, soyez prêts~! Comme on marche dans la vie~! Pas nerveux, pas énervé, pas effrayé, soyez prêt~!}~»}

(Les parents de Harry n'avaient de fait chanté que \emph{ces vers-là} de la chanson de Tom Lehrer, et Harry vivait dans ignorance bienheureuse du reste de cette chanson.)

La posture de McGonagall s'était légèrement adoucie -- essentiellement au moment où Harry qu'il était destiné pour Serdaigle.


«~À quelle sorte de \emph{contingence} imaginez-vous que ce kit vous puisse vous préparer, \emph{jeune homme}~?

--- L'une de mes camarades de classe se fait mordre par un horrible monstre, et alors que je fouille frénétiquement dans ma bourse en peau de Moke à la recherche de quelque chose qui pourrait l'aider, elle me regarde tristement, et dans son dernier souffle me dit~: “\emph{Pourquoi n'étais-tu pas prêt~?}” Alors que ses yeux se ferment pour mourir, je sais que jamais elle ne me pardonnera…~»


Harry entendit la vendeuse laisser échapper un petit cri d'effroi~; il leva les yeux et vit qu'elle le braquait du regard, les lèvres fermement serrées.
Puis la jeune femme fit volte-face et fuit vers le fond du magasin.

\emph{Qu'est-ce qu…?}

McGonagall se pencha pour prendre les mains de Harry dans les siennes, doucement mais fermement, et entraîna Harry hors de la rue principale du Chemin de Traverse, le menant dans une impasse bordée par deux magasins, pavée de briques sales et fermée par un mur de terre noire compactée.

De sa hauteur, la sorcière pointa sa baguette en direction de la rue principale et prononça~: «~\emph{Sourdinam}~». Un écran de silence s'abattit autour d'eux, bloquant tous les bruits de la rue.

\emph{Qu'est-ce que j'ai fait de mal…}

McGonagall se tourna pour regarder Harry.
Elle n'arborait pas le visage d'une adulte en colère envers un enfant désobéissant.
Son expression était lissée, contrôlée.

«~Vous devez vous rappeler, M. Potter, que ce pays était en guerre il n'y a pas dix ans de cela. Tout le monde ici a perdu quelqu'un, et parler d'amis mourants dans vos bras… n'est pas à prendre à la légère.

--- Je, je ne voulais pas…~» 
La déduction s'abattit comme une pierre dans l'imagination exceptionnellement vive de Harry.
Il venait de parler de quelqu'un rendant son dernier souffle -- puis la vendeuse s'était enfuie -- et la guerre avait pris fin dix ans auparavant, donc cette fille devait avoir huit ou neuf ans tout au plus quand, quand…
«~Je suis désolé, je ne voulais pas…~»
Harry sentit sa gorge se serrer, il se détourna pour fuir le regard de McGonagall, mais il y avait un mur en terre en travers de son chemin et il n'avait pas encore sa baguette magique. «~Je suis désolé, je suis désolé, je suis \emph{désolé}~!~»

Un lourd soupir s'éleva derrière lui. «~Je sais que vous l'êtes, M. Potter.~»

Harry osa jeter un coup d'œil derrière lui.
McGonagall avait juste l'air triste à présent.
«~Je suis désolé~», dit à nouveau Harry, se sentant misérable.
«~Est-ce qu'une situation similaire vous…~» et Harry referma ses lèvres et se plaqua la main sur la bouche par sécurité.

Le visage de la sorcière s'attrista un peu plus.
«~Vous devez apprendre à penser avant de parler, M. Potter, ou risquez de passer votre vie avec bien peu d'amis.
Ça a été le sort de nombreux Serdaigle, et j'espère que ce ne sera pas le vôtre.~»

Harry aurait juste voulu s'enfuir en courant; sortir une baguette magique et effacer toute l'histoire de la mémoire de McGonagall, être à nouveau avec elle devant le magasin, \emph{faire que ça n'ait pas eu lieu}.

«~Mais pour répondre à votre question, M. Potter, non, rien de \emph{tel} ne m'est jamais arrivé.
Indéniablement, j'ai vus des amis exhaler leur dernier soupir, plus d'une fois.
Mais aucun d'entre eux ne m'a jamais maudite en mourant, et je n'ai jamais pensé qu'ils pourraient ne pas me pardonner.
Qu'est-ce qui vous a poussé à \emph{dire} une chose pareille, M. Potter~? Qu'est-ce qui vous a poussé à même y \emph{penser}~?~»

% Des larmes coulaient le long des joues de Harry. «~Je suis désolé, je n'aurais jamais rien dû dire, je suis désolé…~»
%
% McGonagall prit une courte inspiration. «~Je \emph{sais} que vous êtes désolé. Ce que je ne comprends pas, c'est pourquoi un enfant de onze ans \emph{pense} à ces choses-là. Avez-vous vraiment décidé d'acheter un kit de soin à cinq gallions pour le transporter dans une bourse à quinze gallions parce que vous êtes convaincu qu'autrement vos camarades de classes vont vous \emph{maudire en mourant}~?

--- Je, je,  je, Harry avala sa salive. C'est juste que j'essaie toujours d'imaginer la pire chose qui puisse arriver~», et peut-être qu'il avait aussi voulu blaguer un peu, mais il aurait préféré se couper la langue plutôt que de l'avouer.

--- Quoi? Mais \emph{Pourquoi}~?

--- Pour que je puisse empêcher que ça ait lieu~!

--- M. Potter…~» la voix de McGonagall resta en suspens.  Puis elle soupira, et s'accroupit à côté de lui.
«~M. Potter, dit-elle, gentiment à présent, ce n'est pas votre responsabilité que de prendre soin des étudiants de Poudlard. C'est la mienne.
Je ne laisserai rien de mal arriver, ni à vous ni à qui que ce soit d'autre.
Poudlard est l'endroit le plus sûr pour les enfants sorciers de tout le monde magique, et madame Pomfresh a une infirmerie en parfait état de fonctionnement.
Vous n'aurez pas besoin d'un kit de soin, qui plus est un coûtant cinq gallions.

--- Mais \emph{si}~! éclata Harry. \emph{Aucun endroit} n'est parfaitement sûr~! Et si mes parents font une crise cardiaque ou ont un accident quand je rentrerai à Noël -- Madame Pomfresh ne sera pas là, j'aurai besoin d'avoir mon propre kit de soin…

--- Par la barbe de Merlin, \emph{qu'est-ce qu}…~» dit McGonagall. Elle se leva et regarda Harry, partagée entre l'irritation et la préoccupation.
«~Il ne faut pas penser à des choses aussi terribles, M. Potter~!~»

Lorsqu'il entendit cela, l'expression de Harry vira vers l'amertume.
«~Si, il le \emph{faut}~! Si vous n'y pensez pas, ce n'est pas simplement que vous allez vous retrouver blessé, vous finissez par blesser les autres~!~»

McGonagall ouvrit la bouche, puis la ferma. Elle se frotta l'arête du nez, l'air pensive.
«~M. Potter… si je vous proposais un moment d'écoute… y a-t-il quelque chose dont vous voudriez me parler~?

--- À propos de quoi~?

--- À propos de la raison pour laquelle vous êtes convaincu que vous devez toujours être sur vos gardes contre les terribles choses qui pourraient vous arriver.~»



Harry la fixa, perplexe. C'était un axiome qui allait de soi. «~Eh bien…~» dit-il lentement. Il essaya d'organiser ses pensées. Comment \emph{pouvait}-il expliquer cela à une sorcière-professeure, alors qu'elle n'avait même pas les bases~?

% STOP HERE

«~Les chercheurs Moldus ont découvert que les gens sont toujours très optimistes, par rapport à la réalité.
Par exemple, ils disent que quelque chose va prendre deux jours et ça en prend dix, ou ils disent que ça va prendre deux mois et ça prend plus de trente-cinq ans.
Dans une expérience, des chercheurs ont demandé à des étudiants les durées maximum avant lesquelles ils étaient sûrs à 50~\%, 75~\% et 99~\% qu'ils auraient terminé leurs devoirs, et seuls 13~\%, 19~\% et 45~\% des étudiants ont terminé dans les temps qu'ils avaient donnés.
Ils se sont rendu compte que c'était parce que lorsqu'on demande à un groupe de gens leur estimation dans le meilleur des cas, si tout se passe le mieux possible, et à un autre groupe leur estimation dans le cas moyen, si tout se passe comme d'habitude, on reçoit des réponses qui sont statistiquement impossible à distinguer.
Vous voyez, si vous demandez à quelqu'un ce à quoi il s'attend dans le cas \emph{normal}, il visualise ce qui semble être le plus probable à chaque étape du parcours -- que tout se déroule comme prévu, sans surprise.
Mais en réalité, puisque plus de la moitié des étudiants n'ont pas fini au moment où ils étaient pourtant certains à 99~\% que ce serait le cas, cela veut dire que la réalité est habituellement légèrement pire que le “pire des cas possibles.”
C'est ce qu'on appelle l'\emph{illusion de la planification}, et la meilleure façon de la corriger est de vous demander combien de temps vous avez mis pour faire certaines tâches la dernière fois que vous avez essayé les faire.
On appelle cela utiliser le point de vue extérieur au lieu du point de vue intérieur.
Mais quand vous faites quelque chose de nouveau, vous ne pouvez pas utiliser cette méthode, alors vous devez être vraiment, vraiment, vraiment pessimiste.
En gros, tellement pessimiste que la réalité finit par être \emph{meilleure} que ce à quoi vous vous attendiez environ aussi souvent qu'elle finit par être pire.
C'est en fait \emph{très difficile} d'être \emph{suffisamment} pessimiste pour se retrouver avec de bonnes chances de \emph{dépasser} la réalité.
Par exemple, je fais un gros effort pour être sinistre et imaginer qu'une de mes camarades se fait mordre, mais ce qui se passe réellement c'est que les Mangemorts survivants attaquent l'école toute entière pour s'en prendre à moi. Mais le bon côté des choses c'est que…

--- Arrêtez~», dit McGonagall.

Harry s'arrêta. Il était juste sur le point de faire remarquer qu'au moins ils savaient que le Seigneur des Ténèbres n'attaquerait pas puisqu'il était mort.

«~Je pense que je n'ai peut-être pas été assez claire, dit McGonagall, son accent écossais redoublant de précaution. Y a-t-il quoi que ce soit qui vous soit arrivé \emph{personnellement} et qui vous ait effrayé, M. Potter~?

--- Ce qui m'est arrivé ne constitue que des éléments anecdotiques, expliqua Harry. Cela n'a pas le même poids qu'un article de journal scientifique, répliqué, évalué par des pairs, sur un essai randomisé contrôlé, avec de nombreux participants, des effets de grande ampleur et statistiquement significatif.~»

McGonagall pinça l'arête de son nez, inhala puis exhala.
«~Je voudrais tout de même que vous m'en parliez, dit-elle.

--- Euh…~» dit Harry. Il prit une profonde inspiration.
«~Il y avait eu quelques vols dans mon quartier, et ma mère m'avait demandé de ramener une poêle qu'elle avait empruntée aux voisins deux pâtés de maisons plus loin, et j'ai dit que je ne voulais pas y aller parce que je risquais de me faire agresser, et elle a dit “Harry, ne dis pas des choses pareilles~!”
Comme si y penser allait \emph{faire} que ça ait lieu, et donc que si je n'en parlais pas, je serais en sécurité.
J'ai essayé de lui expliquer pourquoi cela ne me rassurait pas, et elle m'a quand même forcé à rapporter la poêle.
J'étais trop jeune pour savoir à quel point il était statistiquement improbable qu'un voleur me prenne pour cible, mais j'étais suffisamment âgé pour savoir que ne pas penser à quelque chose ne l'empêchait pas d'avoir lieu, donc j'avais vraiment peur.

--- Rien d'autre~?~» dit McGonagall après une pause, lorsqu'il fut apparent que Harry avait terminé. «~Il n'y a rien d'\emph{autre} qui vous soit arrivé~?

--- Je sais que ça n'a pas \emph{l'air} bien important, se défendit Harry.
Mais c'était un de ces moments critiques d'une vie, vous voyez~?
Je veux dire que je \emph{savais} que ne pas penser à quelque chose ne l'empêchait pas d'avoir lieu, je le \emph{savais}, mais je pouvais voir que maman, elle, le pensait vraiment.~»
Harry s'arrêta, luttant contre la colère qui recommençait à monter alors qu'il se remémorait ce moment.
«~Elle ne \emph{voulait pas écouter}. J'ai essayé de lui dire, je l'ai \emph{suppliée} de ne pas m'envoyer dehors, et elle \emph{en a rit}.
Tout ce que je disais, elle le considérait comme une sorte de blague…~»
Harry força sa colère noire à redescendre.
«~C'est à ce moment là que je me suis rendu compte que tous ceux qui étaient censés me protéger étaient en réalité fous, et qu'ils ne m'écouteraient pas, peu importe que je les supplie, et que je ne pourrai jamais vraiment compter sur eux pour quoi que ce soit d'importance.~»
Parfois les bonnes intentions ne suffisaient pas, parfois il fallait être sain d'esprit…

Il y eut un long silence.

Harry prit le temps de respirer profondément et de se calmer.
Il n'y avait aucune raison de se mettre en colère.
Il n'y avait \emph{aucune} raison de se mettre en colère.
\emph{Tous} les parents étaient comme ça, \emph{aucun} adulte n'était prêt à s'abaisser suffisamment pour se mettre au même niveau qu'un enfant et l'écouter, ses parents génétiques n'auraient pas été différents.
La santé mentale était une petite étincelle dans la nuit, une rare exception infinitésimale à la règle qui est la folie, il était donc futile de se mettre en colère.

Harry ne s'aimait pas quand il était en colère.

«~Merci d'avoir partagé cela, M. Potter~», dit McGonagall au bout d'un moment.
Elle avait une expression distraite (presque exactement la même que celle qui était apparue sur le propre visage de Harry pendant qu'il faisait ses expériences avec la bourse en peau de Moke, mais il aurait fallu que Harry se voie dans un miroir pour pouvoir s'en rendre compte).
«~Je vais avoir besoin d'y réfléchir.~» Elle se tourna vers l'entrée de l'impasse et leva sa baguette…

«~Euh, dit Harry, peut-on aller chercher le kit de soins maintenant~?~»

McGonagall s'interrompit, et se retourna pour le regarder fermement. «~Et si je dis non, que c'est trop cher et vous n'en aurez pas besoin, que se passe-t-il~?~»

Le visage de Harry se couvrit d'amertume.
«~Exactement ce que vous pensez, professeure McGonagall. \emph{Exactement} ce que vous pensez.
J'en conclus que vous êtes une autre adulte folle à laquelle je ne peux pas parler, et je commence à élaborer un plan pour me procurer un kit de soins autrement.

--- Je suis votre gardienne pour cette sortie, commença McGonagall, une pointe de menace dans la voix. Je ne vous \emph{permettrai pas} de me faire du chantage.

--- Je comprends~», dit Harry. Il réussit à ne pas laisser transparaître sa rancœur dans sa voix, et ne dit aucune des autres choses qui lui venaient à l'esprit. McGonagall lui avait dit de penser avant de parler. Il ne s'en souviendrait probablement pas le lendemain, mais il pouvait au moins s'en souvenir pendant cinq minutes.

La baguette de McGonagall décrivit un petit cercle au bout de sa main, et l'on entendit à nouveaux les sons du Chemin de Traverse. «~Très bien, jeune homme, dit-elle. Allons acheter ce kit de soins.~»

La mâchoire de Harry tomba de surprise. Puis il se dépêcha de suivre la sorcière, trébuchant presque dans sa précipitation.

\later

Le magasin était tel qu'ils l'avaient laissé, avec des objets identifiables et d'autres inindentifiables présentés sur les étals de bois inclinés, la lueur grise protectrice et la vendeuse de retour à sa position originale.
Elle les regarda alors qu'ils s'approchaient, la surprise visible sur son visage.

«~Je suis désolée~», dit-elle quand ils arrivèrent, et Harry dit presque au même moment~: «~Je vous demande pardon pour…~»

Ils s'interrompirent et se regardèrent, puis la vendeuse eut un petit rire.
«~Je ne voulais pas vous causer d'ennuis avec la professeure McGonagall~», dit-elle.
Sa voix baissa d'un ton, conspiratrice.
«~J'espère qu'elle n'a pas été \emph{trop} horrible avec vous.

--- \emph{Della~!} dit McGonagall scandalisée.

--- Sac d'or~», dit Harry à sa bourse, et il s'adressa de nouveau à la vendeuse pendant qu'il comptait cinq gallions.
«~Ne vous en faites pas, je comprends bien que si elle est horrible avec moi c'est seulement parce qu'elle m'aime.~»

Il donna les gallions à la vendeuse pendant que McGonagall bafouillait quelque chose sans importance.
«~Un Pack de Soins d'Urgence Plus, s'il vous plaît.~»

C'était en fait assez déstabilisant de voir l'Ouverture Élargissante avaler le kit médical qui avait tout de même la taille d'une mallette.
Harry ne pouvait pas s'empêcher de se demander ce qui se passerait s'il essayait de grimper dans la bourse lui-même, étant donné que seule la personne qui y avait mis quelque chose était censée pouvoir la récupérer.

Lorsque la bourse eut fini de… manger… son achat durement gagné, Harry jura avoir entendu un petit rot.
Cela \emph{devait} être un sort rajouté volontairement.
L'hypothèse alternative était trop horrifiante pour être envisagée… en fait Harry n'arrivait même pas à \emph{imaginer} une hypothèse alternative.
Il releva la tête vers McGonagall alors qu'ils recommençaient leur marche le long du Chemin de Traverse. «~Où allons-nous ensuite~?~»

McGonagall pointa du doigt un magasin qui semblait être fait de chair plutôt que de briques et couvert de fourrure plutôt que de peinture.
«~Les petits animaux de compagnie sont autorisés à Poudlard -- vous pourriez avoir une chouette pour envoyer des lettres, par exemple…

--- Est-ce que je peux payer quelque chose comme une noise pour \emph{louer} une chouette quand j'ai besoin d'envoyer du courrier~?

--- Oui, dit McGonagall.

--- Alors je pense catégoriquement \emph{non}.~»

McGonagall hocha la tête, comme si elle cochait mentalement une case.
«~Pourrais-je vous demander pourquoi~?

--- J'ai eu un jour un caillou de compagnie. Il est mort.

--- Vous ne pensez pas pouvoir prendre soin d'un animal domestique~?

--- Je \emph{pourrais}, dit Harry, mais je finirai immanquablement par en faire une obsession, me demandant à longueur de journée si j'ai bien pensé à la nourrir le matin même ou si elle meurt lentement de faim dans sa cage, ne sachant ni où est son maître ni pourquoi il n'y a pas de nourriture.

--- Pauvre chouette, dit McGonagall d'une voix douce. Abandonnée ainsi. Je me demande ce qu'elle ferait.

--- Eh bien, je suppose qu'elle commencerait à avoir vraiment faim et qu'elle essayerait de se libérer de sa cage ou de sa boîte à coups de bec, mais ça ne fonctionnerait probablement pas…~» Harry s'arrêta net.

McGonagall continua, toujours de cette voix douce~: «~Et que lui arriverait-t-il ensuite~?~»

«~Excusez-moi~», dit Harry, il prit McGonagall par la main, doucement mais fermement, et l'entraîna vers une autre ruelle~;
après avoir esquivé tant d'admirateurs que s'en était presque devenu une routine imperceptible.
«~Lancez ce sort de silence s'il vous plaît.

--- \emph{Sourdinam}.~»

La voix de Harry tremblait. «~Cette chouette ne me symbolise \emph{pas}, mes parents ne m'ont \emph{jamais} enfermé dans un placard pour me laisser mourir de faim, je n'ai \emph{pas} de peurs d'abandon et \emph{je n'aime pas le fil de vos pensées, professeure McGonagall~!}~»

La sorcière le regarda gravement.

«~Et quelles seraient ces pensées, M. Potter~?

--- Vous pensez que j'ai été… Harry avait du mal à le dire, que j'ai été \emph{maltraité}~?

--- L'avez-vous été~?

--- \emph{Non~!} cria Harry. Non, jamais~! Vous pensez que je suis \emph{stupide}~?
Je \emph{connais} le concept d'abus infantile, je \emph{sais} ce que sont des attouchements inappropriés et si quoi que ce soit qui y ressemble m'arrivait j'appellerais la police~!
Et je le signalerais au directeur de l'école~!
Et je chercherais le numéro des services sociaux dans l'annuaire~!
Et j'en parlerais à grand-mère et à grand-père et à Mlle Figg~!
Mais mes parents n'ont \emph{jamais} fait quoi que ce soit de ce genre, jamais jamais \emph{jamais}~!
Comment \emph{osez}-vous suggérer une chose pareille~!~»

McGonagall le regarda fixement. «~Il est de mon devoir en tant que directrice adjointe d'investiguer tout signe de maltraitance possible chez les enfants qui sont sous ma garde.~»

La colère de Harry vrillait hors de contrôle, se transformant en pure fureur noire.

«~Ne vous avisez \emph{jamais} de souffler un mot de ces, de ces \emph{insinuations} à qui que ce soit~!
À \emph{personne}, vous m'entendez, McGonagall~?
Une accusation comme celle-ci peut briser des vies et détruire des familles alors même que les parents sont complètement innocents~!
J'ai lu des choses à ce sujet dans les journaux~!~»
La voix de Harry grimpait dans les aigus.
«~Le \emph{système} ne sait pas s'\emph{arrêter}, il ne croit pas les parents \emph{ni} les enfants lorsqu'ils disent qu'il ne s'est rien passé~!
\emph{Je vous interdit de menacer ma famille ainsi~!  Je ne vous laisserai pas détruire mon foyer~!}

% STOP HERE

--- Harry~», fit doucement McGonagall, et elle tendit sa main vers lui…

Harry recula d'un pas rapide, sa main se leva brusquement pour repousser la sienne.

McGonagall se figea, puis retira sa main, et fit un pas en arrière.

«~Harry, tout va bien, dit-elle. Je vous crois.

--- \emph{Vraiment…}~», siffla Harry. La fureur grondait toujours dans ses veines.
«~Ou vous attendez juste d'être seule pour pouvoir remplir un formulaire~?

--- Harry, j'ai vu votre maison. Je vous ai vu avec vos parents. Ils vous aiment. Vous les aimez.
Je vous crois lorsque vous dites que vos parents ne vous ont pas maltraité.
Mais il \emph{fallait} que je pose la question, car il y a quelque chose d'étrange à l'œuvre ici.~»

Harry la fixa froidement. «~Comme quoi~?~»

--- Harry, j'ai vu de nombreux enfants victimes d'abus depuis que je travaille à Poudlard, cela vous briserait le cœur de savoir combien.
Et quand vous êtes joyeux, vous ne vous comportez pas comme l'un de ces enfants, pas du \emph{tout}.
Vous souriez aux étrangers, vous serrez les gens dans vos bras, j'ai mis ma main sur votre épaule et vous n'avez pas bronché.
Mais parfois, juste parfois, vous dites ou faites quelque chose qui vous fait \emph{fortement} penser à… 
quelqu'un qui aurait passé les onze premières années de sa vie enfermé dans une cave.
Pas dans la famille aimante que j'ai vue.~»
McGonagall inclina la tête, elle semblait de nouveau perplexe.

Harry absorba tout cela, analysant ces informations. Sa rage noire commença à s'estomper, car il réalisait qu'on l'écoutait avec respect, et que sa famille n'était pas en danger.

«~Et comment \emph{expliquez}-vous vos observations, professeure McGonagall~?

--- Je ne sais pas, dit-elle. Mais il est possible que quelque chose vous soit arrivé, quelque chose dont vous ne vous souvenez pas.

Harry sentit de nouveau la fureur monter. Cela ressemblait beaucoup trop aux histoires de familles brisées qu'il avait lues dans les journaux.

% STOP HERE

«~Les souvenirs refoulés ce n'est que de la \emph{pseudoscience}~!
Les gens ne répriment \emph{pas} leurs souvenirs traumatiques, ils ne s'en souviennent que \emph{trop bien} pour le restant de leurs vies~!

--- Non, M. Potter. Il existe un sortilège nommé Oubliettes.~»

Harry se figea sur place. «~Un sort qui efface la mémoire~?~»

McGonagall acquiesça. «~Mais pas tous les effets du souvenir, si vous voyez ce que je veux dire, M. Potter.~»

Un frisson lui parcouru la colonne vertébrale. \emph{Cette} hypothèse… n'était \emph{pas} simple à réfuter.

«~Mais mes parents ne pourraient pas faire ça~!

--- Non, effectivement, dit McGonagall. Il faudrait quelqu'un venu du monde magique. Il n'y a… aucun moyen d'en être certain, j'en ai peur.~»

Les compétences rationalistes de Harry se remirent en fonctionnement. «~Professeure McGonagall, à quel point êtes-vous certaine de vos observations, et quelles explications alternatives pourrait-il y avoir~?~»

McGonagall ouvrit ses mains comme pour montrer qu'elle ne tenait rien.
«~Certaine~? Je ne suis certaine de \emph{rien}, M. Potter.
De toute va vie je n'ai jamais rencontré personne qui vous ressemble.
Parfois vous ne paraissez tout simplement pas avoir onze ans ni même être vraiment \emph{humain}.~»

Les sourcils de Harry s'élevèrent vers le ciel…

«~Pardon~! corrigea rapidement McGonagall. Je suis vraiment désolée, M. Potter.
J'essayais juste d'expliquer mon point de vue, ce n'est pas du tout ce que je voulais dire.

--- Au contraire, professeure, dit Harry, souriant lentement.
Je prends cette remarque comme un très grand compliment.
Mais puis-je vous proposer une explication alternative~?

--- Allez-y, je vous en prie.

--- Les enfants ne sont pas censés être beaucoup plus intelligents que leurs parents, dit Harry.
Ou beaucoup plus sains d'esprit, peut-être --
mon père pourrait probablement se montrer plus intelligent que moi, vous savez, s'il \emph{essayait} vraiment, au lieu d'utiliser son cerveau d'adulte pour trouver de nouvelles raisons de ne pas changer sa façon de penser -- Harry s'interrompit.
Je suis trop intelligent, professeure.
Je ne sais pas de quoi parler avec des enfants normaux.
Les adultes ne me respectent pas suffisamment pour me parler.
Et franchement, même s'ils le faisaient, ils n'auraient pas l'air aussi intelligent que Richard Feynman, donc autant lire ce qu'a écrit Richard Feynman.
Je suis \emph{isolé}, professeure McGonagall. J'ai été isolé toute ma vie.
Peut-être que cela induit en partie les mêmes effets que d'être enfermé dans une cave.
Et je suis trop intelligent pour lever les yeux vers mes parents comme sont censés le faire les enfants.
Mes parents m'aiment, mais ils ne se sentent pas tenus de répondre à la raison, et parfois j'ai la sensation que ce sont eux les enfants --
des enfants qui \emph{n'écoutent pas}, et qui ont une autorité absolue sur toute mon existence.
J'essaie de ne pas être trop amer à ce sujet, mais j'essaie aussi d'être \emph{honnête} avec moi-même, et donc, oui, je suis amer.
Et j'ai aussi du mal à gérer ma colère, mais j'y travaille.
Voilà, c'est tout.

--- \emph{C'est tout~?}~»

Harry acquiesça avec ferveur. «~C'est tout.
Professeure McGonagall, j'imagine que, même en Angleterre magique, l'explication normale mérite d'être \emph{prise en considération}, non~?~»

% l'explication normale mérite d'être \emph{prise en considération}, même dans l'Angleterre magique, non~?~»

% STOP HERE
%
\later

Plus tard dans la journée, le soleil descendait sur un ciel d'été et les acheteurs commençaient à disparaître des rues. Certains magasins avaient déjà fermé~; Harry et McGonagall avaient acheté ses manuels chez Fleury et Bott juste avant la fermeture. Il y avait seulement eu une légère explosion quand Harry avait foncé droit vers le mot-clé “Arithmancie” et avait découvert que les livres de septième année ne contenaient rien de plus mathématiquement avancé que la trigonométrie.

Mais pour le moment, les rêves d'opportunités faciles étaient très loin de l'esprit de Harry.

Pour le moment, Harry et McGonagall sortaient de chez Ollivander's, et Harry fixait sa baguette. Il l'agita et produit des étincelles multicolores, ce qui n'aurait vraiment pas dû le choquer particulièrement après tout ce qu'il avait déjà vu, mais malgré tout…

\emph{Je peux faire de la magie.}

\emph{Moi. Comme dans “Moi, personnellement.” Je suis magique~; je suis un sorcier.}

Il avait \emph{sentit} la magie affluer dans son bras, et à cet instant il avait réalisé qu'il avait toujours eu ce sens, qu'il l'avait possédé toute sa vie, le sens qui n'était ni la vue ni le son ni l'odeur ni le goût ni le toucher, mais seulement la magie. Comme d'avoir des yeux, mais de les avoir toujours gardés fermés, et que vous ne vous rendiez pas compte que vous voyez du noir~; et le jour où vous les ouvriez, vous découvriez le monde. Le choc s'était déversé en lui, touchant plusieurs parties de son être, les réveillant, et disparaissant ensuite en quelques secondes~; ne laissant que la certitude qu'il était maintenant un sorcier, l'avait toujours été, et d'une certaine façon, qu'il l'avait toujours su.

Et…

«~\emph{Il est en effet très curieux que vous soyez destiné à cette baguette, sachant que sa sœur, eh bien, sa sœur vous a donné cette cicatrice.}~»

Ça ne \emph{pouvait} pas être une coïncidence. Il y avait des \emph{milliers} de baguettes dans ce magasin. Bon, d'accord, ça \emph{pouvait} être une coïncidence, il y avait six milliards de personnes sur Terre, des coïncidences à une chance sur mille avaient lieu tous les jours. Mais, Théorème de Bayes 101~: toute hypothèse raisonnable impliquant qu'il avait \emph{plus} d'une chance sur mille que Harry se retrouve avec la baguette sœur de celle du Seigneur des Ténèbres avait un avantage.

McGonagall avait simplement dit \emph{comme c'est curieux} et en était restée là, ce qui avait mit Harry en état de choc face à la pure, à l'écrasante \emph{inconscience} des sorciers et sorcières. Harry n'aurait pu, dans aucun monde \emph{imaginable}, simplement faire «~Hmm~» et sortir du magasin sans même \emph{essayer} de trouver une hypothèse expliquant ce qui s'était passé.

Sa main gauche s'éleva et toucha sa cicatrice.

Qu'est-ce qui… \emph{exactement}…

«~Vous êtes un sorcier complet à présent, dit McGonagall. Félicitations.~»

Harry hocha la tête.

«~Et que pensez-vous du monde magique~?

--- C'est étrange, dit Harry. Je devrais être en train de penser à tout ce que j'ai vu de la magie… tout ce que je sais maintenant être possible, et tout ce que je sais maintenant être un mensonge, et tout le travail qui me reste à accomplir avant de vraiment comprendre. Et pourtant je me trouve distrait par de relatives trivialités telles que,~» Harry baissa la voix, «~toute cette histoire de Survivant.~» Il ne semblait y avoir personne aux alentours, mais autant ne pas tenter le sort.

McGonagall \emph{ahema}. «~Vraiment~? Sans blague.~»

Harry hocha la tête. «~Oui. C'est juste… \emph{curieux}. De se rendre compte que vous faites partie de cette grande histoire, la quête pour vaincre le grand et terrible Seigneur des Ténèbres, et c'est déjà \emph{fini}. Terminé. Complètement réglé. Comme si vous étiez Frodon Sacquet, que vous appreniez que vos parents vous avaient emmené à la Montagne du Destin quand vous aviez un an, qu'ils vous avaient fait jeter l'anneau et que vous ne vous en souveniez même pas.~»

Le sourire de McGonagall s'était plus ou moins figé.

«~Vous savez, si j'étais qui que ce soit d'autre, vraiment n'importe qui d'autre, je serais plutôt anxieux à l'idée de vivre à la hauteur de ce démarrage. \emph{Grand dieu Harry, qu'avez-vous fait depuis que vous avez vaincu le Seigneur des Ténèbres~? Votre propre librairie~? C'est super~! Dites-moi, saviez-vous que j'ai donné votre nom à mon enfant~?} Mais j'ai bon espoir que cela ne soit pas un problème.~» Harry soupira. «~Tout de même… c'est presque assez pour me faire espérer qu'il y ait \emph{quelques} détails de cette quête à finir, juste pour que je puisse dire que j'ai vraiment, vous savez, \emph{participé} d'une façon quelconque.

--- Oh~? dit McGonagall sur un ton étrange. Qu'aviez-vous à l'esprit~?

--- Eh bien par exemple, vous avez mentionné que mes parents ont été trahis. Qui les a trahis~?

--- Sirius Black~», dit McGonagall. Elle siffla son nom plus qu'elle ne le prononça. «~Il est à Azkaban. Prison des sorciers.

--- Quelle est la probabilité que Sirius Black s'échappe de prison et que je doive le traquer et le vaincre dans un duel spectaculaire, ou encore mieux, mettre une large prime sur sa tête et me cacher en Australie pendant que j'attends le résultat~?~»

McGonagall cligna des yeux. Deux fois. «~Peu probable. Personne ne s'est jamais échappé d'Azkaban, et je doute qu'\emph{il} soit le premier.~»

Harry était un peu sceptique de ce “\emph{personne} ne s'est \emph{jamais} échappé d'Azkaban”. Mais bon, peut-être qu'avec la magie vous pouviez faire approcher votre prison de 100~\% de perfection, et encore plus si vous aviez une baguette et pas l'autre. La meilleure façon de sortir serait de ne jamais y être entré.

«~Très bien, dit Harry. Ça m'a l'air bien ficelé.~» Il soupira, et gratta sa paume contre sa tête. «~Ou peut-être que le Seigneur des Ténèbres n'est pas \emph{vraiment} mort cette nuit-là. Pas complètement. Son esprit erre, chuchotant aux gens dans leurs cauchemars, qui se répandent dans le monde éveillé, et il cherche à revenir sur les terres des vivants, qu'il a promis de détruire, et maintenant, en accord avec l'ancienne prophétie, lui et moi sommes coincés dans un duel à mort où le gagnant perdra et le perdant gagnera…~»

La tête de McGonagall pivota, et ses yeux dardèrent aux alentours, à la recherche de personnes prêtant l'oreille.

«~Je \emph{plaisante}, Professeur McGonagall~», dit Harry, un peu contrarié. Bon sang, pourquoi devait-elle toujours tout prendre si sérieusement…

Une lente sensation coula doucement jusqu'au fond de l'estomac de Harry.

McGonagall regarda Harry avec un air calme. Un air très, \emph{très} calme. Puis un sourire fut ajouté. «~Bien sûr que vous plaisantez, M. Potter.~»

\emph{Oh crotte.}

Si Harry avait eu besoin de rationaliser l'inférence muette qui venait de flasher dans son esprit, ça aurait été quelque chose comme~: «~Si j'estime la probabilité que McGonagall a fait ce que je viens de voir parce qu'elle s'est contrôlée avec soin, contre la distribution de probabilités pour toutes les choses qu'elle ferait \emph{naturellement} si j'avais fait une mauvaise blague, alors ce comportement est un élément de preuve significatif pointant vers le fait qu'elle cache quelque chose.~»

Mais ce que Harry pensa fut~: \emph{Oh crotte}.

Harry pivota sa propre tête pour scanner la rue. Non, personne dans le coin.

«~Il n'est \emph{pas} mort, c'est ça~? soupira Harry.

--- M. Potter…

--- Le Seigneur des Ténèbres est vivant. \emph{Bien sûr} qu'il est vivant. C'était un \emph{acte} de pur et simple \emph{optimiste} que de seulement \emph{rêver} qu'il en soit autrement. J'ai \emph{dû} perdre la raison, je ne peux pas \emph{imaginer} à quoi je \emph{pensais}. Juste parce que \emph{quelqu'un} a dit que son corps avait été retrouvé \emph{calciné}, je ne peux pas imaginer pourquoi j'ai pu penser qu'il était \emph{mort}. J'ai \emph{clairement beaucoup} à apprendre sur l'art correct du \emph{pessimisme}.

--- M. Potter…

--- Dites-moi au moins qu'il n'y a pas vraiment de prophétie…~» Mais McGonagall lui donnait ce sourire intense et figé. «~Oh, bon sang, mais c'est une \emph{blague}.

--- M. Potter, vous ne devriez pas inventer des choses comme ça.

--- C'est \emph{vraiment} \emph{ça} que vous voulez me dire~? Imaginez ma réaction plus tard, quand j'apprendrai qu'il y avait quelque chose dont j'aurais dû me soucier après tout.~»

Le sourire de McGonagall se flétrit.

Les épaules de Harry s'affaissèrent. «~J'ai un monde entier de magie à analyser. Je n'ai \emph{pas} de temps à consacrer à ça.~»

Puis les deux se turent, et un homme en robe orange et flottante apparut dans la rue et les dépassa lentement. Les yeux de McGonagall le suivirent discrètement. La bouche de Harry bougeait, car il mâchait sa lèvre inférieure, et quelqu'un observant de près aurait remarqué un léger point de sang apparaître.

Lorsque l'homme en robe orange fut loin, Harry parla à nouveau, d'un bas murmure.

«~Allez vous me dire la vérité à présent, Professeur McGonagall~? Et n'essayez pas de prétendre qu'il n'y a rien, je ne suis pas stupide.

--- Vous avez \emph{onze ans}, M. Potter~! dit-elle dans un murmure cassant.

--- Et par conséquent sous-humain. Pardon… pour un moment j'avais \emph{oublié}.

--- Ce sont des affaires importantes et terribles~! Ce sont des \emph{secrets}, M. Potter~! C'est une \emph{catastrophe} que vous, encore un enfant, en sachiez autant~! Vous ne devez le dire à \emph{personne}, vous comprenez~? Absolument personne~!~»

Et, comme cela arrivait parfois quand Harry se mettait \emph{suffisamment} en colère, son sang devint froid au lieu de chaud, et une terrible clarté obscure s'abattit sur son esprit, décrivant toutes les tactiques possibles et jugeant les conséquences avec un réalisme d'acier.

\emph{Fais remarquer que tu as le droit de savoir~: Échec. Les enfants de onze ans n'ont le droit de savoir rien du tout, aux yeux de McGonagall.}

\emph{Dis que vous ne serez plus amis~: Échec. Elle n'accorde pas assez de valeur à ton amitié.}

\emph{Fais remarquer que tu seras en danger si tu ne sais pas~: Échec. Des plans ont déjà été pensés, basés sur ton ignorance. Le déplaisir} certain \emph{de repenser le plan leur semblera bien plus désagréable que la perspective} incertaine \emph{de te voir blessé.}

\emph{La justice et la raison échoueront. Tu dois soit trouver quelque chose que tu as et qu'elle veut, soit quelque chose que tu peux faire et qu'elle craint…}

Ah.

«~Très bien, dans ce cas, Professeur McGonagall, dit Harry d'un ton bas et glacé, on dirait que j'ai quelque chose que vous désirez. Vous pouvez, si vous le souhaitez, me dire la vérité, \emph{toute} la vérité, et en retour je garderai vos secrets. Ou vous pouvez essayer de me garder dans l'ignorance et m'utiliser comme un pion, auquel cas je ne vous devrai rien.~»

McGonagall s'arrêta net au milieu de la rue. Ses yeux flamboyèrent et sa voix se transforma en un sifflement.

«~Comment osez-vous~!

--- \emph{Comment osez-vous~!} chuchota-t-il en retour.

--- Vous me faites \emph{chanter}~?~»

Les lèvres de Harry se tordirent. «~Je vous \emph{offre} une \emph{faveur}. Je vous \emph{donne} une chance de garder \emph{notre} précieux secret. Si vous refusez, j'aurais \emph{tous} les motifs du monde pour aller poser des questions ailleurs, non par rancune envers vous, mais parce que \emph{j'ai besoin de savoir}~! Dépassez votre colère futile envers un \emph{enfant} qui, vous le croyez, se doit de vous obéir, et vous comprendrez que tout adulte sain d'esprit ferait de même~! \emph{Regardez les choses de mon point de vue~! Comment vous sentiriez-vous si c'était VOUS~?}~»

Harry regarda McGonagall, observa sa respiration saccadée. Il se rendit compte qu'il était temps d'adoucir la pression, de la laisser pondérer un moment. «~Vous n'avez pas à décider tout de suite, dit Harry sur un ton plus normal. Je comprendrais si vous vouliez plus de temps pour réfléchir à mon \emph{offre}… mais je vous préviens d'une chose~», dit Harry sa voix devenant plus froide. «~N'essayez pas ce Charme d'Oubliettes sur moi. Il y a quelque temps, j'ai conçu un signal, et je me le suis déjà envoyé à moi-même. Si je trouve ce signal et que je ne me \emph{souviens} pas l'avoir envoyé…~» Harry laissa sa voix traîner d'une façon lourde de sens.

Le visage de McGonagall travaillait sous le coup de divers changements d'expression.

«~Je… je ne pensais pas à vous lancer Oubliettes, M. Potter… mais pourquoi auriez-vous \emph{inventé} un signal si vous ne connaissiez pas l'existence de…

--- J'y ai pensé en lisant un livre de science-fiction Moldu, et je me suis dit, \emph{bon, juste au cas où}… Et non, je ne vous dirai pas le signal, je ne suis pas stupide.

--- Je ne comptais pas vous le demander,~» dit McGonagall. Elle parut se replier sur elle-même, et eut l'air soudain très vieille et très fatiguée. «~Ça a été une journée épuisante, M. Potter. Pourrions-nous prendre votre malle et vous envoyer chez vous~? Je vous fais confiance pour ne pas parler de cette affaire avant que j'aie eu le temps d'y réfléchir. Gardez à l'esprit qu'il n'y a que deux autres personnes au monde qui soient au courant de cette affaire, et ce sont le Directeur Albus Dumbledore et le Professeur Severus Rogue.~»

Donc. De nouvelles informations~; c'était une offre de paix. Harry acquiesça, tourna la tête vers l'avant et commença à marcher à nouveau.

«~Donc maintenant je dois trouver un moyen de tuer un Seigneur des Ténèbres immortel~», dit Harry, et il soupira de frustration. «~J'aurais vraiment aimé que vous me disiez ça \emph{avant} qu'on commence à faire du shopping.~»

\later

Le magasin de malles était plus richement décoré que tout autre magasin que Harry ait visité auparavant~; les rideaux étaient luxueux et ornés de motifs délicats, le sol et les murs étaient faits de bois teint et poli, et les malles occupaient des places d'honneur sur des plate-formes en ivoire poli. Le vendeur était habillé en robe d'une qualité seulement un cran en dessous de celles de Lucius Malfoy, et il parlait avec une politesse huileuse et exquise tant à Harry qu'à McGonagall.

Harry avait posé ses questions, et avait gravité vers une malle de bois lourd, pas polie, mais chaude et solide, gravée avec le motif d'un dragon gardien dont les yeux se déplaçaient pour regarder toute personne s'approchant. Une malle charmée pour être légère, réduire de taille sur commande, et faire pousser des petits tentacules griffus de sa base et se tortiller derrière son maître. Une malle avec deux tiroirs sur chacun de ses quatre côtés qui glissaient pour révéler des compartiments aussi profonds que la malle entière. Un couvercle équipé de quatre cadenas, et chacun d'entre eux révélait un espace intérieur différent. Et -- et c'était la partie importante -- une poignée sur le fond qui glissait et révélait un cadre contenant des marches menant vers une petite pièce éclairée qui, estima Harry, pouvait contenir environ douze étagères.

S'ils faisaient des malles comme cella-là, Harry ne savait pas pourquoi qui que ce soit s'embêtait à posséder une maison.

Cent huit gallions. C'était le prix d'une \emph{bonne} malle, légèrement usée. À cinquante livres le gallion, c'était assez pour s'offrir une voiture usagée. C'était plus cher que la somme de tout ce que Harry avait acheté de sa vie.

Quatre-vingt-dix-sept gallions. C'était ce qui restait dans le sac d'or que Harry avait été autorisé à retirer de chez Gringotts.

McGonagall avait un air chagriné. Après une longue journée de shopping elle n'avait pas eu besoin de demander à Harry combien d'or il restait dans le sac après que le vendeur eut donné son prix, ce qui voulait dire que le Professeur pouvait faire du calcul mental sans crayon ni papier. À nouveau, Harry se rappela à lui-même que \emph{scientifiquement illettré} n'était pas la même chose que \emph{stupide}.

«~Je suis désolée, jeune homme, dit McGonagall. C'est entièrement de ma faute. Je vous proposerais bien de vous ramener à Gringotts, mais la banque est à présent fermée hormis pour ses services d'urgence.~»

Harry prit une profonde inspiration. Il devait devenir un peu en colère pour ce qu'il voulait maintenant essayer, autrement il n'aurait sûrement pas le courage de le faire. Il se dit~: \emph{Elle ne m'a pas écouté, j'aurais pris plus d'or, mais elle ne voulait pas écouter}… Il repensa à la rage noire, plus tôt, et essaya d'en faire revenir un peu. Il visualisa \emph{la personne qu'il avait besoin d'être}, se revêtit de cette personnalité comme d'une robe de sorcier. Concentrant son univers entier sur McGonagall et le besoin qu'il avait de tordre cette conversation à ses fins, il parla.

«~Laissez-moi deviner, dit Harry. Vous pensiez que vous vous donniez une \emph{grande} marge d'erreur, que cent gallions seraient \emph{plus} que suffisants, et c'est pourquoi vous n'avez pas pris la peine de me prévenir quand nous sommes descendus à quatre-vingt-dix-sept.~»

McGonagall ferma les yeux avec résignation.

«~Oui.

--- J'ai anticipé cela, Professeur McGonagall. J'ai anticipé que cela arriverait. Il y a des études montrant que c'est ce qui se passe quand les gens pensent qu'ils \emph{se donnent une grande marge d'erreur}. Si c'était \emph{moi}, j'aurai pris \emph{deux cents} gallions, juste pour être sûr~; il y avait plein d'argent dans cette chambre forte, et j'aurai pu y remettre la monnaie plus tard. Mais je \emph{savais} que vous ne me laisseriez pas. Je savais qu'il était futile de demander. Je savais que vous seriez agacée et peut-être même \emph{énervée} si je vous demandais. Ai-je tort~?

--- Non, dit McGonagall, vous avez raison.~» Sa voix avait une note d'excuse, mais aussi une note d'orgueil personnel, comme si Harry était censé remarquer le grand, l'immense honneur que c'était de voir le \emph{Professeur McGonagall} s'excuser auprès de lui.

--- Vous devriez comprendre, Professeur McGonagall, Harry prononça ces mots avec soin, que c'est pour ça que je ne fais pas confiance aux adultes. Vous pensiez qu'être adulte voulait dire que c'était votre rôle de m'empêcher de prendre trop d'argent dans ma chambre forte. Pas que c'était votre rôle de \emph{vous assurer que le travail soit fait quoi qu'il arrive}.~»

Les yeux de McGonagall s'ouvrirent grand, et elle jeta un regard dur à Harry.

«~Eh bien, Professeur McGonagall, si tout était à refaire, et que je suggérais de prendre cent gallions de plus \emph{juste pour être sûr}, sans justification autre que celle d'être \emph{prêt}, m'écouteriez vous \emph{cette fois}~?

--- J'accepte votre argument, dit McGonagall, Vous n'avez pas besoin de \emph{me} sermonner, jeune homme~!

--- Ah, mais je n'en suis pas encore \emph{arrivé} à mon argument. Connaissez-vous la différence entre quelqu'un qui mérite qu'on lui parle et un simple obstacle, Professeur McGonagall~? De mon point de vue~? Si un adulte pense que m'être supérieur, qu'être au-dessus de moi, qu'obtenir mon obéissance, sont les choses les \emph{plus importantes} pour lui, alors il sera un obstacle. Un \emph{collaborateur potentiel} est quelqu'un qui pense que \emph{faire le travail} est plus important que de s'assurer que je reste à ma place. Laissez-moi vous montrer quelque chose, Professeur McGonagall.~»

Le vendeur de malle les observait avec une fascination non dissimulée, et Harry sortit sa bourse en peau de Moke et dit «~Onze gallions en vrac, s'il vous plaît.~»

Et il y avait de l'or dans la main de Harry.

«~\emph{Où avez-vous obtenu cet…}

--- Dans ma chambre forte, Professeur McGonagall, quand je suis tombé dans ce tas d'or. J'ai fourré de l'argent dans ma poche et j'ai ensuite tenu le sac d'or contre ma poche, pour que les tintements semblent venir de là où il fallait. Car, vous comprenez, je m'attendais depuis le début à ce que cela ait lieu.~»

La bouche de McGonagall était grande, grande ouverte.

«~La question est maintenant… êtes-vous en colère parce que j'ai défié votre autorité~? Ou contente que notre journée se termine par un succès au lieu d'un échec~? Je ne vous demande rien \emph{d'autre} en vous posant cette question. Je ne vous promets ni ne vous demande une coopération dans nos affaires futures. Je veux seulement savoir si vous êtes une \emph{collaboratrice potentielle} ou un obstacle… Minerva.~»

Le vendeur s'étrangla bruyamment.

Et la puissante sorcière resta silencieuse.

«~La discipline \emph{doit} être appliquée à Poudlard, dit-elle après qu'une minute entière se fut écoulée. Pour le bien de \emph{tous} les étudiants. Et cela \emph{doit} inclure la courtoisie et l'obéissance à \emph{tous} vos professeurs.~»

Harry inclina sa tête. «~Je comprends, Professeur McGonagall.~» Mais il était tout de même incroyable que, bizarrement, il semble \emph{beaucoup plus} important d'appliquer la discipline quand \emph{vous} étiez en \emph{haut} de la pile que quand vous étiez en bas… mais Harry ne jugea pas sage d'appuyer sur ce point.

«~Dans ce cas… je vous félicite pour votre grande préparation.~»

Harry voulait applaudir, ou vomir, ou s'évanouir, ou quelque chose. C'était la première fois que ce discours avait jamais fonctionné sur un adulte. C'était la première fois qu'\emph{aucun} de ses discours avait jamais fonctionné sur \emph{qui que ce soit}. Peut-être aussi parce que c'était la première fois qu'il avait quelque chose dont un adulte avait sérieusement besoin, mais tout de même…

Minerva McGonagall, +1 point.

Harry s'inclina, et donna le sac d'or et les onze gallions supplémentaires aux mains de McGonagall. «~Je vous le laisse, madame. Pour ma part, je dois utiliser les toilettes. Puis-je demander où…~»

Le vendeur, onctueux à nouveau, pointa du doigt en direction d'une porte incrustée dans le mur et munie d'une poignée d'or. Alors que Harry s'éloignait, il entendit le vendeur derrière lui dire de sa voix huileuse~: «~Puis-je m'informer de l'identité de cette personne, Madame McGonagall~? J'imagine qu'il est Serpentard -- troisième année peut-être~? -- et d'une importante famille, mais je n'ai pas reconnu…~»

Le claquement de la porte de la salle de bain coupa ses mots, et après que Harry eut identifié le loquet et l'ait mis en place, il s'effondra contre la porte. Son corps entier était baigné d'une sueur qui avait traversé ses vêtements Moldus, mais au moins ça ne se voyait pas sur sa robe. Il se pencha au-dessus de la cuvette or-ivoire, eut quelques haut-le-cœur, mais heureusement rien ne vint.

\later

Ils se tenaient à nouveau dans le jardin du Chaudron Baveur, sur la petite interface couverte de feuilles entre le Chemin de Traverse de l'Angleterre magique et le monde Moldu. C'était une économie \emph{horriblement} découplée… Harry devait aller à une cabine téléphonique et téléphoner à son père une fois de l'autre côté. Il ne devait pas, apparemment, s'inquiéter de voir son bagage volé~; il avait le statut d'objet magique majeur, un type d'objet que les Moldus ne remarqueraient pas. C'était une partie de ce que vous pouviez obtenir dans le monde magique, si vous étiez prêt à payer le prix d'une voiture de seconde main. Harry se demanda si son père serait capable de voir la malle après que Harry la lui eut explicitement montrée.

«~C'est ici que nos chemins se séparent, pour un temps~», dit le Professeur McGonagall. Elle secoua sa tête avec émerveillement. «~Ça a été le jour le plus étrange de ma vie depuis… depuis bien des années. Depuis le jour où j'ai appris qu'un enfant avait vaincu Vous-Savez-Qui. Je me demande maintenant, rétrospectivement, si c'était le dernier jour sensé de ce monde.~»

Oh, comme si \emph{elle} avait à se plaindre de quoi que ce soit. \emph{Vous pensez que votre journée était surréaliste~? Essayez la mienne pour voir}.

«~Vous m'avez grandement impressionné aujourd'hui, lui dit Harry. J'aurais dû penser à vous complimenter à voix haute, je vous donnais des points dans ma tête et tout.

--- Merci, M. Potter, dit McGonagall. Si vous aviez déjà été trié dans une Maison je vous aurais déduit tant de points que vos petits-enfants perdraient encore la Coupe des Maisons.

--- Merci à \emph{vous}, Minerva.~» Il était probablement encore trop tôt pour l'appeler Minny.

Cette femme était peut-être l'adulte le plus sain d'esprit que Harry ait jamais rencontré, en dépit de son manque de savoir scientifique. Harry envisageait même de lui offrir la position de numéro deux dans le groupe qu'il formerait pour combattre le Seigneur des Ténèbres, mais il n'était pas assez idiot pour dire ça à voix haute. \emph{Et quel serait un bon nom pour ce groupe…? Les Mangemangemorts~?}

«~Je vous verrai très bientôt, quand l'école commencera, dit McGonagall. Et, M. Potter, à propos de votre baguette…

--- Je sais ce que vous allez me demander~», dit Harry. Il sortit sa précieuse baguette et, avec un immense pincement de douleur intérieure, la retourna dans sa main. La poignée vers l'extérieur, il la présenta à McGonagall. «~Prenez-la. Je ne comptais pas faire quoi que ce soit, pas une seule petite chose, mais je ne veux pas que vous ayez des cauchemars où je fais exploser ma maison.~»

McGonagall secoua vivement la tête.

«~Oh, non, M. Potter~! On ne fait pas ce genre de choses. Je voulais juste vous prévenir de ne pas \emph{utiliser} votre baguette chez vous, car il y a des moyens de détecter l'usage de la magie chez les mineurs et c'est interdit sans supervision.

--- Ah, dit Harry et il sourit. \emph{Cela} me semble être une règle \emph{très} sensée. Je suis heureux de voir que le monde magique prend ce genre de choses sérieusement.~»

McGonagall le regarda intensément.

«~Vous le pensez vraiment.

--- Oui, dit Harry. Je comprends. La magie est dangereuse et les règles sont là pour une bonne raison. Certaines affaires sont elles aussi dangereuses. Je le comprends. Souvenez-vous que je ne suis pas stupide.

--- J'ai bien peu de chances de l'oublier. Merci, Harry Potter, cela m'aide à me sentir mieux concernant certaines choses au sujet desquelles je vais devoir vous faire confiance. Au revoir pour l'instant.~»

Harry se détourna pour partir, vers le Chaudron Baveur et jusqu'au monde Moldu.

Et alors que sa main touchait la poignée de la porte, il entendit un dernier murmure derrière lui.

«~Hermione Granger.

--- Quoi~? dit Harry, sa main toujours sur la porte.

--- Cherchez une fille de première année nommée Hermione Granger sur le train vers Poudlard.

--- Qui est-elle~?~»

Il n'y eut pas de réponse, et quand Harry se retourna, McGonagall était partie.

\latersection{Après coup~:}

Le Directeur Dumbledore se pencha par-dessus son bureau. Ses yeux pétillants dévisagèrent McGonagall. «~Alors Minerva, qu'avez-vous pensé de Harry~?~»

McGonagall ouvrit sa bouche. Puis elle ferma sa bouche. Puis elle ouvrit à nouveau sa bouche. Aucun mot ne sortit.

«~Je vois, dit Dumbledore avec gravité. Merci pour votre rapport, Minerva. Vous pouvez y aller.~»

%  LocalWords:  ome zahav ahava Aaaaaaarrrgh QX31 ahemmed Sheesh
%  LocalWords:  Aw
