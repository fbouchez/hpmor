\partchapter{Compromis Tabous}{III}

\lettrinepara{A}{u-dessus} de demi-cercles de pierre surélevés, un océan de mains levées.

\hplettrineextrapara
Les Lords et les Dames du Magenmagot vêtus de robes couleur prune marquées d'un M d'argent regardaient avec sévérité et reproche une jeune fille située plus bas, tremblante et enchaînée.
S'ils s'étaient damnés aux yeux d'un système éthique quelconque, il était clair que cet acte ne leur avait donné qu'une plus haute estime d'eux-mêmes.

La poitrine de Harry tremblait à chacune de ses inspirations.
Son côté obscur avait conçu un plan -- puis il était reparti, car un ton trop froid n'aurait pas été à l'avantage de Hermione~; un fait dont le Harry seulement à moitié glacial ne s'était pas rendu compte…

«~La motion est adoptée~», entonna le secrétaire après que l'on eut fini de compter et que les mains levées se soient rabaissées.
«~Le Magenmagot reconnaît la dette de sang due par Hermione Granger à la maison Malfoy pour la tentative de meurtre sur son héritier et la tentative de destruction de sa lignée.~»

Lucius Malfoy avait un sourire lugubre et satisfait.
«~Et maintenant, dit l'homme à la crinière blanche, je dis que sa dette sera payée…~»

Harry serra ses poings et s'écria~:

«~Par la dette que la maison Malfoy doit à la maison Potter~!

--- Silence~! lâcha la femme recouverte de trop de maquillage rose assise à côté du ministre Fudge.
Vous avez déjà suffisamment perturbé ces délibérations~!
Aurors, escortez-le à l'extérieur~!

--- Attendez, dit Augusta Londubat depuis les sièges supérieurs.
De quelle dette s'agit-il~?~»

Les mains de Lucius Malfoy blanchirent autour de sa canne.
«~La maison Malfoy ne vous doit aucune dette~!~»

Ce n'était pas le plus solide des espoirs, car il n'était basé que sur un article de journal écrit par une femme qui avait reçu un sortilège de faux souvenirs, mais Rita Skeeter avait semblé trouver plausible que M. Weasley puisse devoir une dette à James Potter parce que…

«~Je suis surpris que vous l'ayez oublié, dit Harry d'une voix neutre.
C'était sûrement une époque difficile et douloureuse de votre vie, à travailler sous le joug de l'Imperius de Celui-Dont-Il-Ne-Faut-Pas-Prononer-Le-Nom jusqu'à en être libéré par les efforts de la maison Potter.
Par ma mère, Lily Potter, qui a sacrifié sa vie, par mon père, James Potter, qui a aussi sacrifié sa vie, et bien sûr par moi.~»

Il y eut un bref silence dans la Très Ancienne Chambre.

«~Ah, mais quelle excellente remarque, dit la vieille sorcière qui s'était révélée être Mme Bones.
Je suis moi aussi assez surprise de constater que Lord Malfoy a pu oublier un événement d'une telle importance.
Cela doit avoir été un jour tellement heureux pour lui.

--- Oui, dit Augusta Londubat.
Il a dû être tellement reconnaissant.~»

Mme Bones hocha la tête.
«~La maison Malfoy ne pourrait certainement pas nier cette dette -- à moins peut-être que Lord Malfoy ne nous dise qu'il se souvient mal de quelque chose~?
Cela m'intéressait pour des raisons tout à fait professionnelles.
Nous essayons toujours d'obtenir une meilleure idée de ce qui s'est passé pendant cette sombre époque.~»

Les mains de Lucius Malfoy agrippaient sa canne à poignée serpent comme s'il était sur le point de l'abattre, de relâcher le pouvoir qui y était contenu, quel qu'il fut…

Puis il sembla se détendre et un sourire froid prit forme sur son visage.

«~Bien sûr, dit-il avec aisance.
Je confesse ne pas avoir compris au premier abord, mais l'enfant a tout à fait raison.
Cependant je ne pense pas que les deux dettes s'annulent tout à fait -- la maison Potter n'essayait après tout que de se sauver elle-même…

--- C'est faux, dit Dumbledore plus haut.

--- … et par conséquent, entonna Lucius Malfoy, je demande aussi une compensation monétaire pour le rachat de la dette de sang due à mon fils.
La loi dit aussi cela.~»

Harry sentit un étrange tressaillement intérieur.
Cela avait aussi été dans l'article de journal, que M. Weasley avait exigé dix-mille Gallions supplémentaires…

«~Combien~?~»
dit le Survivant.

Lucius avait toujours son sourire froid.
«~Cent-mille Gallions.
Si vous n'en avez pas autant dans votre chambre forte, j'imagine que je devrai accepter billet à ordre pour le reste.~»

Un rugissement de protestations s'éleva du côté Dumbledore de la pièce et même certaines des robes prune du centre semblaient assez choquées.

«~Devrions-nous faire appel au vote du Magenmagot~? dit Lucius Malfoy.
Je pense que peu d'entre nous aimeraient voir la petite meurtrière libérée.
Par vote à main levée, qu'une compensation additionnelle de cent-mille Gallions est nécessaire pour annuler la dette~!~»

Le greffier commença à compter, mais ce résultat-là était lui aussi évident.

Harry restait debout et respirait lourdement.

\emph{Tu ferais mieux de ne même pas avoir à y penser}, dit le Gryffondor interne de Harry d'un ton menaçant.

\emph{C'est un achat substantiel}, remarqua Serdaigle.
\emph{Nous devrions passer beaucoup de temps à y réfléchir.}

Cela n'aurait pas dû être difficile.
Ça n'aurait pas \emph{dû}.
Deux-millions de livres sterling~: ce n'était que de l'argent, et l'argent ne valait que ce qu'il permettait d'acheter…

Il était étrange de constater à quel point on pouvait être psychologiquement attaché à “que de l'argent”, et à quel point il pouvait être douloureux de s'imaginer perdre une chambre forte pleine d'or dont on avait même pas imaginé l'existence seulement un an auparavant.

\emph{Kimball Kinnison n'hésiterait pas}, dit Gryffondor.
\emph{Sérieusement.
Genre, il aurait décidé d'un claquement de doigts.
Tu es quel genre de héros~?
Je te déteste déjà parce que tu as dû y réfléchir pendant plus de 50 millisecondes.}

\emph{C'est la vraie vie}, dit Serdaigle.
Perdre tout son argent est plus douloureux pour les vrais gens de la vraie vie que dans les livres avec de\emph{s héros.}

\emph{Quoi~?} s'indigna Gryffondor.
\emph{Tu es de quel côté~?}

\emph{Je ne recommandais aucune réponse en particulier}, dit Serdaigle, \emph{je le disais seulement parce que c'est vrai.}

\emph{Cent-mille Gallions dépensés autrement pourraient-ils sauver plus d'une vie~?} dit Serpentard.
\emph{Nous avons de la recherche à faire, des batailles à mener, et la différence entre être riche de 40~000 Gallions et être endetté de 60~000 n'est pas triviale…}

\emph{Alors nous utiliserons juste un de nos moyens de refaire rapidement de l'argent et nous le regagnerons}, dit Poufsouffle.

\emph{Je ne suis pas certain que ces moyens fonctionneront}, dit Serpentard, \emph{et beaucoup nécessitent un apport initial…}

\emph{Personnellement}, dit Gryffondor, \emph{je vote pour qu'on sauve Hermione puis qu'on fasse équipe pour tuer notre Serpentard.}

La voix du greffier annonça que le comptage était fini et que la motion avait été adoptée…

Les lèvres de Harry s'ouvrirent.

«~J'accepte votre offre~», dirent-elles sans qu'aucune décision n'ait été prise, exactement comme si le débat interne n'avait été qu'un simulacre, qu'une illusion, et que le véritable contrôleur de la voix n'y avait pas pris part.

Il était clair que Lucius Malfoy ne s'était pas attendu à cette réponse.

Le masque de calme de Lord Malfoy se brisa en mille morceaux, ses yeux s'écarquillèrent et il regarda fixement Harry, figé par la stupéfaction.
Sa bouche s'était légèrement ouverte mais il ne parlait pas, et s'il faisait des bruits étranges, ceux-ci ne pouvaient être entendus par-dessus le rugissement de hoquets simultanés venu du Magenmagot…

Un claquement de pierre fit taire la foule.

«~Non~», dit la voix de Dumbledore.

La tête de Harry eut un mouvement sec et se réorienta en direction du vieux sorcier.

Le visage ridé de Dumbledore était pâle, la barbe d'argent tremblait et il semblait en être en phase terminale d'une maladie incurable.

«~Je -- je suis désolé, Harry, mais ce n'est pas à toi de choisir -- car je suis toujours le gardien de ta chambre forte.

--- \emph{Quoi~?}~» dit Harry, trop stupéfait pour mieux formuler sa réponse.

«~Je ne peux pas te laisser t'endetter auprès de Lucius Malfoy, Harry~!
Je ne peux pas~!
Tu ne sais pas -- tu ne te rends pas compte…~»

\emph{MEURS.}

Harry ne savait même pas quelle partie de lui avait parlé, ça avait peut-être été un vote unanime, la rage pure et la furie s'étaient déversées de lui.
L'espace d'un instant il pensa que la simple force de sa colère se verrait pousser des ailes magiques et s'envolerait pour aller frapper le directeur, pour le faire chuter raide mort du podium…

Mais après que la voix mentale eut parlé, le vieux sorcier se tenait toujours là, à regarder Harry, sa longue baguette noire dans sa main droite, son court bâton dans la gauche.

Et les yeux de Harry allèrent aussi jusqu'à l'oiseau rouge-or dont les serres se reposaient sur l'épaule recouverte de noir de Dumbledore, silencieux à un moment où aucun phénix n'aurait dû l'être.
«~Fumseck, dit Harry d'une voix qui lui sembla étrange, pourrais-tu lui crier dessus de ma part~?~»

L'oiseau enflammé perché sur l'épaule du vieux sorcier ne cria pas.
Peut-être le Magenmagot avait-il exigé qu'un sortilège de silence soit lancé sur la créature, autrement il aurait probablement crié du début à la fin.
Mais Fumseck frappa son maître d'une aile d'or qui alla s'abattre sur la tête du vieux sorcier.

«~Je ne peux pas, Harry~!~»
dit le vieux sorcier, et l'angoisse était audible dans sa voix.
«~Je fais ce que je me dois de faire~!~»

Et Harry sut alors, en regardant l'oiseau rouge-or, ce qu'il se devait lui aussi de faire.
Elle aurait dû être évidente depuis le début, cette solution.

«~Alors je ferais ce que je me dois de faire moi aussi~», dit Harry, le menton levé vers Dumbledore, comme s'ils se tenaient tous deux seuls dans la pièce.
«~Vous comprenez cela vous aussi, n'est-ce pas~?~»

Le vieux sorcier secoua sa tête tremblante.

«~Tu changeras d'avis quand tu seras plus âgé…

--- Je ne parle pas de cela, dit Harry d'une voix qui lui semblait toujours étrange.
Je veux dire que je ne laisserai pas Hermione Granger se faire manger par des Détraqueurs sous quelques circonstances que ce soit.
Point. Peu importe ce qu'une loi quelconque peut en dire et peu importe ce que j'aurai à faire pour l'empêcher.
Dois-je continuer~?~»

Un étrange voix masculine parla, quelque part, loin en dessous~: «~Assurez-vu que la fille soit menée directement à Azkaban et placée sous garde renforcée.~»

Harry attendit en regardant le vieux sorcier puis parla de nouveau.
«~J'irai à Azkaban~», dit-il à celui-ci, comme s'ils étaient seuls au monde, «~avant qu'Hermione ne puisse y être menée, et je commencerai à claquer des doigts.
J'y perdrai peut-être ma vie, mais lorsqu'elle arrivera là-bas, Azkaban n'existera plus.~»

Quelques membres du Magenmagot eurent des hoquets de surprise.

Puis un grand nombre d'entre eux commença à rire.

«~Et comment irais-tu là-bas, petit garçon~? dit une voix parmi les rieurs.

--- J'ai mes propres moyens de déplacement~», dit la voix distante du garçon.
Il garda ses yeux braqués sur Dumbledore, sur le vieux sorcier qui l'observait, choqué.
Harry ne regarda pas directement Fumseck, il ne dévoila pas son plan~; mais il se préparait mentalement à faire appel au phénix pour que celui-ci le transporte, il se préparait à emplir son esprit de lumière et de furie, à appeler l'oiseau de feu de tout son pouvoir, car il devrait peut-être le faire à la seconde où Dumbledore pointerait sa baguette…

«~Le ferais-tu vraiment~?~»
dit le vieux sorcier à Harry, toujours comme s'ils étaient seuls dans la pièce.

La pièce s'emplit à nouveau de silence lorsque tout le monde dirigea un regard stupéfait vers le président sorcier du Magenmagot qui semblait prendre la folle menace parfaitement au sérieux.

Les yeux du vieux sorciers restèrent rivés à Harry.

«~Risquerais-tu tout -- absolument tout -- seulement pour elle~?

--- Oui~», répondit Harry.

\emph{C'est la mauvaise réponse, tu sais}, dit Serpentard.
\emph{Sérieux.}

\emph{Mais c'est la vraie réponse.}

«~Tu n'entendras pas raison~? dit le vieux sorcier.

--- On dirait bien que non~», répondit Harry.

Les regards demeurèrent ancrés l'un dans l'autre.

«~C'est une terrible folie, dit le vieux sorcier.

--- J'en suis conscient, répondit le héros.
Maintenant hors de mon chemin.~»

Une étrange lumière luit dans les vieux yeux bleus.
«~Comme tu le souhaites, Harry Potter, mais saches que ce n'est pas fini.~»

Le reste du monde recommença d'exister.

«~Je retire mon objection, dit le vieux sorcier, Harry Potter peut faire ce qu'il désire~», et le Magenmagot explosa dans un rugissement ahuri et ne put être tut que par un dernier coup du bâton de pierre.

Harry tourna la tête pour regarder Lord Malfoy qui semblait avoir vu un chat se transformer en humain et commencer à manger d'autres chats.
Dire qu'il avait l'air perdu était loin de faire justice à son apparence.

«~Vous le feriez vraiment… dit lentement Lucius Malfoy.
Vous paieriez vraiment cent-mille Gallions pour sauver une Sang-de-Bourbe.

--- Je pense qu'il y a environ quarante-mille Gallions dans ma chambre forte~», dit Harry.
Étrange comme cela provoquait \emph{toujours} plus de douleur intérieure que l'idée de risquer sa vie à plus d'une chance sur deux pour détruire Azkaban.
«~Quant aux autres soixante-mille -- quelles sont les règles, exactement~?

--- Tu ne les devras qu'en sortant de Poudlard, dit le vieux sorcier depuis son perchoir.
Mais j'ai peur que Lord Malfoy n'ait certains droits sur toi avant ce terme.~»

Lucius Malfoy demeurait immobile et regardait Harry, les sourcils froncés.

«~Qui est-elle pour vous, alors~?
\emph{Qu'}est-elle pour vous, pour que vous soyez prêt à payer autant dans le but de l'épargner~?

--- Mon amie,~» dit doucement le garçon.

Les yeux de Lucius Malfoy se plissèrent.

«~Selon le rapport que j'ai reçu, vous ne pouvez pas lancer le Patronus, et Dumbledore le sait.
Le pouvoir d'un seul Détraqueur a failli vous tuer.
Vous n'oseriez pas vous approcher d'Azkaban en personne…

--- C'était en janvier, dit Harry.
Nous sommes en avril.~»

Les yeux de Lucius Malfoy demeurèrent froids et calculateurs.
«~Vous prétendez pouvoir détruire Azkaban et Dumbledore prétend le croire.~»

Harry ne répondit pas.

L'homme aux cheveux blancs se détourna légèrement vers le centre du demi-cercle comme pour s'adresser aux étages supérieurs du Magenmagot.
«~Je retire mon offre~! cria Lord Malfoy.
Je n'accepterai pas la dette de la maison Potter en paiement, pas même pour cent-mille Gallions~!
La dette de sang que la fille doit à la maison Malfoy tient~!~»

De nouveau un rugissement de voix.
«~Déshonorant~! s'écria quelqu'un.
Vous reconnaissez votre dette envers la maison Potter et pourtant vous…~»
puis cette voix se tut abruptement.

«~Je reconnais la dette, mais je ne suis pas strictement obligé par la loi de l'accepter en annulation de l'autre, dit Lucius Malfoy avec un lugubre sourire.
Cette fille ne fait pas partie de la maison Potter~; la dette que je dois à cette maison n'est pas sienne.
Quant au \emph{déshonneur}…~»
Lucius Malfoy s'interrompit.
«~Quant à la grave honte que je ressens au sujet de mon ingratitude envers les Potter qui ont tant fait pour moi…~»
Lucius Malfoy inclina la tête.
«~Puissent mes ancêtres me pardonner.

--- Eh bien, mon garçon~? dit l'homme balafré assis à la droite de Lord Malfoy.
Va donc, et détruis Azkaban~!

--- J'aimerais voir ça, dit une autre voix.
Vendrez-vous des tickets pour le spectacle~?~»

Il va sans dire que Harry ne choisit pas cet instant pour abandonner.

\emph{Cette fille ne fait pas partie de la maison Potter…}

À vrai dire, il avait vu la solution à ce dilemme presque instantanément.

Il aurait peut-être mis plus longtemps s'il n'avait pas récemment surpris plusieurs conversations entre des filles de Serdaigle plus âgées et lu un certain nombre d'articles du Chicaneur.

Il avait néanmoins du mal à l'accepter.

\emph{C'est ridicule} dit une partie de Harry qui venait de se surnommer le Vérificateur de Cohérence Interne\emph{.
Nos actes sont complètement incohérents.
D'abord tu ressens moins de réticence émotionnelle à risquer ta VIE et à probablement MOURIR pour Hermione que de te séparer d'un tas d'or stupide.
Et maintenant tu rechignes juste à te marier~?}

\emph{ERREUR SYSTÈME.}

\emph{Tu sais quoi~?} Dit le Vérificateur de Cohérence Interne, \emph{tu es stupide.}

\emph{Je n'ai pas dit non}, songea Harry.
\emph{Je disais juste ERREUR SYSTÈME.}

\emph{Je vote pour qu'on détruise Azkaban}, dit Gryffondor.
\emph{Ça devra être fait de toute façon.}

\emph{Vraiment, vraiment stupide}, dit le Vérificateur de Cohérence Interne.
\emph{Oh, et puis merde, je prends le contrôle de notre corps.}

Le garçon prit une profonde inspiration, ouvrit la bouche…

À ce stade, Harry Potter avait entièrement oublié l'existence du professeur McGonagall, qui était resté assise pendant tout ce temps et avait subi un certain nombre d'intéressants changements d'expression faciale que Harry n'avait pas regardés parce qu'il était distrait.
Il aurait été trop dur de dire que Harry l'avait oubliée parce qu'il ne la voyait pas comme un PJ.
Il aurait pu être plus généreux de dire que le professeur McGonagall n'était une solution potentielle à aucun de ses problèmes actuels et qu'elle ne faisait donc pas partie de l'univers.

Donc Harry, qui à ce stade avait une bonne dose d'adrénaline dans le sang, sursauta de façon assez visible lorsque le professeur McGonagall dont les yeux étincelaient maintenant d'un espoir impossible et dont les joues étaient couvertes de larmes à moitié sèches bondit et s'écria~: «~\emph{Avec moi, M. Potter~!}~» et qui sans attendre de réponse descendit quatre à quatre les marches des escaliers qui menaient à la plateforme inférieure où attendait une chaise de métal noir.

Il mit un moment à le faire, mais Harry courut après elle, même s'il mit plus longtemps qu'elle à atteindre l'étage le plus bas après qu'elle eut sauté au-dessus de la dernière moitié des marches d'un mouvement étrangement félin et qu'elle ait atterri alors que le trio d'Aurors à l'air abasourdi pointait déjà ses baguettes vers elle.

«~Mlle Granger~! s'écria le professeur McGonagall.
Pouvez-vous parler~?~»

Comme pour le professeur McGonagall, on pouvait d'une certaine façon dire que Harry avait oublié l'existence de Hermione Granger, car il avait incliné son cou en arrière pour regarder vers le haut plutôt que vers le bas et parce qu'il ne l'avait considérée comme une solution à aucun de ses problèmes actuels.
Quoi qu'il fût loin d'être certain, et il n'était même pas probable du tout, que la situation eut été améliorée de quelque façon que ce soit si Harry s'était souvenu de regarder Hermione ou de penser à ce qu'elle devait ressentir.

Harry atteint le bas des escaliers et vit Hermione de pied en cap…

Sans y penser, sans pouvoir même s'en empêcher, Harry ferma les yeux.
Mais il avait vu.

Sa robe autour de son cou, inondée de larmes.

La façon dont elle avait détourné les yeux à \emph{sa} vue.

Et l'œil du souvenir et de l'empathie, qu'on ne pouvait jamais refermer, qui ne pouvait pas détourner le regard, sut qu'Hermione venait de raconter la pire honte de sa vie face aux nobles d'Angleterre magique, face au professeur McGonagall, face à Dumbledore et à Harry~; et qu'elle avait été condamnée à Azkaban, où elle serait exposée aux ténèbres, au froid et à ses pires souvenirs jusqu'à devenir folle et à mourir~; et qu'elle avait ensuite entendu que Harry allait perdre tout son argent et s'endetter pour la sauver et qu'il allait peut-être même sacrifier sa vie

avec le Détraqueur quelques pas à peine derrière elle

elle n'avait rien dit…

«~O-oui, murmura la voix de Hermione Granger.
Je p-peux parler.~»

Harry rouvrit les yeux et vit son visage qui le regardait à présent.
Ce visage ne communiquait rien qui puisse laisser comprendre ce qu'il pensait qu'Hermione ressentait, car les visages ne pouvaient rien dire d'aussi compliqué, et tout ce que les muscles faciaux pouvaient faire, c'était de se contorsionner et de se nouer.

«~H-H-Harry, j-je suis tellement, je suis tellement…

--- La ferme, suggéra Harry.

--- d-d-désolée…

--- Si tu ne m'avais pas rencontré dans le train tu n'aurais aucun ennui à l'heure qu'il est.
Alors la ferme, dit Harry Potter.

--- Arrêtez de vous comporter comme des idiots, tous les deux~», dit le professeur McGonagall de son ferme accent écossais (il était étrange de constater à quel point l'accent aidait).
«~M. Potter, tendez votre baguette pour que les doigts de Mlle Granger puissent la toucher.
Mlle Granger, répétez après moi.
Sur ma vie et ma magie…~»

Harry fit ce qu'on lui avait dit et avança sa baguette afin qu'elle touche les doigts de Hermione, puis la faible voix de cette dernière dit~:

«~Sur ma vie et ma magie…

--- Je jure de servir la maison Potter…~»
dit le professeur McGonagall.

Et Hermione dit, sans attendre d'instruction supplémentaire, les mots s'écoulant d'elle à toute vitesse~: «~Je jure de servir la maison Potter, d'obéir à son maître ou à sa maîtresse, de me tenir à leur droite, de combattre selon leurs ordres et de les suivre où ils iront jusqu'au jour de ma mort.~»

Tous ces mots avaient été déballés dans une expiration désespérée, avant que Harry ne puisse penser ou dire quoi que ce soit, au cas où il aurait été assez fou pour interrompre.

«~M. Potter, répétez ces mots, dit le professeur McGonagall.
Moi, Harry, héritier et dernier descendant des Potter, accepte que vous me serviez jusqu'à la fin du monde et de sa magie.~»

Harry prit une profonde inspiration et dit~:

«~Moi, Harry, héritier et dernier descendant des Potter, accepte que vous me serviez jusqu'à la fin du monde et de sa magie.

--- C'est fait, dit le professeur McGonagall.
Bien joué.~»

Harry leva les yeux et vit que tout le Magenmagot, dont il avait oublié l'existence, les regardait fixement.

Puis Minerva McGonagall qui, bien qu'elle n'en ait pas toujours l'air, \emph{était} directrice de Gryffondor, leva les yeux vers l'endroit où Lucius Malfoy se tenait, et elle lui dit, face à tout le Magenmagot~: «~Je regrette chaque point que je t'ai donné en Métamorphose, espèce de vil petit vers.~»

Quoi que Lucius ait été sur le point de répondre, cela fut réduit au silence par un coup du court bâton dans la main de Dumbledore.
«~Ahem~! dit le vieux sorcier sur son podium de pierre noire.
Cette séance a déjà duré un certain temps et si elle ne s'achève pas bientôt, certains pourraient manquer leur déjeuner.
La loi en la matière est claire.
Vous avez déjà voté sur les termes de l'échange, et Lord Malfoy ne peut légalement le décliner.
Comme nous avons de beaucoup dépassé le temps qui nous est alloué et en accord avec la dernière décision des survivants du quatre-vingt-huitième Magenmagot, je lève la séance.~»

Le vieux sorcier frappa trois fois de son bâton de pierre.

«~Idiots~!~»
s'écria Lucius Malfoy.
Les cheveux blancs se secouaient autant que si un vent les avait agités et le visage en dessous était rendu pâle par la furie.
«~Vous pensez pouvoir vous en tirer avec ce que vous avez fait aujourd'hui~?
Vous pensez que cette fille peut essayer de tuer mon fils et s'en sortir sans dommages~?~»

La femme-crapaud au maquillage rose dont Harry ne pouvait se remémorer le nom se leva de son siège.
«~Allons, mais bien sûr que non, dit-elle avec un sourire écœurant.
Après tout, cette fille est \emph{toujours} une meurtrière, et je pense que le ministère s'intéressera de très près à ses actes -- il semble fort peu sage qu'on l'autorise à déambuler dans les rues, après tout…~»

Harry en eut assez.

Sans finir d'écouter, il pivota sur ses talons et marcha à grands pas vers…

L'horreur que lui seul pouvait vraiment voir, l'absence de couleur et d'espace, la blessure dans le monde au-dessus de laquelle flottait une cape en lambeaux~; très imparfaitement gardée par un écureuil à l'éclat lunaire et par un moineau d'argent voltigeant.

Son côté obscur avait lui aussi remarqué, lorsqu'il avait parcouru l'intégralité de la pièce à la recherche de toute chose pouvant être utilisée comme arme, que l'ennemi avait été assez sot pour mettre un Détraqueur en présence de Harry.
C'était bel et bien une arme puissante, mais une que Harry manierait peut-être mieux que ses supposés maîtres.
Après tout, Azkaban avait un jour vu Harry dire à douze Détraqueurs de faire volte-face et de partir, et ils étaient partis.

\emph{Les Détraqueurs sont la Mort et le Patronus marche en ayant des pensées heureuses au lieu de penser à la Mort.}

Si la théorie de Harry était correcte, cette phrase seule était tout ce qui était nécessaire pour faire éclater les Patronus des Aurors comme des bulles de savon et s'assurer que personne à portée d'oreille ne pourrait en lancer un autre.

\emph{Je vais annuler les Patronus et empêcher d'autres Patronus d'être lancés.
Puis mon Détraqueur, plus rapide que n'importe quel balai volant, va Embrasser tous ceux ici qui ont voté pour envoyer une fille de douze ans à Azkaban.}

Dire cela pour mettre en place les attentes et patienter jusqu'à ce que les gens comprennent et rient.
Puis dire la fatale vérité, et lorsque les Patronus des Aurors se seraient volatilisés, confirmant les dires de Harry, alors soit l'\emph{anticipation} que les autres auraient au sujet du vide sans esprit soit les menaces de destruction de Harry pousseraient le Détraqueur à obéir.
Ceux qui avaient cherché à faire des compromis avec les ténèbres seraient consumés par elle.

C'était l'autre solution que son côté obscur avait inventée.

Ignorant les halètements de surprises qui s'élevaient derrière lui, Harry passa entre les deux Patronus et s'avança à un unique pas de la Mort.
Sa peur, qui n'était plus entravée, jaillit autour de lui comme un tourbillon, comme s'il venait de mettre le pied à côté du siphon d'une baignoire géante dont on aurait été en train de vider l'eau~; mais sans les faux Patronus pour obscurcir la nature de leur interaction, Harry pouvait atteindre le Détraqueur autant que celui-ci pouvait l'atteindre lui.
Il regarda droit dans le vide attracteur et…

\emph{la Terre au milieu des étoiles}

tout son triomphe d'avoir sauvé Hermione

\emph{un jour, la réalité dont tu es une ombre cessera d'exister}

Harry se saisit de toutes les émotions argentées qui alimentaient son Patronus, les \emph{poussa} vers le Détraqueur, s'attendit à ce que l'ombre de la Mort le fuie…

… et tout en faisant cela, il leva les mains et s'écria~: «~\shout{Bouh}~!~»

Le vide battit vivement en retraite jusqu'à être acculé contre la pierre noire.

Un silence mortel emplissait la salle.

Harry se détourna du néant vide et leva les yeux vers l'endroit où se tenait la femme-crapaud.
Elle était pâle sous son maquillage rose et sa bouche s'ouvrait et se fermait comme celle d'un poisson.

«~Je vous fais cette seule offre, dit le Survivant.
Je n'apprends jamais que vous avez interféré avec moi ou les miens.
Et vous n'apprenez jamais pourquoi le monstre mangeur d'âme intuable a peur de moi.
Maintenant asseyez-vous et fermez-la.~»

La femme-crapaud retomba sur son banc sans un mot.

Harry leva les yeux encore plus haut.

«~Une devinette, Lord Malfoy~! s'écria le Survivant à travers la Très Ancienne Chambre.
Je sais que vous n'étiez pas à Serdaigle, mais essayez quand même de répondre à celle-là~!
Qu'est-ce qui détruit des Seigneurs des Ténèbres, fait peur aux Détraqueurs et vous doit soixante-mille Gallions~?~»

L'espace d'un instant, Lord Malfoy se tint là, les yeux légèrement écarquillés~; puis son visage retomba à un mépris calme et sa réponse fut fraîche et maîtrisée~:

«~Me menacez-vous ouvertement, M. Potter~?

--- Je ne vous menace pas, dit le Survivant.
Je vous fais \emph{peur}.
Il y a une différence.

--- Assez, M. Potter, dit le professeur McGonagall.
Nous allons déjà être assez en retard pour le cours de Métamorphose de cet après-midi.
Et revenez ici, vous terrifiez encore ce pauvre Détraqueur.~»
Elle se tourna vers les Aurors.
«~M. Kleiner, si vous voulez bien~!~»

Harry revint jusqu'à eux et l'Auror à qui on avait parlé s'avança, appuya un court bâton de métal noir sur la chaise de métal noir et marmonna un inaudible mot révocatoire.

Les chaînes glissèrent aussi souplement qu'elles étaient venues, Hermione se releva de sa chaise aussi rapidement qu'elle en était capable, courut et tituba à moitié de quelques pas.

Harry tendit les bras…

… Hermione bondit et tomba à moitié dans les bras du professeur McGonagall en commençant à sangloter de façon hystérique.

\emph{Hhmpf}, dit une voix à l'intérieur de Harry.
\emph{Je pensais qu'on l'avait mérité, celle-là.}

\emph{Oh, tais-toi.}

Le professeur McGonagall tenait Hermione avec une telle force qu'on aurait pu croire que c'était une mère tenant sa fille, ou peut-être sa petite fille.
Au bout de quelques instants, les sanglots de Hermione ralentirent et s'arrêtèrent.
Le professeur McGonagall changea brusquement de posture et renforça sa prise~: les mains de la fille pendaient mollement et ses yeux étaient fermés…

«~Elle s'en remettra, M. Potter, dit doucement le professeur McGonagall à l'attention de Harry mais sans le regarder.
Elle a juste besoin de quelques heures dans l'un des lits de Mme Pomfresh.

--- Très bien, dans ce cas, dit Harry.
Amenons-la chez Mme Pomfresh.

--- Oui, dit Dumbledore en descendant jusqu'au bas des escaliers de pierre noire.
Rentrons tous chez nous.~»
Ses yeux bleus étaient braqués sur ceux de Harry, durs comme des saphirs.

\later

Les Lords et les Dames du Magenmagot quittent leurs bancs de bois comme ils sont venus, l'air plutôt nerveux.

La vaste majorité d'entre eux pense~: “Le Détraqueur avait peur du Survivant~!”

Certains des plus astucieux se demandent déjà comment cela affectera la délicate balance du pouvoir du Magenmagot -- si une nouvelle pièce est apparue sur le plateau de jeu.

Presque aucun d'entre eux ne pense quelque chose comme~: “Je me demande comment il a fait ça”.

C'est la réalité du Magenmagot~: nombre sont nobles, nombre sont de riches magnats des affaires, quelques-uns ont obtenu leur statut par d'autres moyens.
Certains sont stupides.
La plupart sont rusés en affaires et en politique mais leur ruse est restreinte.
Quasiment aucun d'entre eux n'a suivi la voie du mage de pouvoir.
Ils n'ont pas lu de tomes anciens à la recherche de vérités trop puissantes pour rester à découvert et déguisées en casse-têtes, étudié de vieux parchemins en chasse de véritables magies cachées au milieu de cent contes de fée.
Lorsqu'ils ne regardent pas un contrat d'endettement, ils abandonnent toute la ruse qu'ils possèdent et se relaxent dans une absurdité confortable.
Ils croient aux Reliques de la Mort, mais ils croient aussi que Merlin a combattu le terrifiant Totoro et a emprisonné le Ree.
Ils savent (parce que cela aussi fait partie de la légende standard) qu'un puissant sorcier doit savoir distinguer la vérité entre cent mensonges plausibles.
Mais il ne leur est jamais venu à l'esprit de faire de même.

(Pourquoi pas~?
Mais alors pourquoi des sorciers dotés d'un statut et d'une fortune assez grande pour se consacrer à quasiment n'importe quelle entreprise choisiraient de passer leurs vies à se battre pour le contrôle de monopoles lucratifs ou d'importation d'encres~?
Le directeur de Poudlard s'intéresserait à peine à la question~: bien sûr que la plupart des gens ne devraient pas être des mages de pouvoir, tout comme la plupart des gens ne devraient pas être des héros.
Le professeur de Défense pourrait expliquer, longuement et cyniquement, pourquoi leurs ambitions sont si triviales~; pour lui non plus, il n'y a là aucune d'énigme.
Seul Harry Potter, en dépit de tous les livres qu'il a lus, est incapable de comprendre~; pour le Survivant, les choix des Lords et Dames semblent incompréhensibles~: pas ceux qu'une bonne personne ferait ni ceux qu'une mauvaise personne ferait non plus.
Alors lequel des trois est-il le plus sage~?)

Donc, quelle qu'en soit la raison, la majeure partie du Magenmagot n'a jamais suivi la voie qui mène à la puissance magique~; ils ne recherchent pas ce qui est caché.
Pour eux, il n'y a pas de \emph{pourquoi}.
Il n'y a pas d'explication.
Il n'y a pas de causalité.
Le Survivant, qui appartenait déjà à moitié au magistère de la légende, y a été entièrement promu, et il existe maintenant un fait brut, simple et sans explication~: le Survivant effraie les Détraqueur.
Dix ans plus tôt, on leur a dit qu'un enfant de un an avait vaincu le plus terrible Seigneur des Ténèbres de leur génération, peut-être pire Seigneur des Ténèbres à avoir jamais vécu~; et ils ont aussi accepté cela.

On n'est pas censé remettre ce genre de chose en question (ils le savent implicitement).
Si le plus terrible Seigneur des Ténèbres de l'histoire se confronte à un bébé innocent -- allons, comment pourrait-il ne \emph{pas} être vaincu~?
Le rythme de la pièce l'exige.
On est censé applaudir, pas se lever de son siège de spectateur et dire “Pourquoi~?”.
C'est le concept même de l'histoire~: que le Seigneur des Ténèbres est abattu par un petit enfant, et si vous comptez remettra ça en question, alors autant ne même pas assister à la pièce.

L'idée ne leur vient pas de remettre en question l'application d'un tel raisonnement aux événements dont ils ont eux-mêmes été les témoins dans la Très Ancienne Chambre.
De fait, ils ne sont pas conscients du fait qu'ils appliquent un raisonnement fait pour la fiction à la vie réelle.
Quant à examiner le Survivant avec la même logique prudente que celle dont ils useraient pour une alliance politique ou un accord commercial -- quel cerveau ferait \emph{ça} quand une partie du magistère de la légende est en jeu~?

Mais il y en a quelques-uns, très rares, assis sur ces bancs de bois, qui ne réfléchissent \emph{pas} comme cela.

Il existe un tout petit groupe au Magenmagot qui a lu des parchemins à moitié désintégrés, a écouté l'histoire de ces choses qui sont arrivées au cousin du frère de quelqu'un, pas pour passer le temps, mais au sein d'une quête de pouvoir et de vérité.
Ils ont déjà noté la nuit de Godric's Hollow telle que racontée par Albus Dumbledore comme étant un événement anormal et potentiellement important.
Ils se sont demandé pourquoi cela s'est produit, si cela s'est produit, et si non, pourquoi Dumbledore ment.

Et lorsqu'un garçon de onze ans se lève et dit «~Lucius Malfoy~» de cette voix froide et adulte puis continue à prononcer des mots qu'on ne s'attendrait tout simplement pas à voir émaner d'un enfant en première année à Poudlard, ils ne laissent pas ce fait se glisser dans le flou sans loi des légendes et des préambules de pièces de théâtre.

Ils notent que c'est un indice.

Ils l'ajoutent à la liste.

Cette liste commence à avoir l'air relativement alarmante.

Ça n'aide pas particulièrement quand le garçon crie “\shout{Bouh}~!” à un Détraqueur et que le corps en décomposition s'appuie contre le mur derrière lui et que son horrible et douloureuse voix grince~: “\emph{Faites-le partir d'ici.}”

%  LocalWords:  Kleiner Hmpfh Totoro Ree
