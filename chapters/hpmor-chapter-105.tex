\partchapter{La Vérité}{II}

\lettrinepara{T}{\emph{om}} \emph{Jedusor.}

\hplettrineextrapara
Les mots résonnèrent dans l'esprit de Harry, produisant d'éphémères échos, des motifs brisés qui cherchaient à se compléter avant de s'effondrer.

\emph{Tom Jedusor est une}

\emph{Tom Jedusor était}

\emph{Énigme}

Mais Harry avait d'autres priorités.

Le professeur Quirrell avait un pistolet braqué sur lui.

Et étrangement, Lord Voldemort n'avait pas encore tiré.

La voix de Harry ressembla à un coassement.
«~Qu'attendez-vous de moi~?

--- Te voir mourir, répondit le professeur Quirrell, certainement pas. Si je l'avais voulu, tu serais mort depuis longtemps. La bataille fatidique en Lord Voldemort et le Survivant est un pur produit de l'imagination de Dumbledore. Je sais où se trouve ta maison familiale et je sais ce qu'est un fusil à lunette. Tu aurais été mort avant d'avoir touché une baguette. J'espère que tu comprends ça, Tom.

--- C'est limpide~», murmura Harry. Son corps tremblait encore, exécutait des programmes plus appropriés à la fuite face à un tigre qu'au lancer de délicats sortilèges ou à la \emph{réflexion}. Mais il put en partie deviner ce que celui qui tenait l'arme désirait de lui, une question que l'autre espérait voir posée~; et Harry s'exécuta. «~Pourquoi m'appelez-vous Tom~?~»

Le professeur Quirrell le regardait fixement. «~\emph{Pourquoi} est-ce que je t'appelle Tom~? Réponds. Tu n'as pas tout à fait l'intellect que j'espérais, mais il devrait te suffire pour répondre.~»

La bouche de Harry parut connaître la réponse avant même que son cerveau ne se penche sur la question. «~C'est votre nom. Notre nom. C'est Lord Voldemort, ou c'était, ou…~»

Le professeur Quirrell hocha la tête. «~Mieux. Tu as déjà vaincu le Seigneur des Ténèbres, pour la première et la dernière fois. Je n'ai laissé qu'un vestige de Harry Potter et éliminé ce qui différenciait nos esprits, ce qui nous permet de vivre dans le même monde. Maintenant qu'il est clair que ce combat entre nous n'est qu'un mensonge, tu es libre d'agir raisonnablement et selon ce qui a de l'importance à tes yeux. Ou pas.~» Le pistolet fut légèrement relevé, et des perles de sueur apparurent sur le front de Harry. «~Lâche ta baguette. \emph{Maintenant}.~»

Harry s'exécuta.

«~Éloigne-t'en~», dit le professeur Quirrell.

Harry obéit.

«~Tends une main vers ton cou, dit le professeur Quirrell, et enlève ton Retourneur de Temps en ne touchant que la chaîne. Place-le au sol, et éloigne-t'en aussi.~»

Harry le fit aussi. Même en état de choc, son esprit cherchait encore une façon de faire pivoter le Retourneur de Temps, de vaincre par un coup soudain~; mais il savait que le professeur Quirrell se mettait déjà à sa place, imaginait les différentes possibilités…

«~Enlève ta bourse et place-la au sol, puis éloigne-toi.~»

Ce qu'il fit.

«~Très bien, dit le professeur de Défense. Maintenant, il est temps pour moi d'obtenir la Pierre Philosophale. Je compte emmener ces quatre élèves de première année après leur avoir fait oublier leurs souvenirs les plus récents, pour qu'ils se souviennent toujours de leur but initial. Je contrôlerai Rogue et je lui ferai garder cette porte. Après cette journée de travail, je compte tuer Rogue pour ses trahisons envers mon autre identité. Et sache que j'ai des otages. J'ai déclenché un sortilège qui va tuer des centaines d'élèves de Poudlard, y compris nombre de tes amis. Si je parviens à obtenir la pierre, je pourrai interrompre ce sortilège. Sinon, ou si je choisis de ne pas le faire, des centaines d'élèves mourront.~» La voix du professeur Quirrell était toujours douce. «~Est-ce que ce sont là des choses qui t'importent, petit~? Je serais heureux de t'entendre dire non, mais ce serait trop espérer.

--- Je voudrais,~» parvint à dire Harry à travers l'horreur, le chagrin et les lames qui, rompant un lien émotionnel, provoquaient la douleur d'un membre qu'on ampute, «~que vous ne fassiez rien de tout ça, Professeur.~» \emph{Pourquoi, Professeur Quirrell, pourquoi est-ce que ça se termine comme ça~? Je ne… je ne veux pas admettre que c'est vrai…}

«~Très bien, dit le professeur Quirrell. Je t'autorise à me donner quelque chose que je désire.~» Il agita l'arme en signe d'encouragement. «~C'est un privilège rare, petit. Lord Voldemort n'a pas l'habitude de négocier.~»

Une partie de Harry se lança dans une quête frénétique pour trouver ce qui pourrait avoir plus d'importance, aux yeux de Lord Voldemort ou de professeur Quirrell, que la mort de Severus ou quelques jeunes otages.

Une autre partie de lui, celle qui n'avait jamais cessé de réfléchir, connaissait déjà la réponse.

«~Vous attendez déjà quelque chose de moi~». Il s'exprimait à travers les meurtrissures de son âme torturée. «~Qu'est-ce que c'est~?

--- Que tu m'aides à obtenir la Pierre Philosophale.~»

Harry déglutit bruyamment. Il n'arrivait pas à empêcher ses yeux de faire l'aller-retour entre le visage du professeur Quirrell et le pistolet.

Il savait bien que le héros d'un livre de contes aurait dit “Non”, mais ici, dans la réalité, dire “Non” semblait absurde.

«~Je suis déçu de voir que tu as besoin d'y réfléchir, dit le professeur Quirrell. Tu devrais évidemment m'obéir, puisque j'ai un ascendant total sur toi. Je te l'ai déjà appris~: dans cette situation, tu devrais faire semblant de perdre. Il n'y a rien à espérer gagner dans la lutte, sinon de la souffrance. Et tu aurais dû éviter de perdre ma confiance en prenant soin de me répondre plus vite.~» Le professeur Quirrell l'observa avec curiosité. «~Peut-être Dumbledore t'a-t-il rabâché des âneries sur la noblesse de la bravoure~? Je trouve cette moralité amusante, elle est si simple à manipuler. Je t'assure que je peux faire passer la bravoure pour une turpitude morale, et il serait sage de te soumettre avant que je ne te le prouve.~» Le pistolet demeura dirigé vers Harry, mais d'un geste de son autre main, le professeur Quirrell fit léviter Tracey Davis. Elle pivota lentement, ses membres tendus…

… puis elle redescendit juste au moment où une nouvelle dose d'adrénaline atteignait le cœur de Harry.

«~Choisis, dit le professeur Quirrell. Je commence à perdre patience.~»

\emph{J'aurais dû répondre tout de suite, avant qu'il ne manque d'arracher ses jambes à Tracey… non, le directeur a dit qu'il ne fallait pas montrer à Lord Voldemort que les menaces envers les proches peuvent fonctionner, car ça ne faisait qu'encourager ce dernier… Sauf que ce qu'il vient de dire, ce n'est pas une} menace, c'est juste le genre de choses qu'il a l'habitude de faire\emph{.}

Harry inspira plusieurs fois, profondément. La partie de lui qui avait automatiquement continué de fonctionner hurlait au reste qu'il ne \emph{pouvait pas se permettre de rester en état de choc}. Un choc avait une durée finie et n'empêchait pas les neurones de poursuivre leur activité~; la seule chose capable de le faire s'effondrer, malgré un cerveau en état de marche, c'était le modèle qu'il avait de lui-même, le fait de \emph{croire} qu'il allait s'effondrer…

«~Je n'ai pas l'intention d'éprouver votre patience~», dit-il d'une voix brisée. Tant mieux. Lord Voldemort lui donnerait peut-être plus de temps s'il le croyait en état de choc. «~Mais je n'ai jamais entendu dire que Lord Voldemort respectait sa parole.

--- Une inquiétude compréhensible, dit le professeur Quirrell. Il y a une solution simple, que je comptais utiliser de toute façon. \parsel{Le sserpents ne peuvent pas mentir}. Et, comme la stupidité me déplaît au plus haut poins, je te conseille de ne pas me demander de m'expliquer. Tu vaux mieux que ça, et je n'ai pas de temps à consacrer au genre de conversations que les gens ordinaires s'infligent les uns aux autres.~»

Harry déglutit une fois de plus. Les serpents ne peuvent pas mentir. «~\parsel{Troiss et trois font ssix.}~» Il venait d'essayer de dire que trois plus trois était égal à cinq, et le mot \emph{six} lui avait échappé.

«~Bien. Le véritable plan de Salazar Serpentard, lorsqu'il s'ensorcela, lui et ses enfants, était d'assurer que ses descendants pourraient se faire confiance en dépit des intrigues qu'ils mèneraient contre d'autres.~» Le professeur Quirrell avait adopté le ton professoral qu'il utilisait en cours de magie de bataille comme il aurait mis un masque bien connu, mais le pistolet demeura pointé vers Harry. «~L'Occlumancie ne trompe pas le Fourchelangue comme elle trompe le Veritaserum~; tu es libre de vérifier. Maintenant, écoute-moi bien. \parsel{Ssuis moi, promets-moi de m'aider à obtenir la pierre, et je laissserai ces enfants ici et indemnes. Mais les otages ssont réels, des centaines d'élèves mourront ce ssoir à moins que je n'interrompe ce que j'ai déclenché. J'épargnerai les otages ssi je réusssis à obtenir la pierre.} Et souviens-toi, souviens-toi bien de ceci~: \parsel{Il n'exisste à ma connaisssance aucun moyen de me vaincre~; perdre la pierre ne m'empêchera pas de revenir et ma colère n'épargnera ni toi, ni les tiens.} Aucune des réactions hâtives que tu contemples ne te permettront de gagner, petit. Mais je reconnais ta capacité à m'agacer, et je te conseille d'éviter de le faire.

---Vous avez dit, répondit Harry d'une voix qui lui sembla étrange, que les pouvoirs de la Pierre Philosophale n'étaient pas ceux décrits par la légende. Vous me l'avez dit en Fourchelangue. Avant d'accepter de vous aider, je veux connaître le véritable pouvoir de la Pierre.~» Si la réponse ressemblait à «~permet d'établir un contrôle absolu sur l'univers~», alors \emph{rien} ne justifiait d'augmenter, ne serait-ce que légèrement, les chances que Lord Voldemort entre en sa possession.

«~Ah~», dit le professeur Quirrell, et il sourit. «~Tu réfléchis. Voilà qui est mieux, et pour te récompenser, je vais t'offrir une autre raison de coopérer. La vie éternelle et la jeunesse, la création d'or et d'argent. Suppose que la Pierre offre vraiment cela. Dis-moi, petit. Quel est son pouvoir~?~»

Peut-être était-ce l'adrénaline qui se rendait enfin utile. Peut-être était-ce d'avoir entendu qu'une réponse existait et que les indices n'étaient pas des mensonges. «~Elle permet de rendre les Métamorphoses permanentes.~»

Lorsqu'il entendit ses propres paroles, Harry se tut soudain.

«~Correct, dit le professeur Quirrell. Donc, celui qui détient la Pierre Philosophale peut Métamorphoser des humains.~»

L'esprit déjà en lambeaux de Harry reçut un nouveau choc lorsqu'il comprit quelle raison venait de lui être offerte.

«~Tu as volé le corps de Mlle Granger et tu l'as Métamorphosé en une chose sans valeur apparente, dit le professeur Quirrell. Quelque chose que tu dois garder sur toi, pour maintenir la Métamorphose. Ah, je vois que tu regardes l'anneau à ton doigt, mais Mlle Granger n'est bien sûr pas le petit joyau qui y est serti. Ce serait trop évident. Non, je pense que Mlle Granger est l'anneau lui-même et que l'aura magique du joyau Métamorphosé masque la magie qui émane de l'anneau.

--- Oui~», se força à répondre Harry. Cette fois, c'était un mensonge. Il avait délibérément regardé l'anneau. Comme il s'était attendu à ce que quelqu'un s'intéresse à cet anneau, il avait tenté de provoquer les choses afin de pouvoir prouver son innocence une fois de plus, mais personne ne l'avait mis en doute… peut-être Dumbledore avait-il simplement senti que l'acier ne recelait aucune magie.

«~Fort bien, dit le professeur Quirrell. Maintenant suis-moi, aide-moi à obtenir la Pierre, et je ressusciterai Hermione Granger pour toi. Sa mort a eu une mauvaise influence sur toi, et je serais heureux de la défaire. Si je ne me trompe pas, c'est là ce que tu désires le plus. J'ai souvent été bon envers toi, et je serais heureux de l'être une fois de plus.~» Une professeure Chourave aux yeux ternes s'était relevée et avait pointé sa baguette vers Harry. «~\parsel{Aide-moi à obtenir la pierre de métamorphose et je ferai tout mon posssible pour rendre à la fille-amie une vie durable et véritable. ÇCela dit, petit, je perds vite patiençce, et tu n'as pas envie de connaître la ssuite.}~» Cette dernière phrase fut sifflée d'un ton qui laissait imaginer un serpent sur le point de frapper.

\later

Malgré tout.

Malgré tout, malgré le chaos, malgré les chocs successifs, le cerveau de Harry demeurait et ses circuits complétaient les motifs pour lesquels ils avaient été mis en place.

Harry savait que, faite à celui que l'on menace d'une arme, cette offre était trop belle pour être vraie.

À moins d'avoir \emph{désespérément} besoin de son aide pour sortir la Pierre Philosophale du miroir magique.

Il n'avait plus le temps d'échafauder des plans, mais il songea que si le professeur Quirrell était prêt à aller jusque-là pour obtenir son aide, \emph{l'idéal} aurait été d'exiger que le professeur Quirrell promette de ne plus jamais tuer personne en échange, mais Harry était convaincu que le professeur répondrait “Ne sois pas ridicule”~; et ils n'avaient pas le temps d'avoir une conversation ordinaire, il lui fallait déterminer à l’avance la requête à la fois la plus sûre et la plus ambitieuse possible.

Le professeur Quirrell plissa les yeux, écarta les lèvres…

«~Si je vous aide, dit Harry, vous devez promettre que vous ne comptez pas vous en prendre à moi ensuite. Je veux que vous épargniez le professeur Rogue et tous les habitants de Poudlard pendant au moins une semaine. Et je veux des réponses, je veux savoir ce qui se passe depuis le début, et tout ce que vous savez quant à ma nature.~»

Les yeux bleus et pâles le regardaient, détachés.

\emph{Je pense qu'on aurait vraiment pu trouver quelque chose de mieux à demander}, dit le côté Serpentard de Harry. «~\emph{Mais j'imagine qu'on était vraiment pressés, alors quoi qu'il arrive, ce sera toujours utile d'avoir des réponses.}~»

Harry n'écoutait pas la voix. Lorsqu'il avait entendu ses propres mots, adressés à l'homme au pistolet, des frissons avaient parcouru son échine.

«~C'est sous ces conditions que tu m'aideras à obtenir la Pierre~?~» dit le professeur Quirrell.

Harry hocha la tête, incapable de répondre.

«~\parsel{J'acçcepte}, siffla le professeur Quirrell. \parsel{Aide-moi et tu auras les réponsses à tes quesstions, tant qu'elles concernent le passsé et pas mes plans futurs. Je ne compte pas te faire du mal un jour, physiquement ou magiquement, tant que tu feras de même. Je ne tuerai perssonne à l'école pendant une ssemaine, ssauf ssi j'y ssuis forçcé. Maintenant, promets-moi que tu n'esssaieras de prévenir perssonne ni de m'échapper. Promets de faire de ton mieux pour m'aider à obtenir la Pierre. Et je rendrai la vie et la ssanté à la fille-amie~; et ni moi ni les miens ne chercherons plus à lui faire du mal.}~» Un sourire retors. «~\parsel{Promets, petit, et le marché ssera conclu.}

--- Je promets~», murmura Harry.

\emph{QUOI~?} hurla le reste de son esprit.

\emph{Euh, il a toujours un pistolet braqué sur nous}, fit remarqua Serpentard. \emph{On n'a pas vraiment le choix. On essaie juste d'en tirer le meilleur parti.}

\emph{Espèce de bâtard}, répondit Poufsouffle. \emph{Tu crois que c'est ce qu'Hermione aurait voulu~? On parle de Lord Voldemort, là. Est-ce qu'on sait même combien de personnes il a tuées~? Combien d'autres il tuera~?}

\emph{Je rejette l'idée qu'on accepte un compromis avec Lord Voldemort pour sauver Hermione}, dit Serpentard. \emph{Puisqu'il y a de toute façon un pistolet braqué sur nous. Et papa et maman voudraient qu'on fasse ce qu'il dise et qu'on ne se mette pas en danger.}

Le professeur Quirrell le regardait toujours.

«~Répète toute la promesse en Fourchelangue, petit.

--- \parsel{Je vous aiderai à obtenir la Pierre… je ne peux pas vous promettre que je ferai de mon mieux, le cœur n'y ssera pas. Mais je compte esssayer. Je ne ferai rien qui rissque de vous agaçcer ssans raison. Je n'irai pas chercher de l'aide ssi je pense que vous tuerez l'aide ou que les otages mourront. Je ssuis désolé, professseur, mais çc'est le mieux que je puissse faire.}~» L'esprit de Harry se mettait en place, se recomposait suite à sa décision. Il resterait avec le professeur Quirrell, il le suivrait jusqu'à la Pierre, il sauverait les élèves pris en otage et… et… et ensuite, tout ce qu'il savait, c'est qu'il continuerait de réfléchir.

«~Vous êtes vraiment désolé~?~» le professeur Quirrell semblait amusé. «~Je pense que ça devra suffire. Garde deux autres choses à l'esprit~: \parsel{Je compte arrêter le directeur lui-même, ss'il devait apparaître devant nous}. Et aussi ça~: je te demanderai parfois de me dire en Fourchelangue si tu m'as trahi. \parsel{Le marché est conclu}.~»

\later

Suite à cela, le professeur Chourave ramassa la baguette de Harry et l'enveloppa d'une étoffe scintillante qu'elle plaça au sol avant de pointer sa propre baguette vers Harry. Ce n'est qu'alors que le professeur Quirrell abaissa son arme, qui sembla disparaître dans ses robes, et ramassa la baguette de Harry, qu'il rangea elle aussi.

On enleva la véritable Cape d'Invisibilité qui drapait le corps endormi de Lesath Lestrange, et le professeur Quirrell prit la Cape, la bourse de Harry et son Retourneur de Temps.

Puis le professeur Quirrell lança un sortilège d'Oubliettes de masse, puis un sortilège de faux souvenir de masse sur tous les élèves présents~; celui qui laissait les victimes combler eux-mêmes les trous laissés par le sortilège. Puis le professeur Chourave fit léviter les enfants endormis. Elle avait l'air exaspéré ou préoccupé, comme s'il y avait eu un accident de Botanique.

Le professeur Quirrell revint au maître des potions, se pencha, et plaça sa baguette sur le front du professeur Rogue~: «~\emph{Alienis nervus mobile lignum}.~»

Puis il recula et commença à agiter les doigts de sa main gauche comme s'il manipulait une marionnette suspendue à des fils.

Le professeur Rogue se releva prestement et se plaça à nouveau devant la porte du couloir.

«~\emph{Alohomora},~» dit le professeur Quirrell en dirigeant sa baguette vers la porte interdite. Il avait l'air de s'amuser beaucoup. «~Me feras-tu l'honneur, petit~?~»

Harry déglutit. Il avait très envie d'y réfléchir à deux fois, et même à trois fois.

Il était étrange de se retrouver en train de mal agir, pas d'agir de façon égoïste, mais de \emph{mal agir} à un degré plus profond.

Mais l'homme derrière lui tenait le pistolet~; il était réapparu suite à l'hésitation de Harry.

Ce dernier posa sa main sur le heurtoir, inspira profondément plusieurs fois, organisa ses pensées du mieux qu'il pouvait. Vas-y, ne te fais pas tirer dessus, ne laisse pas les otages mourir, sois là pour optimiser les choses, guette des opportunités, reste capable de les saisir. Ce n'était pas le bon choix, mais tous les autres lui avaient semblé pires.

Harry poussa la porte interdite et entra.
%  LocalWords:  om Alienis nervus
