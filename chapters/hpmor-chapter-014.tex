\chapter{L'inconnu et l'inconnaissable}

% STOP HERE
\lettrinepara[ante=«~]{E}{ntrez}~» dit la voix étouffée du professeur McGonagall.

\hplettrineextrapara
Harry s'exécuta.

Le bureau de la directrice adjointe était propre et bien organisé~; sur le mur immédiatement adjacent au bureau de McGonagall se trouvait un labyrinthe de cagibis en bois de toutes formes et de toutes tailles, avec des rouleaux de parchemins fourrés dans la plupart, et il était très clair que, par une méthode inconnue, McGonagall savait exactement à quoi correspondait chaque cagibi, même si personne d'autre ne le savait. Un unique parchemin se trouvait sur le bureau, autrement entièrement vide. Derrière le bureau se trouvait une porte fermée et munie de plusieurs serrures.

McGonagall se tenait assise sur un tabouret et avait l'air perplexe -- lorsqu'elle vit Harry, ses yeux s'écarquillèrent légèrement, avec peut-être une légère note d'appréhension.

«~M. Potter~? dit le professeur McGonagall. De quoi s'agit-il~?~»

Le cerveau de Harry était vide. Le jeu l'avait chargé de venir ici et il s'était attendu à ce qu'\emph{elle} ait quelque chose en tête…

«~M. Potter~?~» dit le professeur McGonagall, et elle commençait à avoir l'air légèrement agacée.

Le cerveau paniqué de Harry se rappela alors heureusement qu'il y \emph{avait} quelque chose d'important dont il comptait s'entretenir avec le professeur McGonagall. Quelque chose d'important et qui ne lui ferait pas perdre son temps.

«~Euh… dit Harry. S'il existe des sorts qui permettraient de s'assurer que personne ne nous écoute…~»

Le professeur McGonagall se leva de sa chaise, ferma solidement la porte menant vers l'extérieur, et commença à sortir sa baguette et à prononcer des sorts.

C'est à ce moment que Harry se rendit compte qu'il faisait face à une opportunité inestimable d'offrir un Hilari-Thé au professeur McGonagall et il ne pouvait pas croire qu'il y pensait sérieusement et tout irait bien le soda disparaîtrait après quelques secondes et il dit à cette partie de lui-même de \emph{la fermer}.

Ce qu'elle fit, et Harry commença à organiser mentalement ce qu'il allait dire. Il n'avait pas prévu d'avoir cette discussion \emph{si tôt}, mais puisqu'il était là…

Le professeur McGonagall acheva un sort dans une langue qui semblait beaucoup plus vieille que le Latin, puis elle s'assit à nouveau.

«~Très bien, dit-elle d'une voix basse. Personne ne nous écoute.~» Son visage était plutôt tendu.

\emph{Oh, c'est vrai, elle s'attend à ce que je la fasse chanter avec la prophétie pour obtenir des informations.}

Euh, Harry s'en occuperait un autre jour.

«~C'est à propos de l'Incident avec le Choixpeau magique,~» dit Harry (le professeur McGonagall cligna des yeux). «~Euh… je pense qu'un sort a été ajouté au Choixpeau magique sans qu'il s'en rende compte, quelque chose qui se déclenche quand le Choixpeau magique dit Serpentard. J'ai entendu un message et je suis plutôt certain que les Serdaigle ne sont pas censés l'entendre. Ça a eu lieu au moment où le Choixpeau a quitté ma tête et que j'ai senti la connexion se couper. Ça ressemblait à la fois à de l'anglais et à un sifflement,~» il y eut une brusque inspiration venant de McGonagall, «~et ça disait~: Salutations de Serpentard à Serpentard, si tu souhaites percer mes secrets, parle à mon serpent.~»

Le professeur McGonagall resta assise là, bouche bée, regardant Harry comme si deux têtes supplémentaires venaient de lui pousser.

«~Et donc…~» dit lentement le professeur McGonagall, comme si elle ne pouvait croire aux mots qui s'échappaient de ses propres lèvres, «~vous avez décidé de venir immédiatement m'en parler.

--- Ben, oui, bien sûr~», dit Harry. Il n'y avait aucun besoin d'admettre le temps qu'il avait mis à avoir eu cette idée. «~Plutôt que de, disons, faire moi-même des recherches à ce sujet, ou de le dire à un des autres enfants.

--- Je… vois, dit le professeur McGonagall. Et si, disons, vous découvriez l'entrée de la légendaire Chambre des Secrets de Salazar Serpentard, une entrée que vous et vous seul pourriez ouvrir…

--- Je fermerais l'entrée et vous en informerais afin qu'une équipe d'archéologues magiciens expérimentés puisse être mise sur pied, dit immédiatement Harry. Puis j'ouvrirais à nouveau l'entrée et ils entreraient en faisant bien attention à ce qu'il n'y ait rien de dangereux. Peut-être que je m'y rendrais plus tard pour jeter un coup d'œil, ou s'ils avaient besoin de moi pour ouvrir autre chose, mais ce serait après que la zone fut déclarée sûre et qu'il y ait des photographies montrant à quoi tout ressemblait avant que les gens ne commencent à piétiner ce lieu historique inestimable.~»

Le professeur McGonagall resta assise là, bouche béé, regardant Harry comme s'il venait de se transformer en chat.

«~C'est évident quand on n'est pas un Gryffondor, dit Harry avec gentillesse.

--- Je pense, dit le professeur McGonagall d'une voix plutôt étouffée, que vous sous-estimiez \emph{grandement} la rareté du sens commun, M. Potter.~»

Ça avait l'air plausible. Même si… «~Un Poufsouffle aurait dit la même chose.~»

McGonagall marqua une pause, sonnée.

«~\emph{C'est} vrai.

--- Le Choixpeau magique m'a proposé Poufsouffle.~»

Elle cligna des yeux comme si elle ne pouvait croire ses propres oreilles.

«~Il a \emph{vraiment} fait ça~?

--- Oui.

--- M. Potter~», dit McGonagall, et sa voix était maintenant très basse, «~il y a cinq décennies que pour la dernière fois un étudiant est mort entre les murs de Poudlard, et je suis maintenant certaine que c'est il y a cinq décennies que quelqu'un a entendu ce message pour la dernière fois.~»

Un frisson parcourut Harry. «~Alors je ferai \emph{très} attention de ne prendre \emph{aucune} initiative \emph{d'aucune sorte} concernant cette affaire sans d'abord vous consulter, professeur McGonagall.~» Il marqua une pause. «~Et puis-je suggérer d'assembler les personnes les plus compétentes que vous puissiez trouver et que vous voyez s'il est possible d'enlever ce sort supplémentaire du Choixpeau magique… et si vous ne pouvez pas faire ça, alors peut-être d'ajouter un \emph{autre} sort, un sort de Sourdinam qui s'active brièvement juste quand le Choixpeau est ôté de la tête d'un étudiant, ça pourrait constituer une bonne alternative. Et voilà, plus d'étudiants morts.~» Harry hocha la tête de satisfaction.

Le professeur McGonagall avait l'air encore plus stupéfaite, si une telle chose était possible.

«~Il n'est pas \emph{possible} que je vous décerne suffisamment de points pour ceci sans vous décerner la Coupe des Quatre Maisons du même coup.

--- Hmm, dit Harry. Hmm. Je préférerais ne pas gagner \emph{autant} de points que ça.~»

Le professeur McGonagall le regardait maintenant étrangement. «~Pourquoi pas~?~»

Harry avait quelques difficultés à trouver les bons mots.

«~Parce que ce serait juste trop triste, vous ne trouvez pas~? Comme… comme quand j'essayais encore d'aller à l'école moldue et à chaque fois qu'il y avait un projet de groupe, je faisais tout moi-même parce que les autres n'auraient fait que me ralentir. Je suis heureux de gagner beaucoup de points, et même plus que n'importe qui d'autre, mais si je gagne assez de points pour que ce soit décisif dans la course à la Coupe des Quatre Maisons, alors c'est comme si je portais Serdaigle sur mon dos, et ce serait trop triste.

--- Je vois…~» dit le professeur McGonagall d'un ton hésitant. Il était visible que cette façon de penser ne lui était pas du tout familière. «~Mettons alors que je vous décerne seulement cinquante points~?~»

Harry secoua encore la tête.

«~Ce ne serait pas juste pour les autres enfants si je gagne beaucoup de points en faisant des choses d'adultes auxquelles je peux participer et pas eux. Comment Terry Boot pourrait-il gagner cinquante points pour avoir fait mention d'un murmure qu'il aurait entendu venir du Choixpeau magique~? Ce ne serait pas juste du tout.

--- Je vois pourquoi le Choixpeau magique vous a proposé Poufsouffle,~» dit le professeur McGonagall. Elle le regardait avec un étrange respect.

Harry avala de travers. Il avait honnêtement pensé qu'il n'était pas assez bon pour Poufsouffle. Que le Choixpeau magique avait juste essayé de le fourrer n'importe où sauf à Serdaigle, même dans une maison dont il posséderait pas les vertus…

Le professeur McGonagall souriait à présent.

«~Et si j'essayais de vous donner \emph{dix} points…~?

--- Allez-vous expliquer d'où viennent ces dix points, si quelqu'un pose la question~? Il pourrait y avoir beaucoup de Serpentard, et je ne parle pas des enfants à Poudlard, qui seraient vraiment vraiment \emph{vraiment} en colère s'ils apprenaient que le sort avait été enlevé du Choixpeau magique et découvraient qui était mêlé à ça. Donc je pense que la meilleure preuve de bravoure, c'est le secret absolu. Pas besoin d'me remercier m'dame, la vertu est sa propre récompense.

--- Ainsi soit-il, dit le professeur McGonagall, mais j'ai quelque chose de très spécial à vous donner. Je vois que je me suis trompée sur votre compte, M. Potter. Attendez ici s'il vous plaît.~»

Elle se leva, alla jusqu'à la porte fermée à loquet, agita sa baguette, et une espèce de voile flou surgit autour d'elle. Harry ne pouvait ni voir ni entendre ce qui se passait. Quelques minutes plus tard, le flou disparut et que le professeur McGonagall se tenait là, face à lui, avec la porte derrière elle semblant n'avoir jamais été ouverte.

Et le professeur McGonagall tenait en main un collier, une fine chaîne d'or portant en son centre un cercle d'argent, qui était l'armature d'un sablier. Dans son autre main se trouvait une brochure pliée. «~C'est pour vous~», dit-elle.

Wow~! Il allait avoir une espèce de super objet magique comme récompense pour sa quête~! Apparemment, le truc de refuser les pièces d'or jusqu'à finalement obtenir un objet magique marchait dans la vraie vie, pas seulement dans les jeux vidéo.

Harry accepta son nouveau collier, souriant. «~Qu'est-ce que c'est~?~»

Le professeur McGonagall inspira. «~M. Potter, c'est un objet qui est habituellement prêté uniquement aux enfants qui se sont montrés hautement responsable, afin de les aider à se dépêtrer de leurs complexes horaires de cours.~» McGonagall hésita, comme si elle allait ajouter quelque chose.

«~Je \emph{dois} insister, M. Potter, sur le fait que la véritable nature de cet objet est \emph{secrète} et que vous ne devez \emph{pas} en parler aux autres élèves, ou les laissez voir que vous l'utilisez. Si vous ne jugez pas cela acceptable, alors vous pouvez me le rendre maintenant.

--- Je sais garder des secrets, dit Harry. Alors qu'est-ce que ça fait~?

--- Du point de vue des autres élèves, ceci sera un Portillon tournant, qui est utilisé pour traiter une maladie magique rare et non contagieuse nommée Duplication Spontanée. Vous le portez sous vos vêtements, et bien que vous n'ayez aucune raison de le montrer à qui que ce soit, vous n'avez pas non plus de raison de le traiter comme un secret honteux. Les Portillons tournants n'ont aucun intérêt. Comprenez-vous, M. Potter~?~»

Harry hocha la tête, son sourire devenant plus large. Il voyait là le travail d'un Serpentard \emph{compétent}.

«~Et quel est son \emph{véritable} effet~?

--- C'est un Retourneur de Temps. Chaque tour de son sablier vous renvoie une heure en arrière dans le temps. Donc si vous l'utilisez pour reculer de deux heures chaque jour, vous devriez pouvoir aller vous coucher tous les jours à la même heure.~»

La suspension consentie de l'incrédulité de Harry explosa en morceaux.

\emph{Vous me donnez une machine à remonter le temps pour traiter mes troubles du sommeil.}

\emph{Vous me donnez une MACHINE À REMONTER LE TEMPS pour traiter mes TROUBLES DU SOMMEIL.}

\emph{Vous \shout{me donnez une machine à remonter le temps} dans le but de \shout{traiter mes troubles du sommeil.}}

«~Ehehehehhheheh…~» dit la bouche de Harry. Il tenait maintenant le collier loin de lui comme si c'était une bombe prête à exploser. Enfin, non, pas comme si c'était une bombe prête à exploser, ça ne \emph{commençait} même pas à exprimer la sévérité de la situation. Harry tenait le collier loin de lui comme si c'était une machine à remonter le temps.

\emph{Dites-moi, professeur McGonagall, saviez-vous que la matière normale voyageant à rebrousse-temps ressemble comme deux gouttes d'eau à de l'antimatière~? Eh bien oui, c'est le cas~! Saviez-vous qu'un kilogramme d'antimatière entrant en contact avec un kilogramme de matière s'annihilera dans une explosion équivalente à 43 millions de tonnes de dynamite~? Vous rendez-vous compte que je pèse moi-même 41 kilogrammes et que le souffle produit laisserait \shout{un cratère géant là où avant se trouvait l’Écosse~?}}

«~Excusez-moi, parvint à dire Harry, mais ça a l'air vraiment vraiment \emph{vraiment} \emph{VRAIMENT DANGEREUX}~!~» La voix de Harry ne devint pas tout à fait un hurlement, il lui aurait été absolument impossible de crier assez pour rendre justice à la situation, alors pas la peine d'essayer.

Le professeur McGonagall l'observa avec une affection pleine de tolérance.

«~Je suis heureuse que vous preniez cela sérieusement, M. Potter, mais les Retourneurs de Temps ne sont pas \emph{si dangereux que ça}. Autrement, nous n'en donnerions pas aux enfants.

--- Vraiment, dit Harry. Ahahahaha. Bien sûr que vous ne donneriez pas de machine à remonter le temps à des enfants si c'était dangereux, mais à \emph{quoi} pouvais-je bien penser~? Donc juste pour que tout soit bien clair, éternuer sur cet engin ne me renverra \emph{pas} au Moyen Âge où je roulerai sur Gutenberg avec une calèche, empêchant ainsi les Lumières d'avoir lieu~? Parce que, voyez-vous, je déteste quand ce genre de choses m'arrive.~»

Les lèvres de McGonagall se tordaient comme elles le faisaient lorsqu'elle essayait de ne pas sourire. Elle offrit à Harry la brochure, mais Harry tenait précautionneusement le collier de ses deux mains et le gardait à l'œil pour s'assurer qu'il n'était pas sur le point de tourner.

«~ Ne vous en faites pas,~» dit McGonagall après une courte pause, une fois qu'il fut devenu clair que Harry ne comptait pas bouger, «~il est impossible que cela arrive, M. Potter. Le Retourneur de Temps ne peut être utilisé pour revenir plus de six heures en arrière. Il ne peut être utilisé plus de six fois par jour.

--- Oh, bien, c'est très bien. Et si quelqu'un me bouscule, le Retourneur de Temps ne se cassera \emph{pas} et n'enferma \emph{pas} le château de Poudlard et ses occupants dans une boucle infinie de jeudis.

--- Eh bien, ils \emph{peuvent} être fragiles… dit McGonagall. Et j'ai entendu dire que d'étranges choses se passent s'ils sont cassés. Mais rien de \emph{tel}~!

--- Peut-être, dit Harry lorsqu'il put à nouveau parler, que vous devriez poser une sorte de \emph{coque protectrice} à vos machines à remonter dans le temps, plutôt que de \emph{laisser le verre exposé}, pour \emph{empêcher que ce genre de chose n'arrive}.~»

McGonagall semblait assez sonnée. «~C'est une excellente idée. J'en informerai le Ministère.~»

\emph{Ça y est, c'est officiel maintenant, ils l'ont ratifié au Parlement, tous les habitants du monde magique sont complètement stupides.}

«~Et bien que je déteste devenir tout à coup très \shout{philosophique},~» Harry essaya désespérément d'abaisser sa voix en dessous du hurlement, «~mais personne n'a-t-il pensé aux \shout{implications} de revenir six heures en arrière et de faire quelque chose qui change le temps, ce qui, en gros, \shout{effacerait toutes les personnes affectées} et les \shout{remplacerait par des versions différentes}…

--- Oh, vous ne pouvez pas \emph{changer} le cours du temps~! l'interrompit le professeur McGonagall. Grands dieux, M. Potter, pensez-vous que nous autoriserions les étudiants à les utiliser si une chose \emph{pareille} était possible~? Et si quelqu'un essayait de changer les résultats de ses examens~?~»

Harry prit son temps pour absorber cela. Ses mains relâchèrent juste un peu leur poigne sur la chaîne du sablier. Comme s'il ne tenait pas une machine à remonter dans le temps, juste une tête nucléaire enclenchée.

«~Donc… dit lentement Harry. On remarque que l'univers… se trouve être cohérent, d'une façon ou d'une autre, même si le voyage temporel y est possible. Si moi et mon futur moi interagissent, alors je verrai la même chose dans les deux moi, bien que, lors de mon premier passage, mon futur moi agisse en sachant déjà parfaitement tout ce qui, de mon point de vue, n'a pas encore eu lieu…~» la voix de Harry resta en suspens face aux lacunes de la langue anglaise.

«~Correct, je pense, dit le professeur McGonagall. bien que l'on conseille \emph{en effet} aux sorciers d'éviter d'être vus par leur soi passés. Si, par exemple, vous assistiez à deux cours en même temps et que vous deviez vous croiser, la première version devrait se mettre sur le côté et fermer ses yeux à un moment choisi -- vous avez déjà une montre, très bien -- afin que le futur vous puisse passer. Tout est là dans la brochure.

--- Ahahahaa. Et que se passe-t-il quand quelqu'un \emph{ignore} ce conseil~?~»

Le professeur McGonagall se pinça les lèvres.

«~J'ai cru comprendre que ça pouvait être assez déroutant.

--- Et ça ne, disons, ça ne crée pas de paradoxe qui détruit l'univers.~»

Elle sourit avec tolérance.

«~M. Potter, je pense que je me souviendrais en avoir entendu parler si \emph{ce} genre de chose avait déjà eu lieu.

--- \shout{Ce n'est pas rassurant~! Aucun de vous n'a donc jamais entendu parler du biais anthropique~? Et qui est l'idiot qui le premier a construit une de ces choses~?}~»

Le professeur McGonagall riait franchement. C'était un son heureux et agréable qui ne semblait pas à sa place sur ce visage dur.

«~Vous vivez un de ces “vous venez de vous transformer en chat”, n'est-ce pas, M. Potter~? Vous n'avez probablement pas envie de savoir ça, mais c'est délicieusement attachant.

--- Se transformer en chat n'est CERTAINEMENT PAS comparable à ça. Vous savez, jusqu'à maintenant, j'avais cette horrible pensée réprimée à l'arrière de mon esprit disant que la seule réponse possible était que mon univers entier était une simulation informatique, comme dans le livre \emph{Simulacron 3}, mais maintenant \emph{même cette possibilité est exclue} parce que ce petit jouet N'EST PAS UNE FONCTION CALCULABLE PAR UNE MACHINE DE TURING~! Une machine de Turing pourrait simuler “revenir à un moment défini du passé et calculer un futur différent à partir de ce point”, une machine avec oracle pourrait prendre en compte le comportement d'arrêt des machines d'ordre inférieur, mais ce que vous décrivez est une réalité qui, d'une façon ou d'une autre, parviendrait à se calculer de façon cohérente en une seule fois, en utilisant des informations qui n'ont pas… encore… eu lieu…~»

La compréhension frappa Harry d'un coup de massue.

Tout concordait à présent. Ça avait \emph{enfin} un sens.

«~\shout{Alors c'est comme ça que l'Hilari-thé fonctionne}~! Bien \emph{sûr}~! Le sort ne \emph{force} pas les choses amusantes à avoir lieu, il provoque juste \emph{l'impulsion de le boire} avant que des choses amusantes n'aient lieu~! Je suis tellement idiot, j'aurais dû m'en rendre compte quand j'ai ressenti le besoin de boire de l'Hilari-Thé avant le deuxième discours de Dumbledore, et que je n'en ai \emph{pas} bu et me suis ensuite étranglé sur ma propre salive -- boire de l'Hilari-Thé ne crée pas la comédie, la comédie vous pousse à boire l'Hilari-Thé~! J'ai vu que les deux événements étaient corrélés et j'ai présumé que l'Hilari-Thé devait être la cause et que la comédie devait être l'effet parce que je pensais que l'ordre temporel restreignait le lien de causalité et que les graphes de causalité devaient être acycliques, \shout{mais tout concorde si on dessine les flèches causales comme allant \emph{en arrière dans le temps}~!}~»

La compréhension frappa Harry d'un \emph{second} coup de massue.

Il parvint à rester discret cette fois-ci, et ne fit qu'un petit bruit étranglé, comme aurait fait un bébé chat mourant, tandis qu'il comprenait qui avait déposé la note sur son lit ce matin.

Les yeux du professeur McGonagall brillaient.

«~Après votre diplôme, ou peut-être même avant, vous \emph{devrez} enseigner quelques-unes de ces théories moldues à Poudlard, M. Potter. Elles ont l'air vraiment fascinantes, même si elles sont toutes fausses.

--- Glehhahhh…~»

Le professeur McGonagall lui offrit quelques plaisanteries de plus, lui demanda de faire quelques promesses supplémentaires, lesquelles il accepta de faire, dit quelque chose concernant le fait qu'il ne devait pas parler aux serpents quand on pouvait l'entendre, lui rappela de lire la brochure, et puis, sans savoir comment, Harry se retrouva hors de son bureau avec la porte solidement fermée derrière lui.

«~Gaahhhrrrraa…~» dit Harry.

Mais oui, il \emph{était} estomaqué.

Particulièrement par le fait que, sans la Farce, il aurait très bien pu ne jamais obtenir le Retourneur de Temps.

Ou le professeur McGonagall le lui aurait-elle donné tout de même, mais plus tard dans la journée, quand il se serait décidé à l'interroger sur son trouble du sommeil ou à lui parler du message du Choixpeau magique~? Et aurait-il alors désiré se faire une blague à lui-même qui l'aurait conduit à obtenir le Retourneur de Temps \emph{plus tôt}~? Et donc la seule possibilité \emph{cohérente} était-elle celle dans laquelle la Farce commençait avant même qu'il ne se réveille ce matin~?

Harry se retrouva à envisager, pour la première fois de sa vie, que la réponse à sa question était peut-être \emph{inconcevable}. Que puisque son cerveau contenait des neurones qui allaient uniquement en avant dans le temps, il n'y avait \emph{rien} que son cerveau puisse faire, aucune opération qu'il puisse effectuer, qui reproduirait l'opération d'un Retourneur de Temps.

Jusqu'à ce moment, Harry avait vécu suivant la maxime de E.T. Jaynes selon laquelle, si vous ignoriez quelque chose d'un phénomène, c'était un fait au sujet de votre propre esprit, pas un fait au sujet du phénomène lui-même~; que votre incertitude était un fait vous concernant, pas un fait concernant ce au sujet de quoi vous étiez incertain~; que l'ignorance existait dans l'esprit, pas dans la réalité~; qu'une carte vide ne correspondait pas à un territoire vide. Il y avait des questions mystérieuses, mais une réponse mystérieuse était une contradiction en soi. Un phénomène pouvait être mystérieux \emph{pour} une personne en particulier, mais il ne pouvait y avoir de phénomène mystérieux en soi. Vénérer un mystère sacré, c'était vénérer sa propre ignorance.

Alors Harry avait regardé la magie et avait refusé d'être intimidé. Les gens n'avaient aucun sens de l'Histoire, ils apprenaient la chimie, la biologie et l'astronomie, et pensaient que ces domaines avaient toujours été au cœur de la science, qu'ils n'avaient \emph{jamais été} mystérieux. Les étoiles avaient un jour été des mystères. Lord Kelvin avait un jour dit de la nature de la vie et de la biologie -- la réponse des muscles à la volonté humaine et la transformation des graines en arbres -- que c'était un mystère “infiniment au-delà” de la portée de la science (pas seulement un peu au-delà, dites-vous bien, mais \emph{infiniment} au-delà. Lord Kelvin avait pris un sacré plaisir à \emph{ignorer quelque chose}). Chaque mystère jamais résolu avait commencé par être un casse-tête, depuis l'aube de l'humanité jusqu'au moment où quelqu'un l'avait résolu.

Maintenant, pour la première fois, il faisait face à la perspective d'un mystère qui menaçait d'être \emph{permanent}. Si le temps ne fonctionnait pas selon des réseaux de causalité acycliques, alors Harry ne comprenait pas ce que “cause” et “effet” voulaient dire~; et si Harry ne comprenait pas les causes et les effets alors il ne comprenait pas de quoi la réalité pouvait bien être composée~; et il était entièrement possible que son cerveau humain ne \emph{puisse} jamais le comprendre, parce que son cerveau était fait de \emph{neurones démodés fonctionnant en temps linéaire}, qui s'étaient avérés n'être qu'un sous-ensemble appauvri de la réalité.

Le bon côté des choses, c'était que l'Hilari-Thé, qui avait auparavant semblé tout-puissant et tout-incroyable, s'était révélé beaucoup plus simple à expliquer. Ce que Harry avait raté \emph{simplement} parce que la vérité était totalement hors de son champ d'hypothèses et de tout ce que son cerveau, de par son évolution, aurait pu être amené à comprendre. Mais maintenant il \emph{avait} vraiment compris. Probablement. Ce qui était relativement encourageant. Plus ou moins.

Harry jeta un coup d'œil à sa montre. Il était presque 11h, il était allé se coucher la nuit dernière à 1h, donc il devrait normalement aller se coucher à 3h. Donc pour aller se coucher à 22h et se réveiller à 7h, il devrait revenir de cinq heures en arrière. Ce qui voulait dire que s'il voulait revenir à son dortoir aux alentours de 6h, avant que quiconque ne soit réveillé, il ferait mieux de se dépêcher et…

Même \emph{rétrospectivement} Harry ne comprenait pas comment il était parvenu à accomplir \emph{la moitié} des choses nécessaires à la réalisation de la Farce. D'où était venue la \emph{tarte}~?

Harry commençait à avoir vraiment peur du voyage dans le temps.

D'un autre côté, il se devait d'admettre que ça \emph{avait été} une opportunité irremplaçable. Une farce que vous ne pouviez vous faire à vous-même qu'une fois dans une vie, six heures avant que vous n'appreniez l'existence des Retourneurs de Temps.

C'était en fait \emph{encore plus} déroutant, maintenant qu'il y pensait. Le Temps lui avait présenté la Farce comme un \emph{fait accompli}, et pourtant c'était indubitablement son propre ouvrage. Le concept, l'exécution, le style d'écriture. Chaque détail, même ceux qu'il ne comprenait pas encore.

Bon aller, le temps filait et il y avait au plus trente heures dans une journée. Harry connaissait \emph{une partie} des choses qu'il avait à faire, et il trouverait peut-être un moyen de faire le reste, comme la tarte, en cours de route. Pas la peine de s'attarder davantage. Ce n'est pas comme s'il pouvait accomplir quoi que ce soit ici, coincé dans le \emph{futur}.

\later


Cinq heures plus tôt, Harry se glissait dans son dortoir, sa robe glissée au-dessus de sa tête en guise de déguisement de fortune, juste au cas où quelqu'un serait déjà debout et risquerait de le voir en même temps que le Harry allongé dans son lit. Il ne voulait pas avoir à expliquer à qui que ce soit son petit problème médical de Duplication spontanée.

Heureusement, tout le monde semblait encore endormi.

Et il semblait aussi y avoir une boîte, emballée dans du papier rouge et vert, avec un ruban doré et brillant, posée à côté de son lit. L'image parfaite et stéréotypée d'un cadeau de Noël, bien que ce ne fût pas Noël.

Harry se glissa jusqu'à elle aussi doucement qu'il le pouvait, juste au cas où quelqu'un aurait son Sourdineur au minimum.

Il y avait une enveloppe attachée à la boîte, scellée par de la cire blanche, sans sceau imprimé.

Harry ouvrit l'enveloppe avec précaution et prit la lettre qui s'y trouvait. La lettre disait~:
\begin{writtenNote}

Ceci est la Cape d'Invisibilité d'Ignotus Peverell, transmise à travers ses descendants, les Potter. Contrairement aux sorts et aux capes de moindre force, elle a le pouvoir de vous garder \underline{caché}, et pas seulement invisible. Votre père me l'a prêtée pour étude peu de temps avant sa mort, et je confesse en avoir fait grand usage durant ces dernières années.

J'ai peur qu'à l'avenir je doive me contenter d'un sortilège de Désillusion. Il est temps que la Cape soit rendue à vous, son légataire. J'avais pensé vous l'offrir comme cadeau de Noël, mais elle souhaitait revenir entre vos mains avant cela. Il semblerait qu'elle s'attende à ce que vous ayez besoin d'elle. Faites-en bon usage.

Vous pensez sans doutes à toutes sortes de farces formidables, semblables à celles que votre père a commises en son temps. Si tous ses méfaits étaient connus, toutes les femmes de Gryffondor se réuniraient pour profaner sa tombe. Je n'essaierai pas d'empêcher l'histoire de se répéter, mais soyez des \underline{plus} attentifs dans votre dissimulation. Si Dumbledore voyait une chance de posséder l'une des Reliques de la Mort, il ne la laisserait jamais échapper à son étreinte.

Un Très Joyeux Noël.

\end{writtenNote}

La note n'était pas signée

\later

«~Attendez,~» dit Harry, s'arrêtant net alors que les autres garçons s'apprêtaient à quitter le dortoir des Serdaigle. «~Désolé, il y a quelque chose d'autre dans ma malle dont j'ai besoin. Je vous rejoindrai pour le petit déjeuner dans quelques minutes.~»

Terry Boot jeta un mauvais menaçant à Harry. «~Tu ferais mieux de ne pas avoir l'intention de fouiller nos affaires.~»

Harry leva une main. «~Je jure ne pas avoir l'intention de faire quoi que ce soit de la sorte à aucune de vos affaires, que je ne compte accéder qu'à des objets m'appartenant, que je n'ai l'intention de faire aucune farce ni quoi que ce soit de douteux à aucun d'entre vous, et que je ne m'attends pas à voir ces intentions changer avant que j'arrive dans la Grande Salle pour le petit déjeuner.~»

Terry fronça les sourcils.

«~Attends, est-ce que…

--- Ne t'inquiète pas~», dit Pénélope Deauclaire, qui était là pour les guider. «~Il n'y avait pas failles. Bien dit, Potter, tu devrais être avocat.~»

Harry Potter cligna des yeux. Ah, oui, \emph{préfète} de Serdaigle.

«~Merci, dit-il. Je crois.

--- Quand tu essaieras de trouver la Grande Salle, tu te perdras.~» Pénélope dit cela sur le ton qu'on utilisait pour énoncer des évidences incontestables.

«~Dès que ça t'arrive, demande à un portrait comment te rendre au rez-de-chaussée. Parle à un autre portait à \emph{l'instant} où tu penses t'être à nouveau perdu. \emph{En particulier} s'il semble que tu vas de plus en plus haut. Si tu es si haut que le plafond du château devrait être en dessous de toi, \emph{arrête-toi} et attends les équipes de recherche. Sinon nous te reverrons trois mois plus tard et tu auras vieilli de deux ans et tu seras habillé d'un pagne et couvert de neige et ça c'est \emph{si tu restes dans le château}.

--- Compris~», dit Harry, avalant difficilement sa salive. «~Euh, ne devriez-vous pas dire ce genre de choses aux élèves dès le début~?~»

Pénélope soupira. «~Quoi, \emph{toutes} ces choses~? Ça prendrait des semaines. Tu l'apprendras au fur et à mesure.~» Elle se tourna pour partir, suivie par les autres élèves. «~Si je ne te vois pas au petit déjeuner dans trente minutes, Potter, je commencerai les recherches.~»

Après que tout le monde fut parti, Harry accrocha la note à son lit -- il l'avait déjà écrite, ainsi que toutes les autres, travaillant dans son niveau caverne avant le réveil des autres. Puis il entra précautionneusement dans le champ d'action du Sourdinam et retira la Cape d'Invisibilité du corps endormi de Harry-1.

Et juste par pure espièglerie, Harry mit la Cape d'Invisibilité dans la bourse de Harry-1, sachant qu'elle serait ainsi déjà dans la sienne.

\later

«~Je vois bien que le message est destiné à Cornelion Flubberwalt~», dit le tableau d'un homme à l'air aristocratique et doté d'un nez à vrai dire parfaitement normal. «~Mais pourrais-je connaître son \emph{origine}~?~»

Harry haussa les épaules avec une impuissance rusée. «~On m'a dit que le message venait d'une voix caverneuse qui émanait d'un trou dans l'air lui-même, un trou qui s'ouvrait sur un abysse flamboyant.~»

\later

«~Hé~!~» dit Hermione avec indignation depuis sa place à l'autre bout de la table de petit déjeuner. «~C'est le dessert de \emph{tout le monde}~! Tu ne peux pas prendre une tarte entière et la mettre dans ta bourse~!

--- Je ne prends pas une tarte, j'en prends deux. Désolé tout le monde, je dois y aller maintenant~!~» Harry ignora les cris d'outrage et quitta la Grande Salle. Il avait besoin d'arriver en Botanique un peu en avance.

\later

Le professeur Chourave regarda Harry avec sévérité.

«~Comment savez-\emph{vous} ce que les Serpentard comptent faire~?

--- Je ne peux nommer mes sources, dit Harry. En fait, je vais devoir vous demander de prétendre que cette conversation n'a jamais eu lieu. Faites simplement comme si vous les croisiez naturellement pendant que vous alliez quelque part, ou quelque chose de ce genre. J'irai en avance dès que le cours d'Botanique se terminera, je pense que je pourrai distraire les Serpentard jusqu'à ce que vous arriviez. Il n'est pas simple de m'effrayer ni de me malmener, et je ne pense pas qu'ils oseront vraiment faire du mal au Survivant. Cela dit… je ne vous demande pas de courir dans les corridors, mais j'apprécierais si vous ne perdiez pas de temps en chemin.~»

Le professeur Chourave le regarda pendant un long moment, puis son expression s'adoucit.

«~Faites attention à vous, Harry Potter. Et… merci.

--- Soyez sûre de ne pas être en retard, dit Harry. Et souvenez-vous, quand vous arrivez là-bas, vous ne vous attendiez pas à me voir et cette conversation n'a jamais eu lieu.~»

\later

C'était horrible de se voir tirer Neville hors du cercle des Serpentard. Neville avait raison, il avait utilisé trop de force, beaucoup trop de force.

«~Bonjour, dit Harry Potter froidement. Je suis le Survivant.~»

Huit garçons en première année, tous à peu près de la même taille. L'un d'eux avait une cicatrice sur le front et ne se comportait pas comme les autres.

\vskip 0pt plus 4\baselineskip\settowidth{\versewidth}{Le don de nous voir comme les autres nous voient~!} \begin{verse}[\versewidth] Ô si un pouvoir nous offrait\\ Le don de nous voir comme les autres nous voient~!\\ De quelles erreurs cela nous libérerait-il\\ Et de quelles idées imbéciles… \end{verse}

Le professeur McGonagall avait raison. Le Choixpeau magique avait raison. C'était clair une fois qu'on le voyait de l'extérieur.

Il y avait quelque chose qui clochait chez Harry Potter.~
%  LocalWords:  ome Ehehehehhheheh Ahahahaha Ahahahaa Simulacron Glehhahhh
%  LocalWords:  Gaahhhrrrraa fait accompli 1’s frae monie giftie gie
%  LocalWords:  oursel’s
