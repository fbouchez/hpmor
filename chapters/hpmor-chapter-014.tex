% vim:spell:spelllang=fr

\chapter{L'inconnu et l'inconnaissable}

\lettrinepara[ante=«~]{E}{ntrez}~» dit la voix étouffée de McGonagall.

\hplettrineextrapara
Harry s'exécuta.

Le bureau de la directrice adjointe était propre et bien organisé~; sur le mur immédiatement adjacent au bureau de McGonagall se trouvait un dédale de niches en bois de toutes formes et de toutes tailles, la plupart contenant des rouleaux de parchemins, et il semblait très clair que McGonagall savait exactement à quoi correspondait chaque niche, même si personne d'autre n'en était capable.
Un unique parchemin se trouvait sur le bureau, par ailleurs entièrement vide.
On pouvait voir derrière le bureau une porte fermée, protégée par plusieurs serrures.

McGonagall se tenait assise sur un tabouret derrière le bureau, l'air perplexe -- à la vue de Harry, ses yeux s'était légèrement écarquillés, avec peut-être une pointe d'appréhension.

«~M. Potter~? dit McGonagall. De quoi s'agit-il~?~»

Le cerveau de Harry se vida de toute pensée.
Le jeu lui avait donné l'instruction de venir là, il s'était attendu à ce qu'\emph{elle} ait quelque chose en tête…

«~M. Potter~?~» répéta McGonagall, qui commençait à avoir l'air un poil agacée.

Heureusement, le cerveau paniqué de Harry se rappela alors qu'il \emph{avait} quelque chose d'important à discuter avec la professeure.
Quelque chose d'important et digne de son temps.

«~Euh… fit Harry. S'il existe des sorts qui permettent de s'assurer que personne ne nous écoute…~»

McGonagall se leva de sa chaise, ferma soigneusement la porte du bureau, sortit sa baguette et commença à prononcer des enchantements.

C'est à ce moment que Harry se rendit compte qu'il faisait face à une opportunité inestimable et potentiellement unique d'offrir un Hilari-Thé à la professeure et il ne pouvait pas croire qu'il y pensait sérieusement et tout irait bien le soda disparaîtrait après quelques secondes et il enjoint à cette partie de lui-même de \emph{se la fermer}.

Ce qu'elle fit, permettant ainsi à Harry d'organiser mentalement ce qu'il allait dire.
Il n'avait pas prévu d'avoir cette discussion \emph{si tôt}, mais puisqu'il était là…

McGonagall acheva un sortilège qui sonnait bien plus vieux que du latin, puis elle s'assit à nouveau.

«~Très bien, dit-elle d'une voix calme.
Personne ne nous écoute.~» Son visage était plutôt tendu.

\emph{Oh, c'est vrai, elle s'attend à ce que je la fasse chanter pour obtenir des informations sur la prophétie.}

Euh, Harry s'en occuperait un autre jour.

«~C'est à propos de l'incident avec le Choixpeau~», dit Harry.
(McGonagall cligna des yeux.)
«~Euh… je pense qu'un sort a été rajouté au Choixpeau, sans que lui même soit au courant, quelque chose qui se déclenche quand le Choixpeau dit Serpentard.
J'ai entendu un message et je suis relativement certain que les Serdaigle ne sont pas censés l'entendre.
C'était au moment où j'ai retiré le Choixpeau et que j'ai senti la connexion se couper.
Ça ressemblait à la fois à de l'anglais et à un sifflement~», McGonagall inspira brusquement, «~et ça disait~: Salutations de Serpentard à Serpentard, si tu souhaites percer mes secrets, parle à mon serpent.~»

McGonagall se tenait bouche bée, regardant Harry comme si deux têtes supplémentaires venaient de lui pousser.

«~Et donc…~» dit lentement McGonagall, comme si elle ne pouvait croire aux mots qui s'échappaient de ses propres lèvres, «~vous avez décidé de venir immédiatement pour m'en parler.

--- Et bien, oui, évidemment~», dit Harry.
Inutile de préciser le temps qu'il avait mis à avoir cette idée.
«~Plutôt que de, disons, faire moi-même des recherches à ce sujet, ou d'en parler à d'autres enfants.

--- Je… vois, dit McGonagall. Et mettons que, par exemple, vous découvriez l'entrée de la légendaire Chambre des Secrets de Salazar Serpentard, une entrée que vous et vous seul pourriez ouvrir…

--- Je fermerais l'entrée et vous en informerais immédiatement afin qu'une équipe d'archéologues magiciens expérimentés puisse être mise sur pied, dit promptement Harry.
Puis j'ouvrirais à nouveau la chambre et ils entreraient en faisant bien attention à ce qu'il n'y ait rien de dangereux.
Peut-être j'entrerais à mon tour après eux pour jeter un coup d'œil, ou s'il s'avérait qu'ils aient besoin de moi pour ouvrir autre chose, mais ce serait après que la zone soit déclarée sûre et qu'ils aient bien tout photographié pour savoir à quoi cela ressemblait avant que les gens ne commencent à piétiner ce lieu historique inestimable.~»

McGonagall se tenait bouche bée, regardant Harry comme s'il venait de se transformer en chat.

«~C'est évident quand on n'est pas un Gryffondor, dit Harry avec gentillesse.

--- Je pense, dit McGonagall d'une voix étranglée, que vous sous-estimez \emph{grandement} la rareté du bon sens, M. Potter.~»

Cela semblait plausible. Même si… «~Un Poufsouffle aurait dit la même chose.~»

McGonagall marqua une pause, sonnée.
«~\emph{C'est} vrai.

--- Le Choixpeau m'a proposé Poufsouffle.~»

Elle cligna des yeux comme si elle ne pouvait croire ses propres oreilles.
«~Il a \emph{vraiment} fait ça~?

--- Oui.


--- M. Potter~», dit McGonagall, et sa voix était maintenant très basse,
«~la dernière fois qu'un élève est mort dans l'enceinte de Poudlard était il y a cinq décennies,
et je suis maintenant certaine que la dernière fois que quelqu'un a entendu ce message était aussi il y a cinq décennies.~»

Un frisson parcourut Harry.
«~Alors je ferai \emph{très} attention à ne prendre \emph{aucune} initiative \emph{d'aucune sorte} concernant cette affaire sans d'abord vous consulter, professeure McGonagall.~»
Il marqua une pause.
«~Et puis-je suggérer de réunir les personnes les plus compétentes que vous puissiez trouver pour voir s'il est possible d'enlever ce sortilège du Choixpeau… et si ce n'est pas possible, alors peut-être ajouter un \emph{autre} sort, comme un Sourdinam qui s'active brièvement juste au moment de retirer le Choixpeau de la tête d'un élève, ça pourrait corriger le problème.
Et voilà, plus d'élèves morts.~»
Harry hocha la tête avec satisfaction.

McGonagall semblait encore plus stupéfaite, pour peu qu'une telle chose soit imaginable.

«~Il n'est pas \emph{possible} de vous récompenser de suffisamment de points pour ceci sans décerner directement la Coupe des Quatre Maisons à Serdaigle.

--- Hum, fit Harry. Hum. Je préférerais ne pas gagner \emph{autant} de points que ça pour ma maison.~»

McGonagall le regardait étrangement à présent. «~Pourquoi pas~?~»

Harry éprouvait quelques difficultés à trouver les bons mots.
«~Parce que ce serait juste trop triste, vous ne trouvez pas~? Comme… comme quand j'essayais de retourner à l'école du monde moldu et qu'à chaque fois qu'il y avait un projet de groupe, je faisais tout moi-même parce que les autres n'auraient fait que me ralentir.
J'aime gagner beaucoup de points, même si c'est plus que tous les autres, mais si je gagne assez de points pour que ce soit décisif pour la Coupe des Quatre Maisons, alors ce serait comme si je transportais tout Serdaigle sur mes épaules, et ce serait trop triste.

--- Je vois…~» dit McGonagall, hésitante.
Il était visible que cette façon de penser ne lui était pas du tout familière.
«~Mettons alors que je vous décerne seulement cinquante points~?~»

Harry secoua encore la tête.
«~Ce serait injuste envers les autres enfants si je gagne plein de points pour des choses d'adultes auxquelles je peux participer mais pas eux.
Comment Terry Boot est-il censé gagner cinquante points en faisant part d'un chuchotement qu'il aurait entendu du Choixpeau~?
Ce ne serait pas juste du tout.

--- Je vois pourquoi le Choixpeau vous a proposé Poufsouffle~», dit McGonagall.
Elle le regardait avec un étrange respect.

Harry s'étrangla légèrement.
Il avait honnêtement pensé qu'il n'était pas digne de Poufsouffle.
Que le Choixpeau avait juste essayé de le fourrer n'importe où sauf à Serdaigle, dans une maison dont il ne possédait pas les vertus…

McGonagall souriait à présent.
«~Et si j'essayais de vous donner \emph{dix} points…~?

--- Êtes-vous prête à expliquer d'où viennent ces dix points, si quelqu'un pose la question~?
Il y a sûrement de nombreux Serpentard, et je ne parle pas des enfants à Poudlard, qui seraient vraiment vraiment \emph{vraiment} en colère s'ils apprenaient que le sort a été enlevé du Choixpeau et découvraient qui c'est à cause de moi.
Donc je pense que la meilleure preuve de bravoure, c'est le secret absolu.
Pas besoin de me remercier m'dame, la vertu est sa propre récompense.

--- C'est vrai, dit McGonagall, cependant, j'ai quelque chose de très spécial à vous donner.
Je vois que je me suis trompée sur votre compte, M. Potter.
Attendez-moi ici s'il vous plaît.~»

Elle se leva, alla jusqu'à la porte verrouillée, agita sa baguette, et une espèce de voile flou surgit autour d'elle.
Harry ne pouvait ni voir ni entendre ce qui se passait.
Quelques minutes plus tard, le flou disparut révélant la professeure, face à lui, tandis que la porte derrière elle semblait n'avoir jamais été ouverte.

Et McGonagall tenait dans sa main un collier, une fine chaîne d'or portant en son centre un cercle d'argent, dans lequel on pouvait voir un sablier.
Elle tenait dans son autre main une notice d'utilisation pliée en quatre.
«~C'est pour vous~», dit-elle.

Wow~! Il allait recevoir une espèce d'objet magique comme récompense pour sa quête, classe~!
Apparemment, le truc de refuser les récompenses pécuniaires jusqu'à obtenir un objet magique marchait aussi dans la vraie vie, pas seulement dans les jeux vidéo.

Harry accepta son nouveau collier en souriant.
«~Qu'est-ce que c'est~?~»

McGonagall prit une inspiration.
«~M. Potter, c'est un objet qui est habituellement prêté uniquement aux élèves qui se sont montrés hautement responsables, afin de les aider à composer avec des emplois du temps difficiles.~»
McGonagall hésita, comme si elle allait ajouter quelque chose.
«~Je \emph{dois} insister, M. Potter, sur le fait que la véritable nature de cet objet est \emph{secrète} et que vous ne devez \emph{pas} en parler aux autres élèves, ou les laisser voir que vous l'utilisez.
Si vous ne jugez pas cela acceptable, alors vous pouvez me le rendre maintenant.

--- Je sais garder des secrets, dit Harry. Alors qu'est-ce que ça fait~?

% Spimster wicket m'évoque qqchose du gadget / machin / bidule (widget), et médallon se rapproche de médaillon
% (inspiré de la traduction Hébreu)
--- Du point de vue des autres élèves, ceci sera un médallon bidoule, utilisé pour traiter une maladie magique rare et non contagieuse nommée Duplication Spontanée.
Vous le portez sous vos vêtements, et bien que vous n'ayez aucune raison de le montrer à qui que ce soit, vous n'avez pas non plus de raison de le traiter comme un secret honteux.
Les médallons bidoules ne sont pas intéressants.
Est-ce que vous comprenez, M. Potter~?~»

Harry hocha la tête, élargissant son sourire.
Il flairait le travail d'un Serpentard \emph{compétent}.
«~Et qu'est-ce que ça fait \emph{vraiment}~?

--- C'est un retourneur de temps.
Chaque tour de sablier vous renvoie une heure en arrière dans le temps.
Donc si vous l'utilisez pour reculer de deux heures chaque jour, vous devriez pouvoir aller vous coucher tous les jours à la même heure.~»

La suspension d'incrédulité de Harry explosa en morceaux.

\emph{Vous me donnez une machine à remonter le temps pour traiter mes troubles du sommeil.}

\emph{Vous me donnez une \shout{machine à remonter le temps} pour traiter mes \shout{troubles du sommeil.}}

\shout{Vous me DONNEZ UNE MACHINE À REMONTER LE TEMPS pour TRAITER MES TROUBLES DU SOMMEIL.}

«~Ehehehehhheheh…~» fit la bouche de Harry.
Il tenait maintenant le collier loin de lui comme si c'était une bombe prête à exploser.
Enfin, non, pas comme si c'était une bombe prête à exploser, cela ne \emph{commençait} même pas à exprimer la sévérité de la situation.
Harry tenait le collier loin de lui comme si c'était une machine à remonter le temps.

\emph{Dites-moi, professeure McGonagall, saviez-vous que la matière ordinaire voyageant à rebrousse-temps ressemble comme deux gouttes d'eau à de l'antimatière~?
Eh bien oui, c'est le cas~!
Saviez-vous qu'un kilogramme d'antimatière entrant en contact avec un kilogramme de matière s'annihilera dans une explosion équivalente à 43 millions de tonnes de TNT~?
Réalisez-vous que je pèse moi-même 41 kilogrammes et que le souffle produit laisserait \shout{un cratère géant à l'endroit où se trouvait l’Écosse~?}}

«~Excusez-moi, parvint à dire Harry, mais cela me semble vraiment vraiment \emph{vraiment} \emph{VRAIMENT DANGEREUX}~!~»
La voix de Harry ne monta pas tout à fait jusqu'au hurlement, il lui aurait été absolument impossible de crier suffisamment fort pour rendre justice à la situation, alors inutile d'essayer.

McGonagall l'observa avec une tolérance pleine d'affection.

«~Je suis heureuse que vous preniez cela sérieusement, M. Potter, mais les retourneurs de temps ne sont pas \emph{si} dangereux.
Autrement, nous n'en donnerions pas aux enfants.

--- Vraiment, dit Harry.
Ahahahaha.
Bien sûr que vous ne donneriez pas de machine à remonter le temps à des enfants si c'était dangereux, mais à \emph{quoi} pouvais-je bien penser~?
Donc juste pour que tout soit bien clair, éternuer sur cet engin ne me renverra \emph{pas} au Moyen Âge où je renverserai Gutenberg avec une calèche, empêchant ainsi le siècle des Lumières d'avoir lieu~?
Parce que, voyez-vous, je déteste quand ce genre de choses m'arrive.~»

Les lèvres de McGonagall tressaillaient comme elles le faisaient lorsqu'elle essayait de ne pas sourire.
Elle tendit à Harry la notice, mais ce dernier tenait précautionneusement le collier de ses deux mains et le gardait à l'œil pour s'assurer qu'il ne risquait pas de tourner.
«~ Ne vous inquiétez pas~», dit McGonagall après une brève pause, quand elle réalisa que Harry ne comptait pas bouger, «~il est impossible que cela arrive, M. Potter.
Le retourneur de temps ne peut être utilisé pour revenir plus de six heures en arrière.
Il ne peut être utilisé plus de six fois dans une seule journée.

--- Oh, bien, c'est très bien.
Et si quelqu'un me bouscule, le retourneur de temps ne se cassera \emph{pas} et n'enferma \emph{pas} l'entièreté du château de Poudlard dans une boucle de jeudis se répétant sans fin.

--- Eh bien, ils \emph{peuvent} être fragiles… dit McGonagall.
Et j'ai entendu parler que d'étranges choses se passent s'ils sont cassés.
Mais rien de \emph{tel}~!

--- Peut-être, dit Harry lorsqu'il put à nouveau parler, que vous devriez fournir une sorte de \emph{coque protectrice} avec vos machines à remonter le temps, plutôt que de \emph{laisser le verre exposé}, pour \emph{empêcher que ce genre de chose n'arrive}.~»

McGonagall avait l'air impressionnée.
«~C'est une excellente idée, M. Potter.
J'en informerai le Ministère.~»

\emph{Ça y est, c'est officiel maintenant, ils l'ont ratifié au parlement, tous les habitants du monde magique sont complètement stupides.}

«~Et, pardonnez-moi d'insister sur quelques détails \shout{philosophiques}~», Harry tentait désespérément de maintenir le niveau de sa voix en dessous du hurlement, «~mais personne n'a jamais pensé aux \shout{implications} de revenir six heures en arrière et de faire quelque chose qui change le cours temps, ce qui, en gros, \shout{effacerait toutes les personnes affectées} et les \shout{remplacerait par des versions différentes}…

--- Oh, vous ne pouvez pas \emph{changer} le cours du temps~! l'interrompit McGonagall.
Grands dieux, M. Potter, pensez-vous que nous autoriserions les élèves à les utiliser si une chose \emph{pareille} était possible~?
Et si quelqu'un essayait de changer les résultats de ses examens~?~»

Harry prit un temps pour digérer cela.
Il relâcha légèrement son poing qu'il maintenait verrouillé sur la chaîne du sablier.
Comme s'il ne tenait pas une machine à remonter dans le temps, juste une ogive nucléaire amorcée.

«~Donc… dit lentement Harry.
Il apparait juste que l'univers… se trouve être cohérent, on ne sait comment, même si le voyage temporel y est possible.
Si moi et mon futur moi interagissent, alors je verrai la même chose dans les deux moi, bien que, lors de mon premier passage, mon futur moi agisse en sachant déjà parfaitement tout ce qui, de mon point de vue, n'a pas encore eu lieu…~» la voix de Harry resta en suspens, errant dans l'inadéquation du langage.

«~Correct, il me semble, dit McGonagall.
Cependant, il est \emph{fortement} conseillé aux sorciers d'éviter d'être vus par leur soi passé.
Si vous assistez à deux cours en même temps et que vous devez vous croiser, par exemple, la première version devrait se mettre sur le côté et fermer les yeux à un moment choisi -- vous avez déjà une montre, très bien -- afin que le futur vous puisse passer.
Tout est là dans la notice.

--- Ahahahaa. Et que se passe-t-il quand quelqu'un \emph{ignore} ce conseil~?~»

McGonagall fit la moue.

«~J'ai cru comprendre que cela pouvait être assez déroutant.

--- Et ça ne, disons, ça ne crée pas de paradoxe qui détruit l'univers.~»

Elle sourit avec tolérance.
«~M. Potter, je pense que je me souviendrais en avoir entendu parler si \emph{ce} genre de chose avait déjà eu lieu.

--- \shout{Ce n'est pas rassurant~! Aucun de vous n'a donc jamais entendu parler du principe anthropique~?
Et qui est l'idiot qui a construit une de ces choses en premier lieu~?}~»

McGonagall riait franchement.
C'était un son heureux et agréable qui ne semblait pas à sa place sur ce visage sévère.
«~C'est encore un de ces moments “vous venez de vous transformer en chat,” n'est-ce pas, M. Potter~?
Vous n'avez probablement pas envie de savoir ça, mais c'est adorablement mignon.

--- Se transformer en chat n'arrive même pas à la \shout{cheville} de ça.
Vous savez, jusqu'à ce moment présent j'ai toujours eu en arrière pensée cette crainte atroce et réprimée que la seule solution possible était que mon univers entier était une simulation informatique, comme dans le livre \emph{Simulacron 3}, mais maintenant \emph{même cette possibilité est exclue} parce que ce petit jouet \shout{n'est pas calculable par une machine de Turing~!}
Une machine de Turing pourrait simuler “revenir à un moment défini du passé et calculer un futur différent à partir de ce point,” une machine avec oracle pourrait se baser sur l'arrêt ou non de machines d'ordre inférieur, mais ce que vous dites est que la réalité, d'une façon ou d'une autre, parvient à s'auto-calculer de façon cohérente en une seule passe, en utilisant des informations sur ce qui n'a pas… encore… eu lieu…~»

La compréhension s'abattit soudain sur Harry comme un coup de massue.

Cela prenait sens à présent. Cela prenait \emph{enfin} sens.

«~\shout{Mais c'est comme ça que l'Hilari-thé fonctionne}~!
Bien \emph{sûr}~!
Le sort ne provoque \emph{pas} les choses amusantes, il provoque juste \emph{l'impulsion de boire} avant que des choses amusantes aient lieu~!
Je suis stupide, j'aurais dû m'en rendre compte quand j'ai ressenti l'envie de boire de l'Hilari-Thé avant le deuxième discours de Dumbledore, que je n'en ai \emph{pas} bu, et me suis ensuite tout de même étouffé sur ma propre salive -- boire de l'Hilari-Thé ne crée pas la comédie, la comédie vous pousse à boire l'Hilari-Thé~!
J'ai vu que les deux événements étaient corrélés et j'ai présumé que l'Hilari-Thé devait être la cause et que la comédie devait être l'effet parce que je pensais que l'ordre temporel restreignait le lien de causalité et que les graphes de causalité devaient être acycliques, \shout{mais tout concorde si on dessine les flèches causales allant \emph{EN ARRIÈRE DANS LE TEMPS}~!}~»


La compréhension s'abattit sur Harry comme un \emph{second} coup de massue.

Il parvint à rester discret cette fois-ci, et ne fit qu'un petit bruit étranglé, comme l'aurait fait un chaton mourant, tandis qu'il réalisait qui avait déposé la note sur son lit ce matin.

Les yeux de McGonagall luisaient.
«~Après votre diplôme, ou peut-être même avant, il faut \emph{absolument} que vous enseigniez quelques-unes de ces théories moldues à Poudlard, M. Potter.
Elles ont l'air vraiment fascinantes, même si elles sont toutes fausses.

--- Glehhahhh…~»

McGonagall lui offrit quelques plaisanteries de plus, demanda quelques promesses supplémentaires que Harry accepta, dit quelque chose comme ne pas parler aux serpents quand on pouvait l'entendre, lui rappela de lire la notice d'utilisation, et Harry finit par se retrouver hors de son bureau, la porte bien refermée derrière lui.

«~Gaahhhrrrraa…~» dit Harry.

Il était effectivement sidéré.

Notamment car, sans le canular, il aurait très bien pu ne jamais obtenir le retourneur de temps en premier lieu.

Ou McGonagall le lui aurait peut-être donné de toute façon, mais plus tard dans la journée, quand il se serait décidé à lui demander pour son trouble du sommeil ou à lui parler du message du Choixpeau~?
Et est-ce qu'il aurait alors voulu s'auto-organiser un canular, ce qui l'aurait conduit à obtenir le retourneur de temps \emph{plus tôt}~?
Et donc la seule possibilité \emph{cohérente} était celle dans laquelle le canular commençait avant même qu'il se réveille ce matin~?

Harry se retrouva à envisager, pour la première fois de sa vie, que la réponse à cette question était peut-être littéralement \emph{inconcevable}.
Que puisque son cerveau contenait des neurones qui allaient uniquement en avant dans le temps, il n'y avait \emph{rien} que son cerveau puisse faire, aucune opération qu'il puisse effectuer, qui se rapprocherait de l'opération d'un retourneur de temps.

Harry avait jusqu'à présent vécu suivant la maxime de E.T. Jaynes selon laquelle, si vous ignoriez quelque chose d'un phénomène, c'était une information sur votre propre pensée, et non une information sur le phénomène lui-même~;
que votre incertitude était un fait vous concernant, non un fait sur le sujet de votre incertitude~;
que l'ignorance existait dans l'esprit, pas dans la réalité~;
qu'une carte vide ne correspondait pas à un territoire vide.
Il y avait des questions mystérieuses, mais une réponse mystérieuse était une contradiction en soi.
Un phénomène pouvait être mystérieux \emph{pour} une personne en particulier, mais il ne pouvait y avoir de phénomène mystérieux en lui-même.
Vénérer un mystère sacré, c'était simplement vénérer sa propre ignorance.

Harry avait donc contemplé la magie en refusant d'être intimidé.
Les gens n'ont aucun sens de l'histoire, ils apprennent la chimie, la biologie et l'astronomie, en pensant que ces domaines ont toujours été le cœur de la science, qu'ils n'ont \emph{jamais été} mystérieux.
Les étoiles ont été mystérieuses.
Lord Kelvin dit un jour de la nature de la vie et de la biologie -- parlant de la réponse des muscles à la volonté humaine et la transformation des graines en arbres -- que c'était un mystère “infiniment au-delà” de la portée de la science.
(Pas seulement un peu au-delà, notez bien, mais \emph{infiniment} au-delà.
Lord Kelvin ressentait une très forte charge émotionnelle à l'idée de \emph{ne pas savoir quelque chose}.)
Tout mystère résolu a commencé par être une énigme, depuis l'aube de l'humanité jusqu'au jour où quelqu'un l'a percé.

Et là, pour la première fois, il faisait face à la perspective d'un mystère qui menaçait d'être \emph{permanent}.
Si le temps ne fonctionnait pas par réseaux de causalité acycliques, alors Harry ne comprenait pas ce que signifiaient “cause” et “effet”~;
et si Harry ne comprenait pas les causes et les effets alors il ne comprenait pas de quoi la réalité pouvait bien être composée~;
et il était entièrement possible que son cerveau humain ne \emph{puisse} jamais le comprendre, parce que son cerveau était fait de \emph{neurones à l'ancienne dont la temporalité est linéaire}, ce qui s'avérait n'être qu'un sous-ensemble appauvri de la réalité.

Le bon côté des choses, c'était que l'Hilari-Thé, qui lui avait auparavant semblé tout-puissant et tout-incroyable, avait en réalité une explication beaucoup plus simple.
Explication que Harry avait ratée \emph{uniquement} parce que la vérité était totalement hors de son champ d'hypothèses et de tout ce que l'évolution avait conditionné son cerveau à comprendre.
Mais maintenant il \emph{avait} vraiment compris. Probablement.
Ce qui était relativement encourageant. Plus ou moins.

Harry jeta un coup d'œil à sa montre.
Il était presque 11h, il était allé se coucher la nuit dernière à 1h, donc dans l'état naturel des choses il devrait aller se coucher à 3h du matin.
Donc pour aller se coucher à 22h et se réveiller à 7h, il lui fallait revenir de cinq heures en arrière.
Ce qui voulait dire que s'il voulait revenir à son dortoir aux alentours de 6h, avant que quiconque soit réveillé, il ferait mieux de se dépêcher et…

Même \emph{rétrospectivement} Harry ne comprenait pas comment il était parvenu à accomplir \emph{la moitié} des choses nécessaires au canular.
D'où était venue la \emph{tarte}~?

Harry commençait à avoir vraiment peur du voyage dans le temps.

D'un autre côté, il se devait d'admettre que cela \emph{avait été} une opportunité irremplaçable.
Un canular que vous ne pouviez vous faire à vous-même qu'une seule fois dans une vie, dans les six heures suivant le moment où vous appreniez l'existence des retourneurs de temps.

C'était en fait \emph{encore plus} déroutant, maintenant qu'il y pensait.
Le temps lui avait présenté le canular entier comme un \emph{fait accompli}, et pourtant c'était indubitablement son propre ouvrage.
Le concept, l'exécution, le style d'écriture.
Les moindres détails, même ceux qu'il ne comprenait pas encore.

Bon, le temps filait et il n'y avait que trente heures dans une journée.
Harry connaissait \emph{une partie} des choses qu'il avait à faire, et il trouverait sûrement moyen de faire le reste, comme la tarte, en cours de route.
Pas la peine de s'attarder davantage.
Ce n'est pas comme s'il pouvait accomplir quoi que ce soit ici, coincé dans le \emph{futur}.

\later

Cinq heures plus tôt, Harry se glissait dans le dortoir, capuche rabattue sur la tête, déguisement de fortune juste au cas où quelqu'un serait déjà debout et risquerait de le voir en même temps que le Harry allongé dans son lit.
Il ne voulait pas avoir à expliquer à qui que ce soit son petit problème médical de Duplication Spontanée.

Heureusement, tout le monde semblait encore endormi.

Et il semblait aussi y avoir une boîte, emballée dans du papier rouge et vert, avec un ruban doré et brillant, posée à côté de son lit.
L'image parfaite et stéréotypée d'un cadeau de Noël, bien que ce ne fût pas Noël.

Harry s'avança à pas de loup, aussi doucement qu'il le pouvait, juste au cas où quelqu'un aurait réglé son Sourdineur au minimum.

Il y avait une enveloppe attachée à la boîte, scellée par de la cire vierge, sans sceau.\strut

Harry ouvrit l'enveloppe avec précaution et prit la lettre qui s'y trouvait.
La lettre disait~:
\begin{writtenNote}
Ceci est la Cape d'Invisibilité d'Ignotus Peverell, transmise à ses descendants, les Potter.
\strut Contrairement aux sorts et aux capes de moindre puissance, elle a le pouvoir de vous garder \underline{caché}, et pas seulement invisible.
Votre père me l'a prêtée pour l'étudier peu de temps avant sa mort, et je confesse en avoir fait grand et bon usage de nombreuses années.

J'ai peur qu'à l'avenir je doive me contenter d'un sortilège de désillusion.
Il est temps que je vous rende cette cape, car vous en êtes héritier.
J'avais pensé vous l'offrir comme cadeau de Noël, mais elle souhaitait revenir entre vos mains avant. \strut 
Il semblerait qu'elle s'attende à ce que vous ayez besoin d'elle. \strut
Faites-en bon \strut usage\vphantom{l}.

Vous pensez sans doute à toutes sortes de blagues formidables, telles celles que votre père a commises en son temps.
Si la totalité de ses méfaits venait à être connue, toutes les femmes de Gryffondor se réuniraient pour profaner sa tombe.
Je n'essaierai pas d'empêcher l'histoire de se répéter, mais soyez \underline{extrêmement} attentif à ne pas vous révéler.
Si Dumbledore voyait une occasion de posséder l'une des Reliques de la Mort, jamais il ne la laisserait s'échapper jusqu'à sa mort.

Un Très Joyeux Noël.

\end{writtenNote}

La note n'était pas signée.

\later

«~Un instant~», dit Harry, s'arrêtant net alors que les autres garçons s'apprêtaient à quitter le dortoir de Serdaigle.
«~Désolé, il y a quelque chose que je dois faire avec ma malle.
Je vous rejoins pour le petit déjeuner dans quelques minutes.~»

Terry Boot jeta un regard menaçant à Harry.
«~T'as pas intérêt à aller fouiller dans nos affaires.~»

Harry leva une main.
«~Je jure ne pas avoir l'intention de faire quoi que ce soit de la sorte à aucune de vos affaires, que je compte uniquement accéder à des objets m'appartenant, que je n'ai nulle intention de faire de blague ou quoi que ce soit de douteux envers aucun d'entre vous, et que je ne prévois pas de changer d'intention avant d'arriver dans le grand hall pour le petit déjeuner.~»

Terry fronça les sourcils.

«~Attends, est-ce que…

--- Pas d'inquiétude~», dit Pénélope Deauclaire, qui était là pour les guider.
«~Il n'y avait pas d'échappatoire. Bien dit, Potter, tu devrais être avocat.~»

Harry Potter cligna des yeux. Ah, oui, \emph{préfète} de Serdaigle.
«~Merci, dit-il. J'imagine.

--- Quand tu essaieras de trouver le grand hall, tu vas tu perdre.~»
Le ton de Pénélope montrait que c'était une évidence incontestable.
«~Dès que cela t'arrive, demande à un portrait comment te rendre au rez-de-chaussée.
Demande à un autre portait à \emph{l'instant} où tu suspectes t'être de nouveau perdu.
\emph{Surtout} s'il te semble que tu vas de plus en plus haut.
Si tu es plus haut que ce que le château est censé être, \emph{arrête-toi} et attends les équipes de recherche.
Sinon on te retrouvera quatre mois plus tard, plus vieux de cinq mois, vêtu d'un pagne et couvert de neige et ça c'est \emph{si tu restes à l'intérieur du château}.

--- Compris~», dit Harry en déglutissant. «~Euh, est-ce qu'on ne devrait pas dire ce genre de choses aux élèves dès le début~?~»

Pénélope soupira. «~Quoi, dire \emph{tout}~? Ça prendrait des semaines.
Tu apprendras au fur et à mesure.~»
Elle se tourna pour partir, suivie par les autres élèves.
«~Si je ne te vois pas au petit déjeuner dans trente minutes, Potter, je commencerai les recherches.~»

Une fois qu'ils furent tous partis, Harry accrocha la note à son lit -- il l'avait déjà écrite, ainsi que toutes les autres notes, travaillant dans son niveau caverne avant le réveil des autres.
Puis il entra précautionneusement dans le champ du Sourdinam et retira la Cape d'Invisibilité qui recouvrait le corps endormi de Harry-1.

Et par pure espièglerie, Harry mit la Cape d'Invisibilité dans la bourse de Harry-1, sachant qu'elle serait ainsi déjà dans la sienne.

\later

«~Je peux m'assurer que le message sera bien transmis à Cornélion Gommalse~», dit le tableau d'un homme à l'air aristocratique et doté d'un nez à vrai dire parfaitement normal.
«~Mais puis-je savoir d'où le message vennait à l'\emph{origine}~?~»

Harry haussa les épaules, un geste d'impuissance parfaitement maîtrisé.
«~On m'a dit qu'il a été prononcé par une voix caverneuse qui émanait d'un trou dans l'air lui-même, un trou qui s'ouvrait sur un abysse flamboyant.~»

\later

«~Hé~!~» dit Hermione d'un ton indigné depuis sa place à l'autre bout de la table du petit déjeuner.
«~C'est le dessert pour \emph{tout le monde}~!
Tu ne peux pas prendre une tarte toute entière et la mettre dans ta bourse~!

--- Je ne prends pas une tarte, j'en prends deux.
Désolé tout le monde, je dois y aller maintenant~!~»
Harry ignora les cris d'outrage et quitta le grand hall.
Il fallait qu'il arrive en classe de botanique un peu en avance.

\later

La professeure Chourave regardait Harry avec sévérité.

«~Et comment savez-\emph{vous} ce que les Serpentard comptent faire~?

--- Je ne peux nommer mes sources, dit Harry.
En fait, je vais devoir vous demander de prétendre que cette conversation n'a jamais eu lieu.
Faites simplement comme si vous les croisiez naturellement alors que vous alliez quelque part, ou quelque chose de ce genre.
Je me hâterai d'y aller dès que le cours de botanique se terminera, je pense que je pourrai distraire les Serpentard jusqu'à ce que vous arriviez.
Je ne suis pas du genre qu'on effraye ou malmène facilement, et je ne pense pas qu'ils oseront vraiment faire du mal au Survivant.
Cependant… je ne vous demande pas de courir dans les couloirs, mais j'apprécierais si vous ne perdiez pas trop de temps en chemin.~»

Chourave le regarda pendant un long moment, puis son expression s'adoucit.
«~Faites attention à vous, Harry Potter. Et… merci.

--- Essayez juste de ne pas être en retard, dit Harry.
Et souvenez-vous, quand vous arrivez là-bas, vous ne vous attendiez pas à me voir et cette conversation n'a jamais eu lieu.~»

\later

C'était affreux, voir comme il avait lui-même tiré brusquement Neville hors du cercle des Serpentard.
Neville avait raison, il avait utilisé trop de force, beaucoup trop de force.

«~Bonjour, dit Harry Potter froidement. Je suis le Survivant.~»

Huit garçons de première année, tous à peu près de la même taille.
L'un d'eux avait une cicatrice sur le front et ne se comportait pas comme les autres.

\vskip 0pt plus 4\baselineskip\settowidth{\versewidth}{Le don de nous voir comme les autres nous voient~!}
\begin{verse}[\versewidth]\itshape
Ô si un pouvoir nous offrait\\
Le don de nous voir comme les autres nous voient~!\\
De quelles erreurs cela nous libérerait\\
Et de quelles idées imbéciles…
\end{verse}

McGonagall avait raison.
Le Choixpeau avait raison.
C'était clair une fois qu'on le voyait de l'extérieur.

Il y avait quelque chose qui clochait chez Harry Potter.
%  LocalWords:  ome Ehehehehhheheh Ahahahaha Ahahahaa Simulacron Glehhahhh
%  LocalWords:  Gaahhhrrrraa fait accompli 1’s frae monie giftie gie
%  LocalWords:  oursel’s
