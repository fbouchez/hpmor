% vim:spell:spelllang=fr

\chapter{Gratification différée}

\lettrine{D}{rago} arborait une expression sévère, et sa robe à liseré vert semblait on ne sait comment être bien plus solennelle, sérieuse et mieux coupée que celles portées par les deux garçons derrière lui, pourtant exactement les mêmes.

«~Parle, dit Drago.

--- Ouais~!
Parle~!

--- T'as entendu l'boss~!
Parle~!

--- Vous deux, en revanche, \emph{fermez-la}.~»

Le dernier cours de vendredi allait commencer dans le vaste auditorium où les quatre maisons apprenaient la Défense… euh, la Magie de Combat.

Le dernier cours du vendredi.

Harry espérait que ce cours ne serait pas trop éprouvant, et que le génial professeur Quirrell se rendrait compte que ce n'était peut-être pas le meilleur moment pour attirer l'attention sur Harry.
Il avait un peu récupéré, mais…

… mais juste au cas où, il valait probablement mieux s'adonner à un petit moment de détente.

Harry s'enfonça dans sa chaise et octroya un regard empli de solennité à Drago et à ses sbires.

«~Vous demandez notre but~? déclama Harry.
Je puis répondre en deux mots.
C'est la victoire.
La victoire à tout prix -- la victoire au prix de toutes les terreurs -- la victoire, aussi longue et difficile que puisse être la route, car sans victoire il n'est point de…

--- \emph{Parle-moi de Rogue}, siffla Drago.
\emph{Qu'est-ce que tu as fait~?}~»

Harry se débarrassa de sa fausse solennité et jeta un regard plus sérieux à Drago.

«~Tu l'as vu, dit Harry.
Tout le monde l'a vu.
J'ai claqué des doigts.

--- \emph{Harry~!
Arrête de me taquiner~!}~»

Ah, alors il avait été promu à \emph{Harry} maintenant.
Intéressant.
En fait, Harry était à peu près certain que c'était prévu qu'il s'en rendre compte, et qu'il culpabilise s'il n'y répondait pas d'une façon ou d'une autre…

Harry se tapota les oreilles et jeta un regard lourd de sens en direction des sbires.

«~Ils ne parleront pas, dit Drago.

---Drago, dit Harry, je vais être cent pour cent honnête avec toi et te dire que hier je n'ai pas été particulièrement impressionné par l'intelligence de M. Goyle.~»

M. Goyle grimaça, embarrassé.

«~Moi non plus, dit Drago.
Je lui ai expliqué qu'au final je te dois une faveur à cause de cela.~»
(M. Goyle fit une nouvelle grimace.)
«~Mais il y \emph{a} une grande différence entre ce genre d'erreur et l'indiscrétion.
On les a vraiment entraînés à comprendre cela depuis leur enfance.

--- Très bien alors~», dit Harry.
Il baissa la voix même si l'ambiance sonore était déjà devenu floue en présence de Drago.
«~J'ai deviné un des secrets de Severus et lui ai fait un peu de chantage.~»

Drago durcit son expression.
«~Bien, maintenant dis-moi quelque chose que tu n'as pas confié dans le plus grand secret aux idiots de Gryffondor, c'est-à-dire l'histoire que tu \emph{voulais} voir se répandre dans l'école.~»

Harry sourit involontairement, et sut que Drago l'avait remarqué.

«~Que dit Severus~? dit Harry.

--- Qu'il ne s'était pas rendu compte à quel point les jeunes enfants étaient sensibles, dit Drago.
Même à Serpentard~!
Même à \emph{moi}~!

--- Es-tu certain, dit Harry, de vouloir savoir quelque chose que même le directeur de ta maison préférerait que tu ignores~?

--- Oui~», dit Drago sans hésitation.

\emph{Intéressant}.
«~Alors tu vas vraiment commencer par demander à tes sbires de partir, parce que je ne suis pas certain de pouvoir croire tout ce que tu crois à leur sujet.~»

Drago hocha la tête.
«~D'accord.~»

M. Crabbe et M. Goyle n'avaient \emph{vraiment} pas l'air contents.
«~Patron…~»
dit M. Crabbe.

«~Vous n'avez donné aucune raison à M. Potter de vous faire confiance, dit Drago.
Partez~!~»

Ce qu'ils firent.

«~En particulier~», dit Harry, baissant encore plus la voix, «~je ne suis pas \emph{entièrement} certain qu'ils n'iraient pas juste rapporter ce que je dis à Lucius.

--- Père ne \emph{ferait} pas ça~!~»
dit Drago, l'air véritablement effaré.
«~Ils \emph{me} sont dévoués~!

--- Je suis désolé, Drago, dit Harry.
Je ne suis juste pas certain de pouvoir croire tout ce que tu crois au sujet de ton père.
Imagine que ce soit ton secret, et moi qui dise que mon père ne ferait pas une chose pareille.~»

Drago hocha lentement la tête.
«~Tu as raison.
\emph{Je suis} désolé, Harry.
J'avais tort de te demander ça.~»

\emph{Comment ai-je fait pour être} autant \emph{promu~?
Ne devrait-il pas me détester à présent~?}
Harry sentait que la situation était exploitable… il aurait juste souhaité que son cerveau ne soit pas si épuisé.
En temps normal, il aurait adoré s'essayer aux manigances complexes.

«~Bref, dit Harry. Échange.
Je te donne une information qui ne circule pas dans les couloirs, et qui \emph{n'entre} pas en circulation dans les couloirs, en \emph{particulier} dans celui qui mène à ton père, et en retour tu me dis ce que Serpentard et toi pensez de toute cette affaire.

--- Marché conclu~!~»

Alors, comment rendre cela aussi vague que possible… quelque chose qui ne poserait pas de problème même si cela se savait…
«~Ce que j'ai dit était vrai.
J'ai découvert l'un des secrets de Severus, et j'ai fait un peu de chantage.
Mais Severus n'était pas la seule personne impliquée.

--- \emph{Je le savais}~!~» dit Drago, exultant.

Le moral de Harry loupa deux marches dans l'escalier.
Il avait apparemment dit quelque chose de très important et il ne savait pas pourquoi.
Ce n'était pas très bon signe.

«~Bien~», dit Drago qui avait maintenant un grand sourire sur le visage.
«~Alors voilà comment était la réaction à Serpentard.
D'abord, tous les idiots étaient là, “On déteste Harry Potter~!
Allons lui mettre une raclée~!”~»

Harry s'étouffa.
«~Qu'est ce qui \emph{cloche} avec le Choixpeau~?
Ce n'est pas du Serpentard, c'est du \emph{Gryffondor}…

--- Tous les enfants ne sont pas des prodiges~», dit Drago, mais avec un sourire narquois et conspirateur, comme pour sous-entendre qu'il était du même avis.
«~Et cela a pris à peu près quinze secondes pour qu'on leur explique en quoi ce ne serait pas vraiment faire une faveur à Rogue, donc ne t'en fais pas.
Bref, après cela il y a eu la deuxième vague d'idiots, ceux qui disaient~: “On dirait que Harry n'est qu'un bon samaritain de plus après tout.”

--- Et ensuite~?~»
demanda Harry qui souriait même s'il n'avait aucune idée de que pourquoi \emph{cela} était stupide.

«~Et alors les gens réellement intelligents ont commencé à parler.
Il est évident que tu as trouvé un moyen de mettre \emph{beaucoup} de pression à Rogue.
Et bien qu'il y ait plus d'une possibilité… la déduction \emph{suivante} était assez évidente~: ce moyen est en rapport avec l'emprise que Rogue a sur Dumbledore.
J'ai raison~?

--- Pas de commentaire~», dit Harry.
Au moins son cerveau traitait cette partie correctement.
La maison Serpentard \emph{s'était} demandée pourquoi Severus ne se faisait pas virer.
Et ils en avaient conclu que Severus faisait chanter Dumbledore.
Cela pouvait-il être vrai~?
Mais Dumbledore ne semblait pas agir comme si c'était le cas…

Drago continua de parler.
«~Et ce que les personnes intelligentes ont \emph{ensuite} fait remarquer, c'était que si tu pouvais mettre suffisamment de pression à Rogue pour le forcer à laisser tranquille la moitié de Poudlard, alors cela voulait dire que tu avais probablement assez de pouvoir pour te débarrasser entièrement de lui si tu le voulais.
Ce que tu lui as fait subir, c'est une humiliation, tout comme il a essayé de t'humilier -- mais tu nous as laissé notre directeur de maison.~»

Harry laissa son sourire grandir.

«~Ensuite, les personnes \emph{vraiment} intelligentes~», dit Drago, le visage maintenant sérieux, «~sont parties discuter en privé, et quelqu'un a fait remarquer que ce serait très stupide de laisser un ennemi dans les parages comme cela.
Si tu avais pu briser l'emprise de Rogue sur Dumbledore, cela aurait été la chose évidente à faire.
Dumbledore pourrait alors dégager Rogue de Poudlard, il le ferait peut-être même assassiner, et il te serait \emph{très} reconnaissant, et tu n'aurais pas à t'inquiéter de voir Rogue se glisser discrètement la nuit dans ton dortoir avec des potions intéressantes.~»

Le visage de Harry était à présent neutre.
Il n'avait pas pensé à cela et il aurait vraiment, vraiment dû.
«~Et de cela tu as conclu…~?

--- L'emprise de Rogue vient d'un secret de Dumbledore et \emph{tu connais ce secret}~!
Drago exultait.
Il ne peut pas être assez puissant pour totalement détruire Dumbledore, sinon Rogue l'aurait déjà utilisé.
Rogue refuse d'utiliser son emprise pour autre chose que de rester souverain de la maison Serpentard, et même là il n'obtient pas toujours ce qu'il veut, donc il doit avoir ses limites.
Mais cela \emph{doit} être un sacrément bon secret~!
Père essaie de tirer les vers du nez à Rogue depuis des \emph{années}~!

--- Et, dit Harry, maintenant Lucius pense que \emph{je} pourrais le lui dire.
As-tu déjà reçu une chouette…

--- Je l'aurai ce soir, dit Drago en riant.
La lettre dira~», sa voix prit une cadence différente, plus formelle, «~\emph{Mon fils bien-aimé~: je t'ai déjà parlé de la potentielle importance de Harry Potter.
Comme tu t'en es déjà rendu compte, celle-ci est à présent devenue plus grande et plus urgente.
Si tu vois la moindre piste d'amitié ou de pression sur lui, tu dois poursuivre dans cette voie, et toutes les ressources des Malfoy seront à ta disposition si besoin est.}~»

Eh ben.
«~Alors, dit Harry, sans commenter sur la véracité ou non de l'édifice compliqué qu'est ta théorie, laisse-moi juste dire que nous ne sommes pas encore de si bons amis que ça.

--- Je sais~», dit Drago.
Puis son visage devint \emph{très} sérieux, et il baissa la voix même si les sons étaient flous.
«~Harry, t'est-il venu à l'esprit que si tu sais quelque chose que Dumbledore souhaite garder secret, il peut simplement te faire tuer~?
Et cela ferait aussi passer le Survivant du statut de concurrent potentiel à celui de précieux martyr.

--- Pas de commentaire~», dit de nouveau Harry.
Il n'avait pas pensé à cela non plus.
Cela ne \emph{ressemblait} pas au style de Dumbledore… mais…

«~Harry, dit Drago, tu as clairement un talent \emph{incroyable}, mais tu n'as aucun entraînement et pas de mentors et tu fais parfois des choses stupides et \emph{tu as vraiment besoin d'un conseiller qui sait comment s'y prendre ou tu vas te mettre en danger~!}~» Le visage de Drago était plein de ferveur.

«~Ah, fit Harry.
Un conseiller, comme Lucius~?

--- Comme \emph{moi}~! dit Drago.
Je promets de garder tes secrets à l'abri de Père, à l'abri de \emph{tout le monde}, je t'aiderai juste à réaliser ce que tu veux faire~!~»

Waouh.

Harry vit zombie-Quirrell passer les portes de la salle en titubant.

«~Le cours va commencer, dit Harry.
Je vais réfléchir à ce que tu m'as dit, je regrette souvent de ne pas avoir eu la même formation que toi, c'est juste que je ne sais pas comment je peux te faire confiance si rapidement…~»

«~Il ne faut pas, dit Drago, c'est trop tôt.
Tu vois~?
Je te donne des bons conseils même si cela me dessert.
Mais on devrait peut-être \emph{se dépêcher} de devenir des amis proches.

--- Je suis ouvert à cela~», dit Harry, qui essayait déjà de trouver un moyen d'exploiter cela.

«~Un autre conseil~», se pressa d'ajouter Drago tandis que Quirrell se traînait vers son bureau, «~pour l'instant tout le monde à Serpentard se pose des questions à ton sujet, alors si tu nous fais la cour, et il me semble que tu la fais, tu devrais faire quelque chose qui signale ton amitié à Serpentard.
\emph{Bientôt}, du genre aujourd'hui ou demain.

--- Laisser Severus continuer de donner des points supplémentaires à Serpentard n'a pas suffi~?~»
Aucune raison pour que Harry ne s'en attribue pas le mérite.

La prise de conscience fit scintiller les yeux de Drago, puis il dit rapidement~:
«~Ce n'est pas pareil, crois-moi, ça doit être quelque chose d'évident.
Pousse ta rivale, la sang-de-bourbe Granger, dans un mur ou quelque chose du genre, tout le monde à Serpentard comprendra ce que ça veut dire…

--- Ce n'est \emph{pas} comme ça que Serdaigle fonctionne, Drago~!
Si tu dois pousser quelqu'un contre un mur, ça veut dire que ton cerveau est trop \emph{faible} pour le vaincre autrement, et tout le monde à Serdaigle \emph{sait ça}…~»

L'écran sur le pupitre de Harry clignota en s'allumant, provoquant une vague de nostalgie pour la télévision et les ordinateurs.

«~Hum hum…~», fit la voix de Quirrell, qui semblait s'adresser personnellement à Harry depuis l'écran.
«~Merci de vous asseoir.~»

\later

Les enfants étaient tous assis, regardant les écrans-relais sur les pupitres, ou directement la grande plateforme de marbre blanc où se tenait le professeur, penché sur son bureau, sur la petite estrade de marbre sombre.

«~Aujourd'hui, dit Quirrell, j'avais prévu de vous enseigner votre premier sort défensif, un petit bouclier qui était l'ancêtre du \emph{Protego} moderne.
Mais après réflexion, à la lumière des événements récents, j'ai décidé de changer la leçon d'aujourd'hui.~»

Le regard de Quirrell parcourut la rangée de sièges.
Harry grimaça depuis son siège, à la dernière rangée.
Il avait un pressentiment sur l'identité de celui qui allait être appelé.

«~Drago, de la Noble et Très Ancienne Maison Malfoy~», dit Quirrell.

Pfiou.

«~Oui, professeur~?~» répondit Drago.
Sa voix était amplifiée et semblait venir de l'écran-relai sur le pupitre de Harry, qui montrait le visage de Drago en train de parler.
Puis l'écran revint à Quirrell, qui demanda~:

«~Votre ambition est-elle de devenir le prochain Seigneur des Ténèbres~?

--- C'est une question étrange, professeur, dit Drago.
Je veux dire, qui serait assez stupide pour admettre cela~?~»

Quelques élèves rirent, mais pas beaucoup.

«~En effet, dit Quirrell, il est donc inutile de poser cette question, et pourtant je ne serais pas le moins du monde surpris s'il y avait un élève ou deux dans mes cours qui entretiennent l'ambition de devenir le prochain Seigneur des Ténèbres.
Après tout, \emph{je} voulais être le prochain Seigneur des Ténèbres quand \emph{j'}étais un jeune Serpentard.~»

Cette fois le rire était bien plus répandu.

«~Eh bien, c'\emph{est} la maison des ambitieux, après tout, dit Quirrell en souriant.
Ce n'est que plus tard que je ne me suis rendu compte que ce que j'aimais vraiment, c'était la magie de combat, et que ma véritable ambition était de devenir un grand sorcier combattant et d'enseigner un jour à Poudlard.
Quoi qu'il en soit, à l'âge de treize ans, j'ai parcouru toute la section histoire de la bibliothèque de Poudlard, examinant méticuleusement les vies et destins des Seigneurs des Ténèbres passés, et j'ai fait la liste de toutes les erreurs que \emph{je} ne ferai jamais quand \emph{je} serai un Seigneur des Ténèbres…~»

Harry ne put s'empêcher de glousser.

«~Oui M. Potter, très amusant.
Alors dites-moi, à votre avis, quel était le tout premier point de cette liste~?~»

\emph{Génial}.
«~Euh… ne jamais utiliser de méthode compliquée pour se débarasser d'un ennemi quand on peut juste l'Abracadabrer~?

--- Le \emph{terme}, M. Potter, est \emph{Avada Kedavra}~», pour une quelconque raison, le ton du professeur était cinglant, «~et non, cela n'était \emph{pas} sur la liste faite à treize ans.
Voulez-vous réessayer~?

--- Euh… ne jamais se vanter de son plan maléfique auprès de quelqu'un~?~»

Quirrell rit.
«~Ah, \emph{ça} c'était le numéro deux.
Dites-moi, M. Potter, avons-nous lu les mêmes livres~?~»

Cela déclencha plus de rires, avec une pointe de nervosité.
Harry serra les dents en silence.
Nier n'aurait servi à rien.

«~Mais non.
Le \emph{premier} point était~: “Je n'irai pas provoquer des ennemis puissants et vicieux.” L'histoire du monde serait très différente si Mornelithe Falconsbane ou Hitler avaient saisi ce conseil élémentaire.
Maintenant, \emph{si}, M. Potter -- juste \emph{si} vous vous trouviez par hasard entretenir la même ambition que celle que j'avais jeune Serpentard -- alors, j'espère que ce n'est pas votre ambition que de devenir un Seigneur des Ténèbres \emph{stupide}.

--- Professeur, dit Harry en grinçant des dents, je suis un \emph{Serdaigle}, et ce n'est pas mon ambition que de devenir stupide, point final.
Je sais que ce que j'ai fait aujourd'hui était idiot.
Mais ce n'était pas \emph{ténébreux}~!
Ce n'est \emph{pas} moi qui ai porté le premier coup dans ce combat~!

--- M. Potter, vous êtes un idiot.
Mais je l'étais moi aussi à votre âge.
Donc j'ai anticipé votre réponse et altéré la leçon du jour en conséquence.
M. Gregory Goyle, voudriez-vous venir ici s'il vous plaît~?~»

La classe fit silence, surprise.
Harry ne s'y attendait pas non plus.

Et vu sa tête, M. Goyle pas plus que lui.
Il semblait hésitant et inquiet tandis qu'il grimpait sur la plateforme de marbre et s'approchait de l'estrade.

Quirrell se redressa, s'éloignant du bureau sur lequel il s'était appuyé.
Il eut soudain l'air plus fort, ses mains se serrèrent en poings et il adopta une posture d'art martial clairement reconnaissable.

À cette vue, Harry écarquilla les yeux, et il comprit pourquoi M. Goyle avait été appelé.

«~La plupart des sorciers, dit le professeur, ne prennent pas la peine de s'intéresser à ce qu'un Moldu appellerait les arts martiaux.
Une baguette n'est-elle pas plus forte qu'un poing~?
Cette attitude est stupide.
Les baguettes sont tenues par des poings.
Si vous voulez être un grand sorcier combattant, vous \emph{devez} apprendre les arts martiaux à un niveau qui impressionnerait même un Moldu.
Je vais maintenant démontrer une technique d'importance vitale, que j'ai apprise dans un \emph{dojo}, une école moldue d'arts martiaux, et dont je vous parlerai juste après.
Pour le moment…~»
Quirrell fit quelques pas, gardant sa posture, s'avançant vers l'endroit où se tenait M. Goyle.
«~M. Goyle, je vais vous demander de m'attaquer.

--- Professeur~», dit M. Goyle, la voix maintenant autant amplifiée que celle du professeur, «~puis-je vous demander le niveau que…

--- Sixième \emph{dan}.
Vous ne serez pas blessé, et moi non plus.
Et si vous voyez une ouverture, saisissez-la.~»

M. Goyle hocha la tête, l'air vraiment soulagé.

«~Notez, dit Quirrell, que M. Goyle avait peur d'attaquer quelqu'un qui ne connaitrait pas les arts martiaux à un niveau suffisant de peur que lui ou moi ne nous blessions.
L'attitude de M. Goyle est parfaitement correcte, et il gagne trois points Quirrell pour cela.
Maintenant, battez-vous~!~»

Le jeune garçon fondit sur Quirrell, ses poings volaient, et le professeur arrêta chaque coup en dansant à reculons, puis il donna un coup de pied que Goyle bloqua et pivota pour tenter de faire chuter Quirrell d'un balayage de la jambe et Quirrell bondit par dessus et tout allait trop vite pour que Harry comprenne ce qui se passait et soudain Goyle était sur le dos et poussait avec ses jambes et Quirrell \emph{volait} réellement \emph{dans les airs} puis il heurta le sol épaule la première et roula.

«~Stop~!~» cria Quirrell au sol, l'air un peu paniqué.
«~Vous avez gagné~!~»

M. Goyle s'arrêta si brusquement qu'il tituba, trébuchant presque à cause de l'inertie prise en s'élançant tête la première sur Quirrell.
Il semblait stupéfait.

Quirrell arqua le dos et bondit sur ses pieds sans utiliser ses bras avec un étrange mouvement de ressort.

On n'entendait aucun bruit dans la classe, un silence né d'une confusion totale.

«~M. Goyle, dit Quirrell, de quelle technique d'importance vitale ai-je fait la démonstration~?

--- Comment tomber correctement quand on se fait projetter, dit M. Goyle.
C'est une des premières leçons qu'on apprend…

--- Ça aussi~», dit Quirrell.

Il y eut une pause.

«~La technique d'importance vitale que je viens de montrer, dit Quirrell, est comment perdre.
Vous pouvez vous rasseoir, M. Goyle, merci.~»

M. Goyle descendit de la plateforme, quelque peu abasourdi, état que partageait Harry.

Quirrell marcha à son bureau et s'appuya de nouveau dessus.
«~Nous oublions parfois les choses les plus élémentaires parce que nous les avons apprises il y a trop longtemps.
Je me suis rendu compte que j'avais fait de même en planifiant mon cours.
On n'apprend pas aux élèves à projeter avant de leur avoir appris à tomber.
Et je ne dois pas vous apprendre à vous battre tant que vous ne comprendrez pas comment perdre.~»

Le visage du professeur se durcit et il sembla à Harry qu'il voyait dans ses yeux une trace de douleur, un soupçon de tristesse.
«~J'ai appris comment perdre dans un \emph{dojo} en Asie, où vivent, comme les Moldus le savent, tout bon pratiquant d'arts martiaux.
Ce \emph{dojo} enseignait un style qui avait la réputation auprès des sorciers combattants de bien s'adapter aux duels magiques.
Le maître de ce \emph{dojo} -- un vieil homme pour un Moldu -- était le meilleur enseignant de cette technique.
Il n'avait pas la moindre idée que la magie existait, bien sûr.
J'ai candidaté pour y étudier, et fus l'un des rares élèves à être acceptés cette année là parmi tous les candidats.
Il se peut qu'une pointe d'influence spéciale y ait été pour quelque chose.~»

Il y eut quelques rires dans la salle.
Harry ne les partagea pas.
Cela n'était pas du tout correct.

«~Quoi qu'il en soit.
Pendant un de mes premiers combats, après avoir été vaincu d'une façon particulièrement humiliante, j'ai perdu le contrôle de moi-même et j'ai attaqué mon partenaire d'entraînement…~»

\emph{Oh là là~!}

«~…heureusement avec mes poings plutôt qu'avec ma magie.
De façon surprenante, le maître ne me renvoya pas immédiatement.
Mais il me dit que j'avais une faille dans mon tempérament, il me l'expliqua, et je sus qu'il disait vrai.
Il dit alors que je devais apprendre à perdre.~»

Quirrell continua, impassible.

«~Sous son ordre direct, tous les élèves du \emph{dojo} se mirent en ligne.
Un par un, ils m'approchèrent.
Je ne devais \emph{pas} me défendre.
Je n'avais que le droit d'implorer leur grâce.
Un par un, ils me giflèrent, ou me frappèrent, et me firent tomber au sol.
Certains d'entre eux me crachèrent dessus.
Ils me traitèrent de tous les noms dans leur langue.
Et à chacun je devais répondre “j'ai perdu~!” ou d'autres choses comme “arrête-toi, je t'en supplie~!” et “je reconnais que tu es meilleur que moi~!”~»

Harry essayait d'imaginer cela et n'y parvenait tout simplement pas.
Il était impossible qu'une chose pareille soit arrivée au digne professeur Quirrell.

«~J'étais un prodige en magie de combat, même à l'époque.
J'aurais pu tous les tuer avec uniquement de la magie sans baguette.
Mais je ne l'ai pas fait.
J'ai appris à perdre.
Aujourd'hui encore, cela reste dans mon souvenir les heures parmi les plus déplaisantes de mon existence.
Et lorsque j'ai quitté le dojo huit mois plus tard -- ce qui était beaucoup trop tôt, mais je ne pouvais me permettre d'y passer plus de temps -- le maître me dit qu'il espérait que j'avais compris pourquoi cela avait été nécessaire.
Et je lui ai répondu que c'était une des leçons les plus précieuses que j'avais jamais apprises.
Ce qui était vrai, et l'est toujours.~»

Quirrell adopta une expression amère.
«~Vous vous demandez où se trouve cet extraordinaire \emph{dojo}, et si vous pourriez y étudier.
Vous ne le pouvez pas.
Car peu de temps après, un autre aspirant élève arriva en ce lieu caché, sur cette montagne reculée.
Celui-Dont-On-Ne-Doit-Pas-Prononcer-Le-Nom.~»

On entendit de nombreuses personnes reprendre leur respiration.
Harry commença à avoir la nausée, il connaissait la suite.

«~Le Seigneur des Ténèbres entra dans l'école ouvertement, sans déguisement, les yeux rougeoyants et compagnie.
Les élèves tentèrent de lui bloquer la route, et il transplana simplement à travers le barrage.
Ils étaient terrifiés, mais aussi disciplinés, et le maître s'avança.
Le Seigneur des Ténèbres exigea -- il ne demanda pas, il exigea -- qu'on lui enseigne.~»

Le visage de Quirrell se durcit fortement.
«~Peut-être le maître avait-il lu trop de livres répétant le mensonge selon lequel à haut niveau les arts martiaux permettaient de vaincre les démons eux-mêmes.
Quoi qu'il en soit, le maître refusa.
Le Seigneur des Ténèbres lui demanda pourquoi il ne pouvait pas devenir un élève.
Le maître lui répondit qu'il n'avait aucune patience, et c'est là que le Seigneur des Ténèbres lui arracha la langue.~»

Souffle coupé collectif.

«~Vous pouvez imaginer ce qui se passa ensuite.
Les élèves essayèrent de se jeter sur le Seigneur des Ténèbres et tombèrent, paralysés sur place.
Et alors…~»

La voix de Quirrell hésita un moment, puis il reprit.

«~Il existe un Sortilège Impardonnable, nommé Doloris, qui provoque une douleur insupportable.
S'il est maintenu plus de quelques minutes, il produit un état de folie permanent.
Un par un, le Seigneur des Ténèbres Endolorit les élèves du maître jusqu'à la folie, puis il les acheva du Sortilège de la Mort tout en forçant le maître à regarder.
Une fois que tous les élèves eurent été ainsi tués, ce fut le tour du maître.
Je l'ai appris de la bouche du seul élève survivant, laissé en vie par le Seigneur des Ténèbres pour qu'il raconte cette histoire, et qui était un de mes amis…~»

Quirrell se détourna, et lorsqu'il refit face à la classe un instant plus tard, il semblait s'être ressaisi.

«~Les sorciers maléfiques ne savent pas contrôler leur colère, dit calmement le professeur.
C'est un défaut quasi-universel de cette espèce, et quiconque a l'habitude de les combattre apprend rapidement à compter dessus.
Comprenez que le Seigneur des Ténèbres ne gagna \emph{pas}, ce jour-là.
Son but était d'apprendre les arts martiaux, et pourtant il est reparti sans avoir eu une seule leçon.
Il était stupide de sa part de souhaiter que cette histoire soit racontée, car elle ne montre pas sa force, mais plutôt une faiblesse exploitable.~»

Le regard de Quirrell se concentra sur un seul élève de la classe.

— Harry Potter, dit-il.

--- Oui, dit Harry, la voix rauque.

--- Qu'avez-vous fait de mal aujourd'hui, \emph{précisément}~?~»

Harry crut qu'il allait vomir.
«~J'ai perdu le contrôle de ma colère.

--- Ceci n'est \emph{pas} précis, dit Quirrell.
Je vais le décrire avec plus d'exactitude.
Il existe de nombreux animaux ayant ce qu'on appelle des luttes de dominance.
Ils se foncent dessus avec leurs cornes -- essayant de s'assommer l'un l'autre, pas de s'encorner.
Ils se battent avec leurs pattes -- griffes rétractées.
Mais pourquoi avec leurs griffes rétractées~?
Ils auraient certainement une meilleure chance de gagner s'ils utilisaient leurs griffes.
Oui mais alors leur ennemi sortirait peut-être lui aussi ses griffes, et alors, au lieu de résoudre leur lutte de dominance et d'avoir un gagnant et un perdant, ils pourraient tous deux se blesser sérieusement.~»

Quirrell sembla regarder Harry droit dans les yeux depuis l'écran-relai.
«~Ce que vous avez démontré aujourd'hui, M. Potter, c'est que -- à la différence des animaux qui gardent leurs griffes rétractées et acceptent l'issue du combat -- vous ne savez pas perdre une lutte de dominance.
Lorsqu'un \emph{professeur de Poudlard} vous a défié, vous n'avez pas battu en retraite.
Lorsqu'il a semblé que vous risquiez de perdre, vous avez sorti vos griffes, sans vous soucier du danger.
Vous avez \emph{renchéri}, puis vous avez \emph{encore} renchéri.
Cela a commencé par une gifle donnée par le professeur Rogue, qui était évidemment dominant.
Au lieu de perdre, vous l'avez giflé en retour et avez fait perdre dix points à Serdaigle.
Peu de temps après, vous parliez de quitter Poudlard.
Le fait que vous ayez surenchéri encore plus dans une direction inconnue et que vous ayez fini par gagner on ne sait comment ne change rien au fait que vous êtes un idiot.

--- Je comprends~», dit Harry.
Il avait la gorge sèche.
Cela \emph{avait} été précis.
\emph{Effroyablement} précis.
Maintenant que Quirrell l'avait dit, Harry pouvait voir rétrospectivement que c'était une description \emph{parfaitement} fidèle de ce qui s'était passé.
Lorsque des gens avaient un modèle de vous aussi précis que cela, vous pouviez vous demander s'ils n'avaient pas raison sur d'autres sujets, comme sur votre intention de tuer par exemple.

«~M. Potter, la \emph{prochaine} fois que lors d'un duel vous renchérissez au lieu de perdre, vous pourriez perdre \emph{tout} ce que vous avez misé sur la table.
Je ne peux pas deviner ce que vos mises étaient aujourd'hui.
Mais je peux deviner qu'elles étaient bien, bien trop élevées par rapport à la perte de dix points de maison.~»

Comme le destin de l'Angleterre magique, par exemple.
C'était ce qu'il avait misé.

«~Vous allez vous défendre en disant que vous essayiez d'aider tout Poudlard, un but bien plus important, méritant que l'on prenne de grands risques.
C'est un \emph{mensonge}.
Si vous aviez…

--- J'aurais accepté la gifle, attendu, et choisi le meilleur moment pour agir~», dit Harry la voix rauque.
«~Mais alors j'aurais \emph{perdu}.
Je l'aurais laissé me dominer.
C'est ce que le Seigneur des Ténèbres n'a pas pu faire face au maître dont il désirait l'enseignement.~»

Quirrell hocha la tête.
«~Je vois que vous avez parfaitement compris.
Et donc, M. Potter, aujourd'hui, vous allez apprendre à perdre.

--- Je…

--- Je n'accepterai aucune objection, M. Potter.
Il est à la fois évident que vous en avez besoin et que vous êtes suffisamment fort pour le supporter.
Je vous assure que l'expérience ne sera pas aussi brutale que celle que j'ai traversée, bien qu'il soit fort probable que vous vous rappeliez ensuite de ces quinze minutes comme les pires de votre jeune existence.~»

Harry déglutit.
«~Professeur, dit-il d'une petite voix, pourrions-nous faire cela une autre fois~?

--- Non, répondit simplement Quirrell.
Vous n'en êtes qu'au cinquième jour de votre éducation à Poudlard et voyez ce qui s'est déjà produit.
Nous sommes aujourd'hui vendredi, notre \emph{prochain} cours de défense est mercredi.
Samedi, dimanche, lundi, mardi, mercredi…
Non, nous n'avons \emph{pas} le temps d'attendre.~»

Il y eut quelques rires, mais très peu.

«~Veuillez considérer cela comme un ordre de votre professeur, M. Potter.
Je tiens à préciser que dans le cas contraire, je ne vous enseignerai aucun sort offensif, parce que je finirais alors par apprendre que vous avez sévèrement blessé voire tué quelqu'un.
Malheureusement, j'ai entendu dire que vos doigts sont déjà de puissantes armes.
Ne les claquez à aucun moment durant ce cours.~»

Plus de rires épars, aux sonorités plutôt nerveuses.

Harry se sentait au bord des larmes.
«~Professeur, si vous faites quoi que ce soit ressemblant à ce dont vous venez de parler, cela va me mettre en colère, et je voudrais vraiment ne pas me remettre en colère aujourd'hui…

--- Le but n'est \emph{pas} d'éviter de se mettre en colère~», expliqua Quirrell, le visage grave.
«~La colère est naturelle.
Vous devez apprendre à perdre même quand vous êtes en colère.
Ou du moins à \emph{faire semblant} de perdre pour pouvoir ensuite \emph{planifier} votre vengeance.
Comme je l'ai fait aujourd'hui avec M. Goyle, à moins bien sûr que l'un d'entre vous pense qu'il \emph{est} réellement meilleur…

--- J'le suis pas~!~»
cria M. Goyle depuis son bureau, l'air paniqué.
«~J'sais que vous avez pas vraiment perdu~!
S'il vous plaît, planifiez pas de vengeance~!~»

Le cœur de Harry se serrait.
Quirrell ne savait pas qu'il avait un mystérieux côté obscur.
«~Professeur, il faut vraiment qu'on en parle après le cours…

--- Nous le ferons~», dit Quirrell. Le ton était celui d'une promesse.
«~Après que vous ayez appris à perdre.~»
Son visage était sérieux.
«~Cela va sans dire que nous allons exclure tout ce qui pourrait vous blesser ou même provoquer une douleur importante.
La douleur viendra de la difficulté à perdre, au lieu de riposter et que le combat dégénère jusqu'à le gagner.~»

Harry ne respirait plus que par halètements courts et paniqués.
Il était plus effrayé que lorsqu'il avait quitté le cours de potions.
«~Professeur, parvint-il à dire, je ne veux pas que vous vous fassiez renvoyer à cause de cela…

--- Cela n'arrivera pas, dit Quirrell, si \emph{vous} leur dites après coup que c'était nécessaire.
Et je vous fais confiance pour cela.~»
Un instant, la voix du professeur se fit très sèche.
«~Croyez-moi, ils ont toléré pire dans leurs couloirs.
Ce cas ne sera exceptionnel que parce qu'il se passe dans une salle de cours.

--- Professeur~», murmura Harry, dont la voix semblait tout de même être répétée partout, «~vous croyez vraiment que si je ne fais pas cela, je pourrais blesser quelqu'un~?

--- Oui, dit simplement Quirrell.

--- Alors~», Harry avait la nausée, «~je vais le faire.~»

Quirrell se tourna vers les Serpentard.
«~Donc… avec l'approbation complète de votre enseignant, et de telle manière que Rogue ne pourra être blâmé pour vos actes… l'un ou l'une d'entre vous veulent-ils faire preuve de dominance sur le Survivant~?
Le malmener, le mettre au sol, l'entendre implorer votre pitié~?~»

Cinq mains se levèrent.

«~Tous ceux qui ont levé la main, vous n'êtes que des idiots finis.
Quelle partie de \emph{faire semblant de perdre} n'avez-vous pas compris~?
Si Harry Potter devient le prochain Seigneur des Ténèbres, il vous pourchassera et vous tuera après avoir obtenu son diplôme.~»

Les cinq mains retombèrent promptement sur leurs pupitres.

«~Je ne le ferai pas~», dit faiblement Harry.
«~Je jure de ne jamais chercher à me venger de ceux qui m'aideront à apprendre à perdre.
Professeur… \emph{s'il vous plaît} pourriez-vous… \emph{arrêter} de faire cela~?~»

Quirrell soupira.
«~Je \emph{suis} désolé, M. Potter.
Je comprends que cela vous dérange, que vous projetiez de devenir le prochain Seigneur des Ténèbres ou non.
Mais ces enfants avaient \emph{eux aussi} une importante leçon de vie à apprendre.
Accepteriez-vous que je vous octroie un point Quirrell en guise d'excuse~?

--- Disons deux~», dit Harry.

Il y eut une vague de rires surpris, ce qui désamorça une partie de la tension ambiante.

«~Marché conclu, dit Quirrell.

--- Et quand j'aurai mon diplôme, je vous pourchasserai et je vous \emph{chatouillerai}.~»

Il y eut plus de rires, cependant Quirrell ne sourit pas.

Harry avait l'impression de lutter contre un anaconda, d'essayer de guider la conversation vers l'étroit chemin qui permettrait aux gens de se rendre compte qu'il n'était pas un Seigneur des Ténèbres après tout…
\emph{pourquoi} Quirrell était-il si suspicieux à son égard~?

«~Professeur~», interpella Drago, la voix non amplifiée, «~ce n'est pas non plus mon ambition que de devenir un Seigneur des Ténèbres stupide.~»

La classe fit silence, choquée.

\emph{Tu n'as pas à faire cela~!} faillit lâcher Harry, mais il se maîtrisa à temps~;
Drago ne souhaitait peut-être pas que l'on sache qu'il faisait cela par amitié pour Harry… ou par désir d'avoir l'air amical…

Appeler \emph{cela} un \emph{désir d'avoir l'air amical} lui fit se sentir petit et mesquin.
Si l'intention de Drago était de l'impressionner, cela marchait à la perfection.

Quirrell regardait Drago d'un air grave.
«~\emph{Vous} craignez de ne pas pouvoir faire semblant de perdre, M. Malfoy~?
Que ce défaut qui caractérise M. Potter vous caractérise également~?
Votre père vous a \emph{certainement} mieux éduqué que cela.

--- Lorsqu'il s'agit de parler, peut-être~», dit Drago, maintenant sur l'écran-relai.
«~Pas lorsqu'il s'agit d'être malmené et mis au sol.
Je veux être aussi fort que vous, professeur.~»

Quirrell leva les sourcils, qui restèrent ainsi.
«~J'ai peur, M. Malfoy, dit-il après un moment, que les arrangements prévus pour M. Potter, qui impliquent quelques Serpentard un peu plus âgés à qui l'on expliquera \emph{plus tard} à quel point ils étaient stupides, ne fonctionneraient pas pour vous.
Mais, en tant que professionnel, mon opinion est que vous êtes déjà très solide.
Si je devais apprendre que vous avez échoué, comme M. Potter a échoué aujourd'hui, je ferai les préparatifs appropriés et m'excuserai auprès de vous et de toute personne que vous auriez blessée.
Je ne pense cependant pas que cela sera nécessaire.

--- Je comprends, professeur~», dit Drago.

Quirrell regarda la classe.
«~Quelqu'un d'autre souhaite-t-il devenir fort~?~»

Quelques élèves regardèrent autour d'eux nerveusement.
Certains, pensa Harry depuis le dernier rang, semblaient ouvrir la bouche, mais sans rien dire.
Au final, personne ne s'avança.

«~Drago Malfoy sera le général d'une des armées de cette année, dit Quirrell, s'il daigne s'impliquer dans cette activité extrascolaire.
Et maintenant, M. Potter, si vous voulez bien vous avancer.~»

\later

\emph{Oui}, avait dit le professeur Quirrell, \emph{cela doit se faire devant tout le monde, devant vos amis, car c'est là que Rogue vous a confronté et c'est là que vous devez apprendre à perdre}.

Et maintenant les élèves de première année observaient, 
dans un silence imposé magiquement, et priés à la fois par Harry et le professeur de ne pas intervenir.
Hermione avait détourné le visage, mais elle n'avait rien dit ni même lancé de regard particulier, peut-être parce qu'elle avait été là en cours de potions, elle aussi.

Harry se tenait sur un tapis bleu clair, comme on aurait pu en trouver dans un dojo moldu, que Quirrell avait installé au sol en prévision du moment où Harry y serait projeté.

Harry était effrayé par ce qu'il risquait de faire.
Si Quirrell avait raison à propos de son intention de tuer…

La baguette de Harry reposait sur le bureau du professeur, non pas parce que Harry connaissait des sorts qui lui auraient permis de se défendre, mais parce que sinon (pensait Harry), il aurait pu essayer de la fourrer dans l'orbite de quelqu'un.
Sa bourse était aussi posée là, contenant son retourneur de temps, à présent protégé mais toujours potentiellement fragile.

Harry avait imploré le professeur de lui transfigurer des gants de boxe pour emprisonner ses mains.
Quirrell lui avait silencieusement lancé un regard compréhensif, et avait refusé.

\emph{Je ne viserai pas leurs yeux, je ne viserai pas leurs yeux, je ne viserai pas leurs yeux, ce serait la fin de ma vie à Poudlard, je serais arrêté}, se chanta Harry à lui-même, essayant de marteler la pensée dans son cerveau, espérant qu'elle resterait là si jamais son intention de tuer prenait le dessus.

Le professeur revint, accompagné de treize Serpentard de différentes années mais tous plus âgés.
Harry reconnut l'un d'entre eux, celui qu'il avait entarté.
Deux autres présents lors de cette confrontation étaient aussi là.
Celui qui avait dit d'arrêter, qu'ils ne devraient vraiment pas faire cela, était absent.

«~Je répète, dit Quirrell sévèrement, Potter \emph{ne doit pas} être réellement blessé.
Tout \emph{accident} sera considéré comme un acte délibéré.
Vous avez compris~?~»

Les Serpentard acquiescèrent en souriant.

«~Alors n'hésitez pas à remettre le Survivant à sa place~», dit le professeur, avec un sourire malicieux que seuls les élèves de première année comprirent.

Une sorte de consentement mutuel avait placé l'entarté en tête du groupe.

«~Potter, dit Quirrell, je vous présente M. Peregrine Derrick.
Il est meilleur que vous et il va bientôt vous le montrer.~»

Derrick s'avança et le cerveau de Harry poussa un cri discordant, il ne faut pas s'enfuir, il ne faut pas se défendre…

Derrick s'arrêta à une distance d'environ un bras de Harry.

Harry n'était pas encore en colère, juste effrayé.
Parce qu'il faisait face à un jeune adolescent plus grand que lui d'au moins cinquante centimètres, dont les muscles étaient clairement visibles, avec du duvet sur la lèvre supérieur, et un horrible sourire anticipatoire.

«~Demandez-lui de ne pas vous faire de mal, dit Quirrell.
Peut-être que s'il vous trouve suffisamment pathétique, il décidera que vous n'en valez pas la peine et qu'il s'en ira.~»

Les Serpentard face à Harry commencèrent à rire.

«~S'il te plaît~», dit Harry, la voix vacillante, «~ne me, fais, pas mal…

--- Ça ne semblait pas très sincère~», dit Quirrell.

Le sourire de Derrick s'agrandit.
Ce crétin lourdaud prenait un air très supérieur et…

… la température sanguine de Harry commença à chuter…

«~S'il te plaît, ne me fais pas mal~», essaya à nouveau Harry.

Quirrell secoua la tête.
«~Au nom de Merlin, comment êtes-vous parvenu à faire sonner cela comme une insulte, Potter~?
Vous ne pouvez vous attendre qu'à une seule réponse possible de la part de M. Derrick.~»

Derrick s'avança et bouscula délibérément Harry.

Harry recula de quelques pas, et, avant même de pouvoir s'en empêcher, se redressa tel un bloc de glace.

«~Faux, dit le professeur, faux, faux, faux.

--- Tu m'as bousculé, Potter, dit Derrick. Excuse-toi.

--- Je suis désolé~!

--- Tu n'as pas l'\emph{air} désolé~», dit Derrick.

Les yeux de Harry s'écarquillèrent d'indignation, il \emph{avait} réussi à prendre un ton suppliant…

Derrick le poussa avec force, et Harry tomba à quatre pattes sur le tapis.

La trame bleue semblait onduler dans son champ de vision.

Il commençait à douter de la motivation réelle qui poussait Quirrell à lui enseigner cette prétendue \emph{leçon}.

Il sentit un pied se poser sur sa hanche, et l'instant suivant une forte poussée l'envoya s'étaler sur le dos.

Derrick rit.
«~C'est \emph{vraiment amusant}~», dit-il.

Tout ce qu'il avait à faire était de dire que c'en était assez.
Et d'aller tout raconter au directeur.
Ce serait la fin du \emph{professeur de défense} et de son infortuné passage à Poudlard et… cela mettrait McGonagall en colère, mais…

(Une image de la professeure lui apparut dans un flash, elle n'avait pas l'air en colère, seulement triste…)

«~Maintenant, dites-lui qu'il vaut mieux que vous, Potter. » C'était la voix de Quirrell.

--- Tu vaux, mieux, que, moi.~»

Harry commença à se relever et Derrick lui mit le pied sur la poitrine et le repoussa sur le tapis.

Le monde devenait aussi transparent que du cristal.
Les actions possibles et leurs conséquences s'étiraient devant Harry avec une clarté absolue.
L'imbécile ne s'attendait pas à ce qu'il riposte, un rapide coup dans les parties l'étourdirait assez longtemps pour…

«~Essayez encore~», dit Quirrell, et d'un mouvement soudain et très rapide, Harry roula sur le tapis, bondit sur ses pieds et virevolta face à son véritable ennemi, le professeur de défense…

Quirrell dit~: «~Vous n'avez aucune patience.~»

Harry vacilla.
Son esprit, expert en pessimisme, lui dessina l'image d'un vieux sage rabougri, du sang s'écoulant de sa bouche après que Harry lui eut arraché la langue…

Un instant plus tard, Derrick poussa à nouveau Harry sur le tapis et s'assit sur lui, expulsant l'air de ses poumons.

«~Arrête~! hurla Harry.
S'il te plaît, arrête~!

--- Mieux, dit le professeur. Ça avait même l'air sincère.~»

Ça \emph{l'avait} été.
C'est ça qui était horrible, qui le rendait malade~: ça \emph{avait} été sincère.
Harry haletait, la peur et la colère froide se répandaient toutes deux en lui…

«~Perdez, dit Quirrell.

--- J'ai, perdu, parvint à dire Harry.

---J'aime bien ça, dit Derrick toujours juché sur lui.
Perds encore un peu.~»

\later

Des mains poussaient Harry, l'envoyant trébucher d'un bout à l'autre du cercle de Serpentard, jusqu'à une autre paire de mains qui le poussait à nouveau.
Cela faisait longtemps que Harry avait arrêté d'essayer de ne pas pleurer, et il essayait juste de ne pas tomber à présent.

«~T'es quoi, Potter~? dit Derrick.

--- Un loser, j'ai perdu, j'abandonne, vous avez gagné, vous êtes m-meilleurs, que moi, arrêtez s'il vous plaît…

Harry trébucha sur un pied et alla s'écraser au sol, ses mains ne parvenant pas tout à fait à le rattraper.
Il fut étourdi pendant un moment, puis tenta de se relever à nouveau…

--- \emph{Assez~!}~» fit la voix du professeur, suffisamment tranchante pour couper de l'acier.
«~Éloignez-vous de M. Potter~!~»

Harry vit l'air surpris sur leurs visages.
Un frisson dans ses veines, qui s'en était allé et venu, sourit avec une froide satisfaction.

Puis Harry s'effondra sur le tapis.

Quirrell parla, et on entendit s'étrangler les Serpentard plus âgés.

«~Et je crois que l'héritier de Malfoy voudrait aussi vous dire quelques mots~», conclut le professeur.

Drago commença à parler, d'une voix presque aussi tranchante que celle du professeur, elle avait la cadence que Drago utilisait pour imiter son père, et il disait des choses telles que \emph{auriez pu mettre en danger la Maison Serpentard} et \emph{qui sait combien d'alliés dans cette école} et \emph{absence totale de jugeote, sans parler de la ruse} et \emph{voyous stupides}, \emph{bons qu'à être des larbins} et quelque chose au fond du cerveau de Harry, en dépit de tout ce qu'il savait, désigna Drago comme étant un allié.

Harry avait mal partout, il était probablement plein de bleus, il avait froid, son esprit était complètement épuisé.
Il essaya de penser à la chanson de Fumseck, mais sans la présence du phénix, il n'arrivait pas à se souvenir de la mélodie, et quand il essaya de l'imaginer, il ne semblait pas capable de penser à autre chose qu'à un gazouillis d'oiseau.

Puis Drago se tut et Quirrell dit aux Serpentard plus âgés qu'ils pouvaient disposer, et Harry ouvrit les yeux et se mit assis non sans difficulté,
«~Attendez~», dit-il, forçant les mots à franchir ses lèvres, «~je voudrais leur dire quelque chose… à eux…

--- Attendez M. Potter~», dit froidement Quirrell aux Serpentard sur le pas de la porte.

Harry se mit debout en vacillant.
Il faisait attention à ne pas regarder en direction de ses camarades de classe.
Il ne voulait pas voir la façon dont ils le regardaient en ce moment.
Il ne voulait pas voir leur pitié.

Alors au lieu de cela, Harry regarda les Serpentard qui l'avaient malmené, et semblaient choqués.
Ils le fixèrent en retour, le visage plein d'appréhension.

Son côté obscur, durant les moments où il avait le contrôle, s'était accroché à l'image de cet instant et avait continué à faire semblant de perdre.

Harry commença~: «~Que personne ne…

--- Stop, dit Quirrell.
Si c'est ce que je pense que c'est, merci d'attendre après leur départ.
Ils en entendront parler plus tard.
Nous avons tous nos leçons à apprendre, M. Potter.

--- Très bien, dit Harry.

--- Vous. Filez.~»

Les Serpentard s'enfuirent et la porte se referma derrière eux.

«~Que personne ne cherche à se venger d'eux, dit Harry d'une voix rauque.
C'est une requête auprès de quiconque se considère mon ami.
J'avais une leçon à apprendre, et ils m'y ont aidé, ils avaient leur leçon à apprendre aussi, c'est fini.
Si vous racontez cette histoire, assurez-vous de raconter cette partie aussi.~»

Harry se tourna vers Quirrell.

«~Vous avez perdu~», dit Quirrell d'une voix qui pour la première fois était douce.
C'était étrange venant du professeur, comme si sa voix n'aurait pas dû être capable de faire cela.

Harry \emph{avait} perdu.
Il y avait eu des moments où la colère froide avait totalement disparu, remplacée par la peur, et pendant ces moments, il avait supplié les Serpentard, et il avait été sincère…

«~Et êtes-vous toujours en vie~?~»
demanda Quirrell, toujours avec cette étrange douceur.

Harry parvint à hocher la tête.

«~Toutes les défaites ne sont pas comme celle-là, dit le professeur.
Il y a des compromis et des capitulations négociées.
Il existe d'autres moyens de calmer les caïds.
C'est tout un art de manipuler les autres en les laissant être dominants.
Mais en premier lieu, la défaite doit être \emph{envisageable}.
Vous rappellerez-vous la façon dont vous avez perdu~?

--- Oui.

--- Serez-vous capable de perdre~?

--- Je… pense…

--- Je le pense aussi.~»
Quirrell s'inclina si bas que ses cheveux épars touchèrent presque le sol.
«~Félicitations M. Potter, vous avez gagné.~»

Il n'y eut pas de source unique, de premier à agir, l'applaudissement démarra à l'unisson comme un immense coup de tonnerre.

Harry était bouleversé, cela se voyait sur son visage.
Il risqua un regard vers ses camarades et il vit dans leurs yeux non pas de la pitié mais de l'admiration.
Les applaudissements venaient de Serdaigle et de Gryffondor et de Poufsouffle et même de Serpentard, probablement parce que Drago Malfoy applaudissait lui aussi.
Certains élèves se levaient de leurs chaises et la moitié de Gryffondor se tenait debout sur les pupitres.

Ainsi se trouvait Harry, chancelant, se laissant flotter dans la houle du respect qu'ils éprouvaient pour lui, se sentant plus fort et peut-être même un peu soigné.

Quirrell attendit que les applaudissements faiblissent.
Cela prit un bon moment.

«~Surpris, M. Potter~?~» dit Quirrell.
Il avait l'air amusé.
«~Vous venez de découvrir que le monde réel ne fonctionne pas \emph{toujours} comme dans vos pires cauchemars.
Oui, si vous aviez été un pauvre petit garçon anonyme en train de se faire maltraiter, alors ils vous auraient probablement respecté encore moins après cela.
Ils auraient eu même pitié de vous tandis qu'ils vous réconfortaient depuis leurs hautains perchoirs.
C'\emph{est} la nature humaine, j'en ai bien peur.
Mais \emph{vous}, qu'ils voyaient déjà comme une figure de pouvoir,
ils vous ont vu affronter votre peur, et continuer à lui faire face, même si vous auriez pu partir à n'importe quel moment.
Avez-vous eu une moindre opinion de \emph{moi} lorsque je vous ai raconté avoir délibérément enduré me faire cracher dessus~?~»

Harry sentit sa gorge se serrer et réprima frénétiquement sa sensation.
Il ne faisait pas suffisamment confiance à ce respect miraculeux pour se remettre à pleurer devant lui.

«~Votre réussite \emph{extraordinaire} pendant ce cours mérite une récompense extraordinaire, Harry Potter.
Merci de l'accepter avec mes compliments, au nom de ma Maison, et souvenez-vous à partir d'aujourd'hui que tous les Serpentard ne sont pas les mêmes.
Il y a des Serpentard, et il y a des Serpentard.~»
Quirrell souriait assez largement en disant cela.
«~Cinquante-et-un points pour Serdaigle.~»

Il y eut un silence choqué, puis un chaos indescriptible éclata chez les élèves de Serdaigle, qui hurlaient, sifflaient et acclamaient.

(Et au même moment, Harry sentit que quelque chose n'était pas \emph{correct}, McGonagall avait raison, il aurait \emph{dû} y avoir des conséquences, il aurait dû y avoir un prix à payer, on ne pouvait pas simplement remettre les choses à leur place comme avant…)

Mais Harry vit les visages transportés de joie des Serdaigle et il sut qu'il lui serait impossible de dire non.

Son cerveau fit une suggestion.
C'était une bonne suggestion.
Harry n'en croyait pas ses méninges que son cerveau soit encore d'aplomb, et encore moins qu'il puisse produire de bonnes suggestions.

«~Professeur~», dit Harry, aussi clairement qu'il le pouvait à travers sa gorge brûlante.
«~Vous êtes tout ce qu'un membre de votre Maison devrait être, et je pense que vous devez être exactement ce que Salazar Serpentard avait à l'esprit lorsqu'il participa à la fondation de Poudlard.
Je vous remercie, vous et votre Maison,~» Drago hochait la tête très doucement et faisait de petits cercles de son index, \emph{continue}, «~et je pense que cela mérite trois acclamations pour Serpentard.
Avec moi, tout le monde~?~»
Harry marqua une pause.
«~\emph{Hourrah}~!~»
Seule une minorité parvint à le rejoindre au premier essai.
«~\emph{Hourrah}~!~»
Cette fois la plupart des Serdaigle participa.
«~\emph{Hourrah}~!~»
C'était presque tout Serdaigle, quelques Poufsouffle éparpillés, et près d'un quart de Gryffondor.

La main de Drago se transforma en un pouce levé, léger et discret.

La plupart des Serpentard étaient sous le choc.
Quelques-uns fixaient Quirrell avec stupeur.
Blaise Zabini regardait Harry d'un air intrigué et calculateur.

Le professeur Quirrell s'inclina.
«~Merci à \emph{vous}, Harry Potter~», dit-il, gardant son large sourire.
Il se tourna vers la classe.
«~Et maintenant, croyez-le ou non, nous avons encore une demi-heure avant la fin du cours, et c'est assez pour présenter le bouclier simple.
M. Potter, bien sûr, est dispensé et va profiter d'un repos bien mérité.

--- Je peux…

--- Idiot~», dit affectueusement Quirrell.
La classe riait déjà.
«~Vos camarades pourront vous l'apprendre plus tard, ou je vous donnerai un cours particulier si cela s'avère nécessaire.
Mais \emph{maintenant}, vous allez sortir par la troisième porte à gauche à l'arrière de cette estrade, où vous trouverez un lit, un assortiment d'en-cas exceptionnellement délicieux, et quelques lectures extrêmement légères tirées de la bibliothèque de Poudlard.
N'emportez rien avec vous, et en particulier pas vos manuels scolaires.
Allez-y maintenant.~»

Harry s'en alla.
%  LocalWords:  raco Mornelithe Falconsbane
