\partchapter{Accomplissement de soi}{I}

\lettrinepara{L}{\emph{'hésitation}} \emph{est toujours facile, rarement utile.}

\hplettrineextrapara
Ainsi avait parlé le professeur de Défense, et même si Harry pouvait bien pinailler sur des détails du proverbe, il était suffisamment au fait des faiblesses Serdaigle pour savoir qu'il valait mieux essayer de \emph{répondre} à ses propres pinailleries.
Certains projets requéraient-ils qu'on attende~?
Oui, nombre d'entre eux nécessitaient que l'action soit \emph{différée}, mais ce n'était pas la même chose que \emph{d'hésiter} avant de \emph{choisir}.
Différer non pas parce qu'on savait quel était le bon moment pour faire ce qu'il y avait à faire mais parce qu'on ne pouvait pas se décider~: cela n'était nécessaire à aucun plan ingénieux.

Avait-on parfois besoin de plus d'information pour pouvoir faire un choix~?
Oui, mais cela aussi pouvait devenir une excuse pour différer~; et il serait \emph{tentant} de le faire, une fois face à un choix entre deux alternatives douloureuses et que \emph{l'absence de choix} permettrait temporairement d'éviter la douleur.
Alors on choisirait une information difficile à obtenir et on prétendrait qu'il était impossible de choisir sans~; de là l'excuse.
Mais si on savait de \emph{quelle} information on avait besoin, si on savait \emph{quand} et \emph{comment} on l'obtiendrait et ce qu'on \emph{ferait} dans chaque éventualité, alors le soupçon qu'il s'agissait d'une excuse pour pouvoir hésiter était moins fort.

Si vous \emph{n'étiez pas} juste en train d'hésiter, vous auriez dû pouvoir choisir \emph{à l'avance} ce que vous feriez une fois en possession des informations supplémentaires dont vous disiez avoir besoin.

Si le Seigneur des Ténèbres existait \emph{vraiment}, serait-il malin de suivre le plan du professeur Quirrell qui consistait à faire jouer le rôle du Seigneur des Ténèbres à un imposteur~?

Non. Certainement pas.
Absolument pas.

Et s'il savait \emph{de source sûre} que le Seigneur des Ténèbres \emph{n'existait plus}… alors dans \emph{ce} cas…

Le bureau du professeur de Défense était petit~; aujourd'hui en tout cas.
Il avait changé depuis la dernière visite de Harry, la pierre était devenue plus sombre, plus polie.
Derrière le bureau du professeur se tenait la seule bibliothèque qui avait toujours décoré la pièce, une grande bibliothèque qui s'étirait presque du sol au plafond et dotée de sept étagères de bois vides.
Il avait un jour vu le professeur Quirrell récupérer un livre de ces étagères vides mais ne l’y avait jamais vu en remettre un.

Le serpent vert ondulait au-dessus de la chaise placée derrière le bureau du professeur de Défense, ses yeux sans paupières braqués sans ciller vers Harry, presque au niveau des yeux de ce dernier.

Ils étaient à présent gardés par vingt-deux sortilèges, tous ceux qui pouvaient être lancés dans Poudlard sans attirer l'attention du directeur.

«~\parsel{Non~»}, siffla Harry.

Le serpent vert pencha la tête et l'inclina légèrement sur le côté~; ce geste ne colportait aucune émotion, du moins aucune que les talents de Fourchelangue de Harry pouvaient lui révéler.
«~\parsel{Raison de refuss~?}~» dit le serpent vert.

«~\parsel{Trop rissqué}~», dit simplement Harry.
C'était vrai, que le Seigneur des Ténèbres existe encore ou pas.
Se forcer à décider à l'avance lui avait permis de se rendre compte qu'il avait juste utilisé la question restée sans réponse comme une excuse pour hésiter~; la bonne décision était la même dans un cas comme dans l'autre.

Les yeux noirs et creusés du serpent semblèrent briller de noirceur l'espace d'un instant, et la bouche écailleuse s'ouvrit grand pour révéler les crochets.

«~\parsel{Pensse que tu as tiré mauvaise leççon d'échec passsé, garççon.
Mes plans pas habitués à échouer, et dernier sse sserait déroulé ssans accroc ssans ta bêtise.
Bonne leççon esst de ssuivre les insstructions données par ton Sserpentard pluss âgé et pluss ssage, de freiner tes pulssions ssauvages.}

--- \parsel{Leççon apprise esst de ne pas esssayer des plans qui donneraient à pensser à amie fille-enfant que je ssuis maléfique ou à ami garççon-enfant que je ssuis sstupide,}~» rétorqua Harry.
Il avait prévu une réponse plus hésitante que cela mais les mots lui avaient échappé.

Harry ne perçut pas le \parsel{sssssss} qui émana du serpent comme un mot mais comme de la rage pure.
Un instant plus tard~:

«~\parsel{Tu leur as dit…}

--- \parsel{Bien ssûr que non~!
Mais ssais çce qu'ils diraient.}~»

Il y eut un long silence et le serpent ondula tout en regardant Harry~; une fois de plus, aucune émotion détectable ne lui parvint et Harry se demanda à quoi pouvait bien penser le professeur Quirrell qui prenait si longtemps à cogiter.

«~\parsel{Tu te préoccupes ssérieusement de çce que çces deux-là penssent~?} dit le dernier sifflement du serpent.
\parsel{Cses deux-là ssont de vrais petiots, pas comme toi.
Ne pourraient ss'occuper d'affaires d'adultes.}

--- \parsel{Auraient peut-être mieux réusssi que moi}, siffla Harry.
\parsel{Ami garççon-enfant aurait interrogé sur dessseins ssecrets avant de conssentir à ssauver femme…}

--- \parsel{Heureux que tu comprennes çcela maintenant,} siffla le serpent d'un ton froid.
\parsel{Cherche toujours çce que l'autre a à gagner.
Puis apprends à toujours chercher ce que tu peux y gagner.
Ssi mon plan ne te convient pas, en quoi conssiste le tien~?}

--- \parsel{Ssi néçcesssaire -- resster à l'école ssix ans et étudier.
Poudlard ssemble bon endroit où ss'attarder.
Livres, amis, nourriture étrange mais goûtue.}~» Harry voulut glousser mais aucun geste Fourchelangue n'existait pour le genre de rire qu'il souhaitait exprimer.

Les crevasses des yeux du serpent étaient quasiment noires.
«~\parsel{Façcile à dire maintenant.
Cseux comme toi et moi ne tolèrent pas l'emprisonnement.
Tu perdras patiençce longtemps avant sseptième année, peut-être avant fin de çcelle-çci.
Je prévoirai en consséquençce.}~»

Et avant que Harry ne puisse siffler un mot de plus en Fourchelangue, la forme humaine du professeur était de nouveau dans sa chaise.
«~Alors, M. Potter~», dit le professeur de Défense, sa voix aussi calme que s'ils n'avaient discuté de rien d'importance, comme si toute la conversation n'avait pas eu lieu, «~j'entends dire que vous avez commencé à pratiquer le duel.
Pas la variante pour bons à rien dotée de \emph{règles}, j'espère~?~»

\later

Hannah Abbott avait l'air plus énervée qu'Hermione ne l'avait jamais vue (à part le jour du phénix, celui où Bellatrix Black s'était échappée, mais on n'aurait dû prendre ce jour-là en compte pour personne).
Pendant le dîner, la Poufsouffle s'était rendue jusqu'à la table Serdaigle, avait tapoté Hermione sur l'épaule et l'avait presque traînée à l'écart…

«~Neville et Harry Potter apprennent à se battre en duel avec M. Diggory~! lâcha Hannah dès qu'elles furent à quelques pas de la table.

--- Qui~? dit Hermione.

--- \emph{Cédric Diggory~!} dit Hannah.
C'est le capitaine de notre équipe de Quidditch, un général d'armée, il suit \emph{tous} les cours facultatifs, il a les meilleures notes, et j'ai entendu dire que des professeurs particuliers lui apprennent à se battre en duel chaque été, et un jour il a même battu \emph{deux} élèves de septième année, et même certains professeurs l'appellent le Super Poufsouffle, et le professeur Chourave dit qu'on devrait tous l'ému, euh, l'émuder ou quelque chose comme ça et…~»

Après que Hannah se fut enfin interrompue pour avaler un peu d'air (la liste avait continué pendant longtemps), Hermione parvint à placer un mot.

«~Soldat Soleil, Hannah Abbott~! dit Hermione.
On se \emph{calme}.
On ne va pas se battre contre le général Diggory, n'est-ce pas~?
D'accord, Neville s'entraîne pour nous battre, mais on peut étudier nous aussi…

--- Tu ne comprends pas~?~»
glapit Hannah, élevant la voix à un niveau supérieur à celui auquel elles auraient dû maintenir leur conversation si elles avaient voulu rester hors de portée d'oreille des Serdaigle qui les regardaient.
«~Neville n'étudie pas pour nous battre \emph{nous}~!
Il s'entraîne pour pouvoir se battre contre \emph{Bellatrix Black~!} Ils vont nous passer au travers comme un Cognard dans une pile de pancakes~!~»

Le général Soleil jeta un regard à son soldat.
«~Écoute, dit Hermione, je ne pense pas que quelques semaines de pratique puissent transformer qui que ce soit en un combattant invincible.
Et on \emph{sait} déjà gérer les combattants invincibles.
On concentrera notre feu sur lui et il tombera comme Drago l'a fait.~»

Le regard de la Poufsouffle était un mélange d'admiration et de scepticisme.

«~Tu n'es même pas, genre, \emph{inquiète}~?

--- Non mais \emph{franchement~!}~» dit Hermione.
Il était parfois difficile d'être la seule personne sensée de toute son année.
«~Tu n'as jamais entendu le proverbe~: la seule chose à craindre est la peur elle-même~?

--- \emph{Quoi~?} dit Hannah.
C'est n'importe quoi, et les Moremplis tapis dans les ténèbres, et être victime d'un Imperius, et les horribles accidents de Métamorphose, et…

--- Je \emph{voulais dire}~», continua Hermione alors que l'exaspération se laissait entendre dans sa voix maintenant devenue plus forte, car elle entendait ce genre de choses depuis le \emph{début de la semaine}, «~qu'on pourrait attendre \emph{d'avoir été} \emph{vraiment} \emph{écrasés} par la légion du Chaos avant de nous mettre à avoir aussi peur d'eux et \emph{est-ce que tu viens de marmonner “Gryffondor”~?}~»

Quelques instants plus tard, Hermione retournait à sa chaise avec un doux sourire gravé au fer rouge sur son jeune visage, ce n'était pas le terrible regard froid du côté obscur de Harry mais c'était l'expression la plus effrayante dont \emph{elle} était capable.

Harry Potter était \emph{mort}.

\later

«~C'est dingue~», s'étrangla Neville avec le peu de souffle qu'il put utiliser avant de ne plus en avoir.

«~C'est \emph{génial}~!~»
dit Cédric Diggory.
Les yeux du Super Poufsouffle étincelèrent avec un enthousiasme frénétique, aussi brillants que la sueur sur son front, alors que ses pieds tapaient le sol au rythme d'une de ses danses de duel.
Ses pas habituellement légers étaient devenus plus lourds, et ça avait peut-être un rapport avec les poids en métal métamorphosés qu'ils avaient tous attachés à leurs bras, leurs jambes et leur poitrine.
«~Où allez-vous pêcher de \emph{telles} idées, M. Potter~?

--- Un étrange vieux magasin… à Oxford… où je ne retournerai… plus… jamais.~»
\emph{Blam.}
%  LocalWords:  essitation ing emudate
