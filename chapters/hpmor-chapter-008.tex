% vim:spell:spelllang=fr

\chapter{Biais de confirmation}

\lettrine{P}{ersonne} n'avait demandé d'aide, c'était ça le problème.
Ils étaient juste restés là à parler, manger, ou regarder en l'air, pendant que leurs parents papotaient.
Pour quelque étrange raison que ce soit, personne ne s'était assis pour lire un livre, ce qui voulait dire qu'elle ne pouvait pas juste s'asseoir à côté d'eux et sortir son propre livre.
Et même après qu'elle eut pris l'audacieuse initiative de s'asseoir et continuer sa troisième lecture de \emph{L'Histoire de Poudlard}, personne n'avait semblé enclin à venir s'asseoir à côté d'elle.

À part aider les gens à faire leurs devoirs, ou autre chose dont ils auraient besoin, elle ne savait pas vraiment comment entrer en contact avec d'autres personnes.
Elle ne se \emph{sentait} pas particulièrement timide.
Elle se considérait comme une fille qui prend les choses en main.
Et pourtant, malgré cela, s'il n'y avait pas une requête du style «~Je n'arrive à me souvenir de comment on pose une division~», c'était juste trop \emph{bizarre} d'aller voir quelqu'un pour lui dire… quoi~?
Elle n'avait jamais pu comprendre quoi.
Et il ne semblait exister aucun mode d'emploi à ce propos, ce qui était ridicule.
Toutes ces histoires autour de “rencontrer des gens” ne lui avaient jamais parues très sensées.
Pourquoi était-ce à \emph{elle} d'endosser toute la responsabilité, alors qu'il y avait deux personnes impliquées~?
Pourquoi les adultes ne l'aidaient-ils jamais~?
Elle aurait tant voulu qu'une autre fille vienne \emph{la} voir et lui dise~: «~Hermione, le professeur m'a dit que je devais être amie avec toi~».

Mais soyons bien clairs~: Hermione Granger, assise toute seule le premier jour d'école dans l'un des rares compartiments qui étaient encore vides, dans le dernier wagon du train, la porte du compartiment laissée ouverte juste au cas où quelqu'un pour quelque raison aurait voulu lui parler, n'était \emph{pas} triste, morose, mélancolique, déprimée, désespérée, ou obsédée par ses problèmes.
Non, elle était au contraire en train de relire \emph{L'Histoire de Poudlard} pour la troisième fois et trouvait cela plutôt agréable, avec en arrière-pensée juste une pointe de contrariété envers la déraison générale du monde.

On entendit une porte inter-wagon s'ouvrir et se refermer, puis des pas et un étrange bruit qui \emph{rampait} sur le sol du couloir.
Hermione mit \emph{L'Histoire de Poudlard} de côté, se leva et jeta un coup d'œil dans le couloir -- juste au cas où quelqu'un aurait besoin d'aide -- et vit un jeune garçon en robe de sorcier, probablement en première ou deuxième année vu sa taille, l'air assez stupide avec une écharpe enroulée autour de la tête.
Une petite malle se tenait à ses côtés.
Au moment où elle l'aperçu, il frappait à la porte d'un autre compartiment, fermé, puis dit d'une voix légèrement étouffée par l'écharpe~: «~Excusez-moi, je peux vous poser une question rapide~?~»

Elle n'entendit pas la réponse venant de l'intérieur du compartiment, mais après que le garçon eut ouvert la porte, elle crut l'entendre dire -- à moins qu'elle ait mal compris -- «~Quelqu'un ici connaît-il les six quarks, ou bien là où je pourrais trouver une fille de première année nommée Hermione Granger~?~»

Après que le garçon eut refermé la porte du compartiment, Hermione demanda~: «~Je peux t'aider~?~»

Le visage voilé se tourna vers elle, et la voix répondit~:
«~Seulement si tu peux nommer les six quarks ou me dire où je peux trouver une fille de première année nommée Hermione Granger.

---  \emph{Up, down, strange, charm, truth, beauty}, et pourquoi la cherches-tu~?~»

C'était difficile d'être certaine à cette distance, mais il lui sembla que le garçon souriait largement sous son écharpe.
«~Ah, donc \emph{tu} es une fille de première année nommée Hermione Granger, dit la jeune voix étouffée.
Et dans le train pour Poudlard, rien que ça.~»
Le garçon commença à marcher vers elle et son compartiment, et sa malle rampa derrière lui.
«~Techniquement parlant, j'avais juste eu la consigne de te \emph{chercher}, mais il me semble probable que je suis aussi censé te parler ou t'inviter à rejoindre mon groupe ou obtenir de toi un objet magique essentiel ou découvrir que Poudlard a été construite sur les ruines d'un ancien temple ou quelque chose comme ça.
PJ ou PNJ, là est la question~?~»

Hermione ouvrit la bouche pour répondre, puis se rendit compte qu'elle était incapable d'imaginer une réponse \emph{possible} à…
ce qu'elle venait d'entrendre, quel que soit ce que \emph{c'était}, cependant que le garçon la rejoignait.
Il regarda à l'intérieur du compartiment, hocha la tête d'un air satisfait, et s'assit sur la banquette vide en face de la sienne.
Sa malle se précipita à sa suite, grandit jusqu'à atteindre trois fois sa taille initiale et se blottit contre celle d'Hermione d'une manière étrangement troublante.

«~Assieds-toi, je t'en prie, dit le garçon, et peux-tu fermer la porte derrière toi, s'il te plaît~?
Ne t'inquiètes pas, je ne mords personne qui ne m'a pas mordu en premier.~»
Il était déjà en train d'enlever l'écharpe d'autour de sa tête.

L'idée que ce garçon pensait qu'elle avait \emph{peur} de lui fit qu'elle envoya la porte coulissante se bloquer dans la cloison avec plus de force que nécessaire.
Elle se retourna vivement et se retrouva face à un visage jeune avec des yeux verts brillants et rieurs, ainsi qu'une vilaine cicatrice rouge foncé gravée sur le front, ce qui semblait lui rappeler quelque chose dans un coin de sa tête, mais pour le moment elle avait d'autres préoccupations.
«~Je n'ai pas dit que j'étais Hermione Granger~!

--- \emph{Je} n'ai pas dit que tu avais \emph{dis} que tu étais Hermione Granger, j'ai juste dit que tu étais Hermione Granger.
Si tu veux savoir comment je le sais, c'est parce que je sais tout.
Bonsoir mesdames et messieurs, mon nom est Harry James Potter-Evans-Verres ou Harry Potter pour faire plus court, je sais qu'au moins pour \emph{toi}, cela ne te dira probablement rien…

Le cerveau de Hermione fit enfin le rapprochement.
La cicatrice sur son front, la forme en éclair.
«~Harry Potter~!  Tu es dans \emph{Histoire de la magie moderne}, dans \emph{Grandeur et décadence de la magie noire}, et dans \emph{Les Grands Évènements de la sorcellerie au XX\textsuperscript{e} siècle}.~»
C'était à vrai dire la première fois de toute sa vie qu'elle \emph{rencontrait} quelqu'un qui était dans un \emph{livre}, et c'était une sensation plutôt étrange.

Le garçon cligna trois fois des yeux.
«~Je suis dans des \emph{livres}~? Attends, bien sûr que je suis dans des livres… en voilà une pensée étrange.

--- C'est pas vrai, tu ne savais pas~? dit Hermione. J'aurais cherché tout ce que je pouvais si ça avait été moi.~»

Le garçon répondit assez sèchement. «~Mademoiselle Granger, cela fait moins de 72 heures que j'ai visité le Chemin de Traverse et découvert que j'étais célèbre.
J'ai passé les deux derniers jours à acheter des livres de science.
\emph{Crois-moi}, je compte bien chercher et trouver tout ce que je peux.
Le garçon hésita.
Qu'est-ce que les livres disent sur moi~?~»

L'esprit d'Hermione Granger fit un bond en arrière, elle ne s'était pas rendu compte qu'on l'interrogerait sur \emph{ces} livres donc elle ne les avait lus qu'une seule fois, mais c'était il y a seulement un mois donc le contenu était encore frais dans sa tête.
«~Tu es le seul à avoir survécu au sortilège de la Mort et tu es donc appelé le Survivant.
Tu es l'enfant de James Potter et Lily Potter (nom de jeune fille Evans), né le 31 juillet 1980.
Le 31 octobre 1981 le Seigneur des Ténèbres Celui-Dont-On-Ne-Doit-Pas-Prononcer-Le-Nom même si je n'ai pas compris pourquoi a attaqué ta maison.
Tu as été retrouvé en vie avec une cicatrice sur le front au milieu des ruines de la maison de tes parents non loin des restes calcinés du corps de Tu-Sais-Qui.
Le président-sorcier du Magenmagot Albus Percival Wulfric Brian Dumbledore t'a envoyé quelque part, personne ne sait où.
\emph{Grandeur et décadence de la magie noire} prétend que tu as survécu grâce à l'amour de ta mère, que ta cicatrice contient tous les pouvoirs du Seigneur des Ténèbres et que les centaures te craignent, mais \emph{Les Grands Évènements de la sorcellerie au XX\textsuperscript{e} siècle} ne mentionne rien de tel et \emph{Histoire de la magie moderne} prévient qu'il existe beaucoup de théories fumeuses à ton sujet.~»

La mâchoire du garçon pendait béante.
«~Est-ce qu'on t'a dit d'attendre Harry Potter dans le train pour Poudlard, ou quelque chose dans le genre~?

--- Non, dit Hermione. Qui t'a parlé de \emph{moi}~?

--- La professeure McGonagall, et je crois que je comprends pourquoi. Est-ce que tu as une mémoire eidétique, Hermione~?~»

Hermione secoua la tête.
«~Elle n'est pas photographique, j'ai toujours rêvé qu'elle le soit, mais j'ai dû lire mes manuels scolaires cinq fois avant de les avoir tous mémorisés.

--- Vraiment, s'étrangla légèrement le garçon.
J'espère que ça ne te dérange pas si je teste ça -- ce n'est pas que je ne te crois pas, mais comme dit le proverbe~: “Fais confiance, mais vérifie.”
Aucune raison de laisser le doute planer quand je peux juste faire l'expérience.~»

Hermione sourit, un peu arrogante.
Elle adorait les tests. «~Vas-y.~»

Le garçon mit une main dans la bourse qu'il portait sur le côté et dit «~\emph{Potions magiques} d'Arsenius Beaulitron.~»
Lorsqu'il retira sa main, il tenait le livre qu'il venait de nommer.

Instantanément, Hermione voulut posséder l'une de ces bourses plus qu'elle n'avait jamais voulu autre chose.

Le garçon ouvrit le livre au hasard et lut.
«~Si tu faisais de \emph{l'huile d'affûtage}…

--- Je peux \emph{voir} la page d'ici, tu sais~!~»

Le garçon inclina le livre pour qu'elle ne puisse plus voir, et tourna à nouveau les pages.

«~Si tu préparais une \emph{potion d'escalade araignée}, quel serait l'ingrédient à ajouter après la soie d'Acromentule~?

--- Après avoir mis la soie, attendre jusqu'à ce que la potion ait pris exactement la teinte du ciel sans nuage à l'aube, 8 degrés au dessus de l'horizon et 8 minutes avant que le haut du soleil ne devienne visible.
Tourner huit fois dans le sens anti-horaire et une fois dans le sens horaire, puis ajouter huit larmes de crotte de nez de licorne.~»

Le garçon referma le livre d'un coup sec et le remit dans sa bourse, qui l'avala en faisant un petit rot.
«~Bien bien bien \emph{bien} bien bien. Je voudrais vous faire une proposition, Mademoiselle Granger.

--- Une proposition~?~» dit Hermione avec méfiance.
Les filles n'étaient pas sensées répondre à ce genre de choses.

C'est aussi à ce moment qu'Hermione se rendit compte de l'autre chose -- enfin, une des choses -- qui était étrange chez ce garçon.
Apparemment, les gens qui étaient \emph{dans} des livres \emph{parlaient} aussi comme des livres.
C'était une découverte pour le moins surprenante.

Le garçon mit la main dans sa bourse et dit «~canette de soda~», et récupéra un cylindre vert fluo.
Il le lui tendit et dit~: «~Puis-je t'offrir quelque chose à boire~?~»

Hermione accepta le soda poliment.
À vrai dire elle \emph{avait} un peu soif à présent.
«~Merci beaucoup, dit Hermione alors qu'elle décapsulait la canette. C'était ça ta proposition~?~»

Le garçon toussa. «~Non~», dit-il. Et juste quand Hermione commença à boire, il compléta~: «~Je voudrais que tu m'aides à conquérir l'univers.~»

Hermione finit de boire et abaissa la canette. «~Non merci, je ne suis pas maléfique.~»

Le garçon la regarda avec surprise, comme s'il s'était attendu à une autre réponse.
«~Eh bien, je parlais un peu rhétoriquement, dit-il.
Au sens du projet baconien, tu sais, pas du pouvoir politique.
“La réalisation de toutes les choses possibles,” et ainsi de suite.
Je veux mener des études expérimentales sur les sortilèges, comprendre les lois sous-jacentes, amener la magie dans le domaine de la science, fusionner les mondes magique et Moldu, élever le niveau de vie de toute la planète, faire avancer l'humanité de plusieurs siècles, découvrir le secret de l'immortalité, coloniser le système solaire, explorer la galaxie, et, le plus important, comprendre ce que c'est que tout ce bazar, parce que tout cela est manifestement impossible.~»

Cela avait l'air un peu plus intéressant. «~Et~?~»

Le garçon la regarda avec incrédulité.

«~\emph{Et}~? Ce n'est pas \emph{assez}~?

--- Et qu'est-ce que tu veux de moi~? dit Hermione.

--- Je veux que tu m'aides dans mes recherches, bien sûr.
Avec ta mémoire encyclopédique associée à mon intelligence et à ma rationalité, nous aurons fini le projet baconien en un rien de temps, et par “un rien de temps” je veux dire probablement au moins trente-cinq ans.~»

Hermione commençait à trouver ce garçon agaçant.
«~Je ne t'ai rien vu faire d'intelligent.
Peut-être que \emph{je} te laisserai m'aider dans \emph{mes} recherches.~»

Le compartiment était à présent assurément silencieux.

«~Donc, tu me demandes de démontrer mon intelligence~», fit le garçon après une longue pause.

Hermione acquiesça.

«~Je te préviens que défier mon ingénuité est un projet dangereux, qui a tendance à rendre la vie assez surréelle.

--- Je ne suis toujours pas impressionnée~», dit Hermione.  Elle porta à ses lèvres la cannette de soda vert fluo.

«~Bien, peut-être que \emph{ceci} t'impressionnera~», dit le garçon.
Il se pencha en avant et la regarda avec intensité.
«~J'ai déjà fait quelques expériences et je me suis rendu compte que je n'ai pas besoin de baguette, je peux faire survenir ce que je veux juste en claquant des doigts.~»

Il le dit juste au moment où Hermione était en train de déglutir, et elle s'étouffa, toussa et expulsa le liquide vert fluo.

Sur sa robe de sorcière neuve, encore jamais portée, le tout premier jour d'école.

Hermione cria. Vraiment. C'était un son aigu qui, dans le compartiment fermé, ressemblait à une sirène d'alerte.

«~Aaaaaah~! Mes vêtements~!

--- Pas de panique~! dit le garçon. Je peux arranger ça. Regarde~!~»
Il leva une main et claqua des doigts.

«~Tu vas…~» puis elle baissa les yeux pour regarder ses vêtements.

Le liquide vert était encore là, mais alors même qu'elle le regardait, il commença à disparaître, à s'effacer, et en seulement quelques instants c'était comme si elle ne s'était jamais rien renversé dessus.

Hermione fixa le garçon qui arborait à présent un sourire, satisfait de lui.

De la magie silencieuse sans baguette~!  À \emph{son} âge~? Alors qu'il avait obtenu ses manuels scolaire seulement \emph{trois jours} plus tôt~?

Puis elle se souvint de ce qu'elle avait lu, et eut un mouvement de recul.
\emph{Tout le pouvoir du Seigneur des Ténèbres~! Dans sa cicatrice~!}

Elle se leva hâtivement. «~Je, je, j'ai besoin d'aller aux toilettes, attends-moi ici…~» il fallait qu'elle trouve un adulte il fallait qu'elle leur dise…

Le sourire du garçon disparu. «~Je t'ai juste joué un tour, Hermione. Je suis désolé, je ne voulais pas te faire peur.~»

Hermione s'arrêta, la main sur la poignée de la porte.  «~Un \emph{tour}~?

--- Oui, dit le garçon.
Tu m'as demandé de démontrer mon intelligence.
Donc j'ai fait quelque chose d'apparemment impossible, ce qui est toujours une bonne manière de frimer.
Je ne peux pas \emph{réellement} faire tout ce que je veux juste en claquant des doigts.
Le garçon s'interrompit.
Du moins, je ne \emph{pense} pas pouvoir le faire, je n'ai jamais vraiment essayé.~»
Le garçon leva sa main et claqua à nouveau des doigts.
«~Eh non, pas de banane.~»

Hermione n'avait jamais été aussi déconcertée de toute sa vie.

Le garçon sourit à nouveau en voyant la tête que faisait Hermione.
«~Je t'avais bien \emph{prévenue} que défier mon ingénuité risquait de rendre ta vie surréaliste.
J'espère que tu t'en souviendras la prochaine fois que je te mettrai en garde.

--- Mais, mais, balbutia Hermione. Qu'est-ce que tu as \emph{fait} alors~?~»

Le garçon se mit à l'observer d'un œil calculateur comme jamais elle ne l'avait vu chez quelqu'un de son âge.
«~Tu penses pouvoir être une scientifique à toi toute seule, avec ou sans mon aide~?
Alors voyons comment \emph{tu} enquêtes sur un phénomène déroutant.

--- Je…~» Le cerveau d'Hermione bloqua un instant.
Elle adorait les tests, mais jamais on ne l'avait testée \emph{ainsi} auparavant.
Elle essaya frénétiquement de passer en revue tout ce qu'elle pouvait avoir lu sur ce que les scientifiques étaient censés faire.
Son cerveau sauta quelques vitesses, trouva quelque chose à moudre, et cracha les instructions nécessaires à la réalisation d'un projet scientifique à d'école primaire~:

\emph{\\
Étape~1~: Formuler une hypothèse.\\
Étape~2~: Faire une expérience pour tester l'hypothèse.\\
Étape~3~: Évaluer les résultats.\\
Étape~4~: Faire un poster en carton.\\
}

L'étape 1 était de formuler une hypothèse.
C'est-à-dire, essayer de penser à quelque chose qui \emph{pourrait} avoir eu lieu à l'instant.

«~Très bien. Mon hypothèse est que tu as enchanté ma robe pour faire disparaître tout ce qui pourrait couler dessus.

--- Très bien, dit le garçon, c'est ta réponse~?~»

Le choc se dissipait et le cerveau d'Hermione recommençait à fonctionner correctement.
«~Attends, ce n'est pas possible.
Je ne t'ai pas vu toucher ta baguette ni prononcer le moindre sortilège, alors comment aurais-tu pu faire un enchantement~?~»

Le garçon attendait, le visage neutre.

«~Mais imaginons que toutes les robes du magasin soient \emph{déjà} enchantées pour les garder propres~; ce serait un enchantement très utile pour des vêtements.
Tu as découvert cela en renversant \emph{toi-même} quelque chose sur ta robe.~»

Le garçon souleva les sourcils.
«~\emph{Est-ce} ta réponse~?

--- Non, je n'ai pas fait l'étape 2, “Faire une expérience pour tester l'hypothèse.”~»

Le garçon referma la bouche et commença à sourire.

Hermione regarda la canette de soda qu'elle avait automatiquement placée dans le porte-canette près de la fenêtre.
Elle l'attrapa, regarda à l'intérieur et vit qu'il en restait environ un tiers.

«~Bon, dit Hermione, l'expérience que je veux faire est d'en verser sur ma robe pour voir ce qui se passe, et ma prédiction est que la tache disparaîtra.
Seulement si ça ne marche \emph{pas}, ma robe sera tachée, et je n'ai pas envie que ça arrive.

--- Fais-le sur la mienne, dit le garçon, comme ça tu n'as pas à t'inquiéter que ta robe soit tachée.

--- Mais…~» dit Hermione. Il y avait quelque chose qui \emph{clochait} avec cette façon de penser, mais elle ne savait pas comment le formuler précisément.

«~J'ai des robes de rechange dans ma malle, dit le garçon.

--- Mais tu n'as nulle part où te changer~», objecta Hermione. Puis elle révisa son opinion. «~En même temps, je suppose que je pourrais sortir et fermer la porte…

--- J'ai aussi un endroit où me changer dans ma malle.~»

Hermione regarda la malle, qui, elle commençait à le soupçonner, était assez particulière comparée à la sienne.

«~Très bien, dit Hermione, puisque tu le proposes~», et elle versa délicatement un peu de soda vert sur un bord de la robe du garçon.
Puis elle fixa la tache du regard, essayant de se rappeler combien de temps le liquide avait mit à disparaître sur la sienne.~»

Et la tache verte disparut~!

Hermione poussa un soupir de soulagement, cela signifiait notamment qu'elle n'était pas face à tous les pouvoirs magiques du Seigneur des Ténèbres.

Eh bien, l'étape 3 était d'évaluer les résultats, mais dans ce cas cela consistait juste à voir que la tache avait disparu.
Et elle supposait qu'elle pouvait sauter l'étape 4, celle du poster.
«~Ma réponse est que les robes sont ensorcelées pour rester propre.

--- Pas vraiment~», dit le garçon.

La déception fut comme un coup de poignard.
Hermione aurait vraiment voulu ne \emph{pas} ressentir cela, le garçon n'était pas un professeur, mais c'était quand même un test et elle avait mal répondu à une question, ce qui lui faisait toujours la sensation de recevoir un coup de poing dans le ventre.

(Le fait qu'elle n'ait jamais laissé cela l'arrêter ni même interférer dans son amour pour les tests vous disait presque tout ce que vous aviez besoin de savoir sur Hermione Granger).

«~Ce qui est triste, dit le garçon, c'est que tu as probablement fait exactement ce que ton livre dit qu'il faut faire.
Tu as fait une prédiction qui distinguerait entre une robe enchantée et une robe non enchanté, tu l'as testée, et tu as rejeté l'hypothèse nulle, à savoir que la robe n'était pas enchantée.
Mais à moins de lire les meilleurs des meilleurs, les livres ne t'apprendront pas à faire de la science \emph{correctement}.
Je veux dire suffisamment bien pour \emph{vraiment} trouver la bonne réponse, et pas juste pondre une de ces publications supplémentaires dont papa se plaint toujours.
Alors laisse-moi essayer d'expliquer -- sans te donner la réponse -- l'erreur que tu viens de faire, et je te donnerai une autre chance.~»

Elle commençait à être irritée par le ton oh-si-supérieur du garçon, alors qu'il n'était qu'un enfant de onze ans comme elle, mais cela était secondaire : il fallait qu'elle sache pourquoi elle s'était trompée.
«~D'accord.~»

Le regard du garçon se fit plus intense.
«~Voici un jeu basé sur une expérience célèbre nommée la tâche “2-4-6.”
J'ai une \emph{règle} -- que je connais, mais pas toi -- que certains triplets de nombres respectent, mais pas d'autres.
2-4-6 est un exemple de triplet qui respecte la règle.
D'ailleurs… je vais écrire la règle, comme ça tu sauras qu'elle est figée, et te la donner pliée en deux.
Ne regarde pas s'il te plaît, puisque tu m'as montré tout à l'heure que tu sais lire à l'envers.~»

Le garçon dit «~papier~» et «~porte-mine~» à sa bourse, et elle ferma les yeux pendant qu'il écrivait.

«~Voilà~», dit le garçon, lui tendant un morceau de papier plié en quatre. «~Mets ça dans ta poche~», et elle s'exécuta.

«~Maintenant, voilà comment le jeu fonctionne, dit le garçon. Tu me donnes un triplet de nombres, et je te dis “oui” si les trois nombres respectent la règle, et “non” s'ils ne la respectent pas.
Je suis la Nature, la règle est une de mes lois, et tu es en train de m'étudier.
Tu sais déjà que 2-4-6 donne un “oui.”
Une fois que tu as effectué tous les tests supplémentaires que tu souhaites -- que tu m'as soumis autant de triplets que tu juges nécessaire -- tu peux t'arrêter pour essayer de deviner la règle.
À ce moment, tu as le droit de déplier la feuille pour voir si tu as réussi.
Tu as compris le jeu~?

--- Bien sûr que j'ai compris, dit Hermione.

--- C'est parti.

--- 4-6-8 dit Hermione.

--- Oui, dit le garçon.

--- 10-12-14, dit Hermione.

--- Oui, dit le garçon.~»

Hermione essaya de pousser son esprit un peu plus loin. Il lui semblait qu'elle avait fait tous les tests dont elle avait besoin, et pourtant ça ne pouvait pas être aussi facile, non~?

«~1-3-5.

--- Oui.

--- Moins 3, moins 1, plus 1.

--- Oui.~»

Hermione ne voyait rien de plus à faire.
«~La règle est que les nombres doivent augmenter de deux à chaque fois.

--- Maintenant suppose que je te dise, dit le garçon, que ce test est plus difficile qu'il n'y paraît, et que seulement 20~\% des adultes le réussissent.~»

Hermione fronça les sourcils. Qu'est-ce qu'elle avait loupé~?
Puis, soudain, elle pensa à un test qu'elle avait encore besoin de faire.

«~2-5-8~! dit-elle triomphante.

--- Oui.

--- 10-20-30~!

--- Oui.

--- La vraie réponse est que les nombres doivent augmenter de la \emph{même} quantité à chaque fois. Ça n'a pas besoin d'être 2.

--- Très bien, dit le garçon, prends le papier et regarde si tu as réussi.~»

Hermione sortit le papier de sa poche et le déplia.

\emph{Trois nombres réels en ordre croissant, du plus petit au plus grand.}

Hermione resta bouche bée.
Elle ressentait un profond sentiment d'injustice, que le garçon était un sale petit menteur pourri et tricheur, mais lorsqu'elle rejoua la scène dans sa tête elle ne put rien trouver qui le mette en défaut dans les réponses qu'il avait données.

«~Tu viens de découvrir le “biais de confirmation,” dit le garçon.
Tu avais une règle en tête, et tu n'as fait que à penser à des triplets qui feraient dire “oui” à la règle.
Mais tu n'as pas essayé de tester un seul triplet qui lui aurait fait dire “non.”
En fait, tu n'as eu \emph{aucun} “non,” donc la règle aurait tout aussi bien pu être simplement “trois nombres, n'importe lesquels.”
C'est un peu comme ces gens qui imaginent des expériences pour confirmer leurs hypothèses au lieu d'essayer d'imaginer des expériences qui tentent de les infirmer -- ce n'est pas exactement la même erreur, mais presque.
Il faut apprendre à voir le côté négatif, regarder la partie sombre.
Lorsqu'on fait cette expérience, seuls 20~\% des adultes trouvent la bonne réponse.
Et la plupart des autres inventent des hypothèses fantastiquement compliquées dans lesquelles ils ont parfaitement confiance, après avoir fait tant expériences qui se sont déroulés comme ils s'y attendaient.

--- Maintenant, continua le garçon, veux-tu te frotter à nouveau au problème initial~?~»

Il la regardait très attentivement à présent, comme si c'était là le \emph{vrai} test.

Hermione ferma les yeux et essaya de se concentrer.
Elle transpirait sous sa robe.
Elle avait la curieuse sensation qu'on ne lui avait jamais demandé de réfléchir autant pour un test, peut-être même que c'était la \emph{première} fois qu'on lui avait jamais demandé de réfléchir pour un test.

Quelle autre expérience pouvait-elle réaliser~?
Elle avait une Chocogrenouille, elle pourrait essayer de la frotter un peu sur la robe et voir si \emph{ça} disparaissait~?
Mais ça n'était pas tout à fait une pensée négative et tortueuse comme le garçon semblait attendre d'elle.
En fait elle s'attendait encore à un “oui,” que la tache de Chocogrenouille disparaisse, alors qu'elle aurait dû chercher à obtenir un “non.”

Donc… selon son hypothèse… dans quelle condition le soda devrait-il… ne \emph{pas} disparaître~?

«~J'ai une expérience à faire, dit Hermione.
Je vais renverser un peu de soda par terre, et voir s'il ne \emph{disparaît pas}.
Est-ce que tu as du papier essuie-tout dans ta bourse, pour que je puisse éponger si ça ne marche pas~?

--- J'ai des serviettes de table~», dit le garçon, le visage toujours neutre.

Hermione prit la canette, et versa un peu de soda sur le sol.

Quelques secondes plus tard, il disparut.

C'est alors qu'elle réalisa, et fut prise d'une furieuse envie de se frapper:
«~Bien sûr~! C'est \emph{toi } qui m'as donné le soda~!
Ce n'est pas la robe qui est ensorcelée, c'était le soda depuis le début!~»

Le garçon se leva et s'inclina solennellement, un large sourire aux lèvres.
«~Dans ce cas… pourrais-je vous aider dans vos recherches, Hermione Granger?

--- Je, euh…~» Hermione était encore sous le coup l'euphorie, mais elle ne savait pas bien comment répondre à \emph{cela}.

Ils furent interrompus par un léger coup frappé à la porte, faible, hésitant, voire \emph{réticent}.

Le garçon se détourna, regarda par la fenêtre, et dit~: «~J'ai enlevé mon écharpe, tu peux répondre~?~»

C'est à ce moment-là qu'Hermione comprit pourquoi le garçon -- non, le garçon-qui-avait-survécu, le Survivant, Harry Potter -- se promenait une écharpe sur la tête, et elle se sentit un peu stupide de ne pas s'en être rendu compte plus tôt.
C'était en fait assez étrange, car elle aurait cru que Harry Potter était le genre de garçon qui se montrerait fièrement au monde entier~; et l'idée lui vint qu'il pourrait en réalité être plus timide qu'il n'en avait l'air.

Lorsqu'Hermione ouvrit la porte, elle fut accueillie par un jeune garçon tremblant qui ressemblait exactement à la façon dont il avait frappé à la porte.

«~Excuse-moi, dit le garçon d'une toute petite voix, je suis Neville Londubat. Je cherche mon crapeau de compagnie, je, je ne l'ai vu nulle part dans ce wagon… tu aurais vu mon crapeau~?

--- Non~», dit Hermione, mais sa nature serviable prit automatiquement les commandes.
«~As-tu regardé dans les autres compartiments~?

--- Oui, murmura le garçon.

--- Alors il suffit de vérifier tous les autres wagons, dit vivement Hermione.
Je vais t'aider.
Au fait, je m'appelle Hermione Granger.~»

Le garçon semblait sur le point de s'évanouir de gratitude.

«~Une seconde~», s'était la voix de \emph{l'autre} garçon -- Harry Potter.
«~Je ne suis pas certain que ce soit la meilleure façon de procéder.~»

Neville était au bord des larmes, et Hermione fit volte-face, énervée.
Si Harry Potter était le genre de personne prête à abandonner un petit garçon juste parce qu'il n'aimait pas être interrompu…
«~Comment ça~? Pourquoi \emph{pas}~?

--- Eh bien, dit Harry Potter, ça va prendre beaucoup de temps de vérifier tout le train manuellement, et on pourrait quand même rater le crapeau.
Et si on ne le trouve pas avant d'arriver à Poudlard, Neville aura des ennuis.
Ça serait beaucoup plus sensé d'aller directement à la voiture de tête, où se trouvent les préfets, et de demander leur aide directement.
C'est la première chose que j'ai faite quand je t'ai cherchée, Hermione, mais ils ne savaient pas.
Par contre, ils ont peut-être des sorts ou des objets magiques qui faciliteraient grandement la recherche d'un crapeau.
On n'est que des première année.~»

Cette idée… \emph{était} très sensée.

«~Penses-tu pouvoir te rendre tout seul à la voiture des préfets?  demanda Harry Potter.
J'ai quelques petites raisons de vouloir ne pas trop montrer mon visage.~»

Neville s'étrangla soudain et fit un pas en arrière.
«~Je me souviens de cette voix~!
Tu es l'un des Seigneurs du Chaos~!
\emph{Tu es celui qui m'a donné du chocolat!}~»

Quoi~? Quoi quoi \emph{quoi}~?

Harry Potter se détourna de la fenêtre et se leva dans un élan théâtral. «~Moi, \emph{jamais~!} s'indigna-t-il.
Ai-je l'air d'être un de ces malfaiteurs qui donnerait des bonbons à un enfant?~»

Neville écarquilla les yeux.
«~\emph{Tu} es Harry Potter~? \emph{Le} Harry Potter~? \emph{Toi}~?

--- Non, juste \emph{un} Harry Potter, nous sommes trois à bord de ce train…~»

Neville poussa un petit cri aigu et s'enfuit.
On entendit s'éloigner des pas affolés, suivis du son d'une porte de wagon s'ouvrant et se refermant.

Hermione se laissa tomber sur le banc.
Harry Potter ferma la porte et s'assit à côté d'elle.

«~Pourrais-tu s'il te plaît m'expliquer ce qui se passe~?~» dit Hermione d'une voix faible.
Elle se demandait si traîner avec Harry Potter impliquait d'être en permanence autant déboussolée.

«~Oh, eh bien ce qui s'est passé c'est que Fred, George et moi avons vu ce pauvre petit garçon à la gare -- la femme qui l'accompagnait s'était éloignée un moment, et il avait l'air vraiment effrayé, comme s'il s'attendait à se faire attaquer par des Mangemorts ou quelque chose comme ça.
Tu connais ce dicton qui dit que la peur est souvent pire que l'objet de la peur, donc je me suis dit que ça serait bénéfique pour ce gamin que son pire cauchemar devienne réalité, et qu'il se rendre compte que ce n'était finalement pas aussi terrible que ce qu'il craignait…~»

Hermione se tenait assise, la bouche grande ouverte.

«…et Fred et George ont lancé ce sort pour que nos écharpes sur nos visages s'assombrissent et deviennent floues, comme si nous étions des rois morts-vivants dans des linceuls…~»

Elle n'aimait pas du tout la tournure de cette histoire.

«…et une fois qu'on lui a donné tous les bonbons que j'avais achetés, on s'est mis à dire: “Donnons-lui de l'argent~! Ha ha ha~! Voilà des noises, garçon~! Prends une mornille d'argent~!”
tout en dansant autour de lui et en riant d'un air maléfique.
Je pense qu'il y avait quelques personnes autour qui voulaient intervenir, mais l'effet du témoin les a retenues assez de temps pour qu'elles voient ce qu'on faisait, et ensuite elles étaient bien trop déconcertées pour faire quoi que ce soit.
Il a fini par murmurer “allez-vous en” d'une toute petite voix, c'est là qu'on est partis tous les trois en hurlant, comme quoi la lumière nous brûlait ou quelque chose du genre.
Avec un peu de chance, il aura moins peur de se faire harceler à l'avenir.
Ça s'appelle la thérapie par désensibilisation, au fait.~»

Ok, la tournure de cette histoire était \emph{très différente} que ce qu'elle avait imaginé.

% STOP HERE
Le feu d'indignation brûlante qui était l'un des moteurs principaux de Hermione s'éveilla en vrombissant, même si une part d'elle \emph{voyait bien} ce qu'il avait essayé de faire.

«~C'est affreux~! \emph{Tu} es affreux~! Ce pauvre garçon~! Tu as fait quelque chose de \emph{méchant}~!

--- Je pense que le mot que tu cherches est \emph{amusant}, et quoi qu'il en soit tu poses la mauvaise question. La question est~: cela a-t-il fait plus de bien que de mal, ou plus de mal que de bien~? Si tu as le moindre argument en rapport avec \emph{cette} question, je serai heureux de l'entendre, mais je n'accepterai aucune autre critique tant que \emph{celle-}ci n'aura pas été réglée. Je suis tout à fait d'accord, ce que j'ai fait \emph{a l'air} affreux, méchant, et brutalisant, vu qu'un petit garçon effrayé est concerné et ainsi de suite, mais ce n'est certainement pas le problème clé, non~? Au fait, ça s'appelle le \emph{conséquentialisme}, et ça signifie qu'un acte n'est pas bon ou mauvais parce qu'il a \emph{l'air} bon ou mauvais, ou autre chose du genre~; la seule question est celle du résultat final -- quelles sont les conséquences.~»

Hermione ouvrit la bouche pour dire quelque chose de \emph{cinglant}, mais il semblait malheureusement qu'elle avait négligé l'étape où elle pensait à quelque chose à dire avant d'ouvrir la bouche. Tout ce qu'elle trouva fut~:

«~Et s'il a des \emph{cauchemars}~?

--- Honnêtement je ne pense pas qu'il avait besoin de notre aide pour avoir des cauchemars, et s'il a des cauchemars à propos de \emph{ça}, alors ce sera des cauchemars où les monstres vous donnent du chocolat et \emph{ça} c'était notre but \emph{initial}.~»

Le cerveau de Hermione hoquetait de confusion à chaque fois qu'elle essayait de se mettre correctement en colère. «~Ta vie est-elle toujours si inhabituelle~?~» dit-elle enfin.

Le visage de Harry Potter rayonna de fierté.

«~Je la \emph{rends} inhabituelle. Tu observes le résultat de beaucoup de travail et d'huile de coude.

--- Donc…~» dit Hermione, et elle se tut avec maladresse.

«~Donc, dit Harry Potter, quelle est l'étendue exacte de tes connaissances scientifiques~? Je sais résoudre des équations, je connais un peu de théorie de probabilité Bayésienne et de théorie de la décision et beaucoup de science cognitive, et j'ai lu le \emph{Cours de Physique de Feynman} (du moins le volume 1) et \emph{Jugement dans l'Incertitude~: Heuristiques et Biais} et \emph{Langage dans la Pensée et l'Action} et \emph{Influence et Manipulation} et \emph{Choix Rationnel dans un Monde Incertain} et \emph{Gödel, Escher, Bach} et \emph{Un pas plus loin} et…~»

Le quiz et le contre-quiz qui suivirent durèrent plusieurs minutes avant d'être interrompus par un autre coup timide frappé à la porte. «~Entrez~» dirent-ils presque au même instant, et la porte glissa pour révéler Neville Londubat.

Neville pleurait \emph{vraiment} cette fois. «~J'ai été à la voiture de tête et j'ai trouvé un p-préfet, mais il m'a d-dit que les préfets ne devaient pas être dérangés pour des petites affaires telles que des crapeaux m-manquants.~»

Le Survivant changea d'expression. Ses lèvres devinrent une ligne fine. Sa voix, lorsqu'il parla, était froide et sinistre.

«~Quelles étaient ses couleurs~? Vert et argent~?

--- N-non, son badge était r-rouge et or.

--- \emph{Rouge et Or~!} s'écria Hermione. Mais ce sont les couleurs de \emph{Gryffondor}~!~»

Harry Potter \emph{siffla} en entendant ça, un son effrayant qui aurait pu venir d'un serpent et fit tressaillir Neville et Hermione. «~Je \emph{suppose}, cracha Harry Potter, que trouver le crapeau d'un quelconque première année n'est pas assez \emph{héroïque} pour mériter un préfet de \emph{Gryffondor}. Viens Neville, \emph{je} vais venir avec toi cette fois, et on verra si le Survivant obtient plus d'attention. D'abord nous trouverons un préfet qui sait jeter des sorts, et si ça ne marche pas, nous trouverons un préfet qui n'a pas peur de se salir les mains, et si \emph{ça} ne marche pas, je commencerai à recruter parmi mes fans et si nous le devons nous démonterons ce train boulon par boulon.~»

Le Survivant se leva, attrapa la main de Neville, et Hermione se rendit compte dans un hoquet mental qu'ils avaient presque la même taille, et pourtant une partie d'elle-même insistait sur le fait que Harry Potter faisait au moins trente centimètres de plus, et Neville quinze de moins.

«~\emph{Reste~!}~» lui lâcha Harry Potter -- non, en fait, à sa \emph{malle} -- et il ferma la porte fermement et s'en fut.

Elle aurait probablement dû y aller avec eux, mais pendant un bref moment Harry Potter était devenu si effrayant qu'elle était plutôt contente de ne pas l'avoir suggéré.

L'esprit de Hermione était maintenant si embrouillé qu'elle ne pensait même pas pouvoir lire «~Histoire~: Son Poudlard~». Elle avait l'impression qu'un rouleau compresseur lui était passé dessus et l'avait transformée en pancake. Elle n'était pas sûre de ce qu'elle pensait, ou de ce qu'elle ressentait, et encore moins de pourquoi. Elle resta juste assise à la fenêtre et regarda le paysage en mouvement.

%  LocalWords:  NPC Eek Judgment
