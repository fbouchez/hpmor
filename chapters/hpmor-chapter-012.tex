% vim:spell:spelllang=fr

{
  \let\specialmark\footnotemark
  \chapter{Contrôle des pulsions}
}

\footnotetext{NdT: le chapitre onze de la version originale contient des alternatives «~omaké~» (extra hors canon).
J'ai choisi de déplacer ce chapitre en fin d'ouvrage pour éviter d'interrompre le flot de l'histoire principale.}

\lettrinepara[ante=«~]{T}{urpin,} Lisa~!~»

\hplettrineextrapara
Chuchote chuchote chuchote Harry Potter chuchote chuchote Serpentard chuchote chuchote non sérieusement c'est quoi ce chuchote chuchote chuchote…

«~SERDAIGLE~!~»

Harry se joignit aux applaudissements accueillant la jeune fille marchant timidement vers la table de Serdaigle, et dont le liseré de la robe venait de se colorer en bleu foncé.
Lisa Turpin semblait partagée entre son désir de s'asseoir aussi loin de Harry Potter que possible et son désir de courir vers lui, de s'insérer de force à côté de lui, et de commencer à lui arracher des réponses.

Être au centre d'un événement extraordinaire et singulier, et être ensuite Réparti chez la Maison Serdaigle ressemblait fortement à être plongé dans de la sauce barbecue puis jeté dans une fosse pleine de chatons affamés.

«~J'ai promis au Choixpeau magique de ne pas en parler, chuchota Harry pour la énième fois.

--- Oui, vraiment.

--- Non, j'ai vraiment promis au Choixpeau de ne pas en parler.

--- Très bien, j'ai promis au Choixpeau de ne pas parler de \emph{presque tout} et le reste est \emph{privé} tout comme ça l'était pour \emph{vous} alors \emph{arrêtez de me poser la question}.

--- Vous voulez savoir ce qui s'est passé~? Très bien~!
Voilà une partie de ce qui s'est passé~!
J'ai dit au Choixpeau que McGonagall l'avait menacé de lui mettre le feu et il m'a dit de lui répondre qu'elle était une jeunette impudente et qu'elle devrait déguerpir de ses plates-bandes~!

--- Si vous ne croyez pas ce que je vous dis alors \emph{pourquoi est-ce que vous me posez la question}~!

--- Non, je ne sais pas non plus comment j'ai vaincu le Seigneur des Ténèbres~! Prévenez-moi si vous découvrez comment~!

--- \emph{Silence~!} cria la professeure McGonagall depuis le pupitre de la table d'honneur.
\emph{Pas de discussions avant la fin de la cérémonie de la Répartition~!}~»

Le volume sonore diminua brièvement pendant que chacun attendait de voir si elle faisait des menaces spécifiques et crédibles, puis les chuchotements recommencèrent.

Alors, l'ancien à barbe argentée se leva de son large fauteuil doré, souriant chaleureusement.

Silence instantané.
Quelqu'un donna des coups de coude frénétiques à Harry qui continuait de chuchoter, et Harry s'interrompit au beau milieu de sa phrase.

Le vieil homme souriant se rassit.

\emph{Note à moi-même~: ne pas faire le malin avec Dumbledore.}

Harry essayait encore de digérer tout ce qui s'était passé durant l'Incident avec le Choixpeau magique.
Un évènement en particulier, et non des moindres, avait eu lieu à l'instant où Harry avait enlevé le Choixpeau de sa tête~;
il avait alors entendu un léger murmure qui semblait venir de nulle part, quelque chose qui sonnait étrangement comme de l'Anglais et un sifflement en même temps, quelque chose qui avait dit~:
«~\parsel{Ssalutations de Sserpentard à Sserpentard~: ssi tu veux chercher mes ssecrets, parle à mon sserpent.}~»

Harry avait en gros deviné que ce n'était pas censé faire partie du processus officiel de la Répartition.
Et que c'était un bout de magie supplémentaire mis en place par Salazar Serpentard pendant la fabrication du Choixpeau.
Et que le Choixpeau lui-même n'était pas au courant.
Et que cela se déclenchait quand le Choixpeau disait «~SERPENTARD~», plus ou moins certaines autres conditions.
Et qu'un Serdaigle tel que lui n'était \emph{vraiment, vraiment pas censé avoir entendu cela.}
Et que s'il parvenait à trouver une manière fiable de faire jurer le secret à Drago pour qu'il puisse l'interroger à ce sujet, ce serait un moment parfait pour avoir de l'Hilari-Thé à portée de main.

\emph{Mec, tu prends la décision de ne pas suivre le chemin d'un Seigneur des Ténèbres et l'univers commence à s'amuser avec toi à l'instant même où le Choixpeau quitte ta tête.
Il y a des jours où ça ne paie pas de se battre contre le destin.
Peut-être que j'attendrai demain avant de commencer ma résolution de ne pas devenir un Seigneur des Ténèbres.}

«~GRYFFONDOR~!~»

Ron Weasley reçu \emph{beaucoup} d'applaudissements, et pas seulement de Gryffondor.
La famille Weasley était apparemment très aimée dans le coin.
Après un moment, Harry sourit et commença à applaudir avec les autres.

Mais après tout, il n'y avait pas de meilleur jour qu'aujourd'hui pour se détourner du Côté Obscur.

Que l'univers et le destin aillent se faire voir.
Il allait lui montrer, au Choixpeau.

«~Zabini, Blaise~!~»

Pause.

«~SERPENTARD~!~» cria le chapeau.

Harry applaudit aussi Zabini, ignorant les étranges coups regards qu'il recevait de la part de tout le monde, y compris de Zabini.

Aucun autre nom ne fut appelé après celui-là, et Harry se rendit compte que «~Zabini, Blaise~» semblait très proche de la fin de l'alphabet.
Génial, donc maintenant il avait \emph{uniquement} applaudi Zabini… bon, tant pis.

Dumbledore se leva à nouveau et commença à se diriger vers l'estrade.
Apparemment ils allaient avoir droit à un discours…

Et Harry fut frappé par l'inspiration, celle d'un test expérimental \emph{brillant}.

Hermione avait dit que Dumbledore était le plus puissant des sorciers en vie, non~?

Harry mit sa main dans sa bourse et murmura~: «~Hilari-Thé~».

Pour que l'Hilari-Thé fonctionne, il faudrait qu'il fasse dire à Dumbledore quelque chose de \emph{si} ridicule dans son discours que même dans l'état de préparation mentale où se trouvait Harry, il s'étoufferait \emph{malgré tout}.
Du genre~: aucun étudiant de Poudlard ne serait autorisé à porter de vêtement de l'année, ou que tout le monde allait être transformé en chat.

Mais si \emph{une seule personne au monde} pouvait résister au pouvoir de l'Hilari-Thé, cela ne pouvait être que Dumbledore.
Donc si ça marchait, cela signifiait que l'Hilari-Thé était littéralement \emph{invincible}.

Harry tira sur la languette de la canette sous la table, essayant d'être le plus discret possible.
Cela fit un petit \emph{pschitt}.
Quelques têtes se tournèrent vers lui, mais se retournèrent rapidement tandis que…

«~Bienvenue~! Bienvenue pour une nouvelle année à Poudlard~!~» dit Dumbledore, rayonnant vers les élèves, les bras grands ouverts, comme si rien ne pouvait lui faire plus plaisir que de les voir tous ici.

Harry prit une première gorgée d'Hilari-Thé et abaissa la canette.
Il avalerait le soda petit à petit et essaierait de ne pas s'étouffer \emph{quoi que dise Dumbledore}…

«~Avant de commencer le banquet, je voudrais dire quelques mots.
Et les voici~: tapis tapis boum boum lampe lampe lampe~! Merci~!~»

Tout le monde applaudit en acclamant, et Dumbledore se rassit.

Harry restait assis, figé. Du soda ruisselait des coins de sa bouche.
Il avait au moins réussi à s'étouffer \emph{discrètement}.

Il n'aurait vraiment, vraiment, \emph{vraiment} jamais dû faire cela.
Incroyable à quel point cela devenait une \emph{évidence}, \emph{une seconde} après qu'il soit \emph{trop tard}.

Rétrospectivement, il aurait probablement dû remarquer que quelque chose n'allait pas lorsqu'il avait pensé à la possibilité que tout le monde soit transformé en chat… ou même avant cela, se rappelant sa note mentale de ne pas faire le malin avec Dumbledore… ou sa nouvelle résolution d'accorder plus de considération aux autres… ou peut-être s'il avait eu \emph{la moindre petite once de bon sens}…

C'était sans espoir.
Il était corrompu jusqu'au trognon.
Gloire au Seigneur des Ténèbres Harry.
On ne pouvait pas combattre le destin.

Quelqu'un demandait à Harry s'il allait bien.
(Les autres avait commencé à se servir dans les plats, qui étaient magiquement apparus sur la table, allons bon.)

«~Oui ça va, dit Harry. Excuse-moi. Euh. Était-ce un… discours \emph{normal} pour le directeur~?
Aucun de vous… n'avait l'air… très surpris…

--- Oh, Dumbledore est cinglé, bien sûr~» dit un Serdaigle un peu plus âgé qui s'était assis à côté de lui et s'était présenté avec un quelconque prénom dont Harry n'allait probablement pas se rappeler.
«~Très drôle, sorcier incroyablement puissant, mais complètement cinglé.~» 
Il marqua une pause.
«~Plus tard j'aimerais te demander pourquoi du liquide vert a coulé de tes lèvres et a ensuite disparu, même si je suppose que tu as promis au Choixpeau de ne pas parler de ça non plus.~»

Au prix d'un grand effort, Harry s'empêcha de baisser les yeux vers la canette incriminante d'Hilari-Thé qu'il avait en main.

Au final, l'Hilari-Thé n'avait pas arbitrairement \emph{matérialisé} un gros titre du Chicaneur sur Drago et lui.
Drago avait donné des explications qui semblaient montrer que tout avait eu lieu… naturellement~?
Comme si l'Hilari-Thé avait \emph{altéré l'histoire pour que tout concorde~?}

Harry se visualisait mentalement en train de se frapper le front contre la table.
\emph{Bam, bam, bam} faisait sa tête dans son esprit.

Un autre étudiant baissa sa voix jusqu'à chuchoter.
«~Il paraît que Dumbledore est secrètement un cerveau génial qui tire les ficelles, et qu'il utilise sa folie comme couverture pour que personne ne le soupçonne.

--- J'ai entendu ça aussi~», chuchota un troisième élève, accompagné par des hochements de tête furtifs tout autour de la table.

% Harry ne put s'empêcher de réagir.
Cela ne pouvait qu'attirer l'attention de Harry.

«~Je vois~», chuchota Harry, baissant sa voix à son tour.
«~Donc tout le monde sait que Dumbledore est secrètement un génie manipulateur.~»

La plupart des élèves acquiescèrent.
Un ou deux semblèrent soudain pensifs, notamment l'élève plus âgé assis à côté de Harry.

\emph{Est-on certain que c'est la table des Serdaigle~?}
Harry parvint à ne pas poser cette question à voix haute.

«~Génial~! chuchota Harry. Si tout le monde le sait, personne ne soupçonnera que c'est un secret~!

--- Exactement~», chuchota un élève, avant de froncer les sourcils.
«~Attends, il y a un truc qui ne colle pas…~»

\emph{Note à moi-même~: le 75\textsuperscript{e} centile des élèves de Poudlard, c'est-à-dire la Maison Serdaigle, n'est pas la formation pour enfants surdoués la plus exclusive au monde.}

Mais au moins il avait découvert une information importante aujourd'hui.
L'Hilari-Thé était omnipotent.
Et \emph{cela} voulait dire…

Harry cligna des yeux de surprise au moment où sa tête fit le lien évident.

…\emph{cela} voulait dire que dès qu'il aurait appris un sort permettant d'altérer temporairement son sens de l'humour, il pourrait faire survenir \emph{n'importe quoi} en faisant en sorte de ne trouver \emph{que cette seule chose} suffisamment surprenante pour s'en étouffer, puis en buvant une canette d'Hilari-Thé.

\emph{Eh bien c'était un voyage assez court vers la divinité.
Même moi je m'attendais à ce que ça prenne plus de temps que mon premier jour d'école}.

Maintenant qu'il y pensait, il venait aussi de complètement saccager Poudlard même pas dix minutes après avoir été Réparti.

Harry éprouvait un certain regret à cette pensée -- Merlin seul savait ce qu'un directeur fou était capable de faire à ses sept prochaines années de scolarité -- mais il ne pouvait \emph{s'empêcher} de ressentir aussi une pointe de fierté.

Demain. Demain, au plus tard, il arrêterait de descendre la pente menant au Seigneur des Ténèbres Harry.
Une perspective qui semblait plus effrayante minute après minute.

Et en même temps, étrangement, de plus en plus attrayante.
Une partie de son esprit visualisait déjà les uniformes de ses sbires.

«~Mange~», grogna l'élève plus âgé assis à côté de lui, lui donnant un coup de coude dans les côtes.
«~Ne pense pas. Mange.~»

Harry se mit à remplir automatiquement son assiette avec ce qui se trouvait en face de lui, des saucisses bleues avec des petits morceaux brillants, allons bon.

«~À quoi est-ce que tu pensais, le Choip…~» commença Padma Patil, une autre Serdaigle de première année.

«~On n'importune pas pendant les repas~! dirent en chœur au moins trois personnes.
Règle de la Maison~! ajouta un autre. Autrement on mourrait tous de faim.~»

Harry se rendit compte qu'il espérait vraiment, \emph{vraiment} que, si astucieuse soit-elle, sa nouvelle idée ne marche pas \emph{réellement}.
Que l'Hilari-Thé fonctionne d'une autre façon et n'ait pas \emph{réellement} des pouvoirs omnipotents d'altération de la réalité.
Non pas qu'il ne \emph{veuille} pas devenir omnipotent.
C'était juste qu'il ne pouvait supporter l'idée de vivre dans un univers qui fonctionnerait vraiment comme cela.
Il y avait quelque chose d'\emph{indigne} à réaliser son ascension via l'utilisation astucieuse d'une boisson gazeuse.

Mais il \emph{allait} le vérifier expérimentalement.

«~Tu sais, dit d'un ton aimable l'élève plus âgé assis à côté de lui, nous avons un système pour forcer les gens comme toi à manger, veux-tu découvrir ce que c'est?~»

Harry laissa tomber et commença à manger sa saucisse bleue.
C'était plutôt bon, surtout les morceaux brillants.

Le dîner s'acheva avec une rapidité surprenante.
Harry avait essayé de goûter à au moins un petit bout de toutes les nouvelles nourritures étranges qu'il avait vues.
Sa curiosité ne supportait pas l'idée de \emph{ne pas savoir} quel était le goût de quelque chose.
Dieu merci ce n'était pas un restaurant où il fallait commander un seul plat et où vous ne sauriez jamais le goût de toutes les autres choses qui étaient au menu.
Harry \emph{détestait} cela, c'était comme une chambre de torture conçue pour quiconque avait une étincelle de curiosité~: \emph{Découvre un seul et unique mystère de tous ceux de la liste, ha ha ha~!}

Puis ce fut l'heure du dessert, pour lequel Harry avait complètement oublié de garder un peu de place.
Il abandonna après avoir goûté un petit morceau de tarte à la mélasse.
Toutes ces choses allaient certainement repasser au moins une fois avant la fin de l'année.

Bon, qu'y avait-il sur sa liste de tâches, mis à part les activités scolaires ordinaires~?

\emph{Tâche n°1: faire des recherches sur les sorts d'altération de la pensée pour pouvoir tester l'Hilari-Thé et voir j'ai vraiment trouvé un chemin menant à l'omnipotence.
En fait, faire des recherches sur tous les types de magie de la pensée que je peux trouver.
La pensée est la fondation de notre pouvoir en tant qu'humains, donc toute magie l'affectant est la plus importante des magies.}

\emph{Tâche n°2~: en fait, c'est la “Tâche n°1” et l'autre est la “Tâche n°2”.
Parcourir les bibliothèques de Poudlard et de Serdaigle, se familiariser avec le système et s'assurer d'avoir lu au moins tous les titres de livres.
Deuxième passage~: lire toutes les tables des matières.
Se coordonner avec Hermione qui a une mémoire bien meilleure que la mienne.
Se renseigner pour savoir s'il existe un système de prêt inter-bibliothèques à Poudlard et voir si on peut tous les deux, surtout Hermione, visiter aussi ces bibliothèques.
Si d'autres Maisons ont des bibliothèques privées, découvrir comment y accéder légalement ou comment s'y introduire.}

\emph{Option 3a~: faire jurer le secret à Hermione et commencer les recherches sur “De Serpentard à Serpentard~: si tu veux chercher mes secrets, parle à mon serpent.”
Problème~: cela semble hautement confidentiel, et ça pourrait prendre un moment avant de tomber par hasard sur un livre contenant un indice.}

\emph{Tâche n°0~: regarder ce qui existe comme sorts de recherche-et-obtention-d'information~; s'il y en a.
La magie de bibliothèques n'est pas aussi importante que la magie de la pensée, mais elle a une priorité bien plus élevée.}

\emph{Option 3b~: chercher un sort pour forcer Drago à garder le secret, ou vérifier de façon magique la sincérité de la promesse de Drago de garder le secret (Veritaserum~?), puis l'interroger sur le message de Serpentard…}

À vrai dire… Harry avait un mauvais pressentiment sur l'option 3b.

Maintenant qu'il y réfléchissait, il n'était pas très confiant dans l'option 3a non plus.

Harry se remémorait le pire moment qu'il avait vécu de sa vie, ces longues secondes d'horreur glacée sous le Choixpeau, quand il pensait avoir déjà échoué.
Il avait souhaité revenir dans le passé juste quelques minutes et changer quelque chose, n'importe quoi avant qu'il ne soit trop tard…

Et puis il s'était trouvé qu'il n'était finalement pas trop tard.

Vœu exaucé.

On ne pouvait pas changer l'histoire.
Mais on pouvait faire correctement dès le départ.
Faire quelque chose différemment au \emph{premier} essai.

Toute cette affaire avec Serpentard, rechercher ses secrets… ça ressemblait terriblement au genre de chose qu'on se rappellerait des années plus tard et au sujet de laquelle on dirait~: “Et c'est \emph{là} que les choses ont commencé à mal tourner.”

Et il souhaiterait désespérément avoir la capacité de revenir dans le temps et de faire un autre choix…

Vœu exaucé. Et maintenant~?

Harry sourit lentement.

C'était une pensée plutôt \emph{contre-intuitive}… mais…

Mais il \emph{pourrait}, il n'y avait aucune règle disant qu'il ne pouvait pas, il \emph{pourrait} prétendre n'avoir jamais entendu ce petit chuchotement.
Laisser l'univers continuer exactement comme il l'aurait fait si ce moment crucial ne s'était jamais produit.
Vingt ans plus tard, c'est exactement ce qu'il aurait souhaité avoir fait vingt ans plus tôt, et il se trouvait que vingt ans avant vingt ans plus tard, c'était maintenant.
Modifier le passé lointain était facile, il suffisait d'y penser au bon moment.

Ou alors… c'était encore \emph{plus} contre-intuitif… il pourrait même en informer, oh, disons \emph{la professeure McGonagall}, plutôt que Drago \emph{ou} Hermione.
Et elle pourrait réunir quelques personnes compétentes pour retirer ce petit sort supplémentaire du Choixpeau.

Mais oui. Cela semblait être une idée \emph{remarquablement} bonne maintenant que Harry y avait \emph{pensé}.

Tellement évidente rétrospectivement, et pourtant, Harry n'avait juste pas imaginé les options 3c et 3d.

Harry se décerna +1 point dans son programme anti-Seigneur-des-Ténèbres-Harry.

Le Choixpeau lui avait joué un tour terriblement cruel, mais du point de vue conséquentialiste, on ne pouvait critiquer les résultats.
Cela dit, cela lui avait également donné une meilleure idée de ce à quoi pouvait ressembler le point de vue des victimes.

\emph{Tâche n°4~: s'excuser auprès de Neville Londubat.}

OK, c'était parti maintenant, il n'avait qu'à continuer comme ça.
\emph{Et chaque jour, à chaque moment, je me rapproche de la Lumière, de la Lumière…}

La plupart des gens autour de Harry avaient également fini de manger à présent, et les plats à desserts commencèrent à disparaître, ainsi que les assiettes sales.

Lorsque toutes les assiettes eurent disparu, Dumbledore se leva à nouveau de son siège.

Harry sentit une envie irrépressible de boire un autre Hilari-Thé.

\emph{Tu veux RIRE}, pensa Harry à l'intention de cette partie de lui-même.

Cependant, l'expérience ne comptait pas si elle n'était pas répliquée, n'est-ce pas~?
Et le mal était déjà fait, non?
Ne voulait-il pas voir ce qui allait se produire \emph{cette} fois-ci~?
N'était-il pas \emph{curieux}~?
Et si le résultat s'avérait différent~?

\emph{Eh, je parie que tu es la partie de mon cerveau qui m'a poussé à jouer ce canular à Neville Londubat}.

Euh, peut-être~?

\emph{Et n'est-ce pas} immensément \emph{évident que si je fais cela je vais le regretter une seconde après qu'il soit trop tard~?}

Ben…

\emph{Ouais. Donc, NON.}

«~Hum hum~», fit Dumbledore depuis l'estrade, caressant sa longue barbe d'argent.
«~Juste quelque mots de plus maintenant que nous sommes tous nourris et désaltérés.
J'ai quelques informations de début de trimestre à vous donner.

«~Pour les premières années, notez que la forêt bordant l'enceinte de Poudlard est interdite à tous les élèves.
C'est pour cela qu'on l'appelle la Forêt interdite.
Si son accès était autorisé on l'appellerait la Forêt autorisée.~»

Clair et direct. \emph{Note à moi-même~: La Forêt interdite est interdite.}

«~M. Rusard, le concierge, m'a demandé de vous rappeler qu'il est interdit d'utiliser la magie dans les couloirs entre deux cours.
Hélas, nous savons tous que ce qui \emph{devrait être} et ce qui \emph{est} sont deux choses différentes.
Merci de garder cela à l'esprit.~»

Euh…

«~Les essais de Quidditch auront lieu la deuxième semaine de cours.
Si vous désirez jouer pour l'équipe de votre Maison, merci de contacter Madame Bibine.
Si vous désirez réformer l'intégralité du Quidditch, merci de contacter Harry Potter.~»

Harry inhala sa propre salive ce qui lui déclencha une quinte de toux tandis que tous les yeux se tournaient vers lui.
Mais comment \emph{diable}~!
Il n'avait jamais croisé les yeux de Dumbledore… du moins il le \emph{pensait}.
Et en tous cas, il ne pensait certainement pas au Quidditch~!
Il n'en avait parlé à personne hormis à Ron Weasley et il ne \emph{pensait} pas que Ron en aurait parlé à qui que ce soit… ou alors Ron avait-il été se plaindre auprès d'un professeur~?
\emph{Mais comment…}

«~De plus, je dois vous dire que cette année, le couloir du troisième étage côté droit est hors limites pour quiconque souhaite ne pas mourir d'une mort très douloureuse.
Il est gardé par une série sophistiquée de pièges dangereux et potentiellement mortels, et il est impossible que vous les franchissiez tous, en particulier si vous êtes en première année.~»

Harry ne ressentait plus rien à ce stade.

«~Et finalement je présente mes plus profonds remerciements au professeur Quirinus Quirrell pour avoir héroïquement accepté le poste de professeur de Défense contre les Forces du Mal de Poudlard.~»
Dumbledore balaya les étudiants d'un regard perçant.
«~J'espère que tous les élèves présenteront au professeur Quirrell la plus grande des courtoisies ainsi que la plus grande \emph{tolérance} due au service extraordinaire qu'il vous rend ainsi qu'à cette école, et que vous \emph{ne nous importunerez pas} de \emph{plaintes insignifiantes} à son propos, à moins que \emph{vous} ne vouliez essayer de faire son travail.~»

C'était \emph{quoi} ça?

«~Je cède maintenant la place au nouveau membre de notre école, le professeur Quirrell, qui souhaiterait dire quelques mots.~»

Le jeune homme mince et nerveux que Harry avait rencontré au Chaudron Baveur s'avança lentement jusqu'au pupitre, jetant des regards apeurés dans toutes les directions.
Harry entrevit l'arrière de sa tête, et il semblait que le Professeur Quirrell devenait déjà chauve en dépit de sa jeunesse apparente.

«~Je me demande ce qui cloche chez \emph{lui}~», murmura l'élève un peu plus âgé assis à côté de Harry.
D'autres commentaires similaires furent discrètement échangés autour de la table.

Quirrell arriva au pupitre et se tint là, clignant des yeux.
«~Ah… dit-il. Ah…~» puis son courage sembla l'avoir totalement abandonné, et il resta là, silencieux, pris de tremblements occasionnels.

«~Génial, chuchota le voisin de Harry, on dirait que cela va être encore \emph{longue} année de cours de Défense…

--- Salutations, mes jeunes apprentis, dit Quirrell d'un ton sec et assuré.
Nous savons tous que Poudlard a une certaine tendance à \emph{l'infortune} dans ses choix pour ce poste, et nul doute que nombreux sont ceux qui parmi vous se demandent déjà quelle malédiction s'abattra sur moi cette année.
Je vous assure que cette malédiction ne sera pas celle de l'incompétence. Il sourit légèrement.
Croyez-le ou non, j'ai depuis longtemps désiré m'essayer au poste de professeur de Défense contre les Forces du Mal, ici à l'école de sorcellerie de Poudlard.
Le premier à enseigner ce cours était Salazar Serpentard lui-même, et il était de coutume jusqu'au quatorzième siècle que les plus grands sorciers combattants de toutes confessions s'essaient à enseigner ici.
On compte parmi les anciens professeurs de Défense non seulement le légendaire héros vagabond Harold Shea, mais aussi \emph{ouvrez les guillemets} l'immortelle \emph{fermez les guillemets} Babayaga, oui, je vois certains d'entre vous frissonner à l'évocation de son nom bien qu'elle soit morte depuis six-cent ans.
Cela devait être une époque intéressante pour être élève à Poudlard, vous ne pensez pas~?~»

Harry avala péniblement sa salive, essayant de contenir la soudaine vague d'émotions qui avait menacé de le submerger quand le professeur Quirrell avait commencé à parler.
Les tons précis de sa voix lui rappelaient un certain conférencier d'Oxford, et Harry commençait à vraiment intégrer le fait qu'il n'allait pas revoir sa maison, ni sa maman, ni son papa avant Noël.

«~Vous êtes habitués à voir le poste de Défense tenu par des incompétents, des vauriens et des malchanceux.
Pour quiconque doté d'un sens de l'histoire, sa réputation est tout autre.
Tous ceux qui ont enseigné ici n'ont pas forcément fait partie des meilleurs, mais les meilleurs ont tous enseigné à Poudlard.
En telle auguste compagnie, et après tant d'années à anticiper ce jour, j'aurais honte de viser moins haut que la perfection.
Et j'ai donc bien l'intention que chacun de vous se souvienne pour toujours de cette année comme celle du \emph{meilleur} cours de Défense que vous ayez jamais eu.
Ce que vous apprendrez cette année vous servira à jamais et sera une fondation solide dans l'art de la Défense, quels que soient vos enseignants passés et futurs.~»

Le visage de Quirrell devint sérieux.
«~Nous avons \emph{beaucoup} de terrain à rattraper, et peu de temps pour le parcourir.
J'ai par conséquent l'intention de m'éloigner des enseignements habituels de Poudlard sur divers degrés, ainsi que d'introduire des activités du soir.~» 
Il marqua une pause.
«~Si ce n'est pas suffisant, je pourrais peut-être trouver de nouvelles façons de vous motiver.
Vous êtes les élèves que j'attends depuis longtemps, et vous \emph{donnerez} le meilleur de vous-mêmes dans ce cours de Défense attendu depuis longtemps.
J'ajouterais bien une terrible menace, comme “Ou sinon vous souffrirez horriblement,” mais ce serait tellement cliché, vous ne trouvez pas~?
Je m'enorgueillis d'être plus créatif que cela. Merci.~»

Et la vigueur et la confiance semblèrent s'écouler hors de Quirrell.
Sa bouche s'ouvrit toute grande, comme s'il s'était soudain trouvé face à un public inattendu, et il se retourna vers son siège dans un tressaillement convulsif, puis traîna les pieds jusqu'à celui-ci, voûté comme s'il était sur le point de s'effondrer sur lui-même et d'imploser.

«~Il a l'air un peu bizarre, chuchota Harry.

--- Bah, dit l'élève voisin. T'as encore rien vu.~»

Dumbledore revint au pupitre.

«~Et maintenant, dit Dumbledore, avant d'aller au lit, chantons la chanson de l'école~!
Choisissez tous votre air favori et vos paroles favorites, et c'est parti~!~»

%  LocalWords:  urpin slytherin 3a 3b 3c 3d nothin
