\chapter{Une journée très faiblement probable}

\lettrine{C}{haque} centimètre carré d'espace mural est caché derrière un pan de bibliothèque. Chaque bibliothèque a six étagères et atteint presque le plafond. Sur certaines on trouve entassés des livres et manuels~: science, mathématiques, histoire… Sur d'autres ce sont des livres de science-fiction sur deux rangées, celle du fond surélevée à l'aide de vieilles boîtes de mouchoirs\footnotemark{} ou de tasseaux de bois pour que l'on puisse voir les livres cachés par la rangée de devant. Et ce n'est pourtant pas suffisant. Les livres débordent jusque sur les tables et des canapés et forment de petits tas sous les fenêtres.
\authorsnotetext{Je fais cela chez moi.}

C'est le salon de la maison où vivent l'éminent professeur Michel Verres-Evans, sa femme Mme Pétunia Evans-Verres, et leur fils adoptif~: Harry James Potter-Evans-Verres.

Une lettre se trouve sur la table du salon, ainsi qu'une enveloppe non timbrée en parchemin jauni, adressée à \emph{M. H. Potter} à l'encre vert-émeraude.

Le professeur et sa femme se parlent sèchement, mais sans crier. Le professeur considère que cela ne serait pas civilisé.

«~Tu plaisantes~», dit Michel à Pétunia. Le ton montrait sa crainte qu'elle soit au contraire tout à fait sérieuse.

«~Ma sœur était une sorcière~», répéta Pétunia. Elle était inquiète, mais n'en démordait pas. «~Son mari était un sorcier.~»

«~C'est absurde~!~l'interrompit Michel.  Ils étaient à notre mariage… Ils venaient à la maison pour Noël…

--- Je leur avais demandé de ne rien te dire, murmura Pétunia, mais c'est vrai~! J'ai vu des choses…~»

Le professeur leva les yeux au ciel.

«~Ma chérie, nous savons tous les deux que tu ne connais pas bien la littérature sceptique.  Tu ne te rends sûrement pas compte à quel point il est facile pour un prestidigitateur aguerri de sembler accomplir l'impossible. Te souviens-tu que j'ai appris à Harry comment tordre des petites cuillères~? Et s'ils semblent pouvoir deviner tes pensées c'est de la lecture à froid, une technique de mentaliste…

--- Je ne parle pas de cuillères tordues…

--- De quoi s'agissait-il, alors~?~»

Pétunia se mordit la lèvre.

«~Je ne peux pas te le dire. Tu penserais que je suis… Elle déglutit. Écoute. Michel. Je n'ai pas toujours été… ainsi…~» Elle montrait son corps et sa silhouette mince.
«~C'est Lily qui m'a… améliorée. Enfants, Lily était plus jolie que moi, c'est elle qui avait toutes les attentions, et je la jalousais. Quand on a appris qu'en plus de cela elle avait également reçu des \emph{pouvoirs magiques}, est-ce que tu imagines ce que j'ai pu ressentir~? Alors je l'ai \emph{suppliée}, durant des années je l'ai suppliée d'utiliser sa magie pour que je puisse être jolie moi aussi. Si je n'avais pas sa magie, j'avais au moins le droit d'être jolie~!~»

Pétunia avait les larmes aux yeux.

«~Mais Lily refusait, elle inventait les excuses les plus ridicules, comme quoi ce serait la fin du monde, ou bien qu'un centaure l'aurait mise en garde… je la haïssais, elle et ses excuses minables. Était-ce si terrible de vouloir être gentille avec sa sœur~? Une fois adulte, j'étais tout juste diplômée, je sortais avec ce garçon, Vernon Dursley. Il était gros mais c'était le seul garçon qui voulait bien me parler.  Il s'est mit un jour à me parler d'enfants, son premier fils s'appellerait Dudley. Ce jour-là, je me souviens avoir pensé~: \emph{quel genre de parent appellerait son enfant Dudley Dursley~?} Ce fut comme si je voyais ma vie future défiler devant moi, et je n'ai pas pu le supporter. De désespoir j'ai écrit à ma sœur, pour la supplier de m'aider une dernière fois, sans cela autant…~»

Elle se tut.

«~Enfin, continua-t-elle d'une petite voix, elle a cédé.  Elle m'a prévenu que c'était dangereux mais m'en fichais, j'ai bu sa potion et j'ai été malade pendant des semaines. Mais une fois remise, ma peau avait perdu ses boutons, j'avais enfin des formes et… j'étais belle. Les gens étaient \emph{gentils} avec moi. Sa voix se brisa. Après cela, je ne pouvais plus haïr ma sœur. Surtout quand j'ai appris ce que sa magie lui avait fait…

--- Ma chérie, dit tendrement Michel, tu es tombée malade, tu t'es étoffée pendant ta convalescence, et ta peau s'est arrangée toute seule. Ou bien tomber malade a modifié tes habitudes alimentaires…

--- C'était une sorcière, répéta Pétunia, je l'ai vu.

--- Pétunia~», rétorqua Michel, dont le ton laissait transparaitre une pointe d'exaspération. «~Tu \emph{sais} que c'est impossible. Dois-je vraiment t'expliquer pourquoi~?~»

Pétunia se tordait les mains, elle semblait être sur le point de pleurer. «~Mon amour, je sais que je ne peux pas remporter ce genre de débat avec toi, mais s'il te plaît, il faut que tu me fasses confiance sur ce point là…

--- \emph{Papa~! Maman~!}~»

Ils se turent et se tournèrent vers Harry, semblant avoir oublié qu'une troisième personne se trouvait là, avec eux.

Harry inspira lentement. «~Maman, \emph{vos} parents n'avaient pas de pouvoirs magiques, n'est-ce pas~?

--- Non, répondit Pétunia d'un air étonné.

--- Alors personne chez vous ne connaissait la magie quand Lily a reçu sa lettre. Comment est-ce qu'ils \emph{vous} ont convaincus~?

--- Ah… dit Pétunia, ils n'ont pas fait qu'envoyer une lettre. Ils ont envoyé un professeur de Poudlard. Il…~» elle regarda brièvement Michel. «~Il nous a fait une démonstration de sa magie.

--- Alors inutile de vous battre~», dit Harry d'un ton ferme. Il espérait contre toute attente que cette fois, juste pour une fois, ils l'écouteraient. «~Si c'est vrai, on n'a qu'à faire venir un professeur de Poudlard, voir cette magie de nos propres yeux et papa devra admettre que c'est bien réel. Et sinon, alors maman admettra que la magie n'existe pas.  C'est à ça que sert la méthode expérimentale au lieu d'essayer de trancher les questions par le débat.~»

Le professeur se retourna pour regarder son fils d'un air condescendant comme à son habitude. «~Voyons Harry… Vraiment, de la \emph{magie}~? Je pensais que \emph{toi} au moins, tu aurais l'intelligence de ne pas prendre ça au sérieux, même si tu n'as que dix ans. Il n'y a rien de moins scientifique au monde que la magie~!~»

Harry serra les lèvres avec amertume. On s'occupait bien de lui, probablement mieux que la plupart des parents biologiques ne le faisaient avec leurs propres enfants. Harry avait été inscrit aux meilleures écoles primaires, et quand les choses s'étaient mal passées, on avait recruté des tuteurs venus de ce puits sans fond que sont les doctorants affamés. On l'avait toujours encouragé à étudier ce qui l'intéressait le plus, acheté tous les livres qu'il désirait, soutenu à tous les concours mathématiques et scientifiques auxquels il avait participé. On lui donnait tout ce qu'il souhaitait, dans la limite du raisonnable, sauf, peut-être, la plus petite once de respect. On pouvait difficilement attendre d'un professeur de biochimie enseignant à Oxford qu'il suive les conseils d'un petit garçon. L'écouter pour montrer qu'on s'intéresse à lui, bien évidemment~; voilà ce que font les \emph{bons parents}, et si vous vous considérez comme un \emph{bon parent}, c'est ce qu'il faut faire. Mais prendre un enfant de dix ans \emph{au sérieux}~? Certainement pas.

Parfois, une envie irrépressible venait à Harry de hurler sur son père.

«~Maman, dit Harry, si tu veux vraiment remporter ce débat avec papa, cherche dans le chapitre deux du Cours de physique de Feynman. Tu y trouveras une citation sur les philosophes qui ont toujours beaucoup à dire sur ce que la science exige, et qui pourtant ont tort, car la seule règle en science, c'est que l'observation est l'arbitre final---il suffit d'observer le monde et de rendre compte de ce que l'on y a vu. Heu… je n'arrive pas à retrouver de tête une source pour expliquer que l'un des idéaux de la science est de régler les désaccords par l'expérimentation plutôt que par le débat…~»

Sa mère baissa les yeux sur lui et lui sourit. «~Merci Harry.  Mais…~» elle regarda alors son mari dans les yeux. «~Je ne veux pas remporter un débat. Je veux que… que mon mari écoute sa femme qui l'aime et qu'il lui fasse pour une fois confiance…~»

Harry ferma brièvement les yeux. \emph{C'était sans espoir}. Ses deux parents étaient des causes perdues.

Ses parents recommençaient à nouveau une de \emph{ces} disputes où sa mère essayait de faire que son père se sente coupable, tandis que ce dernier essayait de faire que sa mère se sente stupide.

«~Je monte dans ma chambre~», annonça Harry dont la voix tremblait un peu. «~Essayez de ne pas trop vous disputer à ce propos, papa et maman.  On aura la réponse bien assez vite, non~?~»

--- Bien sûr Harry~», répondit son père, tandis que sa mère lui envoyait un baiser rassurant, et ils continuèrent leur dispute pendant que Harry grimpait l'escalier jusqu'à sa chambre.

Il ferma la porte derrière lui et tenta de réfléchir.

Ce qui était drôle est qu'il aurait \emph{dû} être du côté de son père. Personne n'avait jamais vu la moindre preuve que la magie existe, et d'après maman un monde magique entier existerait. Comment qui que ce soit pourrait garder une telle chose secrète~? Par magie~? C'était extrêmement suspicieux comme excuse.

Cela semblait pourtant un cas simple à traiter~: maman soit faisait une blague, soit mentait, soit était folle -- en ordre croissant du moins au plus horrible. Si elle avait envoyé la lettre elle-même, cela expliquerait son arrivée dans la boîte aux lettres malgré l'absence de timbre. Un peu de folie était bien, bien moins improbable que d'avoir un univers qui fonctionne réellement ainsi.

Sauf que quelque chose en Harry était profondément convaincu que la magie était bien réelle, et cela depuis l'instant même où il avait posé les yeux sur la lettre envoyée présumément par l'École de Sorcellerie de Poudlard.

Il se frotta le front en grimaçant. \emph{Ne crois pas tout ce que tu penses}, disait l'un de ses livres.

Mais cette étrange certitude… Harry se rendit compte qu'il \emph{s'attendait} à ce qu'effectivement un professeur de Poudlard se présente, qu'il agite une baguette dans les airs et en fasse sortir sa magie. Cette étrange certitude ne faisait absolument aucun effort pour se protéger d'une falsification -- aucune excuse préparée à l'avance au cas où aucun professeur ne viendrait, ou au cas où le professeur ne serait capable de rien d'autre que tordre des petites cuillères.

\emph{D'où viens-tu, étrange petite prédiction~?} Harry dirigea cette pensée vers son cerveau. \emph{Pourquoi crois-je en ce que je crois~?}

Habituellement, Harry savait plutôt bien répondre à cette question, mais dans ce cas particulier, il n'avait \emph{aucune idée} de ce à quoi son cerveau pouvait bien penser.

Harry haussa mentalement les épaules. Une plaque de métal sur une porte invite à la pousser, une poignée sur une porte invite à la tirer, et une hypothèse testable ne demande qu'une chose~: qu'on la teste.

Il se saisit d'une feuille de papier et commença à écrire.

\begin{writtenNote}
\letterAddress{Chère directrice adjointe,}
\end{writtenNote}

Harry s'interrompit pour réfléchir, puis jeta la feuille au profit d'une autre et fit sortir un autre millimètre de graphite de son critérium. Cela méritait un effort calligraphique plus méticuleux.

\begin{writtenNote}
\letterAddress{Chère directrice adjointe Minerva McGonagall,}
\letterAddress{Ou à toute personne qui serait concernée,}

J'ai récemment reçu votre lettre d'admission à Poudlard, adressée à M. H. Potter.  Peut-être ignorez-vous que mes parents biologiques, James Potter et Lily Potter (née Evans) sont décédés. J'ai été adopté par la sœur de Lily, Pétunia Evans-Verres, et par son mari, Michel Verres-Evans.

Je désire ardemment me rendre à Poudlard, si tant est qu'un tel lieu existe réellement. Seule ma mère Pétunia dit connaître l'existence de la magie, mais est elle-même incapable de la pratiquer. Mon père est quand à lui extrêmement sceptique. Je suis moi-même incertain.  J'ignore par ailleurs où me procurer les livres et l'équipement listés dans votre lettre.

Ma mère a mentionné que vous aviez envoyé un représentant de Poudlard chez Lily Potter (Lily Evans à cette époque) afin de démontrer la réalité de la magie à sa famille et, je suppose, d'aider Lily à se procurer ses fournitures scolaires. Ce serait une aide précieuse si vous pouviez faire de même pour ma propre ma famille.

\letterClosing[Veuillez agréer, madame, monsieur, l'expression de mes sentiments les plus sincères,]{Harry James Potter-Evans-Verres.}
\end{writtenNote}

Harry ajouta son adresse puis plia la lettre et la mit dans une enveloppe adressée à Poudlard. Quelques instants de réflexion le poussèrent à se procurer une bougie et à faire fondre de la cire sur le rabat de l'enveloppe dans laquelle il grava H.J.P.E.V.  de la pointe d'un canif. Quitte à sombrer dans la folie, autant que ce soit avec classe.

Il ouvrit alors la porte de sa chambre et redescendit les escaliers. Son père était dans le salon et lisait un livre de mathématiques avancées pour montrer à quel point il était intelligent~; sa mère était dans la cuisine et préparait l'un des plats favoris de son mari pour montrer à quel point elle était aimante. Aucun des deux ne semblait disposé à communiquer avec l'autre. Aussi effrayantes que les disputes puissent être, \emph{ne pas se disputer} était étonnamment encore pire.

«~Maman, Harry rompit le silence pesant, je vais tester l'hypothèse. Heu… selon ta théorie, comment envoie-t-on une chouette à Poudlard~?~»

Sa mère se détourna de l'évier pour le regarder interloquée. «~Je… je ne sais pas, je crois qu'il faut avoir sa propre chouette magique.~»

Cela aurait dû éveiller des soupçons chez Harry. \emph{Ah, donc il n'y a aucun moyen de tester ta théorie~?} Mais l'étrange certitude en Harry semblait lui avoir mis des œillères.

«~Bon, la lettre est bien arrivée jusqu'ici, dit Harry, alors je vais l'agiter dehors en appelant ``Lettre pour Poudlard~!'' pour voir si une chouette vient la ramasser. Papa, tu veux venir regarder~?~»

Son père secoua minutieusement la tête et continua sa lecture.  \emph{Bien sûr}, songea Harry. La magie était une de ces croyances honteuses à laquelle seules des personnes stupides pouvaient croire~; si son père s'abaissait à \emph{tester} l'hypothèse, ou même simplement \emph{observer} le test, ce serait déjà \emph{s'associer} à celle-ci…

Ce n'est qu'en sortant par la porte de derrière menant au jardin qu'Harry réalisa que, si jamais une chouette venait \emph{vraiment} récupérer la lettre, il aurait bien du mal à convaincre son père.

\emph{Mais… Bon… ça ne peut pas} réellement \emph{arriver, non~? Quoi que mon cerveau ait l'air de penser, si une chouette vient vraiment prendre cette enveloppe, je vais avoir des soucis beaucoup plus sérieux que l'opinion de papa.}

Harry inspira profondément et leva l'enveloppe en l'air.

Il déglutit.

S'écrier \emph{Lettre pour Poudlard~!,} au milieu du jardin, avec une enveloppe au bout de son bras tendu était assez… embarrassant, réflexion faite.

\emph{Non. Je vaux mieux que papa. J'utiliserai la méthode scientifique, même si je me sens stupide.}

«~Lettre…~» commença Harry, mais une espèce de croassement sortit de sa gorge.

Il rassembla son courage et cria vers le ciel vide~: «~\emph{Lettre pour Poudlard~! Est-ce que je peux avoir une chouette~?}~»

«~Harry~?~» fit une voix de femme déconcertée. C'était l'une des voisines.

Harry baissa sa main plus rapidement que s'il l'avait mise dans un feu et cacha l'enveloppe dans son dos comme un dealer de drogue pris sur le fait, le visage rouge de honte.

Le visage d'une vieille femme émergea par-dessus la clôture, des cheveux grisonnants s'échappaient de la résille qu'elle portait sur la tête.  C'était Mme Figg, qui venait parfois faire du baby-sitting. «~Qu'est-ce que tu fais, Harry~?~»

--- Rien du tout, dit-il d'une voix étranglée, je… je teste juste une théorie idiote…

--- Est-ce que Poudlard t'a envoyé ta lettre d'admission~?~»

Harry se figea sur place.

«~Oui~», répondirent ses lèvres quelques instants plus tard.  «~J'ai reçu une lettre de Poudlard. Ils disent qu'ils veulent avoir reçu ma chouette avant le 31 juillet, mais…

--- Mais tu n'as \emph{pas} de chouette. Pauvre petit~!  Mais à quoi \emph{pensent}-t-ils, pour t'envoyer la lettre standard.~»

Un bras ridé passa par-dessus la clôture et tendit la main. À peine capable de réfléchir, Harry lui donna l'enveloppe.

«~Laisse la moi, mon petit, dit Mme Figg, et je ferai venir quelqu'un en un rien de temps.~»

Et son visage disparut derrière la clôture.

Après un très long silence dans le jardin, on entendit la voix d'un jeune garçon sortant de sa torpeur~: «~pardon~?~»
