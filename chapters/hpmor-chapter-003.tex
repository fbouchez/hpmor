\chapter{Comparer la réalité à ses alternatives}

\lettrine[ante=«~]{S}{eigneur} Dieu, dit le barman en dévisageant Harry, est-ce… se pourrait-il que…~?~»

Harry se pencha par-dessus le bar du Chaudron Baveur du mieux qu'il pu, le zinc atteignant à peu près le haut de ses sourcils. Une question \emph{pareille} méritait qu'il donne le meilleur de lui-même.

«~Suis-je… pourrais-je être… peut-être… sait-on jamais… si je \emph{ne} suis \emph{pas}… mais alors la question est… \emph{qui~?}

--- Bénie soit mon âme, murmura le vieux barman, Harry Potter… quel honneur.~»

Harry cligna des yeux puis se reprit.  «~Eh bien, oui, vous êtes perspicace~; la plupart des gens ne s'en rendent pas compte si vite…

--- Ça suffit~», dit McGonagall.  Sa main se resserra sur l'épaule de Harry. «~Laisse le garçon tranquille, Tom, tout cela est nouveau pour lui.

--- Mais c'est lui~? chevrota une vieille femme, c'est Harry Potter~?~» Elle se leva en faisant racler sa chaise.

«~Doris…~» avertit McGonagall, tout en balayant la pièce d'un regard suffisant pour intimider n'importe qui.

«~Je veux seulement lui serrer la main~» murmura la vieille femme. Elle s'inclina profondément et brandit une main ridée. Harry, se sentant plus confus et gêné qu'il ne l'avait jamais été auparavant, la serra avec précaution.  Des larmes coulèrent des yeux de la vieille femme sur leurs mains serrées. «~Mon petit-fils était un Auror,~lui murmura-t-elle,~mort en soixante-dix-neuf. Merci, Harry Potter. Loué soit le ciel.

--- De rien~», dit Harry, passé en pilote automatique, puis il tourna la tête vers McGonagall et lui jeta un regard à la fois terrifié et implorant.

McGonagall frappa le sol du pied tuant la ruée dans son œuf. Le bruit produit remplaça pour Harry la référence précédente qu'il avait pour la définition «~claquement du tonnerre~», et tous se figèrent.

«~Nous sommes pressés,~» dit McGonagall d'une voix parfaitement, absolument normale.

Ils sortirent du bar sans difficulté.

«~Professeure~?~» dit Harry, une fois qu'ils furent dans l'arrière-cour. Son intention était de lui demander ce qu'il s'était passé, mais il s'entendit étrangement poser une toute autre question~: «~Qui était l'homme dans l'angle du bar~? Celui très pâle et qui clignait de l'œil~?

--- Hmm~?~» dit McGonagall, l'air un peu étonnée~; peut-être qu'elle non plus ne s'était pas attendue à cette question. «~C'était le Professeur Quirinus Quirrell, qui enseignera le cours de défense contre les forces du mal à Poudlard cette année.

--- J'ai eu une sensation des plus étranges, comme si je le connaissais… Harry se frotta le front, … et qu'il ne fallait pas que je lui serre la main.~» Comme rencontrer quelqu'un qui avait été votre ami, autrefois, avant que quelque chose ne tourne radicalement mal… ce n'était à vrai dire pas ça du tout, mais Harry n'arrivait pas à trouver les bon mots. «~Et c'était quoi… tout le reste~?~»

McGonagall lui jeta un coup d'œil étrange. «~M.  Potter… savez-vous… que vous a-t-on dit \emph{au juste}… sur la façon dont vos parents sont morts~?~»

Harry lui renvoya un regard ferme. «~Mes parents sont vivants et bien portants, merci, et ils ont toujours refusé de me parler de la façon dont mes parents \emph{génétiques} sont morts, ce dont je déduis que cela ne devait pas être glorieux.

--- Une loyauté admirable,~» dit McGonagall. Sa voix se fit plus basse~: «~Mais cela me peine de vous entendre parler ainsi. Lily et James étaient des amis.~»

Soudain honteux, Harry détourna le regard. «~Je suis navré, dit-il d'une petite voix, mais j'\emph{ai} un papa et une maman. Et je sais que je ne ferais que me rendre malheureux en comparant la réalité à… quelque chose de parfait qui serait le fruit de mon imagination.

--- C'est étonnamment sage de votre part, dit McGonagall avec douceur, mais vos parents \emph{génétiques} sont morts admirablement en réalité~; en vous protégeant.~»

\emph{En me protégeant~?}

Quelque chose d'étrange serra le cœur de Harry.  «~Que… \emph{s'est-il} passé~?~»

McGonagall soupira. Elle tapota le front de Harry de sa baguette, et sa vision se brouilla un instant. «~Une sorte de déguisement, dit McGonagall, pour que cela ne se reproduise plus, pas avant que vous ne soyez prêt.~» Puis sa baguette virevolta pour aller donner trois petits coups sur un mur en briques…

… où se creusa un trou qui se dilata, s'étira en tremblant pour devenir une immense arche, révélant une longue avenue remplie de magasins et panneaux publicitaires vantant chaudrons et foies de dragons.

Harry ne cligna même pas des yeux. Ce n'était pas comme si quelqu'un venait de se transformer en chat.

Et ils s'avancèrent tous deux dans le monde des sorciers.

On trouvait des bonimenteurs vendant à la criée des Bottes Rebondissantes («~avec du vrai Plaxmol~!~»), des «~couteaux +3~!  fourchettes +2~! et cuillères avec un bonus de +4~!~» Des lunettes qui transforment en vert tout ce que vous regardez, et toute une gamme de confortables fauteuils de salon avec sièges éjectables pour les urgences.

La tête de Harry tournait, tournait sans pouvoir s'arrêter, comme si elle essayait de se dévisser de son cou. C'était comme de déambuler dans la section objets magiques du livre de règles avancées de Donjons \& Dragons (il ne jouait pas au jeu, mais il aimait lire les livres de règles). Harry voulait être sûr de ne rater aucun des objets en vente, au cas où ce serait l'un des trois requis pour enclencher le cycle des \emph{vœux} infinis.

Puis Harry remarqua quelque chose qui le fit inconsciemment s'écarter de la directrice adjointe pour se diriger droit vers un magasin, la devanture faite de briques bleues aux contours couleur bronze. Il fallut que McGonagall se campe juste devant lui pour que Harry revienne à la réalité.

«~M. Potter~?~» dit-elle.

Harry cligna des yeux, puis se rendit compte de ce qu'il venait de faire. «~Désolé~! J'ai oublié pendant un moment que j'étais avec vous et non ma famille.~» Harry esquissa un geste en direction de la vitrine du magasin, arborant en lettres flamboyantes d'une clarté aveuglante et pourtant lointaine~: \emph{Livres Lumineux de Libam}. «~Lorsqu'on passe devant une librairie qu'on l'on ne connaît pas encore, on doit rentrer et jeter un coup d'œil. C'est la règle dans ma famille.

--- C'est la chose la plus Serdaigle que j'ai jamais entendue.

--- Quoi~?

--- Rien. M. Potter, notre première étape sera Gringotts, la banque des sorciers. La chambre forte de votre famille \emph{génétique} s'y trouve, ainsi que l'héritage que vos parents \emph{génétiques} vous ont laissé. Vous allez avoir besoin d'argent pour vos fournitures scolaires, soupira McGonagall, et je suppose qu'un peu d'argent de poche pour acheter quelques livres est envisageable. Cela dit je vous recommande de ne pas faire d'achat compulsif, Poudlard a une bibliothèque très conséquente sur toutes les disciplines de la magie. Et la tour dans laquelle je soupçonne fortement que vous allez vivre possède sa propre bibliothèque plus généraliste. Tout livre que vous achèterez ici sera probablement un doublon.~»

Harry hocha la tête, et ils reprirent leur chemin.

«~Ne vous méprenez pas, c'est une~\emph{excellente} diversion,~» dit Harry, alors que sa tête continuait de pivoter en tous sens, «~probablement la meilleure diversion qu'on ait jamais essayée sur moi, mais ne croyez pas que j'ai oublié notre discussion laissée en suspens.~»

McGonagall soupira. «~Vos parents… votre mère tout du moins… a peut-être été fort sage de ne rien vous dire.

--- Et vous souhaitez que je reste dans cette ignorance bienheureuse~?  Votre plan possède une faille évidente, professeure McGonagall.

--- J'imagine que ce serait assez futile, dit la sorcière d'un air pincé, vu que n'importe quel passant pourrait vous raconter cette histoire. Allons-y.~»

Et elle lui parla de Celui-Dont-On-Ne-Doit-Pas-Prononcer-Le-Nom, le Seigneur des Ténèbres, Voldemort.

«~Voldemort~?~» murmura Harry. Ça aurait dû être amusant, mais ça ne l'était pas. Le nom brûlait avec la clarté d'un diamant, froid, impitoyable, tel un marteau de titane pur s'abattant sur une enclume de chair sans défense. Un frisson parcouru Harry alors qu'il prononçait le nom, et il décida à l'instant même d'utiliser des termes plus sûrs, comme Vous-Savez-Qui.

Le Seigneur des Ténèbres avait mis l'Angleterre magique à feu et à sang, tel un loup enragé, déchirant, déchiquetant la trame de leur vie quotidienne. Les autres pays ne savaient pas sur quel pied danser, hésitant à intervenir, que ce soit par indifférence égoïste ou par peur, car le premier d'entre eux à s'opposer au Seigneur des Ténèbres verrait sa paix devenir la prochaine cible de sa terreur.

(\emph{L'effet du témoin}, pensa Harry, songeant à l'expérience de Latané et Darley qui avait montré que vous aviez plus de chances de recevoir de l'aide si vous faisiez une crise d'épilepsie en présence d'une seule personne plutôt que de trois.  \emph{Diffusion de la responsabilité, chacun espérant que quelqu'un d'autre agisse en premier.})

Les Mangemorts avaient suivi dans le sillage du Seigneur des Ténèbres, et formaient son avant-garde, vautours charognards qui rouvraient les blessures, et serpents pour mordre et affaiblir. Les Mangemorts n'étaient pas aussi terrifiants que le Seigneur des Ténèbres, mais ils étaient terribles~; et ils étaient nombreux. Par ailleurs, les Mangemorts maniaient plus que des baguettes~; ces troupes masquées possédaient richesse, pouvoir politique, secrets et maîtres-chanteurs, tout pour paralyser une société essayant de se protéger.

Un journaliste âgé et respecté, Yermy Wibble, avait plaidé pour une hausse des taxes et une conscription forcée, criant haut et fort qu'il était absurde que la majorité se terre, par peur d'une minorité. Sa peau, uniquement sa peau, avait été retrouvée clouée au mur de la rédaction le lendemain matin, à côté des peaux de sa femme et de ses deux filles. Tout le monde priait pour que quelque chose soit fait, et personne n'osait prendre l'initiative. Le prochain à se démarquer deviendrait le prochain exemple.

Jusqu'au jour où les noms de Lily et James Potter atteignirent le haut de la liste.

Et ces deux-là auraient pu mourir la baguette à la main sans un regret, car \emph{c'étaient} des héros~; mais ils avaient un nouveau-né, leur fils, Harry Potter.

Les larmes montaient aux yeux de Harry. Il les essuya avec colère, ou peut-être désarroi, \emph{Je ne connaissais pas ces gens, pas vraiment, ce ne sont} plus \emph{mes parents à présent, cela n'aurait aucun sens d'être triste pour
eux}…

Harry fondit en larmes dans les bras de la sorcière puis releva la tête.  Voir que des larmes perlaient aussi à ses paupières l'aida à se sentir un peu mieux.

«~Le Seigneur des Ténèbres s'est rendu à Godric's Hollow, murmura McGonagall, vous auriez dû être cachés, mais on vous a trahi. Le Seigneur des Ténèbres a tué James, puis Lily, et enfin il s'est approché de vous, dans votre berceau. Il vous a jeté le sortilège de la Mort, et c'est là que tout a fini. Le sortilège de la Mort est fait de haine pure, et frappe directement l'âme en la séparant du corps. Il ne peut être contré, et quiconque est frappé meurt sur-le-champ. Mais vous avez survécu. Vous êtes la seule personne à avoir jamais survécu. Le sortilège de la Mort a rebondi et frappé le Seigneur des Ténèbres, ne laissant de son corps qu'une carcasse brûlée et pour vous une cicatrice sur le front.  C'était la fin de la terreur, et nous étions libres.  Voilà pourquoi les gens veulent voir votre cicatrice, Harry Potter, et pourquoi ils veulent vous serrer la main.~»

Le torrent de pleurs qui s'était déversé en Harry l'avait vidé de toutes ses larmes. Il ne pouvait pleurer une goutte de plus~; il avait fini.

(Et quelque part, enfoui au fond de sa tête, se trouvait un léger, très léger sentiment de confusion, l'idée que quelque chose dans cette histoire ne collait pas~; tout l'art de Harry aurait dû être de remarquer ce sentiment, mais il était distrait. Car c'est malheureusement presque une règle que c'est lorsqu'on a le plus besoin de son art de rationaliste qu'on est le plus à même de l'oublier.)

Harry se détacha du flanc de McGonagall. «~Je vais… avoir besoin d'y réfléchir,~» dit-il, essayant de maintenir sa voix sous contrôle. Il fixa le bout de ses chaussures. «~Euh, c'est d'accord vous pouvez les appeler mes parents, si vous le souhaitez, vous n'avez pas à dire ``parents génétiques'' ou quoi que ce soit. Je ne vois pas pourquoi je ne pourrais pas avoir deux mères et deux pères.~»

McGonagall restait silencieuse.

Et ils marchèrent ensemble en silence, jusqu'à se trouver devant un grand bâtiment blanc aux vastes portes de bronze et sur lesquelles était gravé \emph{Banque Gringotts}.
