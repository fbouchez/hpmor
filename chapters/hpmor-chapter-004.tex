\chapter[L'hypothèse du marché efficient]{L'hypothèse du marché efficient\protect\footnotemark}
\authorsnotetext{comme certains l'ont fait remarquer, les livres originaux sont inconsistents pour ce qui est du pouvoir d'achat d'un gallion~; je vais choisir une valeur fixe et m'y tenir, même si cinq livres sterling pour un gallion ne colle pas avec sept gallions pour une baguette, ni avec des enfants utilisant des baguettes d'occasion.}

\lettrinepara{D}{es} monceaux de gallions d'or. Des tas de mornilles d'argent. Des amas de noises de bronze.

\hplettrineextrapara
Harry se tenait là, bouche ouverte, à contempler la chambre forte familiale. Il avait tant de questions qu'il ne savait pas par \emph{où} commencer.

Depuis l'ouverture de la porte de la chambre forte, McGonagall le regardait. Elle paraissait négligemment adossée contre le mur, mais ses yeux trahissaient sa tension. Bon, rien de surprenant. Se retrouver plongé devant un énorme tas de pièces d'or constituait un test de personnalité si épuré que c'en était un archétype.

«~Ces pièces sont-elles faites de métal pur~? dit finalement Harry.

--- Quoi~? siffla le gobelin Gripsec qui attendait près de la porte. Remettez-vous en question l'intégrité de Gringotts, M.  Potter-Evans-Verres~?

--- Non, dit Harry d'un air absent, pas du tout, pardon, je me suis mal exprimé. C'est juste que je n'ai aucune idée de comment fonctionne votre système financier. Je demandais si les gallions en général sont faits d'or pur.

--- Bien sûr, dit Gripsec.

--- Et n'importe qui peut les frapper, ou sont-ils produits par un monopole qui collecte ainsi un seigneuriage~?

--- Pardon~?~» dit McGonagall.

Gripsec eut un sourire carnassier, révélant des dents acérées. «~Seul un idiot ferait confiance à de la monnaie non gobeline~!

--- En d'autres termes, dit Harry, les pièces ne sont pas supposées valoir plus que le métal dont elles sont faites~?~»

Gripsec fixait Harry du regard. McGonagall semblait déconcertée.

«~Je veux dire… imaginez que j'arrive ici avec une tonne d'argent. Pourrais-je faire fabriquer une tonne de mornilles~?

--- Contre rémunération, M. Potter-Evans-Verres. Le gobelin l'observait avec des yeux scintillants. Il y aurait une commission. Je me demande… où trouveriez-vous une tonne d'argent~?

--- Je parlais hypothétiquement~», dit Harry. \emph{Pour l'instant, tout du moins.} «~Donc… à combien s'élèverait la commission, en fraction du poids total~?~»

Gripsec fixait Harry intensément. «~Je devrais consulter mes supérieurs…

--- Donnez-moi une estimation, sans que cela engage Gringotts.

--- Un vingtième du métal paierait pour la frappe des pièces.~»

Harry hocha la tête. «~Merci beaucoup, M. Gripsec.~»

\emph{Alors non seulement l'économie des sorciers est presque totalement découplée de l'économie moldue, mais personne ici n'a jamais entendu parler d'arbitrage.} L'économie moldue, plus grande, a un taux d'échange fluctuant entre l'or et l'argent, si bien que chaque fois que le taux entre or et argent des moldus s'écarte de plus de 5~\% du poids de dix-sept mornilles par gallion, l'or ou l'argent aurait dû être drainé hors de l'économie des sorciers jusqu'à ce qu'il devienne impossible de maintenir ce taux de change. Apportez une tonne d'argent, échangez-la contre des mornilles (et payez 5~\%), échangez les mornilles contre des gallions, amenez l'or dans le monde moldu, échangez-le contre plus d'argent que ce que vous aviez au départ, et recommencez.

Le taux de change or-argent des Moldus n'était-il pas aux environs de cinquante pour un~? En tout cas, ce n'était certainement pas dix-sept pour un, pensait Harry. Et il lui semblait que les pièces d'argent étaient de surcroît \emph{plus petites} que les pièces d'or.

Mais après tout, Harry se trouvait dans une banque qui stockait \emph{littéralement} votre argent dans des chambres fortes pleines de pièces d'or, gardées par des dragons, dans laquelle vous deviez vous déplacer et récupérer de la monnaie à chaque fois que vous souhaitiez dépenser de l'argent. Des détails tels que réduire l'inefficacité des marchés grâce à l'arbitrage leur passait probablement au-dessus de la tête. Il fut tenté de railler la grossièreté de leur système financier…

\emph{Mais ce qui était triste, c'était que leur façon de faire était probablement meilleure.}

D'un autre côté, un gestionnaire de portefeuille financier un peu malin pourrait probablement devenir propriétaire de la totalité du monde magique en moins d'une semaine. Harry rangea cette idée quelque part dans un coin de sa tête, au cas où il finirait par manquer d'argent, ou se retrouverait avec une semaine de libre.

En attendant, les montagnes d'or de la chambre forte Potter devraient répondre à ses besoins à court terme.

Harry s'avança et commença à ramasser des pièces d'or d'une main pour les déposer dans l'autre.

Lorsqu'il en fut arrivé à vingt, McGonagall toussota.  «~Je pense que ce sera bien plus qu'assez pour payer vos fournitures scolaires, M. Potter.

--- Hmm~? dit Harry, l'esprit ailleurs. Ne bougez pas, je fais une estimation de Fermi.

--- Une \emph{quoi}~? dit McGonagall, un peu alarmée.

--- C'est un truc de math. Nommé d'après Enrico Fermi. Une façon d'obtenir des résultats approximatifs de tête très rapidement…~»

Vingt gallions d'or pesaient peut-être un dixième de kilo. Et l'or valait, quoi, dix mille livres britanniques au kilo~? Un gallion valait donc environ cinquante livres… Les monticules de pièces d'or semblaient faire environ soixante pièces de haut et vingt pièces dans les autres dimensions de la base, et avaient une forme pyramidale, donc ce serait environ un tiers du cube. Huit mille gallions par tas, environ, et il y avait cinq tas de cette taille, soit quarante mille gallions, équivalents à deux millions de livres sterling.

Pas mal. Harry eut un sourire satisfait mais maussade. Dommage qu'il soit tout juste en train de découvrir le nouveau monde extraordinaire de la magie, et qu'il ne puisse pas prendre le temps d'explorer le nouveau monde extraordinaire de la richesse, qu'une rapide estimation de Fermi avait évalué être environ un milliard de fois moins intéressant.

\emph{Quand même, c'est la dernière fois que je tonds la pelouse pour une pauvre livre.}

Harry tourna le dos à l'immense tas d'or. «~Excusez-moi de poser la question, professeure McGonagall, mais il me semble que mes parents n'avaient pas encore trente ans lorsqu'ils sont morts. Est-ce une somme \emph{habituelle} d'argent à avoir dans sa chambre forte pour un jeune couple de sorciers~?~» Si c'était le cas, alors une tasse de thé coûtait probablement cinq mille livres.  Règle numéro un de l'économie~: vous ne pouvez pas manger l'argent.

McGonagall secoua la tête. «~Votre père était le dernier héritier d'une ancienne famille, M. Potter. Il est aussi possible… McGonagall hésita.~Une partie de cet argent pourrait provenir des récompenses pour la tête de Vous-Savez-Qui, payable à qui le tu… heu… à qui le vaincrait. Ou peut-être ces récompenses n'ont-elles pas encore été récupérées, je ne suis pas sûre.

--- Intéressant… dit lentement Harry, donc une partie de ceci est à moi en un sens. Je veux dire, gagné par moi. En quelque sorte.  Peut-être. Même si je ne m'en souviens pas.~» Les doigts de Harry tapotaient contre la jambe de son pantalon. «~Je me sens moins coupable à l'idée d'en dépenser \emph{une vraiment toute petite fraction~! Ne paniquez pas, professeure McGonagall~!}

--- M. Potter~! Vous êtes mineur, et en tant que tel, vous ne serez autorisés qu'à faire des retraits \emph{raisonnables} de…

--- Je suis \emph{entièrement} pour ce qui est raisonnable~! Je suis complètement favorable à la prudence fiscale et le contrôle de ses impulsions~! Mais j'ai \emph{remarqué} quelques petites choses en chemin qui constitueraient des achats \emph{sensés et matures}…~»

Harry verrouilla son regard à celui de McGonagall, s'engageant dans un duel silencieux.

«~Comme quoi~? dit finalement McGonagall.

--- Des malles dont l'intérieur est plus que grand que leur extérieur~?~»

Le visage de McGonagall devint sévère. «~Ces malles sont \emph{très} onéreuses, M. Potter~!

--- Oui mais… plaida Harry, je suis sûr que j'en voudrais une quand je serai adulte. Et je \emph{peux} me le permettre.  En toute logique, il serait plus sensé d'en acheter une maintenant que plus tard, et d'en avoir l'usage immédiatement. C'est la même quantité d'argent dans un cas comme dans l'autre, n'est-ce pas~? Je veux dire, il m'en \emph{faudrait} une de bonne qualité, avec \emph{beaucoup} de place à l'intérieur, suffisamment pour que je n'ai pas besoin d'en racheter une meilleure plus tard…~» Harry laissa sa phrase en suspens, plein d'espoir.

Le regard de McGonagall ne vacilla pas. «~Et que \emph{conserveriez-vous} au juste dans une telle malle, M. Potter…

--- Des livres.

--- Bien sûr, soupira McGonagall.

--- Vous auriez dû me dire \emph{bien plus tôt} que ce genre d'objet magique existait~! Et que je pouvais me le payer~! Maintenant mon père et moi allons devoir passer les deux prochains jours à parcourir \emph{frénétiquement} toutes les librairies de livres d'occasion à la recherche de vieux manuels scolaires pour avoir une bibliothèque de sciences décente avec moi à Poudlard… et peut-être un petit rayon de science-fiction, si je peux constituer quelque chose de convenable à prix bradés. Ou encore mieux, vous n'allez pas regretter mon offre vous allez voir, laissez-moi juste acheter…

--- \emph{M. Potter~!} Vous pensez pouvoir me \emph{soudoyer}~?

--- Quoi~? \emph{Non}~! Pas comme ça~! Ce que je veux dire c'est que Poudlard pourra garder certains des livres que j'apporterai si vous pensez qu'il pourraient étoffer la bibliothèque. Je vais les acheter pour pas cher, et \emph{je} veux juste qu'ils me soient accessibles.  C'est acceptable de soudoyer les gens avec des livres, non~? C'est une…

--- Tradition familiale.

--- Oui, exactement.~»

McGonagall sembla s'affaisser, abaissant les épaules dans ses vêtements de sorcière. «~J'ai bien peur de ne pouvoir contredire la logique de votre discours, et pourtant j'aurais bien voulu. Je vais vous autoriser à retirer cent gallions de plus, M. Potter. Elle soupira de nouveau. Je \emph{sais} que je vais le regretter, et je le fais quand même.

--- En voilà du caractère~! Et est-ce qu'une “bourse en peau de Moke” fait bien ce que je pense~?

--- Pas autant qu'une malle, dit McGonagall visiblement réticence, mais une bourse en peau de Moke avec un sort de Récupération et un sort d'Extension Indétectable peut contenir un nombre conséquent d'objets qui peuvent ensuite être rappelés par celui qui les y a rangés…

--- Extra~! J'ai absolument besoin d'une de ces bourses. Ce serait le super sac banane ultime de la génialitude~!  La ceinture-gadget sans fond de Batman~! Oubliez le couteau suisse, je pourrais transporter toute une caisse à outils là-dedans~! Ou des \emph{livres}~! Je pourrais avoir mes trois meilleurs livres en cours de lecture sur moi, tout le temps, et en piocher un n'importe où~! Je n'aurais plus jamais à gâcher une seule minute de ma vie~! Qu'en pensez-vous, professeure McGonagall~? Nous parlons de la lecture d'un enfant, la meilleure des raisons possibles.

--- … je suppose que vous pouvez rajouter dix gallions.~»

Gripsec couvait Harry d'un regard empreint de respect, voire de véritable admiration.

«~Et un peu d'argent de poche, comme vous l'avez mentionné plus tôt. Je crois pouvoir me rappeler d'une ou deux autres choses qui auraient tout à fait leur place dans cette bourse.

--- \emph{N'abusez pas, M. Potter.}

--- Oh, mais professeure McGonagall, pourquoi inviter la pluie à la fête~?  Aujourd'hui est assurément un \emph{jour heureux}, celui où je découvre toutes les choses magiques pour la première fois~!  Pourquoi prendre le rôle de l'adulte grognon alors que vous pourriez plutôt sourire et vous remémorer votre enfance innocente, vous attendrissant de voir le ravissement sur mon jeune visage tandis que j'achète quelques jouets, n'utilisant qu'une fraction insignifiante de la fortune que j'ai gagné en battant le plus terrible sorcier que l'Angleterre ait jamais connu, non que je vous accuse d'être ingrate ou quoi que ce soit, mais tout de même, que sont quelques jouets comparés à ça~?

--- \emph{Vous}~», gronda McGonagall. Elle avait un regard si redoutable et terrifiant que Harry couina, fit un pas en arrière, renversa une pile de pièces d'or avec grand fracas et finit par s'étaler dans un tas de gallions. Gripsec soupira et se couvrit le visage de la main. «~Je rendrais un grand service à l'Angleterre magique, M. Potter, et peut-être au monde entier, si je vous enfermais dans cette chambre forte et que je vous laissais ici.~»

Et ils quittèrent la banque sans plus de perturbation.
