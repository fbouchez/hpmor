\chapter{Localiser l'hypothèse}

\lettrine{J}{eudi}.

\hplettrineextrapara
Si on voulait être précis, 7h24 un jeudi matin.

Harry était assis sur son lit, un manuel mollement étalé entre ses mains immobiles.

Harry venait tout juste d'avoir eu l'idée d'un test expérimental \emph{vraiment brillant}.

Ce qui voulait dire qu'il lui faudrait attendre une heure de plus avant d'aller prendre le petit-déjeuner, mais c'était pour ça qu'il avait des barres énergétiques. Non, cette idée devait absolument définitivement être testée tout de suite, immédiatement, maintenant.

Harry écarta les manuels, bondit hors de son lit, courut autour de celui-ci, fit jaillir le niveau caverne de sa malle, descendit les escaliers à la cavalcade, et commença à déplacer ses boites en tous sens (Il fallait vraiment qu'il trouve le temps de défaire ses boites, mais il était en plein milieu de son concours de lecture de manuels avec Hermione et il prenait du retard alors il n'avait pas le temps).

Harry trouva le livre qu'il cherchait et courut en haut des escaliers.

Les autres garçons se préparaient à aller prendre leur petit-déjeuner dans la Grande Salle et à démarrer la journée.

«~Excusez-moi, vous pourriez faire quelque chose pour moi~?~» dit Harry. Tout en parlant, il faisait défiler l'index du livre, trouva la page qui comportait les dix-milles premiers nombres premiers, ouvrit le livre à cette page et le fourra dans les mains d'Anthony Goldstein. «~Choisis deux nombres à trois chiffres dans la liste. Ne me dis pas lesquels, multiplie-les juste, et donne-moi leur produit. Oh, et pourrais-tu faire le calcul deux fois, pour vérifier~? S'il te plaît, assure-toi vraiment que tu as la bonne réponse, je ne suis pas certain de ce qui arrivera à moi ou à l'univers si tu fais une erreur de multiplication.~»

Ce qui en disait beaucoup sur ce qu'était devenue la vie dans ce dortoir en l'espace de quelques jours, c'était qu'Anthony ne prenne même pas la peine de dire «~Mais qu'est-ce qui te prend tout d'un coup~?~» ou «~Ça a l'air vraiment bizarre, pourquoi est-ce que tu me demandes de faire ça~?~» ou «~Qu'est-ce que tu veux dire par “je ne suis pas sûr de ce qui arrivera à l'univers”~?~».

Anthony accepta le livre en silence et sortit un parchemin et une plume. Harry se retourna et ferma les yeux, s'assurant de ne rien voir, se balançant d'avant en arrière et de haut en bas avec impatience. Il s'empara d'un bloc-note et d'un critérium et se prépara à écrire.

«~C'est bon, dit Anthony, cent-quatre-vingt-et-un-mille-quatre-cents-vingt-neuf.«~

Harry écrivit 181~429. Il répéta ce qu'il venait d'écrire, et Anthony confirma.

Puis Harry courut jusqu'au niveau caverne de sa malle, jeta un coup d'œil à sa montre (la montre indiquait 4h28, ce qui voulait dire 7h28) puis ferma les yeux.

Environ trente secondes plus tard, Harry entendit des bruits de pas, puis le bruit du tiroir du niveau caverne de la malle qui se refermait (Harry n'avait pas peur de suffoquer. Un charme de Rafraîchissement d'Air automatique faisait partie des avantages qu'il y avait à acheter une malle de très bonne qualité. La magie était vraiment formidable, pas besoin de se soucier des factures d'électricité).

Et lorsque Harry ouvrit les yeux, il vit exactement ce qu'il avait espéré voir, un bout de papier plié posé au sol~; le cadeau de son futur lui.

Appelons ce bout de papier «~Papier-2~».

Harry déchira un bout de papier de son bloc-notes.

Appelons ce bout de papier «~Papier-1~». C'était, bien sûr, le même bout de papier. On pouvait même voir, si on y regardait de près, que les bords déchirés correspondaient.

Harry passa mentalement en revue l'algorithme qu'il allait suivre.

Si Harry ouvrait Papier-2 et qu'il était vierge, alors il écrirait «~101~$\times$~101~» sur Papier-1, le plierait, étudierait pendant une heure, reviendrait en arrière dans le temps, déposerait Papier-1 (qui deviendrait ainsi Papier-2), et se dirigerait hors du niveau caverne pour rejoindre ses camarades de dortoir au petit-déjeuner.

Si Harry ouvrait Papier-2 et qu'il y avait deux nombres écrits dessus, Harry multiplierait ces deux nombres entre eux.

Si leur produit était égal à 181429, Harry écrirait ces deux nombres sur Papier-1 et renverrait Papier-1 en arrière dans le temps.

Sinon, Harry ajouterait 2 au nombre de droite et écrirait la nouvelle paire de nombres sur Papier-1. À moins que cela ne rende le nombre de droite supérieur à 997, auquel cas Harry ajouterait 2 au nombre de gauche et écrirait 101 à droite.

Et si Papier-2 disait 997~$\times$~997, Harry laisserait Papier-1 vide.

Ce qui voulait dire que la seule boucle temporelle \emph{stable} possible était celle où Papier-2 contenait les deux facteurs premiers de 181429.

Si ça fonctionnait, Harry pourrait utiliser cette technique pour obtenir n'importe quelle sorte de réponse facile à vérifier mais difficile à trouver. Il n'aurait pas \emph{seulement} montré que P=NP quand on avait un Retourneur de Temps, non, la technique était \emph{plus générale} que cela. Harry pourrait l'utiliser pour trouver des combinaisons de cadenas ou n'importe quel genre de mot de passe. Ou même peut-être trouver l'entrée de la Chambre des Secrets de Serpentard, si Harry pouvait deviser une façon systématique de décrire tous les emplacements de Poudlard. Ce serait une technique de triche impressionnante, même comparé aux standards de triche de Harry.

Harry prit Papier-2 dans ses mains tremblantes, et le déplia.

Papier-2 disait, d'une écriture légèrement irrégulière~:

\shout{Ne joue pas avec le temps}

Harry écrivit «~\shout{Ne joue pas avec le temps}~» sur Papier-1 d'une écriture légèrement irrégulière, le plia avec soin, et se résolut à ne plus faire d'expérience vraiment brillante sur le Temps avant d'avoir au moins quinze ans.

Pour autant que Harry le sache, ça avait été le résultat expérimental le plus effrayant de toute l'histoire de la science.

Harry avait eut quelques difficultés à se concentrer sur la lecture de son manuel pendant l'heure suivante.

Et c'est ainsi que le jeudi de Harry avait commencé.

\later

Jeudi.

Si on voulait être précis, 15h32 un jeudi après-midi.

Harry et tous les autres garçons de première année étaient dehors, sur un champ herbeux, avec Madame Bibine, juste à côté de la réserve de balais de Poudlard. Les filles apprendraient à voler séparément. Apparemment, pour une raison inconnue, les filles ne voulaient pas apprendre à voler sur des balais en présence de garçons.

Harry avait été un peu vacillant pendant la journée. Il était incapable d'arrêter de penser à la façon dont cette boucle temporelle stable \emph{et pas une autre} avait bien pu être sélectionnée à partir de ce qui, rétrospectivement, semblait être un large espace de possibilités.

Et aussi~: sérieusement, des \emph{balais}~? Il allait voler sur ce qui était essentiellement un segment~? N'était-ce pas quasiment la forme la plus instable qu'on puisse jamais trouver, à moins d'essayer de se tenir sur une bille~? Qui avait sélectionné \emph{ce} modèle d'engin volant, parmi toutes les formes possibles~? Harry avait espéré que ce ne soit qu'une figure de style, mais non, ils se tenaient devant ce qui ressemblait indiscutablement à d'ordinaires balais de cuisine en bois. Quelqu'un s'était-il juste retrouvé bloqué sur l'idée des balais et n'avait pas pensé à envisager autre chose~? Ça devait être ça. Il n'y avait aucune chance pour que, si on les développait indépendamment l'un de l'autre, les modèles \emph{optimaux} permettant de nettoyer une cuisine et ceux permettant de voler dans les airs se révèlent coïncider exactement.

C'était une belle journée, avec un ciel bleu clair et un soleil éclatant qui ne demandait qu'à entrer dans vos yeux et à vous empêcher de voir quoi que ce soit, s'il se trouvait que vous étiez en train d'essayer de voler dans le ciel. Le champ était propre et sec, sentait presque le brûlé, et semblait étrangement être très, très dur sous les chaussures de Harry.

«~Levez votre main droite au-dessus du balai, ou la main gauche si vous êtes gaucher, s'écria Madame Bibine. Et dites \shout{Debout}~!~»

Tout le monde cria «~\shout{Debout}~!~»

Le balai bondit d'impatience jusqu'à la main de Harry.

Ce qui fit de lui le premier de la classe, pour une fois. Apparemment dire «~\shout{Debout}~!~» était beaucoup plus difficile que ça en avait l'air, et la plupart des balais roulaient partout sur l'herbe ou essayaient de s'écarter discrètement de leur soi-disant futur conducteur.

(Bien sûr, Harry aurait été prêt à parier de l'argent sur le fait qu'Hermione avait réussi au moins aussi bien que lui lorsque ça avait été son tour, plus tôt dans la journée. Il était impossible que \emph{Harry} maîtrise du premier coup quelque chose qui échappe à Hermione, et si une telle chose \emph{existait} et se révélait être \emph{le vol sur balai}, plutôt qu'une activité intellectuelle, Harry préférait encore mourir.)

Il fallut un bon moment avant que tout le monde n'ait un balai devant lui. Madame Bibine montra comment monter dessus, puis déambula entre les élèves, corrigeant les prises et les postures. Apparemment, même parmi les quelques enfants qu'on avait autorisés à voler chez eux, peu avaient appris à le faire correctement.

Madame Bibine passa en revue le champ de garçons et hocha la tête. «~Et maintenant, quand je souffle dans mon sifflet, vous donnez un bon coup d'envoi sur le sol avec vos pieds.~»

Harry déglutit, essayant de réprimer la sensation d'écœurement qu'il avait à l'estomac.

«~Maintenez vos balais stables, élevez-vous de quelques pieds, et revenez directement au sol en vous penchant légèrement en avant. À mon coup de sifflet… trois… deux…~»

L'un des balais piqua vers le ciel, accompagné par les cris d'un jeune garçon -- d'horreur, pas de délectation. Le garçon tournoyait à une effroyable vitesse tout en continuant de monter, et ils ne pouvaient qu'entrapercevoir son visage blanc…

Comme au ralenti, Harry avait bondi, était descendu de son balai, et fouillait sa robe à la recherche de sa baguette, même s'il ne savait pas exactement ce qu'il comptait en faire~; il avait eu exactement deux cours de Sortilèges, et le dernier \emph{avait} été le sortilège de Lévitation, mais Harry n'avait réussi à jeter le sort qu'une fois sur trois et il ne pouvait certainement pas faire léviter des personnes entières…

\emph{S'il y a un pouvoir caché en moi, qu'il se révèle MAINTENANT~!}

«~Reviens, mon garçon~!~» cria Madame Bibine (ce qui devait être l'injonction la plus inutile qu'on puisse imaginer face à un balai incontrôlable, venant d'un \emph{professeur de vol}, et une section totalement automatique du cerveau de Harry ajouta Madame Bibine à sa liste d'idiots).

Et le garçon fut projeté loin du balai.

Il sembla d'abord se déplacer très lentement dans les airs.

«~\emph{Wingardium Leviosa~!}~» hurla Harry.

Le sort échoua. Harry pouvait le sentir échouer.

Il y eut un BAM et un lointain son de craquement, et le garçon gisait au sol, face contre terre, comme un petit monticule posé sur l'herbe.

Harry rengaina sa baguette et courut à pleine vitesse. Il parvint aux côtés du garçon au même moment que Madame Bibine, et Harry fouilla dans sa bourse et essaya de se souvenir de -- oh mon dieu comment s'appelait -- peu importe il essayerait juste «~Pack de Soin~!~» et le pack apparut dans sa main et…

«~Poignet cassé, dit Madame Bibine. Calme-toi mon garçon, il a juste un poignet cassé~!~»

Il y eut une espèce d'embardée mentale tandis que l'esprit de Harry s'extirpait du Mode Panique.

Le Pack de soins d'urgence Plus était ouvert devant lui, et il avait en main une seringue de feu liquide qui aurait maintenu le cerveau du garçon oxygéné si jamais il s'était brisé le cou.

«~Ah…~» dit Harry d'une voix chancelante. Son cœur battait avec tant de force qu'il ne s'entendait presque pas haleter.

«~Un os cassé… d'accord… Fil de Remise~?

--- C'est seulement pour les urgences, jeta Madame Bibine. «~Écarte ça, il va bien~», elle se pencha au-dessus du garçon, lui offrant sa main. «~Allons mon garçon, tout va bien, on se met debout~!

--- Vous n'allez pas sérieusement lui faire à nouveau conduire le balai~?~» dit Harry avec horreur.

Madame Bibine jeta un regard noir à Harry. «~Bien sûr que non~!~» Elle tira sur le bras encore valide du garçon, l'aidant à se remettre sur pied -- Harry fut choqué de se rendre compte que c'était \emph{encore} Neville Londubat, qu'est-ce qui n'allait \emph{pas} chez lui~? -- et elle se retourna vers les enfants qui attendaient tous. «~Aucun d'entre vous ne bougera pendant que j'amène ce garçon à l'infirmerie~! Vous laisserez ces balais où ils sont ou vous serez renvoyés de Poudlard avant d'avoir pu dire “Quidditch”. Allons-y, mon garçon.~»

Et Madame Bibine s'en fut avec Neville, qui serrait son poignet et essayait de contrôler ses reniflements.

Lorsqu'ils furent hors de portée de voix, l'un des Serpentard commença à glousser.

Ce qui fit glousser les autres.

Harry se retourna et les regarda. Ça semblait être le moment opportun pour se graver quelques visages en mémoire.

Et Harry vit que Drago marchait nonchalamment vers lui, accompagné de M. Crabbe et de M. Goyle. M. Crabbe ne souriait pas. M. Goyle en revanche, si, et franchement. Drago lui-même avait une expression très contrôlée qui tressaillait parfois, et Harry en déduit que Drago trouvait la situation hilarante mais ne voyait aucun avantage politique à en rire maintenant plutôt que plus tard, dans les donjons de Serpentard.

«~Eh bien, Potter,~» dit Drago d'une voix grave qui ne portait pas, toujours avec cette expression très contrôlée qui parfois tressaillait, «~je voulais juste te dire que, quand tu tires parti d'une urgence pour démontrer que tu es un leader, il vaut mieux avoir l'air de contrôler totalement la situation, plutôt que, disons, d'avoir l'air d'être complètement paniqué.~» M. Goyle gloussa, et Drago lui jeta un regard de réprimande. «~Mais tu as probablement gagné quelques points quand même. Besoin d'aide pour ranger ce kit de soin~?~»

Harry se tourna pour regarder le Pack de soins, ce qui détourna son visage de celui de Drago. «~Je pense que ça ira,~» dit Harry. Il remit la seringue en place, refit les sangles, et se leva.

Ernie Macmillan arriva juste quand Harry donnait le pack à manger à sa bourse.

«~Merci, Harry Potter, de la part de Poufsouffle, dit solennellement Ernie Macmillan. C'était une bonne idée, et tu as fait de ton mieux.

--- Une bonne idée, certainement, coassa Drago. Pourquoi personne à Poufsouffle n'avait-il sa baguette sortie~? Peut-être que si vous aviez \emph{tous} aidé, et pas juste Potter, vous l'auriez rattrapé. Je pensais que les Poufsouffle étaient censés se serrer les coudes~?~»

Ernie avait l'air d'être déchiré entre la colère et la honte mortelle.

«~On n'y a pas pensé à temps…

--- Ah, dit Drago, pas \emph{pensé} à temps, j'imagine que c'est pour ça qu'il vaut mieux être ami avec un seul Serdaigle qu'avec tous les Poufsouffle.~»

Oh, bon sang, comment Harry allait-il jouer ça… «~Tu n'aides pas vraiment~», dit Harry d'un ton égal. Espérant que Drago interprète cela comme \emph{tu interfères avec mes plans, s'il te plaît tais-toi}.

«~Eh, qu'est-ce que c'est que ça~?~» dit M. Goyle. Il se pencha et ramassa quelque chose de la taille d'une grande bille, une balle de verre qui semblait être pleine d'une brume blanche tourbillonnante.

Ernie cligna des yeux.

«~Le Rapeltout de Neville~!

--- Qu'est-ce qu'un Rapeltout~? demanda Harry.

--- Il devient rouge si on a oublié quelque chose, dit Ernie. Mais il ne dit pas ce qu'on a oublié. Donne-le-moi s'il te plaît, et je le rendrai à Neville plus tard.~» Ernie tendit sa main.

Un sourire apparut soudain sur le visage de M. Goyle, et il se retourna et courut loin d'eux.

Ernie se tint là un moment, surpris, puis il cria «~Eh~!~» et courut après M. Goyle.

Et M. Goyle attrapa un balai, grimpa dessus avec souplesse et prit les airs.

La mâchoire de Harry s'affaissa. Madame Bibine n'avait-elle pas dit qu'il serait \emph{renvoyé}~?

«~\emph{Imbécile}~!~» siffla Drago. Il ouvrit la bouche pour crier…

«~\emph{Eh~!} cria Ernie. C'est à Neville~! \emph{Rapporte-le}~!~»

Les Serpentard commencèrent à acclamer et à siffler.

La bouche de Drago se referma aussi sec. Harry surprit l'air d'indécision qui était brusquement apparu sur son visage.

«~Drago, dit Harry d'une voix basse, si tu n'ordonnes pas à cet idiot d'atterrir, le professeur va revenir et…

--- \emph{Viens le chercher, Poufsouffle~!}~» hurla M. Goyle, et des salves d'acclamations montèrent des Serpentard.

«~Je ne \emph{peux pas}~! chuchota Drago. Tout le monde à Serpentard pensera que je suis \emph{faible}~!

--- Et si M. Goyle est renvoyé, siffla Harry, ton \emph{père} va penser que tu es un \emph{crétin}~!~»

Le visage de Drago était déformé par l'agonie.

À cet instant…

«~Eh, \emph{Serpensale,} cria Ernie, on ne vous a jamais dit que les Poufsouffle se serraient les coudes~? \emph{Baguettes, Poufsouffle~!}~»

Et il y eut soudain beaucoup de baguettes pointées en direction de M. Goyle.

Trois secondes plus tard…

«~\emph{Baguettes, Serpentard~!}~» dirent au moins cinq Serpentard.

Et il y eut plein de baguettes pointées en direction de Poufsouffle.

Deux secondes plus tard…

«~\emph{Baguettes, Gryffondor~!}

--- \emph{Fais quelque chose, Potter~!} chuchota Drago. \emph{Je ne peux pas être celui qui arrête ça il faut que ce soit toi~! Je te devrai une faveur trouve juste quelque chose tu n'es pas censé être brillant~?}~»

Harry mit environ cinq secondes et demie à se rendre compte que quelqu'un allait jeter le Sort d'Attaque Simple Sumérien et que quand ce serait terminé et que les enseignants auraient fini de renvoyer les responsables, les seuls garçons de cette année encore à Poudlard seraient les Serdaigle.

«~\shout{Baguettes, Serdaigle~!}~» cria Michael Corner, qui se sentait apparemment exclu du désastre.

«~\emph{GREGORY GOYLE~!} hurla Harry. \shout{Je te défie à un concours pour la possession du Rapeltout de Neville~!}~»

Il y eut une pause soudaine.

«~Oh, vraiment~?~» dit Drago du coassement le plus fort que Harry avait jamais entendu. «~Ça a l'air intéressant. Quel genre de concours, Potter~?~»

Euh…

L'inspiration de Harry n'était pas allée au-delà de “concours”. Quel genre de concours, il ne pouvait pas dire “échecs” parce que Drago ne pourrait pas accepter sans que ça ait l'air bizarre, il ne pouvait pas dire “bras de fer” parce que M. Goyle l'écrabouillerait…

«~Que pensez-vous de ça~? dit Harry bien fort. Gregory Goyle nous tenons loin l'un de l'autre, et personne d'autre n'a le droit de s'approcher. Nous n'utilisons pas nos baguettes, et personne d'autre non plus. Je ne bouge pas de là où je suis, et lui non plus. Et si je peux mettre la main sur le Rapeltout de Neville, alors Gregory Goyle abandonne toute prétention au Rapeltout et me le donne.~»

Il y eut une autre pause tandis que l'expression de soulagement des gens alentour se transformait en une expression de confusion.

«~Ha, Potter~! dit Drago avec force. J'aimerais bien te voir faire \emph{ça}~! M. Goyle accepte~!

--- C'est parti~! dit Harry.

--- Potter, \emph{quoi~?}~» chuchota Drago, ce qu'il parvint mystérieusement à faire sans bouger ses lèvres.

Harry ne savait pas comment répondre sans bouger les siennes.

Les gens rangeaient leur baguette, et M. Goyle fit une gracieuse descente en piquée jusqu'au sol, l'air plutôt confus. Quelques Poufsouffle commencèrent à s'avancer vers M. Goyle mais Harry leur jeta un regard suppliant et ils reculèrent.

Harry se déplaça vers M. Goyle et s'arrêta lorsqu'ils furent à quelques pas l'un de l'autre, assez loin pour qu'ils ne puissent se toucher.

Doucement, délibérément, Harry rengaina sa baguette.

Tout le monde recula.

Harry déglutit. Il savait en gros ce qu'il \emph{voulait} faire, mais il fallait que ce soit fait de telle façon que personne ne comprenne \emph{ce qu'il avait fait}…

«~Très bien~», dit Harry avec force. Et maintenant…~» Il prit une profonde inspiration et leva une main, les doigts prêts à claquer. Il y eut des glapissements venant de tous ceux qui avaient entendu parler des tartes, c'est-à-dire presque tout le monde. «~\emph{J'en appelle à la démence de Poudlard~! Content content boum boum marais marais marais}~!~» Et Harry claqua des doigts.

De nombreuses personnes eurent un mouvement de peur.

Et rien ne se passa.

Harry laissa le silence s'étirer un moment, se développer, jusqu'à…

«~Euh, dit quelqu'un. C'est tout~?~»

Harry regarda le garçon qui avait parlé.

«~Regarde devant toi. Tu vois le petit bout de terre qui a l'air stérile, sans herbe dessus~?

--- Euh, oui,~» dit le garçon, un Gryffondor (Dean Quelque Chose~?).

«~Creuse ici.~»

Harry était maintenant la cible de regards intrigués.

«~Euh, pourquoi~?

--- Fais-le, c'est tout, dit Terry Boot d'une voix lasse. Pas la peine de demander pourquoi, crois-moi.~»

Dean Quelque Chose s'agenouilla et commença à enlever des poignées de terre.

Après une minute, Dean se releva. «~Il n'y a rien ici~», dit Dean.

Uh. Harry avait compté revenir en arrière dans le temps et enterrer une carte au trésor qui mènerait à une autre carte au trésor qui mènerait au Rapeltout de Neville qu'il aurait mis là après l'avoir repris à M. Goyle…

Puis Harry se rendit compte qu'il existait un moyen beaucoup plus simple, qui ne menaçait pas autant le secret des Retourneurs de Temps.

«~Merci, Dean~! dit Harry avec force. Ernie, pourrais-tu regarder là où Neville est tombé et voir si tu peux trouver son Rapeltout~?~»

Les gens avaient l'air encore plus confus.

«~Fais-le, c'est tout, dit Terry Boot. Il va continuer d'essayer jusqu'à ce que ça marche, et ce qui me fait peur c'est que…

--- \emph{Par Merlin~!}~» s'étrangla Ernie. Il tenait le Rapeltout de Neville. «~Il était \emph{là}~! Exactement là où il est tombé~!

--- \emph{Quoi}~?~» s'écria M. Goyle. Il baissa le regard et vit…

… qu'il tenait toujours le Rapeltout de Neville.

Il y eut une longue pause.

«~Euh, dit Dean quelque chose, ce n'est pas impossible~?

--- C'est une erreur de scénario, dit Harry. Je suis devenu assez bizarre pour distraire l'univers pendant un moment, et il a oublié que Goyle avait déjà ramassé le Rapeltout.

--- Non, attends, je veux dire, c'est \emph{complètement} impossible…

--- Excuse-moi, mais sommes-nous tous ici à attendre de voler sur des balais~? Tout à fait. Alors tais-toi. Bref, une fois que je mettrai ma main sur le Rapeltout de Neville, le concours sera terminé et Gregory Goyle devra abandonner toute prétention sur le Rapeltout qu'il tient dans sa main et me le donner. C'étaient les termes du concours, tu te souviens~?~» Harry tendit une main et fit signe à Ernie. «~Fais-le juste rouler jusqu'ici, puisque personne n'est censé s'approcher de moi.

--- Attends~!~» cria un Serpentard -- Blaise Zabini, Harry n'allait certainement pas oublier ce nom. «~Comment savons-nous que c'est le Rapeltout de Neville~? Tu pourrais avoir fait tomber un \emph{autre} Rapeltout ici…

--- Le Serpentard est fort chez celui-ci, dit Harry en souriant. Mais tu as ma parole que celui que Ernie tient dans sa main est à Neville. Aucun commentaire en ce qui concerne celui de Gregory Goyle.~»

Zabini se tourna vers Drago.

«~\emph{Malfoy~!} Tu ne vas pas le laisser s'en tirer avec ce…

--- Tais-toi, toi~», gronda M. Crabbe, qui se tenait derrière Drago. «~M. Malfoy n'a pas besoin que \emph{tu} lui dises ce qu'il doit faire~!~»

\emph{Bon} laquais.

«~Mon pari était avec Drago, de la Noble et Ancienne Maison Malfoy, dit Harry. Pas avec toi, Zabini. J'ai fait ce que M. Malfoy a dit qu'il aimerait me voir faire, et pour le jugement de ce pari, je m'en remettrai à lui.~» Harry inclina sa tête vers Drago et releva légèrement ses sourcils. Ça devrait permettre à Drago de sauver la face.

Il y eut une pause.

«~Tu promets que c'est \emph{vraiment} le Rapeltout de Neville~? dit Drago.

---Oui, dit Harry. C'est celui qui reviendra à Neville, et c'était le sien au départ. Et celui que tient Gregory Goyle entrera en ma possession.~»

Drago hocha la tête, l'air décidé. «~Alors je ne remettrai pas en question la parole de la Noble Maison de Potter, peu importe que tout cela ait été très étrange. Et la Noble et Ancienne Maison Malfoy tient aussi sa parole. M. Goyle, donnez cela à M. Potter…

--- Hé~! dit Zabini. Il n'a pas \emph{encore} gagné, il n'a pas mis la main sur…

--- Attrape, Harry~!~» dit Ernie, et il jeta le Rapeltout.

Harry attrapa le Rapeltout au vol avec facilité, il avait toujours eu de bons réflexes pour ça. «~Voilà, dit Harry, j'ai gagné…~»

Harry laissa sa phrase en suspens. Toutes les conversations s'arrêtèrent.

Le Rapeltout rougeoyait avec force dans sa main, flamboyant comme un soleil miniature qui faisant danser des ombres sur le sol illuminé.

\later

Jeudi.

Si on voulait être précis, 17h09 un jeudi après-midi, dans le bureau du professeur McGonagall, après les cours de vol (avec une heure supplémentaire pour Harry glissée entre les deux).

Le professeur McGonagall était assise sur son tabouret. Harry sur la sellette, face à son bureau.

«~Professeur, dit fermement Harry, les Serpentard pointaient leurs baguettes vers Poufsouffle, les Gryffondor pointaient leurs baguettes vers Serpentard, un \emph{idiot} de chez Serdaigle a lui aussi dit “baguettes~!”, et j'avais au plus cinq secondes pour empêcher la situation d'exploser~! C'est tout ce que j'ai pu trouver~!~»

Le visage du professeur McGonagall était pincé et empli de colère.

«~\emph{Vous n'utiliserez pas le Retourneur de Temps à de telles fins, M. Potter~!} Le concept de secret n'est-il pas quelque chose que vous comprenez~?

--- Ils ne \emph{savent} pas comment j'ai fait~! Ils pensent juste que je peux faire des choses vraiment bizarres en claquant des doigts~! J'ai fait d'autres choses bizarres qui ne peuvent être faites avec les Retourneurs de Temps, et j'en ferai \emph{encore d'autres}, et \emph{ce} cas précis ne sera qu'un de plus parmi les autres~! Je \emph{devais le faire}, professeur~!

--- Vous ne deviez \emph{pas} le faire~! dit le professeur McGonagall d'un ton cinglant. Tout ce que vous aviez à faire était de faire redescendre ce \emph{Serpentard anonyme} au sol et de faire rengainer les baguettes~! Vous pourriez l'avoir défié à une partie de Bataille Explosive, mais non, vous deviez utiliser le Retourneur de Temps de façon inutile et flagrante~!

--- C'est tout ce que j'ai pu trouver~! Je ne sais même pas ce qu'\emph{est} la Bataille Explosive, ils n'auraient pas accepté une partie d'échecs, et si j'avais choisi le bras de fer, j'aurais perdu~!

--- \emph{Alors vous auriez dû choisir le bras de fer~!}~»

Harry cligna des yeux. «~Mais alors j'aurais \emph{perdu}…~»

Harry s'interrompit.

Le professeur McGonagall avait l'air \emph{très} en colère.

«~Je suis désolé, professeur McGonagall, dit Harry d'une petite voix. Je n'y avais honnêtement pas pensé, et vous avez raison, j'aurais dû, ça aurait été brillant de ma part, mais je n'y ai juste pas pensé du tout…~»

Harry laissa sa phrase en suspens. Il devenait soudain clair qu'il avait eu \emph{beaucoup} d'autres options. Il aurait pu demander à \emph{Drago} de suggérer quelque chose, il aurait pu demander à la foule… son utilisation du Retourneur de Temps avait été inutile et flagrante. Il y avait eu un espace de possibilités immense, pourquoi avait-il choisi \emph{celle}-là~?

Parce qu'il avait vu un moyen de gagner. Gagner la possession d'une babiole sans importance que les professeurs auraient de toute façon reprise à M. Goyle.

Intention de gagner. C'est ça qui l'avait eu.

«~Je suis désolé, dit à nouveau Harry. Pour mon orgueil et ma stupidité.~»

Le professeur McGonagall se passa une main sur le front. Une partie de sa colère sembla se dissiper. Mais sa voix resta très dure.

«~Une autre démonstration de ce genre, M. Potter, et vous me rendrez le Retourneur de Temps. Ai-je bien été claire~?

--- Oui, dit Harry. Je comprends et je suis désolé.

--- Alors, M. Potter, vous serez autorisé à conserver le Retourneur de Temps pour le moment. Et étant donné la taille de la débâcle que vous avez, en effet, évitée, je ne déduirai pas de point à Serdaigle.~»

\emph{Et puis vous ne pourriez pas expliquer pourquoi vous avez déduit les points}. Mais Harry n'était pas assez stupide pour dire ça à voix haute.

«~Plus important, pourquoi le Rapeltout s'est-il déclenché comme ça~? dit Harry. Cela veut-il dire que j'ai été victime d'un sortilège d'Amnésie~?

--- Je suis moi aussi perplexe, dit lentement le professeur McGonagall. Si c'était aussi simple, je pense que les tribunaux utiliseraient les Rapeltouts, et ce n'est pas le cas. J'étudierai la question, M. Potter.~» Elle soupira. «~Vous pouvez y aller, maintenant.~»

Harry commença à se lever de sa chaise, puis s'arrêta. «~Euh, pardon, il y a quelque chose que je voulais vous dire…~»

On put à peine remarquer le tressaillement.

«~De quoi s'agit-il, M. Potter~?

--- C'est à propos du professeur Quirrell…

--- Je suis certaine, M. Potter, que ce n'est rien de très important.~» Le professeur McGonagall prononça ces mots très rapidement. «~Vous avez certainement entendu le directeur dire aux étudiants de ne pas nous importuner avec des complaintes sans importance au sujet du professeur de Défense~?~»

Harry était plutôt déconcerté. «~Mais ça pourrait \emph{être} important, hier j'ai eu cette sensation funeste quand…

--- M. Potter~! J'ai moi aussi une sensation funeste~! Et ma sensation funeste dit que \emph{vous ne devez pas finir cette phrase~!}~»

La bouche de Harry s'ouvrit grande. Le professeur McGonagall avait réussi~; Harry était sans voix.

«~M. Potter, dit le professeur McGonagall, si vous avez découvert quoi que ce soit d'intéressant au sujet du professeur Quirrell, n'hésitez pas à ne pas le partager avec moi ni avec qui que ce soit. Maintenant je pense que vous avez pris assez de mon temps précieux…

--- \emph{Ça ne vous ressemble pas~!} éclata Harry. Je suis désolé, mais ça semble juste \emph{incroyablement} irresponsable~! De ce que j'ai entendu, il y a une sorte de malédiction sur le poste de Défense, et vous \emph{savez} déjà que quelque chose va mal tourner, je pensais que seriez tous en alerte…

--- \emph{Mal tourner}, M. Potter~? \emph{J'espère bien que non}.~» Le visage du professeur McGonagall était parfaitement inexpressif. «~Après que le professeur Blake a été pris dans un placard avec pas moins de trois Serpentard en cinquième année, en février dernier, et qu'un an avant cela, le professeur Summers avait tellement échoué dans sa mission de pédagogue que ses étudiants pensaient qu'un Épouvantard était une sorte de meuble, il serait \emph{catastrophique} qu'un problème concernant le remarquablement compétent professeur Quirrell soit maintenant porté à ma connaissance, ce qui, si je puis me permettre, ferait échouer la plupart de nos étudiants à leurs BUSE ou à leurs ASPIC

--- Je vois~», dit lentement Harry, absorbant le tout. «~Donc en d'autres mots, quoi qu'il se passe avec le professeur Quirrell, vous souhaitez désespérément ne pas l'apprendre avant la fin de l'année scolaire. Et puisque nous sommes pour l'instant en septembre, en ce qui vous concerne, il pourrait parfaitement assassiner le Premier Ministre en direct à la télévision et se tirer d'affaire.~»

Le professeur McGonagall le regarda sans ciller. «~Je suis certaine qu'on ne pourrait jamais m'entendre confirmer une telle affirmation, M. Potter. À Poudlard, nous essayons d'être proactifs en ce qui concerne \emph{tout} ce qui met en danger la réussite scolaire de nos élèves.~»

\emph{Comme des Serdaigle de première année qui ne savent pas la fermer.}

«~Je crois vous avoir très bien compris, professeur McGonagall.

--- Oh, j'en doute, M. Potter. J'en doute grandement.~» Le professeur McGonagall se pencha en avant, son visage à nouveau contracté. «~Puisque vous et moi avons discuté de sujets bien plus sensibles que celui-ci, je vous parlerai franchement. Vous, et vous seul, avez fait état de cette mystérieuse sensation funeste. Vous, et vous seul, êtes un aimant à chaos tel que je n'en ai jamais vu. Après notre petite excursion au Chemin de Traverse, \emph{puis} le Choixpeau Magique, puis le petit épisode d'\emph{aujourd'hui}, je puis prédire que je suis condamnée à m'asseoir dans le bureau du directeur et à entendre une histoire hilarante au sujet du professeur Quirrell dans laquelle vous et vous seul tiendrez un premier rôle, après quoi je n'aurai d'autre choix que de le licencier. J'y suis déjà résignée, M. Potter. Et si ce triste événement a lieu avant les ides de mai, je vous attacherai au portail de Poudlard avec vos propres intestins et je verserai des abeilles de feu dans votre nez. \emph{Maintenant}, m'avez-vous très bien compris~?~»

Harry hocha la tête, ses yeux grands ouverts. Puis, une seconde plus tard~:

«~Et j'ai droit à quoi si j'arrive à faire que ça ait lieu le dernier jour de l'année scolaire~?

--- \emph{Sortez de mon bureau~!}~»

\later

Jeudi.

Il devait y avoir quelque chose de particulier avec les jeudis à Poudlard.

Il était 17h32, jeudi après-midi, et Harry se tenait à côté du professeur Flitwick, devant la grande gargouille de pierre qui gardait l'entrée du bureau du directeur.

Il était à peine revenu du bureau du professeur McGonagall et entré dans les salles d'études de Serdaigle qu'un des étudiants lui avait dit de se présenter au bureau du professeur Flitwick, et là, Harry avait appris que Dumbledore voulait lui parler.

Harry, se sentant plutôt appréhensif, avait demandé au professeur Flitwick si le directeur avait dit de quoi il s'agissait.

Le professeur Flitwick avait haussé les épaules d'un air impuissant.

Dumbledore avait apparemment dit que Harry était bien trop jeune pour invoquer les mots de pouvoir et de folie.

\emph{Content content boum boum marais marais marais}~? avait pensé Harry, mais il ne l'avait pas dit à voix haute.

«~Ne vous en faites pas trop, M. Potter,~» piailla le professeur Flitwick depuis ce qui devait être le niveau de l'épaule de Harry (il était heureux que le professeur Flitwick porte une barbe si immense et si fournie, car c'était difficile de s'habituer à un professeur qui était non seulement plus petit que lui mais parlait aussi d'une voix plus aiguë). «~Le directeur Dumbledore peut sembler un peu étrange, ou très étrange, et même extrêmement étrange, mais il n'a jamais fait le moindre mal à un étudiant, et je ne pense pas qu'il le fera jamais.~» Le professeur Flitwick adressa un sourire encourageant à Harry. «~Gardez cela à l'esprit à tout moment et vous serez certain de ne pas paniquer~!~»

Ça ne l'aida pas.

«~Bonne chance~!~» piailla le professeur Flitwick, et il se pencha au-dessus de la gargouille et dit quelque chose que Harry ne parvint bizarrement pas à entendre (bien sûr, le mot de passe ne serait pas très utile si on pouvait entendre quelqu'un le prononcer). Et la gargouille de pierre fit un pas de côté d'un mouvement très naturel et ordinaire, ce que Harry trouva plutôt choquant, puisque pendant ce temps, la gargouille continua à ressembler à de la pierre solide et inamovible.

Derrière la gargouille se trouvaient plusieurs marches en spirales tournant sur eux-mêmes. Il y avait là quelque chose de troublant et d'hypnotique, et encore plus troublant était le fait que faire \emph{tourner} une spirale n'aurait pas dû vous mener quelque part.

«~Et ça monte~!~» piailla Flitwick.

Harry mit un pied assez nerveux sur la spirale et commença à se déplacer vers le haut pour une raison que son cerveau ne parvenait pas du tout à visualiser.

La gargouille fit un bruit sourd en se remettant en place derrière lui, et les escaliers en spirale continuèrent de tourner et Harry continua de monter, et après une durée plutôt étourdissante Harry se retrouva en face d'une porte de chêne avec un heurtoir en laiton.

Harry tendit la main et tourna la poignée.

La porte s'ouvrit grand.

Et Harry vit la pièce la plus intéressante qu'il ait vue de sa vie.

Il y avait de petits mécanismes métalliques qui vrombissaient ou cliquetaient ou changeaient lentement de forme ou émettaient de petites bouffées de fumée. Il y avait des douzaines de fluides mystérieux dans des douzaines de contenants aux formes étranges, tous bullants, bouillonnants, suintants, changeant de couleur, ou prenant des formes intéressantes qui disparaissaient une demi-seconde après que vous les aviez vues. Il y avait des choses qui ressemblaient à des horloges dotées de nombreuses aiguilles, gravées de chiffres ou de mots dans des langues inconnues. Il y avait un bracelet portant un cristal lenticulaire qui étincelait de mille couleurs, et la statue d'un faucon incrusté dans de l'émail noir. Le mur portait des tableaux de gens endormis, et le Choixpeau Magique était nonchalamment posé sur un porte-chapeaux qui soutenait aussi deux parapluies et trois pantoufles rouges pour pied gauche.

Au milieu de tout ce chaos se trouvait un bureau en chêne noir bien rangé. Devant le bureau se trouvait un tabouret en chêne. Et derrière le bureau se trouvait un trône bien rembourré dans lequel était assis Albus Percival Wulfric Brian Dumbledore, qui était orné d'une longue barbe d'argent, d'un chapeau ressemblant à un immense champignon écrasé, et de ce que des moldus considéreraient être trois couches de pyjamas rose vif.

Dumbledore souriait, et ses yeux radieux pétillaient d'une folle intensité.

Avec une inquiétude certaine, Harry s'assit face au bureau. La porte se referma derrière lui dans un puissant \emph{blam}.

«~Bonjour, Harry, dit Dumbledore.

---Bonjour, monsieur le directeur~», répondit Harry. Alors ils s'appelaient par leurs prénoms~? Dumbledore allait-il lui dire de l'appeler…

«~S'il te plaît, Harry, dit Dumbledore. Monsieur le directeur a l'air si officiel. Appelle-moi juste Dir, pour faire court.

--- D'accord, Dir,~» dit Harry.

Il y eut une courte pause.

«~Sais-tu, dit Dumbledore, que tu es la première personne à m'avoir jamais pris au mot~?

--- Ah…~» dit Harry. Il essaya de contrôler sa voix en dépit des soudaines contractions de son estomac. «~Je suis désolé, je, ah, monsieur le directeur, vous m'avez dit de le faire alors je l'ai fait…

--- Dir, s'il te plaît~! dit gaiement Dumbledore. Et il n'y a pas de raison d'être aussi inquiet, je ne te jetterai pas par la fenêtre juste parce que tu as fait une erreur. Je te donnerai plein d'avertissements avant, si tu fais quelque chose de mal~! Et puis, ce qui compte, ce n'est pas la façon dont les gens vous parlent, mais ce qu'ils pensent de vous.~»

\emph{Il n'a jamais fait le moindre mal à un étudiant, garde cela à l'esprit et tu seras certain de ne pas paniquer.}

Dumbledore sortit une petite boîte en métal et en souleva le couvercle, révélant de petits blocs jaunes. «~Bonbons au citron~?~» dit le directeur.

«~Euh, non merci, Dir~», dit Harry. \emph{Pouvait-on considérer que de glisser du LSD à un étudiant c'était leur faire du mal, ou cela rentrait-il dans la catégorie jeu inoffensif~?}

«~Vous, euh, avez dit quelque chose à mon propos, au sujet du fait que j'étais trop jeune pour invoquer les mots de pouvoir et de folie~?

--- J'ai dit que vous l'étiez sûrement~! dit Dumbledore. Heureusement, les Mots de Pouvoir et de Folie ont été perdus il y a sept siècles et personne n'a la moindre idée de ce qu'ils sont. C'était juste une petite remarque.

--- Ah…~» dit Harry. Il se rendait bien compte que sa bouche était grande ouverte. «~Pourquoi m'avez-vous fait venir ici, alors~?

--- \emph{Pourquoi~?} répéta Dumbledore. Ah, Harry, si je passais mes journées à me demander \emph{pourquoi} je fais les choses, je n'aurais le temps de rien faire~! Je suis quelqu'un de très occupé, tu sais.~»

Harry hocha la tête en souriant. «~Oui, c'était une liste très impressionnante. Directeur de Poudlard, Président du Magenmagot, et Manitou suprême de la Confédération internationale des Mages et Sorciers. Navré de vous poser la question, mais je me demandais, est-il possible d'obtenir plus de six heures si on utilise plus d'un Retourneur de Temps~? Parce que c'est plutôt impressionnant si vous faites tout ça en seulement trente heures par jour.~»

Il y eut une autre courte pause, pendant laquelle Harry continua de sourire. Il était un peu appréhensif, en fait très appréhensif, mais une fois qu'il était devenu clair que Dumbledore s'amusait délibérément avec lui, quelque chose à l'intérieur de Harry avait \emph{absolument refusé} de rester là à tout se prendre comme un empoté sans défense.

«~J'ai bien peur que le Temps n'aime pas être trop étiré, dit Dumbledore après la courte pause, mais nous semblons être trop grands pour lui, et s'ensuit donc une lutte permanente pour faire tenir nos vies dans le Temps.

--- En effet, dit Harry avec une grave solennité. C'est pourquoi il est préférable d'en arriver rapidement à ce qui nous préoccupe.~»

Pendant un moment, Harry se demanda s'il avait été trop loin.

Puis Dumbledore gloussa. «~Droit au fait nous irons.~» Le directeur s'inclina en avant, faisant pencher son chapeau-champignon écrasé et brossant sa barbe contre son bureau. «~Harry, lundi dernier tu as fait quelque chose qui aurait dû être impossible, même avec un Retourneur de Temps. Ou plutôt, impossible avec \emph{seulement} un Retourneur de Temps. Je me demande~: d'où ces deux tartes venaient-elles~?~»

Un choc d'adrénaline traversa Harry. Il avait fait cela en utilisant la Cape d'Invisibilité, celle qui lui avait été donnée dans une boîte de cadeau de Noël, avec une note, et la note avait dit~: \emph{Si Dumbledore voyait une chance de posséder l'une des Reliques de la Mort, il ne la laisserait jamais échapper à son étreinte…}

«~Une conclusion logique, continua Dumbledore, serait que puisqu'aucun des étudiants en première année présents n'étaient capables de jeter un sort pareil, quelqu'un d'autre était présent mais n'a pas été vu. Et si personne ne pouvait la voir, eh bien, il aurait été facile pour cette personne de jeter les tartes. On pourrait de plus soupçonner que puisque tu as le Retourneur de Temps, tu étais la personne invisible~; et que puisque le sort de Désillusion est bien au-delà de tes capacités actuelles, tu as une cape d'invisibilité.~» dit Dumbledore avec un air de conspirateur. «~Suis-je sur la bonne voie jusqu'ici, Harry~?~»

Harry se figea. Il avait l'impression qu'un pur mensonge ne serait pas très malin, et ne l'aiderait peut-être même pas du tout, et il ne trouvait rien d'autre à dire.

Dumbledore agita une main amicale.

«~Ne t'en fais pas Harry, tu n'as rien fait de mal. Les capes d'invisibilité ne sont pas interdites -- j'imagine qu'elles sont assez rares pour que personne n'ait pris le temps de les ajouter à la liste. Mais à vrai dire, je songeais à tout autre chose.

--- Oh~?~» dit Harry de la voix la plus normale qu'il put.

Les yeux de Dumbledore brillèrent d'enthousiasme. «~Tu vois, Harry, après avoir vécu quelques aventures, on commence à attraper le coup. On commence à voir le motif, à entendre le rythme du monde. On commence à entretenir des soupçons \emph{avant} le moment de révélation. Tu es le Survivant, et une cape invisible est parvenue jusqu'à tes mains seulement quatre jours après que tu as découvert notre Angleterre magique. De telles capes ne sont pas à vendre au Chemin de Traverse, mais il y en a \emph{une} qui pourrait trouver son chemin jusqu'à celui destiné à la porter. Et je ne peux donc m'empêcher de me demander si par un étrange hasard tu n'aurais pas trouvé non pas juste \emph{une} cape d'invisibilité, mais \emph{la} Cape d'Invisibilité, l'une des trois Reliques de la Mort, réputée cacher son porteur du regard de la Mort elle-même.~» Le regard de Dumbledore était brillant et avide. «~Pourrais-je la voir, Harry~?~»

Harry déglutit. Il y avait maintenant un raz-de-marée d'adrénaline dans son corps, et c'était totalement inutile, c'était le sorcier le plus puissant du monde, il n'atteindrait même pas la porte, et il n'y avait nulle part à Poudlard où il pourrait se cacher, et il était sur le point de perdre la Cape qui avait été passée de Potter en Potter pendant Merlin savait combien de générations…

Lentement, Dumbledore se radossa à sa chaise. La lueur vive avait quitté ses yeux, et il avait l'air perplexe, et un peu triste.

«~Harry, dit Dumbledore, si tu ne veux pas, tu peux juste dire non.

--- Je peux~? dit Harry d'une voix rauque.

--- Oui, Harry~», dit Dumbledore. Sa voix était triste à présent, et inquiète. «~Il semblerait que tu aies peur de moi, Harry. Puis-je te demander ce que j'ai fait pour mériter ta méfiance~?~»

Harry déglutit. «~Y a-t-il un moyen par lequel vous pouvez jurer, par un serment magique, que vous vous engagez à ne pas prendre ma cape~?~»

Dumbledore secoua lentement la tête. «~Les Vœux \emph{Inviolables} ne doivent pas être utilisés à la légère. Et puis, Harry, si tu ne connais pas encore le sort, tu n'auras que ma parole pour savoir qu'il m'engage. Et tu te rends sûrement compte que je n'ai pas \emph{besoin} de ta permission pour voir la Cape. Je suis assez puissant pour la faire apparaître moi-même, bourse en peau de Moke ou non.~» Le visage de Dumbledore était sérieux.

«~Mais je ne le ferai pas. La Cape est tienne, Harry. Je ne te la prendrai pas. Pas même pour y jeter un coup d'œil, à moins que tu ne décides de me la montrer. C'est une promesse et un serment. Si je devais t'interdire de l'utiliser dans l'enceinte de l'école, je te demanderais d'aller dans ton coffre-fort à Gringotts et de l'y déposer.

--- Ah…~» dit Harry. Il avala sa salive bruyamment, essayant de calmer le flot d'adrénaline et de penser de façon raisonnable. Il décrocha la bourse en peau de Moke de sa ceinture. «~Si vous n'avez vraiment \emph{pas} besoin de ma permission… alors vous l'avez.~» Harry tendit la bourse à Dumbledore et mordit sa lèvre avec force, s'envoyant ce signal au cas où il serait plus tard soumis à un sortilège d'Amnésie.

Le vieux sorcier mit sa main dans la bourse, et sans prononcer un seul mot de récupération, fit apparaître la Cape d'Invisibilité.

«~Ah, souffla Dumbledore. J'avais raison…~» Il fit glisser le canevas de velours noir entre ses mains. «~Vieille de plusieurs siècles, et toujours aussi parfaite qu'au jour de sa création. Au fil du temps, nous avons perdu une grande partie de notre art, et maintenant je ne peux pas créer ce genre de choses~; personne ne le peut. Je peux sentir son pouvoir, comme un écho dans mon esprit, comme une chanson chantée depuis toujours sans personne pour l'entendre…~» Le sorcier releva les yeux. «~Ne la vends pas, dit-il, ne l'offre à personne. Penses-y à deux fois avant de la montrer à quelqu'un, et pèse la question trois fois avant de révéler qu'il s'agit d'une Relique de la Mort. Traite-la avec respect, car il s'agit bel et bien d'un Objet de Pouvoir.~»

Le visage de Dumbledore devint momentanément rêveur…

… puis il rendit la Cape à Harry.

Harry la remit dans sa bourse.

Le visage de Dumbledore était de nouveau sérieux. «~Harry, puis-je te redemander comment tu en es venu à te méfier de moi~?~»

Harry se sentit brusquement honteux.

«~Il y avait une note avec la Cape, dit Harry d'une petite voix. Elle disait que vous essaieriez de vous emparer de la Cape si vous étiez au courant. Mais je ne sais pas qui a laissé cette note, vraiment pas.

--- Je… vois, dit lentement Dumbledore. Eh bien Harry, je ne présumerai pas des motifs de celui ou de celle qui t'a laissé cette note. Qui sait, peut-être cette personne avait-elle les meilleures intentions~? Elle t'a donné la Cape après tout.~»

Harry acquiesça, impressionné par la magnanimité de Dumbledore, et choqué par le net contraste avec sa propre attitude.

Le vieux sorcier continua~:

«~Mais je crois que toi et moi sont des pions de la même couleur. Le garçon qui a enfin vaincu Voldemort, et le vieil homme qui l'a tenu à distance assez longtemps pour que tu interviennes. Je ne te reprocherai pas ta prudence, Harry, nous faisons tous de notre mieux pour être sages. Je te demande seulement d'y penser à deux fois et de peser la question trois fois la prochaine fois que quelqu'un te dit de ne pas me faire confiance.

--- Je suis désolé,~» dit Harry. Il commençait à se sentir minable, il venait plus ou moins d'envoyer bouler Gandalf, et la gentillesse de Dumbledore le faisait se sentir encore plus mal. «~Je n'aurais pas dû me méfier de vous.

--- Hélas Harry, dans ce monde…~» Le vieux sorcier secoua la tête. «~Je ne peux même pas dire que tu as manqué de sagesse. Tu ne me connaissais pas. Et en vérité, il y a certaines personnes à Poudlard en qui tu ferais mieux de ne pas avoir confiance. Peut-être même certains que tu appelles des amis.~»

Harry déglutit. Ça semblait lourd de sous-entendus. «~Comme qui~?~»

Dumbledore se leva de sa chaise, et commença à examiner l'un de ses instruments, un cadran avec huit aiguilles de longueurs différentes.

Après quelques instants, le vieux sorcier parla de nouveau.

«~Il te semble probablement très charmant, dit Dumbledore. Poli… avec toi du moins. Choisissant bien ses mots, peut-être même admiratif. Toujours prêt à aider, à faire une faveur, à prodiguer un conseil…

--- Oh, \emph{Drago Malfoy}~!~» dit Harry, se sentant plutôt soulagé qu'il ne s'agisse pas de Hermione ou de quelqu'un d'autre. «~Oh non, non non non, vous vous trompez complètement, il ne me convertit pas, c'est moi qui le convertis.~»

Dumbledore se figea, toujours face au cadran.

«~Tu \emph{quoi}~?

--- Je vais éloigner Drago du Côté Obscur, dit Harry. Vous savez, faire de lui un gentil.~»

Dumbledore se raidit et se tourna vers Harry. Il arborait l'air le plus effaré que Harry ait jamais vu chez quelqu'un, et encore moins chez quelqu'un portant une longue barbe d'argent.

«~Es-tu certain, dit le vieux sorcier après un moment, que cette bonté que tu vois en lui n'est pas juste un vœu pieux de ta part, Harry~? J'ai peur que ce que tu vois ne soit que l'attrait, l'appât…

--- Hmm, peu probable, dit Harry. Je veux dire, s'il essaie de se faire passer pour un gentil alors il est sacrément nul. Ce n'est pas que Drago soit venu me voir, tout charmant, et que j'ai décidé qu'il devait avoir un noyau de bonté cachée au fond de lui. Je vise sa rédemption à lui en particulier parce qu'il est l'héritier de la Maison Malfoy, et que s'il faut choisir une personne à sauver, c'est forcément lui.~»

L'œil gauche de Dumbledore eut un léger mouvement convulsif.

«~Tu essaies de planter les graines de l'amour et de la gentillesse dans le cœur de Drago parce que tu t'attends à ce que l'héritier des Malfoy te soit utile~?

--- Pas seulement à \emph{moi}~! dit Harry d'une voix indignée. À toute l'Angleterre magique, si ça fonctionne~! \emph{Et} il aura lui-même une vie plus heureuse et mentalement plus saine~! Écoutez, je n'ai pas le temps d'éloigner \emph{tout le monde} du Côté Obscur, alors je dois trouver d'où la Lumière peut rapidement tirer le meilleur avantage…~»

Dumbledore commença à rire. À rire beaucoup plus fort que ce à quoi Harry se serait attendu~; il hurlait presque. Ça semblait presque \emph{indigne}. Un vieux et puissant sorcier se devait de glousser sur des tons caverneux, pas de rire tellement qu'il en manquait d'air. Harry était un jour littéralement tombé de sa chaise en regardant le film \emph{La Soupe au canard} des Marx Brothers, et Dumbledore riait maintenant avec autant d'intensité.

«~Ce n'est pas \emph{si} drôle que \emph{ça}~», dit Harry après un moment. Il commençait de nouveau à douter de la santé mentale de Dumbledore.

Il fallut à Dumbledore un effort visible pour qu'il parvienne à se contrôler de nouveau. «~Ah, Harry, l'un des symptômes de la maladie appelée sagesse est que l'on commence à rire de choses que personne d'autre ne trouve amusantes, parce que quand on est sage, Harry, on commence à comprendre ce qui est vraiment drôle~!~» Le vieux sorcier essuya des larmes de ses yeux. «~Oh oh oh. La volonté du mal ruine en effet souvent le mal, à grand effet.~»

Le cerveau de Harry mit un moment à remettre en place les mots familiers…

«~Hé, c'est une citation de \emph{Tolkien}~! \emph{Gandalf} dit ça~!

--- Plutôt Théoden, dit Dumbledore.

--- Vous êtes \emph{né-Moldu}~? dit Harry choqué.

--- J'ai bien peur que non, dit Dumbledore souriant de nouveau. Je suis né soixante-dix ans avant que ce livre ne soit publié, cher enfant. Mais il semble que mes étudiants nés-Moldus ne pensent souvent de façon similaire. J'ai accumulé pas moins de vingt copies du \emph{Seigneur des Anneaux} et trois collections complètes des œuvres de Tolkien, et je chéris chacun d'entre eux.~» Dumbledore dégaina sa baguette, la leva, et prit la pose. «~\emph{Vous ne passerez pas}~! De quoi j'ai l'air~?

--- Ah~», dit Harry, proche de l'arrêt cérébral complet, «~je pense qu'il vous manque un Balrog.~» Et les pyjamas roses et le chapeau champignon écrasé n'aidaient pas le moins du monde.

--- Je vois.~» Dumbledore soupira et rengaina sombrement sa baguette dans sa ceinture. «~J'ai peur qu'il y ait eu bien peu de Balrogs dans ma vie dernièrement. Ces temps-ci c'est surtout réunions au Magenmagot où j'essaie désespérément de les empêcher d'accomplir la moindre chose, et dîners officiels avec des politiciens étrangers qui se battent pour savoir qui pourra être l'idiot le plus borné. Et être mystérieux, et savoir des choses que je ne devrais pas savoir, prononcer des phrases cryptiques qui ne peuvent être comprises que rétrospectivement, et toutes les autres petites méthodes que les sorciers puissants ont pour s'amuser une fois qu'ils ont quitté la partie de l'histoire qui leur permettait d'être des héros. En parlant de ça Harry, j'ai quelque chose à te donner, quelque chose qui appartenait à ton père.

--- Vraiment~? dit Harry. Dieu, qui l'aurait cru.

--- Oui, c'est vrai, dit Dumbledore. j'imagine que c'était un peu prévisible.~» Son visage devint solennel. «~Néanmoins…~»

Dumbledore revint à son bureau et s'assit tout en ouvrant l'un des tiroirs. Il y farfouilla des deux mains, et, forçant légèrement, tira de celui-ci un objet plutôt grand et à l'air assez lourd qu'il déposa ensuite sur son bureau de chêne avec un bruit sourd.

«~Ceci, dit Dumbledore, était le rocher de ton père.~»

Harry le fixa du regard. Il était gris clair, décoloré, d'une forme irrégulière, aux arêtes tranchantes, et ressemblait tout à fait à un grand rocher ordinaire. Dumbledore l'avait posé de façon à ce qu'il tienne sur sa face la plus grande, mais il oscillait toujours sur le bureau.

Harry releva les yeux.

«~C'est une blague, c'est ça~?

--- Pas du tout~», dit Dumbledore, secouant la tête et prenant un air très sérieux. «~Je l'ai trouvé dans les ruines de la maison de James et Lily, à Godric's Hollow, où je t'ai aussi trouvé~; et je l'ai gardé jusqu'à maintenant, jusqu'au jour où je pourrais te le rendre.~»

Dans le mélange d'hypothèses qui servait à Harry de modèle du monde, la démence de Dumbledore grimpait rapidement à l'échelle des probabilités. Mais il y \emph{avait} toujours une quantité substantielle de probabilité allouée aux autres alternatives…

«~Euh, est-ce un rocher \emph{magique}~?

--- Pas que je sache, dit Dumbledore. Mais je te recommande avec la plus grande insistance que tu le gardes en permanence près de toi.~»

Très bien. Dumbledore était \emph{probablement} dément, mais s'il ne l'était \emph{pas}… eh bien, ce serait juste trop \emph{embarrassant} que de se retrouver dans le pétrin parce qu'il avait ignoré le conseil d'un vieux sorcier impénétrable. Ça devait être en 4\textsuperscript{e} position sur la liste des \emph{100 Façons évidentes d'échouer}.

Harry s'avança et mit la main sur le rocher, essayant de dénicher une prise permettant de le soulever sans se couper. «~Je le mettrai dans ma bourse, alors.~»

Dumbledore fronça les sourcils.

«~Il se pourrait que ce ne soit pas assez proche de ta personne. Et si tu perds ta bourse en peau de Moke, ou qu'on te la vole~?

--- Vous pensez que je devrais juste transporter un gros rocher partout où je vais~?~»

Dumbledore regarda Harry avec sérieux.

«~Ça pourrait s'avérer fort sage.

--- Ah…~» dit Harry. Le rocher semblait plutôt lourd. «~Je pense que les autres élèves me poseront des questions à ce sujet.

--- Dis-leur que je t'ai ordonné de le faire, dit Dumbledore. Personne n'en doutera puisqu'ils pensent tous que je suis fou.~» Son visage était toujours parfaitement sérieux.

--- Euh, pour être honnête, si vous vous mettez à ordonner à vos élèves de transporter de grands rochers, je peux plus ou moins voir pourquoi les gens penseraient une chose pareille.

--- Ah, Harry,~» dit Dumbledore. Le vieux sorcier fit un geste courbe de la main qui semblait inclure tous les mystérieux instruments de la pièce. «~Quand on est jeune, on croit tout savoir, et on croit donc que, si on ne trouve aucune explication à quelque chose, alors aucune explication n'existe. Quand on est plus vieux, on se rend compte que l'univers entier fonctionne selon un rythme et pour une raison, même si nous ne la connaissons pas. C'est seulement notre ignorance que nous prenons pour de la folie.

--- La réalité suit toujours des lois, dit Harry, même si nous ne connaissons pas ces lois.

--- Précisément, Harry, dit Dumbledore. Comprendre cela -- et je vois que tu le \emph{comprends} -- est l'essence de la sagesse.

--- Et donc… \emph{pourquoi} exactement dois-je transporter ce rocher~?

--- À vrai dire, je n'arrive pas à imaginer une raison, dit Dumbledore.

--- … vous ne pouvez pas.~»

Dumbledore acquiesça. «~Mais ce n'est pas parce que je n'arrive pas à en imaginer une qu'il n'y en \emph{a} aucune.~»

Les instruments continuèrent de cliqueter.

«~D'accord, dit Harry, je ne sais même pas si je devrais le dire, mais ce n'est tout simplement pas comme ça qu'il faut prendre en compte notre ignorance admise au sujet de la façon dont l'univers fonctionne.

--- Ah non~?~» dit le vieux sorcier, l'air surpris et déçu.

Harry sentait que la conversation n'allait pas tourner à son avantage, mais il continua tout de même.

«~Non. Je ne sais même pas si cette erreur a un nom officiel, mais si je devais en créer un moi-même, ce serait “privilégier l'hypothèse” ou quelque chose comme ça. Comment décrire ça de façon formelle… hum… imaginez que vous avez un million de boîtes, et une seule de ces boîtes contient un diamant. Et vous avez un carton plein de détecteurs de diamant, et chaque détecteur de diamant s'allume toujours en présence d'un diamant, et s'allume une fois sur deux lorsqu'il est mis à côté d'une boîte vide. Si vous passiez vingt détecteurs sur chacune des boîtes, vous vous retrouveriez, en moyenne, avec une fausse boîte candidate et une vraie boîte candidate. Ensuite, il vous suffirait d'un ou deux détecteurs de plus avant de vous retrouver uniquement avec le vrai candidat. L'idée étant que lorsqu'il y a beaucoup de réponses possibles, la \emph{plupart} des informations dont on a besoin servent à \emph{localiser} la bonne hypothèse parmi les millions de possibilités -- à attirer votre attention vers elle. En comparaison, la quantité d'information nécessaire à choisir entre deux ou trois candidats plausibles est bien plus petite. Donc si vous foncez sans aucune preuve et placez une possibilité en particulier au centre de votre attention, vous sautez le plus gros du travail. Par exemple, vous vivez dans une ville avec un million d'habitants, et il y a un meurtre, et le détective dit, bon, on a aucune preuve, alors a-t-on envisagé la possibilité que c'est Mortimer Snodgrass qui a fait le coup~?

--- C'est lui qui a fait le coup~? dit Dumbledore.

--- Non, dit Harry. Mais plus tard, on apprend que le meurtrier a les cheveux noirs, et Mortimer a les cheveux noirs, donc tout le monde est là, ah, on dirait qu'après tout c'est bien Mortimer qui a fait le coup. Donc c'est injuste pour Mortimer que la police le \emph{place au centre de son attention} sans déjà avoir de bonnes raisons de le soupçonner. Lorsqu'il y a beaucoup de possibilités, le plus gros du travail consiste à \emph{localiser} la bonne réponse -- à commencer à y faire attention. Vous n'avez pas besoin de \emph{preuves} comme celles requises par la science ou par les tribunaux, mais il vous faut une sorte d'\emph{indice}, et cet indice doit établir une distinction entre une possibilité en particulier et les millions d'autres. Sans cela, vous ne pouvez pas juste sortir la bonne réponse de nulle part. Vous ne pouvez même pas sortir une possibilité méritant qu'on s'y attarde de nulle part. Et il y a certainement un million d'autres choses que je pourrais faire autre que de transporter le rocher de mon père partout où je vais. Ce n'est pas parce que j'ignore des choses au sujet de l'univers que je suis incertain au sujet de la façon dont je devrais raisonner lorsque je sais que j'ignore quelque chose. Les lois régissant le raisonnement probabiliste ne sont pas moins trempées dans l'acier que les lois qui gouvernent la bonne vieille logique, et ce que vous venez de faire n'est tout simplement \emph{pas permis}.~» Harry s'interrompit. «~\emph{À moins}, bien sûr, que vous n'ayez un \emph{indice} dont vous ne me parlez pas.

--- Ah~», dit Dumbledore. Il se tapota la joue, l'air pensif. «~Un argument intéressant, c'est certain, mais ne s'effondre-t-il pas lorsque tu fais une analogie entre d'un côté un million de meurtriers potentiels, dont un seul a vraiment commis le meurtre, et de l'autre choisir une façon d'agir quand de nombreuses façons d'agir sont peut-être toutes sages~? Je ne dis pas que transporter le rocher de ton père est l'une des meilleures façons d'agir, seulement qu'il est plus sage de le faire que de ne pas le faire.~»

Dumbledore farfouilla à nouveau dans le même tiroir que précédemment, semblant cette fois s'y enraciner -- mais au moins son bras avait l'air de bouger.

«~Je remarquerai~», dit Dumbledore pendant que Harry essayait encore de trouver quoi répondre à cette réplique totalement inattendue, «~que c'est une idée fausse courante chez les Serdaigle que de croire que tous les enfants intelligents y sont répartis, n'en laissant aucun pour les autres Maisons. Ce n'est pas le cas~; être réparti à Serdaigle indique que l'on est mû par son désir de savoir des choses, ce qui est une qualité toute autre que celle d'être intelligent.~» Le sorcier souriait, penché au-dessus du tiroir. «~Néanmoins, tu \emph{as} l'air intelligent. Pas comme un jeune héros ordinaire, mais plutôt comme un jeune mystérieux sorcier ancien. Je pense que j'ai choisi la mauvaise approche avec toi, Harry, et que tu es peut-être capable de comprendre des choses que peu d'autres pourraient appréhender. Je vais donc être audacieux et t'offrir un \emph{autre} héritage fort spécial.

--- Vous ne voulez pas dire… s'étrangla Harry. Mon père… \emph{possédait un autre rocher~?}

--- Excuse-moi, dit Dumbledore. Je \emph{suis} encore plus vieux et plus mystérieux que toi, et si des révélations doivent être faites, alors \emph{je} les révélerai, merci bien… oh, mais où \emph{est} ce truc~!~» Dumbledore alla plus profond dans le tiroir du bureau, puis plus profond encore. Sa tête et ses épaules et tout son torse disparurent à l'intérieur jusqu'à ce que seules ses hanches et ses jambes dépassent, comme si le tiroir le mangeait.

Harry ne put s'empêcher de se demander combien de choses se trouvaient là-dedans et à quoi ressemblerait un inventaire complet de celui-ci.

Dumbledore surgit enfin hors du tiroir, tenant l'objet de sa recherche, qu'il déposa sur le bureau aux côtés du rocher.

C'était un manuel usagé, aux bords irréguliers et au dos en mauvais état~: \emph{Fabrication de Potions niveau Intermédiaire}, par Libatius Borage. Il y avait l'image d'un flacon fumant sur la couverture.

«~Ceci, dit Dumbledore, était le manuel de Potions de cinquième année de ta mère.

--- Que je devrai transporter partout où j'irai, dit Harry.

--- \emph{Qui contient un terrible secret.} Un secret dont la révélation pourrait être si désastreuse que je dois te demander de prêter serment -- et je te demande de le faire sérieusement, Harry, quoi que tu penses de tout cela -- de ne jamais le dire à quelqu'un ni à quoi que ce soit.~»

Harry examina le manuel de Potions de cinquième année de sa mère qui, apparemment, contenait un terrible secret.

Le problème, c'était que Harry \emph{prenait} les serments très sérieusement. Un vœu était un Vœu Inviolable lorsqu'il était prononcé par un certain genre de personne.

Et…

«~J'ai soif, dit Harry, et ce n'est pas du tout bon signe.~»

Dumbledore négligea totalement de l'interroger au sujet de cette déclaration sibylline. «~\emph{Prêtes-tu} serment, Harry~?~» dit Dumbledore. Ses yeux se plongèrent dans ceux de Harry. «~Sinon, je ne peux pas te le dire.

--- Oui, dit Harry. Je promets.~» C'était le problème quand vous étiez Serdaigle. Vous ne pouviez pas refuser une offre pareille ou votre curiosité vous dévorerait vivant, et tout le monde le savait.

«~Et je prête à mon tour serment, dit Dumbledore, que ce que je suis sur le point de te dire est la vérité.~»

Dumbledore ouvrit le livre, apparemment au hasard, et Harry se pencha pour mieux voir.

«~Vois-tu ces notes~», dit Dumbledore d'une voix si basse que c'était presque un murmure, «~écrites dans les marges du livre~?~»

Harry plissa un peu les yeux. Les pages jaunissantes semblaient décrire quelque chose nommé \emph{potion de la splendeur de l'aigle}, et plusieurs des ingrédients étaient des objets que Harry ne connaissait pas du tout et dont les noms ne semblaient pas venir de l'anglais. Dans la marge se trouvait une annotation gribouillée qui disait~: \emph{Je me demande ce qui se passerait si tu utilisais du sang Thestral à la place des myrtilles~?} et immédiatement en dessous se trouvait une réponse d'une écriture différente~: \emph{Tu serais malade pendant des semaines et tu mourrais peut-être}.

«~Je les vois, dit Harry. Et alors~?~»

Dumbledore indiqua le second gribouillage. «~Cette écriture, dit-il toujours d'une voix basse, est celle de ta mère. Et \emph{cette} écriture~», déplaçant son doigt pour indiquer le premier gribouillage, «~est la mienne. Je me rendais invisible et me glissais dans sa chambre pendant qu'elle dormait. Lily pensait qu'un de ses amis les écrivait et ils avaient des disputes des plus phénoménales.~»

C'est à ce moment exact que Harry se rendit compte que le directeur de Poudlard \emph{était} bel et bien dingue.

Dumbledore le regardait avec sérieux.

«~Comprends-tu les implications de ce que je viens de te dire, Harry~?

--- Ehhh…~» dit Harry. Sa voix avait l'air d'être bloquée. «~Désolé… je… pas vraiment…

--- Ah, eh bien,~» dit Dumbledore, et il soupira. «~J'imagine que ton intelligence a des limites après tout. Il semblerait que mon enthousiasme ait été grandement prématuré. Pourrions-nous tous deux prétendre que je n'ai rien dit d'incriminant~?~»

Harry se leva de sa chaise, un sourire figé sur le visage. «~Bien sûr, dit Harry. Vous savez, il se fait assez tard et j'ai un peu faim, alors je devrais vraiment aller dîner~», et il fonça droit vers la porte.

La poignée semblait être en panne.

«~Tu m'as blessé, Harry~», dit la voix de Dumbledore sur un ton doux venant de juste derrière Harry. «~Ne te rends-tu pas au moins compte que ce que je t'ai dit est un signe de confiance~?~»

Harry se retourna lentement.

Devant lui se trouvait un sorcier très puissant et très dément, à la longue barbe d'argent et au chapeau comme un champignon géant écrasé et portant ce qui pour un Moldu aurait ressemblé à trois couches de pyjamas rose vif.

Derrière lui se trouvait une porte qui ne semblait pas fonctionner pour l'instant.

Dumbledore avait l'air plutôt triste et usé, comme s'il avait eu envie de s'appuyer sur le bâton de sorcier qu'il ne possédait pas. «~Franchement, dit Dumbledore, vous essayez une nouvelle technique au lieu de suivre la même méthode que vous avez toujours suivie pendant cent-dix ans, et les gens commencent à partir en courant.~» Le vieux sorcier secoua la tête avec chagrin.

«~J'attendais mieux de toi, Harry Potter. J'ai entendu dire que tes propres amis te croient fou. Je sais qu'ils ont tort. Ne croiras-tu pas la même chose à mon sujet~?

--- Ouvrez la porte, s'il vous plaît~», dit Harry, sa voix tremblante. «~Si vous voulez que je vous fasse à nouveau confiance un jour, ouvrez la porte.~»

Il y eut le son d'une porte s'ouvrant derrière lui.

«~Il y a d'autres choses que je comptais te dire, dit Dumbledore, et si tu pars maintenant, tu ne sauras pas de quoi il s'agissait.~»

Parfois, Harry \emph{détestait} être un Serdaigle.

\emph{Il n'a jamais fait de mal à un élève}, dit le côté Gryffondor de Harry. \emph{Rappelle-toi juste ça et tu ne paniqueras pas. Tu ne vas pas prendre la fuite juste parce que les choses commencent à être intéressantes~?}

\emph{Tu ne peux pas évincer le directeur comme ça~!} dit la partie Poufsouffle. \emph{Et s'il commence à enlever des points~? Il pourrait rendre ta vie très difficile s'il décide qu'il ne t'aime pas~!}

Et une partie de lui-même, que Harry n'aimait pas beaucoup mais qu'il n'arrivait pas tout à fait à faire taire, pesait les avantages potentiels qu'il y avait à être l'un des rares amis de ce vieux sorcier fou qui se trouvait aussi être directeur, Président et Manitou Suprême. Et malheureusement son Serpentard intérieur semblait attirer les gens vers le Côté Obscur bien mieux que ne le faisait Drago, parce qu'il disait des choses telles que \emph{pauvre vieux, il a l'air d'avoir besoin de quelqu'un à qui parler, on ne dirait pas~?} et \emph{tu ne voudrais pas qu'un homme si puissant se retrouve à accorder sa confiance à quelqu'un de moins vertueux que toi, si~?} et \emph{Je me demande quelles sortes de secrets incroyables Dumbledore pourrait te révéler si vous, tu sais, si vous deveniez amis} et \emph{je parie qu'il a une collection de livres vraaaiment intéressante.}

\emph{Vous êtes tous dingues}, dit Harry à l'attention de cet assemblage, mais il avait été mis en minorité par chaque partie de lui-même.

Harry se retourna, fit un pas vers la porte, tendit la main, et la referma délibérément. C'était un sacrifice gratuit vu qu'il comptait de toute façon rester, et Dumbledore pouvait de toute façon contrôler ses mouvements, mais peut-être que ce geste l'impressionnerait.

Lorsque Harry se retourna encore, il vit que le puissant sorcier fou souriait de nouveau et avait l'air amical. C'était positif, peut-être.

«~Ne le refaites pas s'il vous plaît, dit Harry. Je n'aime pas être piégé.

--- Je \emph{suis} navré d'avoir fait ça, Harry~», dit Dumbledore d'un ton qui ressemblait à des excuses sincères. «~Mais il aurait été épouvantablement mal avisé de te laisser partir sans le rocher de ton père.

--- Bien sûr, dit Harry. Ce n'était pas raisonnable de ma part de m'attendre à ce que la porte s'ouvre avant que j'aie mis les objets de quête dans mon inventaire.~»

Dumbledore sourit et hocha la tête.

Harry marcha jusqu'au bureau, fit coulisser la bourse en peau de Moke jusqu'à l'avant de sa ceinture, et au prix de quelques efforts parvint à lever le rocher avec ses bras d'enfant de onze ans et à le donner à manger à la bourse.

Il pouvait vraiment sentir le poids diminuer lentement alors que l'Ouverture Élargissante mangeait le rocher, et le rot qui s'ensuivit était plutôt bruyant et distinctement rouspéteur.

Le Manuel de Potions de cinquième année de sa mère (qui renfermait un secret à vrai dire assez terrible) suivit peu après.

Puis le Serpentard intérieur de Harry fit une suggestion sournoise visant à s'attirer les bonnes grâces du directeur, ornée d'un baratin parfait destiné à obtenir le soutien du côté Serdaigle.

«~Alors, dit Harry. Euh. Puisque je suis ici, j'imagine que vous ne voudriez pas me faire faire une petite visite de votre bureau~? Je serais assez curieux de savoir ce que sont certaines de ces choses,~» et c'était son euphémisme du mois de septembre.

Dumbledore le fixa un moment, puis acquiesça en faisant une légère grimace. «~Je suis flatté par ton intérêt, dit-il, mais j'ai peur qu'il n'y a pas grand-chose à dire.~» Dumbledore fit un pas de plus vers le mur et tendit le doigt en direction d'un homme endormi.

«~Ce sont les portraits des directeurs de Poudlard.~» Il se tourna et pointa vers son bureau. «~C'est mon bureau.~» Il pointa vers sa chaise. «~C'est ma chaise…

--- Excusez-moi, dit Harry, en fait, je m'interrogeais sur ces…~» Harry pointa en direction d'un petit cube qui chuchotait doucement~: «~blurpe… blurpe… blurpe.

--- Oh, les petites choses gélatineuses~? dit Dumbledore. Elles étaient incluses dans le bureau de directeur et je n'ai absolument aucune idée de ce que la plupart d'entre elles font. Mais \emph{ce} cadran avec huit aiguilles compte le nombre de, disons d'éternuements, faits par les sorcières gauchères de France, et tu ne me croirais pas si je te disais le travail que ça a pris pour le faire fonctionner correctement. Et \emph{celui-là} avec les petits gigoteurs est de mon invention, et Minerva ne va jamais, jamais réussir à comprendre ce qu'il fait.~»

Dumbledore fit un pas vers le porte-chapeaux pendant que Harry finissait d'emmagasiner ces informations. «~Ici, bien sûr, nous avons le Choixpeau Magique, je crois que vous vous êtes rencontrés. Il m'a dit qu'il ne fallait plus jamais qu'il soit mis sur ta tête, quelles que soient les circonstances. Tu n'es que le quatorzième étudiant de l'histoire au sujet duquel il a dit ça, il y a aussi eu Baba Yaga, et je te parlerai des douze autres quand tu seras plus vieux. Ça c'est un parapluie. Ça c'est un autre parapluie.~» Dumbledore fit quelques pas de plus et se retourna, avec un grand sourire sur le visage. «~Et bien sûr, la plupart des gens qui viennent dans mon bureau veulent voir Fumseck.~»

Dumbledore se tenait à côté de l'oiseau sur la plate-forme dorée.

Harry s'approcha, plutôt perplexe.

«~C'est Fumseck~?

--- Fumseck est un phénix, dit Dumbledore. Des créatures magiques très rares et très puissantes.

--- Ah…~» dit Harry. Il baissa sa tête et fixa les petits petits yeux en perle noires qui ne montraient pas le moindre signe d'intelligence.

«~Ahhh…~» dit à nouveau Harry.

Il était presque certain de reconnaître la forme de l'oiseau. C'était assez difficile à rater.

«~Humm…~»

\emph{Dis quelque chose d'intelligent~!} Rugit l'esprit de Harry à sa propre intention. \emph{Ne reste pas là à baragouiner comme un crétin~!}

\emph{Ben qu'est-ce que je peux bien être} censé \emph{dire~?} répondit l'esprit de Harry.

\emph{N'importe quoi~!}

\emph{Tu veux dire, n'importe quoi sauf} «~Fumseck \emph{est un poulet…}~»

\emph{Oui~! N'importe quoi sauf ça~!}

«~Et donc, ah, quelle sorte de magie font les phénix alors~?

--- Leurs larmes ont le pouvoir de guérir, dit Dumbledore. ~»Ce sont des créatures de feu, et ils se déplacent entre les lieux aussi facilement que le feu peut s'éteindre quelque part et se rallumer ailleurs. La tension intense que provoque leur magie innée fait vieillir leur corps très rapidement, et pourtant ils sont les plus éternelles des créatures de ce monde, car lorsque leur corps les abandonne, ils s'immolent dans un jet de flammes et laissent derrière eux un nouveau-né, ou parfois un œuf.~» Dumbledore s'approcha et inspecta le poulet, fronçant les sourcils. «~Hmm… l'air un peu patraque on dirait.~»

Lorsque la phrase percuta pleinement l'esprit de Harry, le poulet était déjà en feu.

Le bec du poulet s'ouvrit, mais il n'eut pas le temps de caqueter une seule fois avant de commencer à flétrir et à se carboniser. L'incendie fut bref, intense, et complètement isolé~; il n'y avait pas d'odeur de brûlé.

Puis le feu mourut seulement quelques secondes après avoir commencé, laissant derrière un petit tas pathétique de cendres sur la plate-forme dorée.

«~N'ai pas l'air si horrifié, Harry~! dit Dumbledore. Fumseck n'a pas eu mal.~» La main de Dumbledore plongea dans sa poche, puis la même main passa dans les cendres et fit surgir un petit œuf jaunâtre. «~Regarde, voilà un œuf~!

--- Oh… waoh… incroyable…

--- Mais nous devrions nous activer~», dit Dumbledore. Laissant l'œuf derrière lui entre les cendres du poulet, il revint à son trône et s'assit. «~C'est presque l'heure du dîner après tout, et nous ne voudrions pas avoir à utiliser nos Retourneurs de Temps.~»

Il y eut une violente lutte de pouvoir au Gouvernement de Harry. Serpentard et Poufsouffle avaient changé de camp après avoir vu le directeur de Poudlard mettre le feu à un poulet.

«~Oui, nous activer, dirent les lèvres de Harry. Et puis, dîner.~»

\emph{Tu baragouines encore comme un crétin} nota le Critique Interne de Harry.

«~Bon, dit Dumbledore. J'ai peur d'avoir une confession à te faire, Harry. Une confession et une excuse.

--- Les excuses, c'est bien.~» \emph{Ça ne veut rien dire~! Mais de quoi je parle~?}

Le vieux sorcier soupira profondément.

«~Tu ne le penseras peut-être plus après avoir compris ce que j'ai à te dire. J'ai bien peur, Harry, de t'avoir manipulé pendant toute ta vie. C'est moi qui t'ai remis à la garde de tes beaux-parents malfaisants…

--- Mes beaux-parents ne sont pas malfaisants~! lâcha Harry. Mes \emph{parents}, je veux dire~!

--- Ils ne le sont pas~?~» dit Dumbledore, l'air surpris et déçu. «~Pas même un peu malfaisants~? Ça ne cadre pas avec…~»

Le Serpentard intérieur de Harry hurla à s'en faire exploser les poumons, TAIS-TOI IDIOT IL T'ENLÈVERA À EUX~!

«~Non, non~», dit Harry, les lèvres figées en une grimace livide, «~j'essayais juste de vous épargner, ils sont à vrai dire très malfaisants…

--- Ils le sont~?~» Dumbledore se pencha en avant, le regardant avec intensité. «~Que font-ils~?~»

\emph{Parle vite} «~ils, ah, je dois faire la vaisselle et laver des problèmes et ils ne me laissent pas lire beaucoup de livres et…

--- Ah, bien, c'est bon à entendre~», dit Dumbledore, se penchant à nouveau en arrière. Il sourit d'un air triste. «~Je te demande pardon pour \emph{ça}, alors. Maintenant, où en étais-je~? Ah, oui. Je suis navré de te dire, Harry, que je suis responsable de presque tout ce qui t'es jamais arrivé de mal. Je sais que ça te mettra probablement très en colère.

--- Oui, je suis très en colère~! dit Harry. Grrr~!

Le Critique Interne de Harry lui décerna promptement le Prix Ultime de Pire Jeu d'Acteur de l'Histoire de Tous les Temps.

«~Et je voulais juste que tu le saches, dit Dumbledore, je voulais te le dire aussitôt que possible, au cas où quelque chose arriverait à l'un de nous deux plus tard, que je suis vraiment, vraiment navré. Pour tout ce qui s'est déjà produit, et pour tout ce qui se produira.~»

De l'humidité scintillait dans les yeux du vieux sorcier.

«~Et je suis très en colère~! dit Harry. Tellement en colère que je veux partir tout de suite à moins que vous n'ayez quelque chose d'autre à me dire~!~»

\emph{PARS avant qu'il ne te mette le feu à toi aussi~!} crièrent Serpentard, Poufsouffle et Gryffondor.

«~Je comprends, dit Dumbledore. Alors une dernière chose, Harry. Tu ne devras \emph{pas} essayer de franchir la porte interdite dans le couloir du troisième étage. Il est impossible que tu traverses tous les pièges, et je ne voudrais pas apprendre que tu t'es fait mal en essayant. Enfin, je doute que tu puisses ne serait-ce qu'ouvrir la première porte, puisqu'elle est fermée à clé et que tu ne connais pas le sort \emph{Alohomora}…~»

Harry fit demi-tour et se précipita vers la sortie à pleine vitesse, la poignée pivota agréablement dans sa main, et un instant après il descendait les escaliers quatre à quatre alors même qu'ils tournaient, ses pieds se faisaient presque des croche-pattes, et un moment plus tard il était en bas et la gargouille faisait un pas sur le côté et Harry se propulsait hors de la cage d'escalier tel un boulet de canon.

\later

Harry Potter.

Il devait y avoir quelque chose de spécial chez Harry Potter.

C'était jeudi pour tout le monde après tout, et ce genre de choses ne semblait arriver à personne d'autre.

Il était 18h21 un jeudi après-midi quand Harry Potter, se propulsant hors de la cage d'escalier tel un boulet de canon et accélérant au maximum, fonça droit sur Minerva McGonagall alors qu'elle prenait un tournant, en chemin vers le bureau du directeur.

Heureusement, aucun d'entre eux n'eut très mal. Comme on le lui avait expliqué plus tôt dans la journée -- alors qu'il refusait de jamais s'approcher à nouveau d'un balai -- le Quidditch avait besoin de Cognards en fer solide juste pour avoir une bonne chance de blesser les joueurs, puisque les sorciers avaient tendance à être beaucoup plus résistants aux chocs que les Moldus.

Harry et le professeur McGonagall finirent tous deux au sol, et les parchemins qu'elle transportait étalés dans le couloir.

Il y eut une pause épouvantable.

«~Harry Potter~», souffla le professeur McGonagall de là où elle était allongée, juste à côté de Harry. Sa voix devint presque un cri. «~\emph{Que faisiez-vous dans le bureau du directeur~?}

--- Rien~! glapit Harry.

--- \emph{Parliez-vous du professeur de Défense} \emph{?}

--- Non~! Dumbledore m'a appelé dans son bureau et il m'a donné ce gros rocher et il m'a dit que c'était celui de mon père et que je devrai le transporter partout~!~»

Il y eut une autre pause épouvantable.

«~Je vois~», dit le professeur McGonagall, sa voix un peu plus calme. Elle se leva, s'épousseta, et jeta un coup d'œil aux parchemins éparpillés, qui bondirent pour former une pile bien ordonnée et filèrent le long du mur du couloir comme s'ils essayaient d'échapper à son regard.

«~Mes amitiés, M. Potter, et je vous demande pardon d'avoir douté de vous.

--- Professeur McGonagall~», dit Harry. Sa voix tremblait. Il poussa contre le sol, se leva, et regarda ce visage sérieux et sain d'esprit. «~Professeur McGonagall…

--- Oui, M. Potter~?

--- Pensez-vous que je devrais le faire~? dit Harry d'une petite voix. Transporter le rocher de mon père partout où je vais~?~»

Le professeur McGonagall soupira. «~J'ai bien peur que ce ne soit entre vous et le directeur.~» Elle hésita. «~Je dirais que complètement ignorer ce que le directeur dit n'est presque jamais une bonne idée. Je \emph{suis} navrée d'apprendre votre dilemme, M. Potter, et si je \emph{peux} vous aider d'une quelconque façon, quoi que vous choisissiez de faire…

--- Euh, dit Harry. À vrai dire je me disais qu'une fois que j'aurais trouvé comment faire, je pourrais métamorphoser le rocher en un anneau et le porter à mon doigt. Si vous pouviez m'enseigner comment maintenir une métamorphose…

--- Vous avez bien fait de m'en parler avant~», dit le professeur McGonagall, son visage devenant légèrement sévère. «~Si vous perdiez le contrôle de la métamorphose, l'annulation vous couperait le doigt et vous fendrait probablement la main en deux. Et à votre âge, même un anneau est une cible trop large pour être maintenue indéfiniment sans que ce soit un sérieux drain de votre magie. Mais je peux vous faire forger un anneau doté d'un emplacement pour un joyau, un \emph{petit} joyau, en contact avec votre peau, et vous pouvez pratiquer avec un objet sûr, comme un marshmallow. Lorsque vous l'aurez maintenu avec succès, même pendant votre sommeil, pendant un mois entier, je vous autoriserai à métamorphoser le, ah, le rocher de votre père…~» Le professeur McGonagall laissa sa phrase en suspens. «~Le directeur a-t-il \emph{vraiment}…

--- Oui. Ah… euh…~»

Le Professeur McGonagall soupira.

«~C'est un peu étrange, même de sa part.~» Elle se baissa et ramassa la pile de parchemins. «~Je suis navrée, M. Potter. Je vous demande à nouveau pardon de ne pas vous avoir fait confiance. Mais maintenant, c'est à moi d'aller voir le directeur.

--- Ah… bonne chance, j'imagine. Euh…

--- Merci, M. Potter.

--- Hmm…~»

Le professeur McGonagall marcha jusqu'à la gargouille, donna le mot de passe sans que Harry puisse l'entendre, et prit place sur les escaliers en spirale tournants. Elle commença à s'élever hors de la vue de Harry, et la gargouille commença à revenir…

«~\emph{Professeur McGonagall le directeur a mis le feu à un poulet~!}

--- Il a \emph{quo…}~»
%  LocalWords:  hursday Remembrall Goyle’s Hufflepuffle Slytherslime Hah
%  LocalWords:  Remembralls Libatius Ehhh reaallly blorple wibblers Ahhh
%  LocalWords:  Umm Grrr wha
