% vim:spell:spelllang=fr

\chapter{Localiser l'hypothèse}

\lettrine{J}{eudi}.

\hplettrineextrapara
Si l'on voulait être précis, jeudi matin à 7h24.

Harry se trouvait assis sur son lit, un livre scolaire mollement retenu par ses mains immobiles.

Harry venait tout juste d'avoir l'idée d'un test expérimental \emph{véritablement brillant}.

Cela signifiait qu'il lui faudrait attendre une heure de plus avant d'aller petit-déjeuner, mais c'était pour ça qu'il avait des barres de céréales.
Non, cette idée devait absolument définitivement être testée tout de suite, immédiatement, maintenant.

Harry mis le livre de côté, bondit hors de son lit, courut autour de son lit, tira le niveau caverne de sa malle, descendit les escaliers quatre à quatre, et commença à déplacer ses cartons de livres.
(Il fallait vraiment qu'il trouve le temps de récupérer des étagères et vider ses cartons, mais il était en plein milieu de sa compétition de lecture de manuels avec Hermione et il prenait du retard donc il n'avait pas le temps.)

Harry trouva le livre qu'il voulait et courut en remontant les escaliers.

Les autres garçons se préparaient à aller prendre leur petit-déjeuner dans le grand hall et à démarrer la journée.

«~Excusez-moi, vous pourriez faire quelque chose pour moi~?~»
demanda Harry.
Tout en parlant, il feuilletait la table des matières du livre, trouva la page contenant les dix mille premiers nombres premiers, ouvrit le livre à cette page et le fourra dans les mains d'Anthony Goldstein.
«~Choisis deux nombres à trois chiffres dans cette liste.
Ne me dis pas lesquels.
Multiplie-les juste entre eux, et donne-moi le produit.
Oh, et pourrais-tu faire le calcul deux fois, pour vérifier~?
S'il te plaît, assure-toi que tu as vraiment la bonne réponse, je ne suis pas certain de ce qui pourrait m'arriver à moi ou à l'univers si tu fais une erreur de multiplication.~»

Cela en disait long sur ce qu'était devenue la vie dans ce dortoir en l'espace de quelques jours qu'Anthony ne prenne même pas la peine de dire
«~Mais qu'est-ce qui te prend tout d'un coup~?~»
ou
«~Ça a l'air vraiment bizarre, pourquoi est-ce que tu me demandes de faire ça~?~»
ou
«~Qu'est-ce que tu veux dire par “je ne suis pas sûr de ce qui risque d'arriver à l'univers”~?~».

Anthony accepta le livre en silence et sortit un parchemin et une plume.
Harry se retourna et ferma les yeux, s'assurant de ne rien voir, se balançant d'avant en arrière et de haut en bas avec impatience.
Il s'empara d'un bloc-note et d'un critérium et se prépara à écrire.

«~C'est bon, dit Anthony, cent quatre-vingt-un mille quatre cent vingt-neuf.~»

Harry écrivit 181\,429.
Il répéta ce qu'il venait d'écrire, et Anthony confirma.

Puis Harry fonça jusqu'au niveau caverne de sa malle, jeta un coup d'œil à sa montre (la montre indiquait 4h28, ce qui voulait dire 7h28) puis ferma les yeux.

Environ trente secondes plus tard, Harry entendit des bruits de pas, suivits par le son du tiroir du niveau caverne de la malle qu'on refermait.
(Harry n'avait pas peur de l'asphyxie.
Un enchantement de Rafraîchissement d'Air automatique faisait partie des avantages qu'il y avait à acheter une malle de très bonne qualité.
La magie était vraiment formidable, elle n'avait pas besoin de se soucier des factures d'électricité.)

Et lorsque Harry ouvrit les yeux, il vit exactement ce qu'il avait espéré voir, un bout de papier plié posé au sol, cadeau de son futur lui.

Appelons ce bout de papier «~papier-2~».

Harry déchira un bout de papier de son bloc-notes.

Appelons ce bout de papier «~papier-1~».
C'était, bien sûr, le même bout de papier.
On pouvait même voir, si on regardait de près, que les bords déchirés correspondaient.

Harry passa mentalement en revue l'algorithme qu'il allait suivre.

Si Harry ouvrait papier-2 et qu'il était vierge, alors il écrirait «~101~$\times$~101~» sur papier-1, le plierait, étudierait pendant une heure, reviendrait en arrière dans le temps, déposerait papier-1 (qui deviendrait ainsi papier-2), et sortirait du niveau caverne pour rejoindre ses camarades au petit-déjeuner.

Si Harry ouvrait papier-2 et qu'il y avait deux nombres écrits dessus, Harry multiplierait ces nombres entre eux.

Si leur produit était égal à 181\,429, Harry réécrirait ces deux nombres sur papier-1 et renverrait papier-1 en arrière dans le temps.

Sinon, Harry ajouterait 2 au nombre de droite et écrirait la nouvelle paire de nombres sur papier-1.
À moins que cela ne rende le nombre de droite supérieur à 997, auquel cas Harry ajouterait 2 au nombre de gauche et écrirait 101 à droite.

Et s'il y avait 997~$\times$~997 sur papier-2, Harry laisserait papier-1 vierge.

Ce qui voulait dire que la seule boucle temporelle \emph{stable} possible était celle où papier-2 contenait les deux facteurs premiers de 181\,429.

Si cela fonctionnait, Harry pourrait utiliser cette technique pour obtenir n'importe quelle sorte de réponse facile à vérifier mais difficile à trouver.
Il n'aurait pas \emph{seulement} montré que P$=$NP quand on avait un retourneur de temps, non, cette astuce était \emph{plus générale} que cela.
Harry pourrait l'utiliser pour trouver des combinaisons de cadenas ou n'importe quel genre de mot de passe.
Ou même peut-être trouver l'entrée de la chambre des secrets de Serpentard, si Harry trouvait une façon systématique de décrire tous les emplacements de Poudlard.
Ce serait une technique de triche extraordinaire, et pourtant Harry avait des attentes élevées en matière de triche.

Harry prit Papier-2 dans ses mains tremblantes, et le déplia.

Papier-2 disait, d'une écriture légèrement tremblotante~:

\quad

\centerline{
\shout{On ne plaisante pas avec le temps}
}
\quad

Harry écrivit «~\shout{On ne plaisante pas avec le temps}~» sur Papier-1, sa main légèrement tremblante, le plia avec soin, et prit la résolution de ne plus faire d'expérience véritablement brillante sur le temps avant d'avoir au moins quinze ans.

De ce que connaissait Harry, cela avait été le résultat expérimental le plus effrayant de toute l'histoire de la science.

Harry eut quelques difficultés à se concentrer sur la lecture de son manuel pendant l'heure suivante.

C'est ainsi que le jeudi de Harry avait commencé.

\later

Jeudi.

Si l'on voulait être précis, jeudi après-midi à 15h32.

Harry et tous les autres garçons de première année étaient à l'extérieur sur un terrain herbeux avec Madame Bibine, se tenant debout juste à côté de la réserve de balais de Poudlard.
Les filles apprendraient à voler séparément.
Apparemment, pour une quelconque raison, les filles ne voulaient pas apprendre à voler sur des balais en présence de garçons.

Harry avait un peu flageolé toute la journée.
Il n'arrivait pas à s'empêcher de se demander comment cette boucle temporelle stable \emph{particulière} avait bien pu être sélectionnée à partir de ce qui, rétrospectivement, semblait être un large espace de possibilités.

Et aussi~: sérieusement, des \emph{balais}~?
Il allait voler sur ce qui était essentiellement un segment de droite~?
N'était-ce pas quasiment la forme la plus instable qu'on puisse jamais trouver, à moins d'essayer de se tenir sur une bille~?
Qui avait sélectionné \emph{cette} forme pour un engin volant, parmi toutes les formes possibles~?
Harry avait espéré que ce ne soit qu'une façon de parler, mais non, ils se tenaient devant ce qui ressemblait indiscutablement à d'ordinaires balais de cuisine en bois.
Quelqu'un avait-il juste bloqué sur l'idée des balais et échoué à envisager autre chose~?
Ça devait être ça.
Il y avait zéro probabilité pour que, si on les développait indépendamment l'un de l'autre, les modèles \emph{optimaux} permettant de nettoyer une cuisine et ceux permettant de voler dans les airs se révèlent coïncider exactement.

C'était une belle journée, avec un ciel bleu clair et un soleil éclatant qui ne demandait qu'à rentrer dans vos yeux et à vous empêcher de voir quoi que ce soit, s'il se trouvait que vous étiez en train d'essayer de voleter dans le ciel.
Le sol était propre et sec, sentait presque la terre cuite, et semblait étrangement être très, très dur sous les chaussures de Harry.

Harry se persuadait en boucle que si l'on attendait du plus petit commun dénominateur d'enfants de onze ans qu'il apprenne à voler cela ne pouvait pas être si difficile.

«~Levez votre main droite au-dessus du balai, ou la main gauche si vous êtes gaucher, s'écria Madame Bibine, et dites \shout{Debout}~!~»

Tout le monde cria «~\shout{Debout}~!~»

Le balai bondit d'impatience dans la main de Harry.

Ce qui le plaça dans le peloton de tête de la classe, pour une fois.
Apparemment dire «~\shout{Debout}~!~»
était beaucoup plus difficile que cela en avait l'air, et la plupart des balais roulaient un peu partout sur l'herbe ou essayaient de s'éloigner discrètement de leur soi-disant futur conducteur.

(Bien sûr, Harry aurait été prêt à parier qu'Hermione avait réussi au moins aussi bien que lui lorsque cela avait été son tour, plus tôt dans la journée.
Il n'y avait rien qu'\emph{il} puisse maîtriser du premier coup et qui déroute Hermione, et si une telle chose \emph{existait} et se révélait être \emph{le pilotage de balai} plutôt qu'une activité intellectuelle, Harry préférait encore mourir.)

Il fallut un bon moment avant que tout le monde ait un balai devant lui.
Madame Bibine leur montra comment monter dessus, puis déambula entre les élèves, corrigeant leurs postures et positions de leurs mains.
Apparemment, même parmi les quelques enfants qu'on avait autorisés à voler chez eux, peu avaient appris à le faire correctement.

Madame Bibine passa en revue l'assemblée de garçons et approuva de la tête.
«~Et maintenant, à mon coup de sifflet, vous donnez un bon coup de pied sur le sol pour vous lancer.~»

Harry déglutit, essayant de réprimer la sensation de nausée venant de son estomac.

«~Maintenez vos balais stables, élevez-vous de quelques dizaines de centimètres, et revenez immédiatement au sol en vous penchant légèrement en avant.
Attention au coup de sifflet… trois… deux…~»

L'un des balais piqua vers le ciel, accompagné par les cris d'un jeune garçon -- d'horreur, pas de délectation.
Le garçon tournoyait à une effroyable vitesse tout en continuant de monter, et ils ne pouvaient qu'entrapercevoir son visage pâle…

Comme au ralenti, Harry avait bondi de son balai et fouillait sa robe à la recherche de sa baguette, même s'il ne savait pas exactement ce qu'il comptait en faire, il avait eu exactement deux cours de sortilèges, et le dernier \emph{avait} été le sortilège de Lévitation, mais Harry n'avait réussi à jeter le sort qu'une fois sur trois et il ne pouvait certainement pas faire léviter des personnes entières…

\emph{S'il y a un pouvoir caché en moi, qu'il se révèle MAINTENANT~!}

«~Reviens, mon garçon~!~»
cria Madame Bibine (ce qui devait être l'injonction la plus inutile qu'on puisse imaginer face à un balai incontrôlable, venant d'une \emph{professeure de vol}, et une section totalement automatique du cerveau de Harry ajouta Madame Bibine à sa liste des imbéciles).

Et le balai désarçonna le garçon.

Il sembla d'abord se déplacer très lentement dans les airs.

«~\emph{Wingardium Leviosa~!}~» hurla Harry.

Le sort échoua.
Harry pouvait le sentir échouer.

On entendit un BAM et un lointain son de craquement, et le garçon gisait au sol, face contre terre, petit tas ramassé.

Harry rengaina sa baguette et courut à pleine vitesse.
Il parvint aux côtés du garçon au même moment que Madame Bibine, et Harry fouilla dans sa bourse et essaya de se souvenir de oh mon dieu comment ça s'appelle peu importe il essayerait juste «~pack de soins~!~»
et le kit apparut dans sa main et…

«~Poignet cassé, dit Madame Bibine.
Calme-toi mon garçon, il a juste le poignet cassé~!~»

Harry eut une sorte d'embardée mentale tandis que son esprit débraya du mode panique.

Le Kit de Soins d'Urgence Plus était ouvert devant lui, il avait en main une seringue de feu liquide, qui aurait maintenu le cerveau du garçon oxygéné s'il s'était brisé la nuque.

«~Ah…~» fit Harry d'une voix chancelante.
Son cœur battait avec tant de force qu'il n'entendait presque pas sa respiration à bout de souffle.
«~Un os cassé… d'accord…
Ficelle de Maintien~?

--- C'est seulement pour les urgences, jeta Madame Bibine.
Range ça, il va bien~», elle se pencha au-dessus du garçon, lui offrant sa main.
«~Allons mon garçon, tout va bien, on se met debout~!

--- Vous n'allez pas sérieusement le faire remonter sur un balai~?~»
fit Harry avec horreur.

Madame Bibine fusilla Harry du regard.
«~Bien sûr que non~!~»
Elle remit d'aplomb le garçon en tirant sur son bras encore valide -- Harry vit avec stupeur que c'était \emph{encore} Neville Londubat, qu'est-ce qui se passe \emph{chez} lui~?
-- et elle se retourna vers les enfants qui observaient la scène.
«~Vous ne bougez pas pendant que j'amène ce garçon à l'infirmerie~!
Laissez ces balais là où il sont si vous ne voulez pas être renvoyés de Poudlard plus vite que le temps de dire “Quidditch.”
Allons-y, mon garçon.~»

Et Madame Bibine accompagna Neville, qui serrait son poignet et essayait de s'empêcher de renifler.

Lorsqu'ils furent hors de portée, l'un des Serpentard commença à ricaner.

Ce qui déclencha d'autres rires de moquerie chez les autres.

Harry se retourna et les regarda.
Cela semblait être un moment opportun pour se graver quelques visages en mémoire.

Puis Harry vit que Drago marchait nonchalamment vers lui, accompagné de M. Crabbe et de M. Goyle.
M. Crabbe ne souriait pas.
M. Goyle en revanche, incontestablement.
Drago lui-même arborait sur son visage une expression très contrôlée avec juste quelques tressaillements, et Harry en déduit que Drago trouvait la situation hilarante mais ne voyait aucun avantage politique à en rire maintenant plutôt que dans les donjons de Serpentard un peu plus tard.

«~Eh bien, Potter~», dit Drago d'une voix grave qui ne portait pas, toujours avec cette expression contrôlée bien que tressaillante, «~je voulais juste te dire que, quand tu tires parti d'une urgence pour montrer ton leadership, il vaut mieux avoir l'air de contrôler totalement la situation, plutôt que, disons, d'avoir l'air complètement paniqué.~»
M. Goyle gloussa, et Drago lui jeta un regard de réprimande.
«~Mais tu as probablement gagné quelques points quand même.
Besoin d'aide pour ranger ce kit de soin~?~»

Harry se tourna pour regarder le kit, ce qui détourna son visage de celui de Drago.
«~Je pense que ça ira,~» dit Harry.
Il remit la seringue en place, referma les loquets, et se leva.

Ernie Macmillan arriva au moment où Harry donnait le kit à manger à sa bourse.

«~Merci, Harry Potter, de la part de Poufsouffle, dit solennellement Ernie Macmillan.
C'était une bonne tentative, et une bonne idée.

--- Une bonne idée, en effet, répéta Drago de sa voix traînante.
Pourquoi personne à Poufsouffle n'a-t-il sorti sa baguette~?
Peut-être que si vous aviez \emph{tous} aidé, et pas juste Potter, vous l'auriez rattrapé.
Je pensais que les Poufsouffle étaient censés se serrer les coudes~?~»

Ernie semblait être déchiré entre la colère et l'envie de mourir de honte.
«~On n'y a pas pensé à temps…

--- Ah, fit Drago, pas \emph{pensé} à temps, j'imagine que c'est pour ça qu'il vaut mieux être ami avec un seul Serdaigle qu'avec tous les Poufsouffle.~»

Enfer, comment Harry était-il supposer jongler avec cela…
«~Tu n'aides pas vraiment~», dit Harry d'un ton léger.
Espérant que Drago interprète cela comme \emph{tu interfères avec mes plans, s'il te plaît ferme-la}.

«~Eh, qu'est-ce que c'est que ça~?~»
dit M. Goyle.
Il se pencha et ramassa quelque chose de la taille d'une très grosse bille, une balle de verre qui semblait pleine d'une brume blanche tourbillonnante.

Ernie cligna des yeux.
«~Le Rapeltout de Neville~!

--- C'est quoi un Rapeltout~? demanda Harry.

--- Il devient rouge si on a oublié quelque chose, dit Ernie.
Ceci dit, il ne dit pas ce qu'on a oublié.
Donne-le-moi s'il te plaît, et je le rendrai à Neville plus tard.~»
Ernie tendit sa main.

M. Goyle fit apparaître un sourire sur ses lèvres, puis se retourna et courut en s'éloignant.

Ernie resta immobile un instant, surpris, puis il cria «~Eh~!~»
et courut après M. Goyle.

Et M. Goyle attrapa un balai, grimpa dessus avec souplesse et décolla.

Harry était ébahi.
Madame Bibine n'avait-elle pas dit qu'il serait \emph{renvoyé}~?

«~\emph{Quel imbécile~!}~» siffla Drago.
Il ouvrit la bouche pour crier…

«~\emph{Eh~!} cria Ernie.
C'est à Neville~!
\emph{Rend-le}~!~»

Les Serpentard commencèrent à acclamer et à siffler.

Drago referma la bouche aussi sec.
Harry vit l'indécision qui était brusquement apparue sur son visage.

«~Drago, dit Harry d'une voix basse, si tu n'ordonnes pas à cet idiot d'atterrir, la professeure va revenir et…

--- \emph{Viens le chercher, Poufsouffle~!}~» hurla M. Goyle, et des salves d'acclamations montèrent des Serpentard.

«~Je ne \emph{peux pas}~! chuchota Drago.
Tout le monde à Serpentard penserait que je suis \emph{faible}~!

--- Et si M. Goyle est renvoyé, siffla Harry, ton \emph{père} va penser que tu es un \emph{crétin}~!~»

L'agonie déformait le visage de Drago.

À cet instant…

«~Eh, \emph{Serpensale}, cria Ernie, on ne t'a jamais dit que les Poufsouffle se serraient les coudes~?
\emph{Baguettes, Poufsouffle~!}~»

Et il y eut soudain beaucoup de baguettes pointées en direction de M. Goyle.

Trois secondes plus tard…

«~\emph{Baguettes, Serpentard~!}~» dirent une demi-dizaine de Serpentard.

Et il y eut beaucoup de baguettes pointées en direction de Poufsouffle.

Deux secondes plus tard…

«~\emph{Baguettes, Gryffondor~!}

--- \emph{Fais quelque chose, Potter~!} chuchota Drago.
\emph{Je ne peux pas être celui qui arrête ça il faut que ce soit toi~!
Je te revaudrai ça trouve juste quelque chose tu n'es pas censé être brillant~?}~»

Dans environ cinq secondes et demie, réalisa Harry, quelqu'un allait lancer un sort d'attaque simple sumérien et quand tout serait terminé et que les enseignants auraient fini de renvoyer toutes les personnes impliquées, les seuls garçons de cette année encore à Poudlard seraient les Serdaigle.

«~\emph{Baguettes, Serdaigle~!}~» cria Michael Corner, qui se sentait apparemment exclu du désastre.

«~\shout{Gregory Goyle~!} hurla Harry.
\emph{Je te lance un défi pour la possession du Rapeltout de Neville~!}~»

Tous s'interrompirent, hésitants.

«~Oh, vraiment~?~»
fit Drago.
Harry n'avait jamais entendu une voix aussi forte et pourtant traînante.
«~Ça a l'air intéressant.
Quel genre de défi, Potter~?~»

Euh…

L'inspiration de Harry n'était pas allée au-delà de “défi.”
Quel genre de défi, il ne pouvait pas dire “échecs” parce que Drago ne pourrait pas accepter sans que cela ait l'air bizarre, il ne pouvait pas dire “bras de fer” parce que M. Goyle l'écrabouillerait…

«~Que pensez-vous de ça~? dit Harry bien fort.
Gregory Goyle et moi nous tenons loin l'un de l'autre, et personne d'autre n'a le droit de s'approcher.
Nous n'utilisons pas nos baguettes, et personne d'autre non plus.
Je ne bouge pas de là où je suis, et lui non plus.
Et si j'arrive à mettre la main sur le Rapeltout de Neville, alors Gregory Goyle abandonnera toute prétention au Rapeltout et me le donnera.~»

Silence de nouveau, tandis que le soulagement visible chez les élèves alentour se transformait en confusion.

«~Ha, Potter~! fit Drago avec force.
J'attends de voir comment tu vas faire \emph{cela}~!
M. Goyle accepte~!

--- Allons-y~! fit Harry.

--- Potter, tu fais \emph{quoi~?}~» chuchota Drago, ce qu'il parvint mystérieusement à faire sans bouger les lèvres.

Harry ne savait pas comment répondre sans bouger les siennes.

Les gens rangeaient leur baguette, et M. Goyle descendit en piqué pour se poser gracieusement au sol, l'air assez confus.
Quelques Poufsouffle commencèrent à s'avancer vers M. Goyle mais Harry leur jeta un regard suppliant et ils reculèrent.

Harry marcha vers M. Goyle et s'arrêta lorsqu'ils furent à quelques pas l'un de l'autre, assez loin pour qu'ils ne puissent se toucher.

Doucement, délibérément, Harry rengaina sa baguette.

Tout le monde recula.

Harry déglutit.
Il savait en gros ce qu'il \emph{voulait} faire, mais il fallait que ce soit fait de telle manière que personne ne comprenne \emph{comment} il l'avait fait…

«~Très bien~», dit Harry avec force.
Et maintenant…~»
Il prit une profonde inspiration et leva une main, prêt à faire claquer ses doigts.
Il y eut des glapissements venant de tous ceux qui avaient entendu parler des tartes, c'est-à-dire presque tout le monde.
«~\emph{J'en appelle à la démence de Poudlard~!
Tapis tapis boum boum lampe lampe lampe~!}~»
Et Harry claqua des doigts.

Beaucoup eurent un petit recul de frayeur.

Et rien ne se passa.

Harry laissa le silence s'étirer un moment, se développer, jusqu'à ce que…

«~Euh, dit quelqu'un.
C'est tout~?~»

Harry regarda le garçon qui avait parlé.

«~Regarde devant toi.
Tu vois le petit bout de terre qui a l'air stérile, sans herbe dessus~?

--- Euh, ouais~», fit le garçon, un Gryffondor (Dean machin-chose~?).

«~Creuse là.~»

On commençait à regarder Harry bizarrement.

«~Euh, pourquoi~?

--- Fais-le, c'est tout, dit Terry Boot d'une voix lasse.
Inutile de demander pourquoi, fais moi confiance là dessus.~»

Dean machin-chose s'agenouilla et commença à enlever des poignées de terre.

Après une minute, Dean se releva.
«~Il n'y a rien ici.~»

Euh. Harry avait prévu de revenir en arrière dans le temps et enterrer une carte au trésor qui mènerait à une autre carte au trésor qui mènerait au Rapeltout de Neville qu'il aurait caché là après l'avoir repris à M. Goyle…

Puis Harry se rendit compte qu'il existait un moyen beaucoup plus simple, qui ne menaçait pas autant le secret des retourneurs de temps.

«~Merci, Dean~! dit Harry d'une voix forte.
Ernie, pourrais-tu regarder là où Neville est tombé et voir si tu peux trouver son Rapeltout~?~»

Les gens semblaient encore plus confus.

«~Fais-le, dit Terry Boot.
Il va continuer jusqu'à ce que ce que quelque chose marche, et ce qui fait peur c'est que…

--- \emph{Par Merlin~!}~» s'étrangla Ernie.
Il tenait le Rapeltout de Neville.
«~Il est \emph{là}~!
Exactement là où il est tombé~!

--- \emph{Quoi~?}~» s'écria M. Goyle.
Il baissa le regard et vit…

… qu'il tenait toujours le Rapeltout de Neville.

Il y eut une assez longue pause.

«~Euh, fit Dean machin-chose, ce n'est pas possible, n'est-ce pas~?

--- C'est une erreur de scénario, dit simplement Harry.
Je me suis rendu suffisamment bizarre pour distraire l'univers pendant un moment, et il a oublié que Goyle avait déjà ramassé le Rapeltout.

--- Non, attends, je veux dire, ce n'est \emph{vraiment} pas possible…

--- Excuse-moi, mais ne sommes-nous pas tous ici à attendre de voler sur des balais~?
Si, c'est le cas.
Alors tais-toi.
Bref, une fois que je mettrai ma main sur le Rapeltout de Neville, le défi sera terminé et Gregory Goyle devra abandonner toute prétention sur le Rapeltout qu'il tient dans sa main et me le donner.
C'étaient les termes du défi, vous vous souvenez~?~»
Harry tendit une main et fit signe à Ernie.
«~Fais-le juste rouler jusqu'ici, puisque personne n'est censé s'approcher de moi, OK~?

--- Attends~!~»
cria un Serpentard -- Blaise Zabini, Harry n'était pas près d'oublier ce nom.
«~Comment savoir que c'est le Rapeltout de Neville~?
Tu pourrais avoir laissé tomber un \emph{autre} Rapeltout ici…

--- Le Serpentard est fort chez celui-ci, dit Harry en souriant.
Mais tu as ma parole que celui que Ernie tient dans sa main est à Neville.
Aucun commentaire en ce qui concerne celui de Gregory Goyle.~»

Zabini se tourna vers Drago.
«~\emph{Malfoy~!} Tu ne vas pas le laisser s'en tirer avec ce…

--- Ferme-la, toi~», gronda M. Crabbe, qui se tenait derrière Drago.
«~M. Malfoy a pas besoin qu'\emph{tu} lui dises quoi faire~!~»

\emph{Bon} sbire.

«~Mon pari était avec Drago, de la Noble et Très Ancienne Maison Malfoy, dit Harry.
Pas avec toi, Zabini.
J'ai fait ce que M. Malfoy a dit qu'il aimerait me voir faire, et je le laisse juge de ce pari.~»
Harry inclina sa tête vers Drago et releva légèrement les sourcils.
Cela devrait permettre à Drago de sauver la face.

Nouvelle pause.

«~Tu promets que c'est \emph{vraiment} le Rapeltout de Neville~? demanda Drago.

--- Oui, fit Harry.
C'est celui qui reviendra à Neville et c'était le sien au départ.
Et celui que tient Gregory Goyle me revient.~»

Drago acquiesça de la tête, l'air décidé.
«~Dans ce cas, je ne remettrai pas en question la parole de la Noble Maison de Potter, peu importe que tout cela ait été très étrange.
Et la Noble et Très Ancienne Maison Malfoy tient aussi sa parole.
M. Goyle, donnez cela à M. Potter…

--- Hé~! dit Zabini.
Il n'a pas \emph{encore} gagné, il n'a pas mis la main sur…

--- Attrape, Harry~!~»
dit Ernie, et il jeta le Rapeltout.

Harry récupéra le Rapeltout au vol avec facilité, il avait toujours eu de bons réflexes pour ça.
«~Voilà, dit Harry, j'ai gagné…~»

Harry laissa sa phrase en suspens.
Toutes les conversations s'arrêtèrent.

Le Rapeltout rougeoyait avec force dans sa main, flamboyant comme un soleil miniature et projetant des ombres au sol en plein jour.

\later

Jeudi.

Si l'on voulait être précis, jeudi après-midi à 17h09, dans le bureau de la professeure McGonagall, après le cours de vol.
(Avec une heure supplémentaire que Harry avait glissée entre les deux.)

McGonagall était assise sur son tabouret.
Harry sur la sellette, face à son bureau.

«~Professeure...~»
 le ton de Harry était tendu, «~Serpentard pointait ses baguettes sur Poufsouffle, Gryffondor pointait ses baguettes sur Serpentard, un \emph{idiot} appela à sortir les baguettes chez Serdaigle, et j'avais peut-être cinq secondes pour empêcher la situation d'exploser façon nitroglycérine~!
C'est tout ce que j'ai réussi à trouver~!~»

McGonagall le regardait furieuse, le visage pincé.

«~\emph{Vous ne devez pas utiliser le retourneur de temps à de telles fins, M. Potter~!}
Qu'est-ce que vous n'avez pas compris dans le concept de secret~?

--- Ils ne \emph{savent} pas comment j'ai fait~!
Ils pensent juste que je peux faire des choses vraiment bizarres en claquant des doigts~!
J'ai fait d'autres choses bizarres qui ne peuvent pas être faites avec un retourneurs de temps, et j'en ferai \emph{encore d'autres}, et \emph{ce} cas précis ne se remarquera pas plus qu'un autre~!
Il \emph{fallait que je le fasse}, professeur~!

--- Non, il ne fallait \emph{pas} que vous le fassiez~! cingla McGonagall.
Tout ce que vous aviez à faire était de faire redescendre ce \emph{Serpentard anonyme} au sol et que les baguettes soient mises de côté~!
Vous auriez pu le défier à une partie de bataille explosive, mais non, il fallait que vous utilisiez le retourneur de temps de manière inutile et flagrante~!

--- C'est tout ce que j'ai réussi à trouver~!
Je ne sais même pas ce que c'\emph{est} la bataille explosive, ils n'auraient pas accepté une partie d'échecs, et si j'avais choisi le bras de fer, j'aurais perdu~!

--- \emph{Alors vous auriez dû choisir le bras de fer~!}~»

Harry cligna des yeux.
«~Mais alors j'aurais \emph{perdu}…~»

Harry s'interrompit.

McGonagall avait l'air \emph{très} en colère.

«~Je suis désolé, professeure, dit Harry d'une petite voix.
Je n'y ai honnêtement pas pensé, et vous avez raison, j'aurais dû, cela aurait été brillant si je l'avais fait, mais je n'y ai juste pas pensé du tout…~»

Harry laissa sa phrase en suspens.
Il lui apparaissait soudainement qu'il y avait \emph{beaucoup} d'autres options.
Il aurait pu demander à \emph{Drago} de suggérer quelque chose, il aurait pu demander à l'assemblée… son utilisation du retourneur de temps avait été inutile et flagrante.
Il existait un espace de possibilités gigantesque, pourquoi avait-il choisi \emph{celle}-là~?

Parce qu'il avait vu un moyen de gagner.
Gagner la possession d'une babiole sans importance que les professeurs auraient de toute façon reprise à M. Goyle.

L'intention de gagner.
C'est ça qui l'avait pris.

«~Je suis désolé, dit de nouveau Harry.
Pour mon orgueil et ma stupidité.~»

McGonagall se passa une main sur le front.
Une partie de sa colère sembla se dissiper.
Mais sa voix resta très dure.
«~Une autre démonstration de ce genre, M. Potter, et vous devrez rendre le retourneur de temps.
Ai-je bien été claire~?

--- Oui, répondit Harry.
Je comprends et je suis désolé.

--- Alors, M. Potter, vous êtes autorisé à conserver le retourneur de temps pour le moment.
Et étant donné l'ampleur de la débâcle que vous avez, en effet, évitée, je ne déduirai pas de point à Serdaigle.~»

\emph{Plus vous ne pourriez pas expliquer pourquoi vous avez déduit les points}.
Mais Harry n'était pas assez stupide pour dire ça à haute voix.

«~Plus important, pourquoi le Rapeltout s'est-il déclenché comme ça~? demanda Harry.
Est-ce que ça veut dire que j'ai été Oublietté~?

--- Je suis moi aussi perplexe, dit lentement McGonagall.
Si c'était aussi simple, je pense que les tribunaux utiliseraient les Rapeltouts, et ce n'est pas le cas.
J'étudierai la question, M. Potter.~»
Elle soupira.
«~Vous pouvez y aller, maintenant.~»

Harry commença à se lever de sa chaise, puis s'arrêta.
«~Euh, pardon, il y a tout de même quelque chose que je voulais vous dire…~»

On pouvait à peine remarquer le tressaillement.
«~De quoi s'agit-il, M. Potter~?

--- C'est à propos du professeur Quirrell…

--- Je suis certaine, M. Potter, que ce n'est rien de très important.~»
McGonagall avait prononcé ces mots avec précipitation.
«~Vous avez certainement entendu le directeur dire aux élèves de ne pas nous importuner de complaintes insignifiantes à propos du professeur de Défense~?~»

Harry était un peu déconcerté.
«~Mais cela pourrait \emph{être} important, hier j'ai eu soudainement une sensation funeste quand…

--- M. Potter~!
J'ai moi aussi une sensation funeste~!
Et ma sensation funeste suggère que \emph{vous ne devez pas finir cette phrase~!}~»

Harry se trouvait bouche bée.
La professeure avait réussi~; Harry était sans voix.

«~M. Potter, dit McGonagall, si vous avez découvert quoi que ce soit d'intéressant au sujet du professeur Quirrell, n'hésitez pas à ne pas le partager avec moi ni avec qui que ce soit.
Maintenant je pense que vous avez usé assez de mon précieux temps…

--- \emph{Cela ne vous ressemble pas~!} éclata Harry.
Je suis désolé, mais cela semble juste \emph{incroyablement} irresponsable~!
De ce que j'ai compris, il y a une sorte de malédiction sur le poste de Défense, et si vous \emph{savez} déjà que quelque chose va mal tourner, je pensais que seriez tous sur le qui-vive…

--- \emph{Mal tourner}, M. Potter~?
\emph{J'espère bien que non}.~»
Le visage de McGonagall était parfaitement neutre.
«~Après que l'on ait surpris le professeur Blake dans une armoire avec pas moins de trois Serpentard de cinquième année, en février dernier, et qu'un an auparavant, le professeur Summers avait tellement échoué dans sa mission éducative que ses élèves pensaient qu'un épouvantard était une sorte de meuble, il serait \emph{catastrophique} qu'un problème concernant le remarquablement compétent professeur Quirrell soit maintenant porté à ma connaissance, ce qui, je me permet de le rappeler, ferait échouer la plupart de nos élèves à leur BUSE ou à leur ASPIC.

--- Je vois~», dit lentement Harry, absorbant le tout.
«~Donc en d'autres mots, quoi qu'il se passe avec le professeur Quirrell, vous souhaitez désespérément ne pas l'apprendre avant la fin de l'année scolaire.
Et puisque nous sommes pour l'instant en septembre, il pourrait assassiner le premier ministre en direct à la télévision sans que vous ne lui en teniez rigueur.~»

McGonagall le regarda sans ciller.
«~Je suis certaine que jamais on ne pourra m'entendre cautionner une telle affirmation, M. Potter.
À Poudlard, nous aspirons à être proactifs en ce qui concerne \emph{tout} ce qui met en danger la réussite scolaire de nos élèves.~»

\emph{Comme des Serdaigle de première année qui n'arrêtent pas de l'ouvrir.}
«~Je crois vous avoir parfaitement comprise, professeure McGonagall.

--- Oh, j'en doute, M. Potter.
J'en doute grandement.~»
McGonagall se pencha en avant, le visage de nouveau tendu.
«~Puisque vous et moi avons discuté de sujets bien plus sensibles que celui-ci, je vous parlerai franchement.
Vous, et vous seul, avez fait état de cette mystérieuse sensation funeste.
Vous, et vous seul, attirez le chaos plus fortement que tout aimant que j'ai jamais vu.
Après notre petite excursion au Chemin de Traverse, \emph{puis} le Choixpeau, puis le petit épisode d'\emph{aujourd'hui}, je puis prédire dès maintenant que je suis condamnée à m'asseoir un jour dans le bureau du directeur pour entendre une histoire hilarante au sujet du professeur Quirrell dans laquelle vous et vous seul tenez le rôle principal, après quoi nous n'aurons pas d'autre choix que de le licencier.
Je m'y suis déjà résignée, M. Potter.
Et si ce triste événement a lieu n'importe quand avant les ides de mai, je vous ligoterai au portail de Poudlard avec vos propres intestins et verserai des scarabées de feu dans votre nez.
\emph{Maintenant}, m'avez-vous parfaitement comprise~?~»

Harry acquiesça de la tête, les yeux écarquillés.
Puis, après une seconde~:
«~Et qu'est-ce que j'aurais si j'arrive à faire que ce soit le dernier jour de l'année scolaire~?

% TODO: fix the orphan line
--- \emph{Sortez de mon bureau~!}~»

\later

Jeudi.

Il devait y avoir quelque chose de particulier avec les jeudis à Poudlard.

Il était 17h32, jeudi après-midi, et Harry se tenait à côté du professeur Flitwick, devant la grande gargouille de pierre qui gardait l'entrée du bureau du directeur.

À peine avait-il fait le trajet retour du bureau de McGonagall aux salles d'études de Serdaigle qu'un des élèves lui avait dit de se présenter au bureau de Flitwick, et là Harry avait appris que Dumbledore voulait lui parler.

Harry avait demandé avec appréhension à Flitwick si le directeur avait dit de quoi il s'agissait, mais ce dernier avait haussé les épaules en signe d'impuissance.
Apparemment, Dumbledore avait dit que Harry était bien trop jeune pour invoquer les mots du pouvoir et de la folie.

\emph{Tapis tapis boum boum lampe lampe lampe}~? avait pensé Harry, mais il ne l'avait pas dit à haute voix.

«~Ne vous en faites pas trop, M. Potter~», piailla Flitwick depuis à peu près la hauteur de l'épaule de Harry.
(Harry était reconnaissant que Flitwick porte une barbe immense et fournie, il avait du mal s'habituer à un professeur qui était non seulement plus petit que lui mais parlait aussi d'une voix plus aiguë).
«~Le directeur Dumbledore peut sembler un peu étrange, ou très étrange, ou même extrêmement étrange, mais il n'a jamais fait le moindre mal à un élève, et je ne pense pas qu'il le fera un jour.~»
Flitwick adressa un sourire encourageant à Harry.
«~Gardez cela à l'esprit tout le temps et vous serez certain de ne pas paniquer~!~»

Cela n'aidait pas.

«~Bonne chance~!~»
piailla Flitwick, qui se pencha vers la gargouille et prononça quelque chose dont Harry ne parvint bizarrement pas à entendre un seul son.
(Bien sûr, le mot de passe ne serait pas très utile si on pouvait entendre quelqu'un le prononcer.)
Et la gargouille de pierre fit un pas de côté d'un mouvement très naturel et ordinaire, ce que Harry trouva plutôt perturbant, puisque la gargouille continua à ressembler à de la pierre solide et inamovible tout ce temps.

Derrière la gargouille se trouvait un escalier en double spirale tournant lentement sur lui-même.
Il y avait là quelque chose de troublant et d'hypnotique, et encore plus troublant était le fait que faire \emph{tourner} une spirale n'était pas censé vous mener quelque part.

«~Allez, monte~!~»
piailla Flitwick.

Harry mit nerveusement un pied sur la spirale et se sentit commencer à monter pour une raison que son cerveau ne parvenait pas du tout à visualiser.

La gargouille se remit en place derrière lui dans un bruit sourd, les escaliers en spirale continuèrent de tourner et Harry continua de monter, puis se trouva bientôt, étourdi, en face d'une porte en chêne munie d'un heurtoir en laiton en forme de griffon.

Harry tendit la main et tourna la poignée.
La porte s'ouvrit.

Et Harry entra dans la pièce la plus intéressante qu'il ait jamais vue de sa vie.

Il y avait des petits mécanismes en métal qui vrombissaient ou faisaient tic-tac ou changeaient lentement de forme ou émettaient de petites bouffées de fumée.
Des dizaines de fluides mystérieux dans des dizaines de contenants aux formes étranges, tous bullants, bouillonnants, suintants, changeant de couleur, ou prenant des formes intéressantes qui disparaissaient une demi-seconde après que vous les aviez vues.
Des choses qui ressemblaient à des horloges dotées de nombreuses aiguilles, gravées de nombres ou de mots dans des langues inconnues.
Un bracelet portant un cristal lenticulaire qui étincelait de mille couleurs, un volatile perché sur une plateforme dorée, une coupe en bois remplie de ce qui semblait être du sang, la statue d'un faucon incrusté dans de l'émail noir.
Sur le mur des tableaux de gens endormis, et le Choixpeau était tout simplement accroché à un portemanteau sur lequel se trouvaient aussi deux parapluies et trois pantoufles rouges pour pied gauche.

Au milieu de tout ce chaos, un bureau en chêne noir impeccable.
Devant le bureau un tabouret en chêne.
Et derrière le bureau un trône bien rembourré contenant Albus Percival Wulfric Brian Dumbledore, orné d'une longue barbe argentée, d'un chapeau ressemblant à un immense champignon écrasé, et de ce qui, pour des yeux moldus, ressemblait à trois couches de pyjama rose vif.

Dumbledore souriait, et ses yeux radieux pétillaient d'une intensité folle.

Harry s'assit face au bureau avec un peu d'appréhension.
La porte se referma derrière lui dans un puissant \emph{blam}.

«~Bonjour, Harry, dit Dumbledore.

--- Bonjour, monsieur le directeur~», répondit Harry.
Alors on partait pour utiliser les prénoms~?
Dumbledore allait-il lui demander de l'appeler…

«~S'il te plaît, Harry, dit Dumbledore.
Monsieur le directeur fait tellement formel.
Appelle-moi juste Dir, pour faire court.

--- D'accord, Dir~», dit Harry.

Il y eut une courte pause.

«~Tu sais, continua Dumbledore, tu es la première personne à le faire réellement~?

--- Ah…~» fit Harry, l'estomac noué, essayant malgré tout de contrôler sa voix.
«~Je suis désolé, je, ah, monsieur le directeur, vous m'avez dit de le faire alors j'ai…

--- Dir, s'il te plaît~! dit gaiement Dumbledore.
Et ne sois pas si inquiet, je ne vais pas te jeter par la fenêtre juste parce que tu fais une erreur.
Je t'enverrai plein d'avertissements avant, si tu fais quelque chose de mal~!
De plus, ce n'est pas la façon dont les gens vous parlent qui compte, mais ce qu'ils pensent de vous.~»

\emph{Il n'a jamais fait le moindre mal à un élève, garde cela à l'esprit et tu seras certain de ne pas paniquer.}

Dumbledore sortit une petite boîte en métal et en souleva le couvercle, révélant de petits blocs jaunes.
«~Bonbon au citron~?~»
demanda le directeur.

«~Euh, non merci, Dir~», répondit Harry.
\emph{Est-ce que refiler du LSD à un élève, c'est lui faire du mal, ou cela rentre-t-il dans la catégorie jeu inoffensif~?}
«~Vous, euh, avez dit quelque chose comme quoi j'étais trop jeune pour invoquer les mots du pouvoir et de la folie~?

--- Tu l'es beaucoup trop~! dit Dumbledore.
Heureusement, les Mots du Pouvoir et de la Folie ont été perdus il y a sept siècles et plus personne n'a la moindre idée de ce qu'ils sont.
C'était juste une petite remarque.

--- Ah…~» fit Harry.
Il se rendait bien compte que sa bouche était restée ouverte.
«~Pourquoi m'avez-vous fait venir ici, alors~?

--- \emph{Pourquoi~?} répéta Dumbledore.
Ah, Harry, si je passais mes journées à me demander \emph{pourquoi} je fais les choses, je n'aurais jamais le temps de rien faire~!
Je suis quelqu'un de très occupé, tu sais.~»

Harry hocha la tête en souriant.
«~Oui, c'est une liste très impressionnante.
Directeur de Poudlard, président du Magenmagot, et manitou suprême de la Confédération internationale des mages et sorciers.
Navré de vous poser la question, mais je me demandais, est-il possible d'obtenir plus de six heures si on utilise plus d'un retourneur de temps~?
Parce que c'est plutôt impressionnant si vous faites tout ça en seulement trente heures par jour.~»

Il y eut une autre courte pause, pendant laquelle Harry continua de sourire.
Il était un peu inquiet, en fait très inquiet, mais une fois qu'il était devenu clair que Dumbledore s'amusait délibérément avec lui, quelque chose à l'intérieur de Harry avait \emph{absolument refusé} de rester là à tout se prendre comme un bloc de gelée sans défense.

«~J'ai bien peur que le temps n'aime pas être trop étiré, dit Dumbledore après la courte pause, et pourtant nous semblons être nous-mêmes un peu trop grands pour lui, c'est donc une lutte permanente pour faire tenir nos vies dans le temps.

--- En effet, acquiesça Harry solennellement.
C'est pourquoi il est préférable d'en venir rapidement à ce qui nous préoccupe.~»

Pendant un moment, Harry se demanda s'il n'avait pas poussé le bouchon un peu trop loin.

Puis Dumbledore gloussa.
«~Droit au fait, allons-y.~»
Le directeur s'inclina en avant, faisant pencher son chapeau-champignon-écrasé et balayant le bureau de sa barbe.
«~Harry, lundi dernier tu as fait quelque chose qui aurait dû être impossible, même avec un retourneur de temps.
Ou plutôt, impossible avec \emph{seulement} un retourneur de temps.
D'où ces deux tartes venaient-elles, je me demande~?~»

Un courant d'adrénaline traversa Harry.
Il avait fait cela en utilisant la Cape d'Invisibilité, celle qui lui avait été donnée emballée comme un cadeau de Noël, avec une note qui disait~: \emph{si Dumbledore voyait une occasion de posséder l'une des Reliques de la Mort, jamais il ne la laisserait s'échapper de son emprise…}

«~Une pensée naturelle, continua Dumbledore, serait que puisqu'aucun des élèves de première année présents n'était capable de jeter un sort pareil, quelqu'un d'autre était présent, mais qu'on ne pouvait voir.
Et si l'on ne pouvait voir cette personne, eh bien, il aurait été facile pour elle de jeter les tartes.
On pourrait de plus soupçonner que, puisque tu as le retourneur de temps, tu étais la personne invisible~; et que puisque le sort de Désillusion est bien au-delà de tes capacités actuelles, tu avais une cape d'invisibilité.~»
Dumbledore sourit avec un air conspirateur.
«~Suis-je sur la bonne voie jusqu'à présent, Harry~?~»

Harry était immobile.
Il lui semblait qu'un mensonge pur et simple ne serait pas très malin, voire que ça ne l'aiderait pas le moins du monde, et il ne trouvait rien d'autre à dire.

Dumbledore agita une main amicale.
«~Ne t'en fais pas Harry, tu n'as rien fait de mal.
Les capes d'invisibilité ne sont pas interdites -- je suppose qu'elles sont suffisamment rares pour que personne n'ait pris le temps de les ajouter à la liste.
Mais à vrai dire, je songeais à tout autre chose.

--- Oh~?~»
fit Harry de la voix la plus normale qu'il put.

Les yeux de Dumbledore brillaient d'enthousiasme.
«~Tu vois, Harry, quand on a vécu quelques aventures, on commence à attraper le coup pour ce genre de choses.
On voit les motifs, on entend le rythme du monde.
On commence à développer des soupçons \emph{avant} le moment de la révélation.
Tu es le Survivant, et il se trouve qu'une cape d'invisibilité est parvenue jusqu'à tes mains seulement quatre jours après ta découverte de l'Angleterre magique.
On ne trouve pas à acheter de telles capes au Chemin de Traverse, mais il y en a \emph{une} qui pourrait trouver elle-même son chemin jusqu'à celui destiné à la porter.
Et je ne peux donc m'empêcher de me demander si par un étrange hasard tu n'aurais non pas juste trouvé \emph{une} cape d'invisibilité, mais \emph{la} Cape d'Invisibilité, l'une des trois Reliques de la Mort, réputée cacher son porteur du regard de la Mort elle-même.~»
Dumbledore avait le regard brillant et avide.
«~Pourrais-je la voir, Harry~?~»

Harry déglutit.
Il avait à présent un raz-de-marée d'adrénaline dans son corps et c'était totalement inutile.
C'était le sorcier le plus puissant du monde, il n'atteindrait même pas la porte, il n'y avait nulle part à Poudlard où il pourrait se cacher, et il était sur le point de perdre la Cape transmise de Potter en Potter pendant on ne savait combien de générations…

Lentement, Dumbledore se radossa à son fauteuil.
Ses yeux avaient perdu leur lueur vive, il avait l'air perplexe et un peu triste.
«~Harry, dit Dumbledore, si tu ne veux pas, tu peux juste dire non.

--- Je peux~? dit Harry d'une voix rauque.

--- Oui, Harry~», dit Dumbledore.
Sa voix était peinée à présent, et inquiète.
«~Il semblerait que tu aies peur de moi, Harry.
Puis-je te demander ce que j'ai fait pour mériter ta méfiance~?~»

Harry déglutit.
«~Y a-t-il un moyen par lequel vous pouvez jurer, par un serment magique, que vous vous engagez à ne pas prendre ma cape~?~»

Dumbledore secoua lentement la tête.
«~Les Serments Inviolables ne doivent pas être utilisés à la légère.
Et puis, Harry, si tu ne connais pas encore le sort, tu n'auras que ma parole qu'il est réellement engageant.
Pourtant tu te rends sûrement compte que je n'ai pas \emph{besoin} de ta permission pour voir la Cape.
Je suis assez puissant pour la récupérer moi-même, bourse en peau de Moke ou non.~»
Le visage de Dumbledore était sérieux.
«~Mais je ne le ferai pas.
La Cape est tienne, Harry.
Je ne te la prendrai pas.
Pas même juste pour y jeter un coup d'œil, à moins que tu ne décides de me la montrer.
C'est une promesse et un serment.
Si je devais t'interdire de l'utiliser dans l'enceinte de l'école, je te demanderais d'aller dans ta chambre forte à Gringotts et de l'y déposer.

--- Ah…~» fit Harry.
Il déglutit fortement, essayant de calmer le flot d'adrénaline et de penser de façon raisonnable.
Il décrocha la bourse en peau de Moke de sa ceinture.
«~Si vous n'avez vraiment \emph{pas} besoin de ma permission… alors vous l'avez.~»
Harry tendit la bourse à Dumbledore et mordit sa lèvre avec force, s'envoyant ce signal à lui-même au cas où il serait plus tard Oublietté.

Le vieux sorcier mit sa main dans la bourse, et sans prononcer un seul mot de récupération, fit apparaître la Cape d'Invisibilité.

«~Ah, souffla Dumbledore.
J'avais raison…~»
Il fit glisser le tissu de velours noir entre ses mains.
«~Vieille de plusieurs siècles, et toujours aussi parfaite qu'au jour de sa création.
Au fil du temps, nous avons tant perdu de notre art, et à présent je ne pourrais créer un tel objet~; personne ne le peut.
Je peux ressentir son pouvoir, comme un écho dans mon esprit, comme une chanson chantée depuis toujours sans personne pour l'entendre…~»
Le sorcier releva les yeux.
«~Ne la vends pas, dit-il, ne l'offre à personne.
Penses-y à deux fois avant de la montrer à quelqu'un, et pèse la question trois fois avant de révéler qu'il s'agit d'une Relique de la Mort.
Traite-la avec respect, car il s'agit bel et bien d'un Objet de Pouvoir.~»

Le visage de Dumbledore devint mélancolique quelques temps…

… puis il rendit la Cape à Harry, qui la rangea dans sa bourse.

Dumbledore arbora de nouveau un visage sérieux.
«~Harry, puis-je te redemander pourquoi tu te méfies tant de moi~?~»

Harry se sentit soudainement honteux.

«~Il y avait une note avec la Cape, fit Harry d'une petite voix.
Elle disait que vous essaieriez de vous emparer de la Cape si vous étiez au courant.
Mais en fait je ne sais pas qui a laissé cette note, vraiment pas.

--- Je… vois, dit lentement Dumbledore.
Eh bien Harry, je ne présumerai pas des motifs de celui ou de celle qui t'a laissé cette note.
Qui sait, peut-être cette personne avait-elle les meilleures intentions~?
Elle t'a donné la Cape après tout.~»

Harry acquiesça, impressionné par la bienveillance de Dumbledore, et embarrassé par le net contraste avec sa propre attitude.

Le vieux sorcier continua~:
«~Mais je crois que toi et moi sommes des pièces de la même couleur sur le plateau de jeu.
Le garçon qui finit par anéantir Voldemort, et le vieil homme qui le tint à distance assez longtemps pour que tu nous sauves.
Je ne te tiendrai pas rigueur de ta prudence, Harry, nous devons tous agir avec autant de sagesse que possible.
Je te demanderai seulement d'y penser à deux fois et de peser la question trois fois la prochaine fois que quelqu'un te dit de ne pas me faire confiance.

--- Je suis désolé,~» dit Harry.
Il se sentait misérable à présent, il venait plus ou moins de réprimander Gandalf, et la gentillesse de Dumbledore renforçait encore plus ce sentiment.
«~Je n'aurais pas dû me méfier de vous.

--- Hélas, Harry, dans ce monde…
Le vieux sorcier secoua la tête.
Je ne peux même pas dire que tu as manqué de sagesse.
Tu ne me connaissais pas.
Et en vérité, il y a certaines personnes à Poudlard en qui tu ferais mieux de ne pas avoir confiance.
Peut-être même certains que tu appelles des amis.~»

Harry déglutit.
Cette phrase était de mauvais augure.
«~Comme qui~?~»

Dumbledore se leva de sa chaise et commença à examiner l'un de ses instruments, un cadran comportant huit aiguilles de longueurs différentes.

Après quelques instants, le vieux sorcier parla de nouveau.
«~Il te semble probablement très charmant, dit Dumbledore.
Poli… envers toi du moins.
S'exprimant bien, peut-être même admiratif.
Toujours prêt à tendre la main, offrir une faveur ou un conseil…

--- Oh, \emph{Drago Malfoy}~!~»
dit Harry, soulagé qu'il ne s'agisse pas de quelqu'un comme Hermione ou autre.
«~Oh non, non non non, vous avez compris de travers, ce n'est pas lui qui m'endoctrine, c'est moi qui le convertis.~»

Dumbledore se figea sur place, face au cadran.
«~Tu \emph{quoi}~?

--- Je vais éloigner Drago du côté obscur, dit Harry.
Vous savez, faire de lui un gentil.~»

Dumbledore se redressa et se tourna vers Harry.
Harry n'avait jamais vu personne avec un air si stupéfait, encore moins chez quelqu'un portant une longue barbe d'argent.
«~Es-tu certain, dit le vieux sorcier après un moment, qu'il est prêt pour cette rédemption ?
Je crains que toute bonté que tu vois en lui ne soit que vœu pieux de sa part -- voire pire, un appât ou un leurre…

--- Euh, peu probable, dit Harry.
Je veux dire, s'il essaie de se faire passer pour un gentil alors il est extrêmement mauvais.
Ce n'est pas que Drago est venu me voir, tout charmant, et que j'ai décidé qu'il devait avoir un noyau de bonté caché au plus profond de lui.
J'ai sélectionné sa rédemption à lui spécifiquement parce qu'il est l'héritier de la Maison Malfoy, et que s'il faut choisir une personne à sauver, c'est forcément lui.~»

Dumbledore tressaillit de l'œil gauche.
«~Tu as l'intention de planter les graines de l'amour et de la gentillesse dans le cœur de Drago parce que tu penses que l'héritier des Malfoy te sera utile~?

--- Pas seulement à \emph{moi}~! s'indigna Harry.
À toute l'Angleterre magique, si cela fonctionne~!
\emph{Et} il aura lui-même une vie plus heureuse et mentalement plus saine~!
Écoutez, je n'ai pas le temps d'éloigner \emph{tout le monde} du côté obscur, alors je dois trouver où la lumière apportera au plus vite le meilleur avantage…~»

Dumbledore commença à rire.
À rire beaucoup plus fort que ce à quoi Harry se serait attendu, il hurlait presque.
C'en était même véritablement \emph{indigne}.
Un ancien, puissant sorcier se devait de glousser d'un ton caverneux, pas de rire à perdre haleine.
Harry était un jour littéralement tombé de sa chaise en regardant le film \emph{La Soupe au canard} des Marx Brothers, et Dumbledore riait maintenant avec autant d'intensité.

«~Ce n'est pas \emph{si} drôle que ça~», dit Harry après un moment.
Il commençait de nouveau à s'inquiéter pour la santé mentale de Dumbledore.

Dumbledore parvint à se ressaisir au prix d'un effort visible.
«~Ah, Harry, un des symptômes de la maladie appelée sagesse est que l'on commence à rire de choses que personne d'autre ne trouve amusantes, parce que quand on devient sage, Harry, on commence à comprendre les blagues~!~»
Le vieux sorcier essuya des larmes aux coins de ses yeux.
«~Ah là là.
La volonté du mal ruine souvent le mal, en effet, tout à fait.~»

Le cerveau de Harry mit un moment pour reconnaître les mots familiers…
«~Hé, c'est une citation de \emph{Tolkien}~!
\emph{Gandalf} dit ça~!

--- Théoden, à vrai dire, dit Dumbledore.

--- Vous êtes \emph{né-Moldu}~? dit Harry, choqué.

--- J'ai bien peur que non, dit Dumbledore souriant de nouveau.
Je suis né soixante-dix ans avant que ce livre ne soit publié, cher enfant.
Mais il semblerait que mes élèves nés-Moldus pensent souvent de façon similaire pour certaines choses.
J'ai accumulé pas moins de vingt copies du \emph{Seigneur des Anneaux} et trois ensembles des œuvres complètes de Tolkien, et je chéris chacun d'entre eux.~»
Dumbledore dégaina sa baguette, la leva devant lui et prit la pose.
«~\emph{Vous ne passerez pas~!}
De quoi j'ai l'air~?

--- Ah~», fit Harry, proche de l'arrêt cérébral complet, «~je crois qu'il vous manque un Balrog.~»
Et le pyjama rose et le chapeau champignon écrasé n'aidaient pas vraiment non plus.

--- Je vois.~»
Morose, Dumbledore soupira et rengaina sa baguette dans sa ceinture.
«~J'ai peur qu'il y ait eu bien peu de Balrogs dans ma vie dernièrement.
Ces temps-ci c'est surtout réunions au Magenmagot où je dois désespérément empêcher tout travail d'avancer, dîners officiels où des politiciens étrangers se battent pour savoir qui sera l'idiot le plus borné.
Et avoir l'air mystérieux, sachant des choses que je ne devrais pas savoir, prononçant des phrases cryptiques qui ne peuvent être comprises que rétrospectivement, et toutes les autres petites techniques que les sorciers puissants ont pour s'amuser une fois qu'ils ont quitté la partie de l'histoire qui leur permettait d'être des héros.
En parlant de cela, Harry, j'ai quelque chose à te donner, quelque chose qui appartenait à ton père.

--- Vraiment~? dit Harry.
Ça alors, qui l'aurait cru.

--- Oui, tout fait, dit Dumbledore. J'imagine que c'était un peu prévisible.~»
Son visage devint solennel. «~Néanmoins…~»

Dumbledore revint à son bureau et s'assit tout en ouvrant l'un des tiroirs.
Il y farfouilla des deux mains, et, avec effort, tira de celui-ci un objet plutôt grand et à l'air assez lourd qu'il déposa ensuite sur son bureau en chêne avec un bruit sourd.

«~Ceci, dit Dumbledore, était le rocher de ton père.~»

Harry fixa le rocher du regard.
Il était gris clair, décoloré, d'une forme irrégulière, aux arêtes tranchantes, et ressemblait tout à fait à un grand rocher ordinaire.
Dumbledore l'avait posé de façon à ce qu'il tienne sur sa face la plus grande, mais il oscillait encore, instable sur le bureau.

Harry releva les yeux.
«~C'est une blague, c'est ça~?

--- Pas du tout~», dit Dumbledore, secouant la tête et prenant un air très sérieux.
«~Je l'ai pris dans les ruines de la maison de James et Lily, à Godric's Hollow, là où je t'ai également trouvé~; et je l'ai gardé jusqu'à maintenant, jusqu'au jour où je pourrais te le donner.~»

Dans le mélange d'hypothèses qui servait à Harry de modèle du monde, la démence de Dumbledore grimpait rapidement à l'échelle des probabilités.
Mais il y \emph{avait} toujours une quantité substantielle de probabilité allouée aux autres alternatives…
«~Euh, est-ce un rocher \emph{magique}~?

--- Pas que je sache, dit Dumbledore.
Mais je te recommande avec la plus grande insistance que tu le gardes en permanence près de toi.~»

Très bien.
Dumbledore était \emph{probablement} dément, mais s'il ne l'était \emph{pas}… eh bien, ce serait juste trop \emph{embarrassant} que de se retrouver dans le pétrin parce qu'on avait ignoré le conseil d'un vieux sorcier impénétrable.
Cela devait être genre en 4\textsuperscript{ème} position sur la liste des \emph{100 façons évidentes d'échouer}.

Harry s'avança et mit ses mains sur le rocher, essayant de dénicher une prise permettant de le soulever sans se couper.
«~Je le mettrai dans ma bourse, alors.~»

Dumbledore fronça les sourcils.
«~Il se pourrait que ce ne soit pas assez proche de ta personne.
Et si tu perds ta bourse en peau de Moke, ou qu'on te la vole~?

--- Vous pensez que je devrais juste transporter un gros rocher partout où je vais~?~»

Dumbledore regarda Harry avec sérieux.
«~Cela pourrait s'avérer fort sage.

--- Ah…~» fit Harry.
Le rocher semblait assez lourd.
«~Il me semble que les autres élèves risquent de me poser des questions à ce sujet.

--- Dis-leur que je t'ai ordonné de le faire, dit Dumbledore.
Personne n'en doutera puisqu'ils pensent tous que je suis fou.~»
Son visage était toujours parfaitement sérieux.

--- Euh, pour être honnête, si vous vous mettez à ordonner à vos élèves de transporter de gros rochers, je peux plus ou moins comprendre pourquoi les gens pensent une chose pareille.

--- Ah, Harry,~» fit Dumbledore.
Le vieux sorcier fit un ample geste de la main, semblant inclure tous les mystérieux instruments de la pièce.
«~Quand on est jeune, on croit tout savoir, et donc on croit que, si on ne trouve aucune explication à quelque chose, alors aucune explication n'existe.
Quand on est plus vieux, on se rend compte que l'univers entier fonctionne selon un rythme et a ses raisons, même si nous ne les connaissons pas.
C'est seulement notre ignorance que nous prenons pour de la folie.

--- La réalité suit toujours des lois, dit Harry, même si nous ne connaissons pas ces lois.

--- Précisément, Harry, dit Dumbledore.
Comprendre cela -- et je vois que tu le \emph{comprends} -- est l'essence de la sagesse.

--- Et donc… \emph{pourquoi} exactement dois-je transporter ce rocher~?

--- À vrai dire, je n'arrive pas à trouver une raison, dit Dumbledore.

--- … vous n'y arrivez pas.~»

Dumbledore acquiesça.
«~Mais ce n'est pas parce que je n'arrive pas à en trouver une qu'il n'y en \emph{a} aucune.~»

Les instruments continuèrent de cliqueter.

«~D'accord, fit Harry, je ne sais même pas si je devrais le dire, mais ce n'est tout simplement pas comme cela que l'on gère notre ignorance admise sur la façon dont l'univers fonctionne.

--- Ah non~?~» fit le vieux sorcier, l'air surpris et déçu.

Harry avait le sentiment que la conversation n'allait pas tourner à son avantage, mais il continua tout de même.
«~Non. Je ne sais même pas si cette erreur a un nom officiel, mais si je devais en créer un moi-même, ce serait “privilégier l'hypothèse” ou quelque chose comme ça.
Comment décrire cela de façon formelle… hum… imaginez que vous avez un million de boîtes, et une seule de ces boîtes contient un diamant.
Et vous avez un carton plein de détecteurs de diamant, et chaque détecteur de diamant se déclenche toujours en présence d'un diamant, mais se déclenche une fois sur deux lorsqu'il est mis à côté d'une boîte vide.
Si vous passiez vingt détecteurs sur chacune des boîtes, vous vous retrouveriez, en moyenne, avec une fausse boîte candidate et une vraie boîte candidate.
Ensuite, il vous suffirait d'un ou deux détecteurs de plus avant de vous retrouver uniquement avec le vrai candidat.
L'idée étant que lorsqu'il y a beaucoup de réponses possibles, la \emph{plupart} des indices dont on a besoin servent à \emph{localiser} la bonne hypothèse parmi des millions de possibilités -- à attirer votre attention vers elle en premier lieu.
En comparaison, la quantité d'information nécessaire pour choisir entre deux ou trois candidats plausibles est bien plus petite.
Donc si vous foncez sans aucune preuve et placez une possibilité en particulier au centre de votre attention, vous sautez le plus gros du travail.
Par exemple, vous vivez dans une ville avec un million d'habitants, et il y a un meurtre, et le détective dit, bon, on n'a absolument aucune preuve, mais a-t-on envisagé la possibilité que ce soit Maurice Malherbe qui ait fait le coup~?

--- C'est lui~? dit Dumbledore.

--- Non, dit Harry.
Mais plus tard, on apprend que le meurtrier a les cheveux noirs, et Maurice a les cheveux noirs, donc tout le monde est là, ah, on dirait qu'après tout c'est bien Maurice le coupable.
Donc c'est injuste pour Maurice que la police le \emph{place au centre de leur attention} sans déjà avoir de bonnes raisons de le soupçonner.
Lorsqu'il y a beaucoup de possibilités, le plus gros du travail consiste à \emph{localiser} la bonne réponse -- à lui donner notre attention.
Vous n'avez pas besoin de \emph{preuves} comme celles requises par la science ou par les tribunaux, mais il vous faut une sorte d'\emph{indice}, et cet indice doit discriminer cette possibilité en particulier parmi les millions d'autres.
Sans cela, vous ne pouvez pas juste matérialiser la bonne réponse à partir de rien.
Vous ne pouvez même pas matérialiser une possibilité méritant qu'on s'y attarde à partir de rien.
Et il y a certainement un million d'autres choses que je pourrais faire plutôt que de transporter le rocher de mon père partout où je vais.
Ce n'est pas parce que j'ignore des choses sur l'univers que je suis incertain de la façon dont je devrais raisonner en présence de mon ignorance.
Les lois du raisonnement probabiliste ne sont pas moins solides que les lois gouvernant la bonne vieille logique, et ce que vous venez de faire n'est tout simplement \emph{pas permis}.~»
Harry marqua une pause.
«~\emph{À moins}, bien sûr, que vous ayez un \emph{indice} dont vous ne m'avez pas parlé.

--- Ah~», fit Dumbledore.
Il se tapota la joue, l'air pensif.
«~Un argument intéressant, c'est certain, mais ne s'effondre-t-il pas lorsque tu fais une analogie entre d'un côté un million de meurtriers potentiels, dont un seul a vraiment commis le meurtre, et de l'autre choisir une façon d'agir quand de nombreuses façons d'agir sont peut-être toutes sages~?
Je ne dis pas que transporter le rocher de ton père est l'une des meilleures façons d'agir, seulement qu'il est plus sage de le faire que de ne pas le faire.~»

Dumbledore plongea à nouveau le bras dans le même tiroir que précédemment, semblant cette fois fouiller l'intérieur -- du moins son bras semblait bouger.
«~Une remarque en passant~», dit Dumbledore pendant que Harry essayait encore de trouver quoi répondre à sa réplique totalement inattendue, «~que c'est une idée reçue courante chez les Serdaigle que tous les enfants intelligents y sont envoyés, n'en laissant aucun pour les autres maisons.
Ce n'est pas le cas~; être Réparti à Serdaigle indique que l'on est poussé par son désir de savoir des choses, ce qui est une qualité toute autre que celle d'être intelligent.~»
Le sorcier souriait tandis qu'il se pencha au-dessus du tiroir.
«~Néanmoins, tu \emph{as} l'air intelligent.
Pas tant comme un jeune héros ordinaire, mais plutôt comme un jeune ancien sorcier mystérieux.
Je pense que j'ai peut-être choisi la mauvaise approche avec toi, Harry, et que tu pourrais comprendre des choses qui échappent à la plupart.
Je vais donc être audacieux, et t'offrir un \emph{autre} héritage bien particulier.

--- Vous ne voulez pas dire… Harry reprit sa respiration.
Mon père… \emph{possédait un autre rocher~?}

--- Excuse-moi, dit Dumbledore.
C'est encore \emph{moi} le plus vieux et plus mystérieux de nous deux, et s'il y a des révélations à faire, alors \emph{je} les révélerai, merci bien… oh, mais où \emph{est} donc ce truc~!~»
Dumbledore glissa son autre bras dans le tiroir du bureau, puis fouilla plus profondément encore.
Sa tête et ses épaules et tout son torse disparurent à l'intérieur si bien que seules ses hanches et ses jambes dépassaient, comme si le tiroir était en train de l'avaler.

Harry ne pouvait s'empêcher de se demander combien de choses se trouvaient là-dedans et à quoi ressemblerait un inventaire complet.

Dumbledore se redressa enfin, hors du tiroir, tenant l'objet de sa recherche, qu'il déposa sur le bureau à côté du rocher.

C'était un livre scolaire usé, aux bords irréguliers et à la reliure fatiguée~: \emph{Manuel intermédiaire de préparation des potions}, par Libatius Borage.
La couverture montrait l'image d'une fiole fumante.

«~Ceci, dit Dumbledore, était le manuel de potions de cinquième année de ta mère.

--- Que je dois transporter partout où je vais, dit Harry.

--- \emph{Qui contient un terrible secret.}
Un secret dont la révélation pourrait être si désastreuse que je dois te demander de prêter serment -- et je te demande de le faire sérieusement, Harry, quoi que tu penses de tout cela -- de ne jamais le répéter à qui ou quoi que ce soit.~»

Harry contempla le manuel de potions de cinquième année de sa mère qui, apparemment, contenait un terrible secret.

Le problème, c'était que Harry \emph{prenait} les serments très au sérieux.
N'importe quel serment était un Serment Inviolable s'il était prononcé par une certaine catégorie de personnes.

Et…

«~J'ai soif, dit Harry, et ce n'est pas du tout bon signe.~»

Dumbledore lui posa exactement zéro question sur sa remarque cryptique.
«~\emph{Promets-tu}, Harry~?~» poursuivit Dumbledore.
Ses yeux se plongèrent dans ceux de Harry.
«~Sinon, je ne peux pas te le dire.

--- Oui, dit Harry. Je promets.~»
C'était le problème quand vous étiez Serdaigle.
Vous ne pouviez pas refuser une offre pareille ou votre curiosité vous dévorerait vivant, et tout le monde était au courant.

«~Et je promets à mon tour, dit Dumbledore, que ce que je vais dire est la vérité.~»

Dumbledore ouvrit le livre, apparemment au hasard, et Harry se pencha pour mieux voir.

«~Vois-tu ces notes~», dit Dumbledore d'une voix si basse que c'était presque un murmure, «~écrites dans les marges du livre~?~»

Harry plissa légèrement les yeux.
Les pages jaunissantes semblaient décrire quelque chose nommé \emph{potion de la splendeur de l'aigle}, Harry ne connaissait même pas la plupart des ingrédients, dont les noms ne semblaient pas d'origine anglaise.
Une annotation était griffonnée dans la marge.
On lisait~: \emph{Je me demande ce qui se passerait si tu utilisais du sang de Sombral à la place des myrtilles~?} et immédiatement en dessous se trouvait une réponse d'une écriture différente~: \emph{Tu serais malade pendant des semaines et pourrais en mourir}.

«~Je les vois, dit Harry.
Et alors~?~»

Dumbledore indiqua le second griffonnage.
«~Cette écriture, dit-il toujours à voix basse, est celle de ta mère.
Et \emph{cette} écriture~», déplaçant son doigt pour indiquer le premier griffonnage, «~est la mienne.
Je me rendais invisible et me glissais dans sa chambre pendant qu'elle dormait.
Lily pensait que c'est un de ses amis qui les écrivait et ils avaient des disputes des plus phénoménales.~»

C'est à ce moment exact que Harry réalisa que le directeur de Poudlard \emph{était} bel et bien cinglé.

Dumbledore le regardait avec sérieux.
«~Comprends-tu les implications de ce que je viens de te dire, Harry~?

--- Ehhh…~» fit Harry.
Sa voix semblait bloquée.
«~Désolé… je… pas vraiment…

--- Ah, bon~», fit Dumbledore, et il soupira.
«~J'imagine que ton intelligence a des limites après tout.
Pourrions-nous tous deux prétendre que je ne t'ai rien dit~?~»

Harry se leva de sa chaise, un sourire figé sur le visage.
«~Bien sûr, dit-il.
Vous savez, il se fait assez tard et j'ai un peu faim, donc il faudrait que j'aille dîner, vraiment~», et il se dirigea droit vers la porte.

La poignée ne tourna absolument pas.

«~Tu me blesses, Harry~», c'était la voix de Dumbledore, un ton calme venant juste de derrière Harry.
«~Ne te rends-tu pas au moins compte que ce que je t'ai dit est un signe de confiance~?~»

Harry se retourna lentement.

Devant lui se trouvait un sorcier très puissant et très dément, à la longue barbe d'argent, au chapeau tel un champignon géant écrasé, et portant ce qui pour un Moldu ressemblait à trois couches de pyjama rose vif.

Derrière lui se trouvait une porte qui ne semblait pas fonctionner pour l'instant.

Dumbledore semblait attristé et fatigué, comme s'il avait envie de s'appuyer sur le bâton de sorcier qu'il ne possédait pas.
«~Franchement, dit Dumbledore, vous essayez une fois quelque chose de nouveau lieu de suivre le même schéma depuis cent-dix ans, et les gens partent tous en courant.~»
Le vieux sorcier secoua la tête avec tristesse.
«~J'espérais mieux de toi, Harry Potter.
J'ai entendu dire que tes propres amis te croient fou toi aussi.
Je sais qu'ils ont tort.
Ne croiras-tu pas la même chose à mon sujet~?

--- S'il vous plaît, ouvrez la porte~», dit Harry la voix tremblante.
«~Si vous voulez que je vous fasse à nouveau confiance un jour, ouvrez la porte.~»

On entendit une porte s'ouvrir derrière lui.

«~Il y avait d'autres choses que je comptais te dire, dit Dumbledore, et si tu pars maintenant, tu ne sauras pas de quoi il s'agissait.~»

Parfois, Harry \emph{détestait} être un Serdaigle.

\emph{Il n'a jamais fait de mal à un élève}, dit le côté Gryffondor de Harry.
\emph{Garde juste cela à l'esprit et tu seras certain de ne pas paniquer.
Tu ne vas pas prendre la fuite juste parce que les choses commencent à être intéressantes, non~?}

\emph{Tu ne peux pas laisser le directeur en plan comme ça~!} dit la partie Poufsouffle.
\emph{Et s'il commence à enlever des points de maison~?
Il pourrait rendre ta vie à l'école très difficile s'il décide qu'il ne t'aime pas~!}

% STOP HERE
Et une partie de lui-même, que Harry n'aimait pas beaucoup mais qu'il n'arrivait pas tout à fait à faire taire, pesait les avantages potentiels qu'il y avait à être l'un des rares amis de ce vieux sorcier fou qui se trouvait aussi être directeur, président-sorcier et manitou suprême.
Et malheureusement son Serpentard intérieur semblait être bien meilleur que Drago pour attirer les gens vers le côté obscur, parce qu'il disait des choses telles que \emph{le pauvre gars, il a l'air d'avoir besoin de quelqu'un à qui parler, tu ne trouves pas~?} et \emph{tu ne voudrais pas qu'un homme si puissant finisse par accorder sa confiance à quelqu'un de moins vertueux que toi, non~?} et \emph{je me demande quelles sortes de secrets incroyables Dumbledore pourrait te révéler si vous… tu sais, si vous deveniez amis} et \emph{je parie qu'il a une collection de livres vraaaiment intéressante.}

% STOP HERE
\emph{Vous êtes tous dingues}, dit Harry à l'attention de cet assemblée, mais il avait été mis en minorité par chaque partie de lui-même.

Harry se retourna, fit un pas vers la porte, tendit la main, et la referma délibérément.
C'était un sacrifice gratuit vu qu'il comptait de toute façon rester, et Dumbledore pouvait de toute façon contrôler ses mouvements, mais peut-être que ce geste l'impressionnerait.

Lorsque Harry se retourna encore, il vit que le puissant sorcier fou souriait de nouveau et avait l'air amical.
C'était positif, peut-être.

«~Ne le refaites pas s'il vous plaît, dit Harry.
Je n'aime pas être piégé.

--- Je \emph{suis} navré d'avoir fait ça, Harry~», dit Dumbledore d'un ton qui ressemblait à des excuses sincères.
«~Mais il aurait été épouvantablement mal avisé de te laisser partir sans le rocher de ton père.

--- Bien sûr, dit Harry.
Ce n'était pas raisonnable de ma part de m'attendre à ce que la porte s'ouvre avant que j'aie mis les objets de quête dans mon inventaire.~»

Dumbledore sourit et hocha la tête.

Harry marcha jusqu'au bureau, fit coulisser la bourse en peau de Moke jusqu'à l'avant de sa ceinture, et au prix de quelques efforts parvint à lever le rocher avec ses bras d'enfant de onze ans et à le donner à manger à la bourse.

Il pouvait vraiment sentir le poids diminuer lentement alors que l'Ouverture Élargissante mangeait le rocher, et le rot qui s'ensuivit était plutôt bruyant et distinctement rouspéteur.

Le Manuel de Potions de cinquième année de sa mère (qui renfermait un secret à vrai dire assez terrible) suivit peu après.

Puis le Serpentard intérieur de Harry fit une suggestion sournoise visant à s'attirer les bonnes grâces du directeur, ornée d'un baratin parfait destiné à obtenir le soutien du côté Serdaigle.

«~Alors, dit Harry.
Euh. Puisque je suis ici, j'imagine que vous ne voudriez pas me faire faire une petite visite de votre bureau~?
Je serais assez curieux de savoir ce que sont certaines de ces choses,~» et c'était son euphémisme du mois de septembre.

Dumbledore le fixa un moment, puis acquiesça en faisant une légère grimace.
«~Je suis flatté par ton intérêt, dit-il, mais j'ai peur qu'il n'y a pas grand-chose à dire.~»
Dumbledore fit un pas de plus vers le mur et tendit le doigt en direction d'un homme endormi.

«~Ce sont les portraits des directeurs de Poudlard.~»
Il se tourna et pointa vers son bureau.
«~C'est mon bureau.~»
Il pointa vers sa chaise.
«~C'est ma chaise…

--- Excusez-moi, dit Harry, en fait, je m'interrogeais sur ces…~»
Harry pointa en direction d'un petit cube qui chuchotait doucement~: «~blurpe… blurpe… blurpe.

--- Oh, les petites choses gélatineuses~? dit Dumbledore.
Elles étaient incluses dans le bureau de directeur et je n'ai absolument aucune idée de ce que la plupart d'entre elles font.
Mais \emph{ce} cadran avec huit aiguilles compte le nombre de, disons d'éternuements, faits par les sorcières gauchères de France, et tu ne me croirais pas si je te disais le travail que ça a pris pour le faire fonctionner correctement.
Et \emph{celui-là} avec les petits gigoteurs est de mon invention, et Minerva ne va jamais, jamais réussir à comprendre ce qu'il fait.~»

Dumbledore fit un pas vers le porte-chapeaux pendant que Harry finissait d'emmagasiner ces informations.
«~Ici, bien sûr, nous avons le Choixpeau Magique, je crois que vous vous êtes rencontrés.
Il m'a dit qu'il ne fallait plus jamais qu'il soit mis sur ta tête, quelles que soient les circonstances.
Tu n'es que le quatorzième élève de l'histoire au sujet duquel il a dit ça, il y a aussi eu Baba Yaga, et je te parlerai des douze autres quand tu seras plus vieux.
Ça c'est un parapluie.
Ça c'est un autre parapluie.~»
Dumbledore fit quelques pas de plus et se retourna, avec un grand sourire sur le visage.
«~Et bien sûr, la plupart des gens qui viennent dans mon bureau veulent voir Fumseck.~»

Dumbledore se tenait à côté de l'oiseau sur la plate-forme dorée.

Harry s'approcha, plutôt perplexe.

«~C'est Fumseck~?

--- Fumseck est un phénix, dit Dumbledore.
Des créatures magiques très rares et très puissantes.

--- Ah…~» dit Harry.
Il baissa sa tête et fixa les petits petits yeux en perle noires qui ne montraient pas le moindre signe d'intelligence.

«~Ahhh…~» dit à nouveau Harry.

Il était presque certain de reconnaître la forme de l'oiseau.
C'était assez difficile à rater.

«~Humm…~»

\emph{Dis quelque chose d'intelligent~!} Rugit l'esprit de Harry à sa propre intention.
\emph{Ne reste pas là à baragouiner comme un crétin~!}

\emph{Ben qu'est-ce que je peux bien être} censé \emph{dire~?} répondit l'esprit de Harry.

\emph{N'importe quoi~!}

\emph{Tu veux dire, n'importe quoi sauf} «~Fumseck \emph{est un poulet…}~»

\emph{Oui~!
N'importe quoi sauf ça~!}

«~Et donc, ah, quelle sorte de magie font les phénix alors~?

--- Leurs larmes ont le pouvoir de guérir, dit Dumbledore.
~»Ce sont des créatures de feu, et ils se déplacent entre les lieux aussi facilement que le feu peut s'éteindre quelque part et se rallumer ailleurs.
La tension intense que provoque leur magie innée fait vieillir leur corps très rapidement, et pourtant ils sont les plus éternelles des créatures de ce monde, car lorsque leur corps les abandonne, ils s'immolent dans un jet de flammes et laissent derrière eux un nouveau-né, ou parfois un œuf.~»
Dumbledore s'approcha et inspecta le poulet, fronçant les sourcils.
«~Hmm… l'air un peu patraque on dirait.~»

Lorsque la phrase percuta pleinement l'esprit de Harry, le poulet était déjà en feu.

Le bec du poulet s'ouvrit, mais il n'eut pas le temps de caqueter une seule fois avant de commencer à flétrir et à se carboniser.
L'incendie fut bref, intense, et complètement isolé~; il n'y avait pas d'odeur de brûlé.

Puis le feu mourut seulement quelques secondes après avoir commencé, laissant derrière un petit tas pathétique de cendres sur la plate-forme dorée.

«~N'ai pas l'air si horrifié, Harry~! dit Dumbledore.
Fumseck n'a pas eu mal.~»
La main de Dumbledore plongea dans sa poche, puis la même main passa dans les cendres et fit surgir un petit œuf jaunâtre.
«~Regarde, voilà un œuf~!

--- Oh… waoh… incroyable…

--- Mais nous devrions nous activer~», dit Dumbledore.
Laissant l'œuf derrière lui entre les cendres du poulet, il revint à son trône et s'assit.
«~C'est presque l'heure du dîner après tout, et nous ne voudrions pas avoir à utiliser nos retourneurs de temps.~»

Il y eut une violente lutte de pouvoir au Gouvernement de Harry.
Serpentard et Poufsouffle avaient changé de camp après avoir vu le directeur de Poudlard mettre le feu à un poulet.

«~Oui, nous activer, dirent les lèvres de Harry.
Et puis, dîner.~»

\emph{Tu baragouines encore comme un crétin} nota le Critique Interne de Harry.

«~Bon, dit Dumbledore.
J'ai peur d'avoir une confession à te faire, Harry.
Une confession et une excuse.

--- Les excuses, c'est bien.~»
\emph{Ça ne veut rien dire~!
Mais de quoi je parle~?}

Le vieux sorcier soupira profondément.

«~Tu ne le penseras peut-être plus après avoir compris ce que j'ai à te dire.
J'ai bien peur, Harry, de t'avoir manipulé pendant toute ta vie.
C'est moi qui t'ai remis à la garde de tes beaux-parents malfaisants…

--- Mes beaux-parents ne sont pas malfaisants~! lâcha Harry.
Mes \emph{parents}, je veux dire~!

--- Ils ne le sont pas~?~»
dit Dumbledore, l'air surpris et déçu.
«~Pas même un peu malfaisants~?
Ça ne cadre pas avec…~»

Le Serpentard intérieur de Harry hurla à s'en faire exploser les poumons, TAIS-TOI IDIOT IL T'ENLÈVERA À EUX~!

«~Non, non~», dit Harry, les lèvres figées en une grimace livide, «~j'essayais juste de vous épargner, ils sont à vrai dire très malfaisants…

--- Ils le sont~?~»
Dumbledore se pencha en avant, le regardant avec intensité.
«~Que font-ils~?~»

\emph{Parle vite} «~ils, ah, je dois faire la vaisselle et laver des problèmes et ils ne me laissent pas lire beaucoup de livres et…

--- Ah, bien, c'est bon à entendre~», dit Dumbledore, se penchant à nouveau en arrière.
Il sourit d'un air triste.
«~Je te demande pardon pour \emph{ça}, alors.
Maintenant, où en étais-je~?
Ah, oui.
Je suis navré de te dire, Harry, que je suis responsable de presque tout ce qui t'es jamais arrivé de mal.
Je sais que ça te mettra probablement très en colère.

--- Oui, je suis très en colère~! dit Harry.
Grrr~!

Le Critique Interne de Harry lui décerna promptement le Prix Ultime de Pire Jeu d'Acteur de l'Histoire de Tous les Temps.

«~Et je voulais juste que tu le saches, dit Dumbledore, je voulais te le dire aussitôt que possible, au cas où quelque chose arriverait à l'un de nous deux plus tard, que je suis vraiment, vraiment navré.
Pour tout ce qui s'est déjà produit, et pour tout ce qui se produira.~»

De l'humidité scintillait dans les yeux du vieux sorcier.

«~Et je suis très en colère~! dit Harry.
Tellement en colère que je veux partir tout de suite à moins que vous n'ayez quelque chose d'autre à me dire~!~»

\emph{PARS avant qu'il ne te mette le feu à toi aussi~!} crièrent Serpentard, Poufsouffle et Gryffondor.

«~Je comprends, dit Dumbledore.
Alors une dernière chose, Harry.
Tu ne devras \emph{pas} essayer de franchir la porte interdite dans le couloir du troisième étage.
Il est impossible que tu traverses tous les pièges, et je ne voudrais pas apprendre que tu t'es fait mal en essayant.
Enfin, je doute que tu puisses ne serait-ce qu'ouvrir la première porte, puisqu'elle est fermée à clé et que tu ne connais pas le sort \emph{Alohomora}…~»

Harry fit demi-tour et se précipita vers la sortie à pleine vitesse, la poignée pivota agréablement dans sa main, et un instant après il descendait les escaliers quatre à quatre alors même qu'ils tournaient, ses pieds se faisaient presque des croche-pattes, et un moment plus tard il était en bas et la gargouille faisait un pas sur le côté et Harry se propulsait hors de la cage d'escalier tel un boulet de canon.

\later

Harry Potter.

Il devait y avoir quelque chose de spécial chez Harry Potter.

C'était jeudi pour tout le monde après tout, et ce genre de choses ne semblait arriver à personne d'autre.

Il était 18h21 un jeudi après-midi quand Harry Potter, se propulsant hors de la cage d'escalier tel un boulet de canon et accélérant au maximum, fonça droit sur Minerva McGonagall alors qu'elle prenait un tournant, en chemin vers le bureau du directeur.

Heureusement, aucun d'entre eux n'eut très mal.
Comme on le lui avait expliqué plus tôt dans la journée -- alors qu'il refusait de jamais s'approcher à nouveau d'un balai -- le Quidditch avait besoin de Cognards en fer solide juste pour avoir une bonne chance de blesser les joueurs, puisque les sorciers avaient tendance à être beaucoup plus résistants aux chocs que les Moldus.

Harry et le professeur McGonagall finirent tous deux au sol, et les parchemins qu'elle transportait étalés dans le couloir.

Il y eut une pause épouvantable.

«~Harry Potter~», souffla le professeur McGonagall de là où elle était allongée, juste à côté de Harry.
Sa voix devint presque un cri.
«~\emph{Que faisiez-vous dans le bureau du directeur~?}

--- Rien~! glapit Harry.

--- \emph{Parliez-vous du professeur de Défense} \emph{?}

--- Non~!
Dumbledore m'a appelé dans son bureau et il m'a donné ce gros rocher et il m'a dit que c'était celui de mon père et que je devrai le transporter partout~!~»

Il y eut une autre pause épouvantable.

«~Je vois~», dit le professeur McGonagall, sa voix un peu plus calme.
Elle se leva, s'épousseta, et jeta un coup d'œil aux parchemins éparpillés, qui bondirent pour former une pile bien ordonnée et filèrent le long du mur du couloir comme s'ils essayaient d'échapper à son regard.

«~Mes amitiés, M. Potter, et je vous demande pardon d'avoir douté de vous.

--- Professeur McGonagall~», dit Harry.
Sa voix tremblait.
Il poussa contre le sol, se leva, et regarda ce visage sérieux et sain d'esprit.
«~Professeur McGonagall…

--- Oui, M. Potter~?

--- Pensez-vous que je devrais le faire~? dit Harry d'une petite voix.
Transporter le rocher de mon père partout où je vais~?~»

Le professeur McGonagall soupira.
«~J'ai bien peur que ce ne soit entre vous et le directeur.~»
Elle hésita.
«~Je dirais que complètement ignorer ce que le directeur dit n'est presque jamais une bonne idée.
Je \emph{suis} navrée d'apprendre votre dilemme, M. Potter, et si je \emph{peux} vous aider d'une quelconque façon, quoi que vous choisissiez de faire…

--- Euh, dit Harry.
À vrai dire je me disais qu'une fois que j'aurais trouvé comment faire, je pourrais métamorphoser le rocher en un anneau et le porter à mon doigt.
Si vous pouviez m'enseigner comment maintenir une métamorphose…

--- Vous avez bien fait de m'en parler avant~», dit le professeur McGonagall, son visage devenant légèrement sévère.
«~Si vous perdiez le contrôle de la métamorphose, l'annulation vous couperait le doigt et vous fendrait probablement la main en deux.
Et à votre âge, même un anneau est une cible trop large pour être maintenue indéfiniment sans que ce soit un sérieux drain de votre magie.
Mais je peux vous faire forger un anneau doté d'un emplacement pour un joyau, un \emph{petit} joyau, en contact avec votre peau, et vous pouvez pratiquer avec un objet sûr, comme un marshmallow.
Lorsque vous l'aurez maintenu avec succès, même pendant votre sommeil, pendant un mois entier, je vous autoriserai à métamorphoser le, ah, le rocher de votre père…~»
Le professeur McGonagall laissa sa phrase en suspens.
«~Le directeur a-t-il \emph{vraiment}…

--- Oui.
Ah… euh…~»

Le Professeur McGonagall soupira.

«~C'est un peu étrange, même de sa part.~»
Elle se baissa et ramassa la pile de parchemins.
«~Je suis navrée, M. Potter.
Je vous demande à nouveau pardon de ne pas vous avoir fait confiance.
Mais maintenant, c'est à moi d'aller voir le directeur.

--- Ah… bonne chance, j'imagine.
Euh…

--- Merci, M. Potter.

--- Hmm…~»

Le professeur McGonagall marcha jusqu'à la gargouille, donna le mot de passe sans que Harry puisse l'entendre, et prit place sur les escaliers en spirale tournants.
Elle commença à s'élever hors de la vue de Harry, et la gargouille commença à revenir…

«~\emph{Professeur McGonagall le directeur a mis le feu à un poulet~!}

--- Il a \emph{quo…}~»
%  LocalWords:  hursday Remembrall Goyle’s Hufflepuffle Slytherslime Hah
%  LocalWords:  Remembralls Libatius Ehhh reaallly blorple wibblers Ahhh
%  LocalWords:  Umm Grrr wha
