\namedpartchapter{L'Expérience de Prison de Stanford}{TSPE}{IX}{Curiosité}

\lettrine{L}{e} balai avaient la réputation d'avoir été inventés à une époque qu'un Moldu aurait qualifiée de moyenâgeuse par une légendaire sorcière du nom de Celestria Relevo prétendument arrière-arrière-arrière petite fille de Merlin.

Celestria Relevo ou la ou les personnes qui avaient vraiment inventé ces enchantements ne connaissaient rien à la mécanique de Newton.

Les balais volants fonctionnaient donc selon les lois de la physique aristotélicienne.

Ils allaient vers là où vous les orientiez.

Si vous vouliez aller vers l'avant, vous pointiez vers l'avant~; vous n'aviez pas à vous inquiéter de maintenir un peu de poussée vers le haut afin d'annuler l'effet de la gravité.

Si vous preniez un virage, toute la vélocité passait dans la direction vers laquelle le balai pointait et vous ne dériviez pas dans le sens de votre élan actuel.

Les balais avaient des vitesses maximales mais pas d'accélérations maximales. Cela n'avait rien à voir avec la résistance de l'air mais était dû au fait que les enchantements d'un balai volant étaient pourvus d'un niveau maximum d'\emph{impetus} Aristotélicien.

Même s'il était assez adroit pour obtenir les meilleures notes en cours de vol du balai, Harry ne l'avait encore jamais \emph{remarqué}. Le fonctionnement de ces derniers était si proche de ce à quoi l'esprit humain \emph{s'attendait instinctivement} que son cerveau avait réussi à \emph{totalement laisser passer leur absurdité physique}. En ce premier jour de leçons de balai Harry avait été distrait par des phénomènes apparemment plus intéressants, tels que des mots écrits sur un papier et une sphère rouge lumineuse. Son cerveau avait alors simplement suspendu son incrédulité, inscrit la réalité des balais volants comme étant acceptée, et avait entrepris de s'amuser sans \emph{penser une seule fois à la question} dont la réponse aurait été évidente. Il est en effet tristement avéré que nous ne pensons jamais qu'à une \emph{petite} fraction de tous les phénomènes que nous rencontrons…

Ceci est l'histoire de la façon dont Harry James Potter-Evans-Verres faillit être tué par son manque de curiosité.

Parce que les fusées ne respectaient \emph{pas} la physique Aristotélicienne.

Les fusées n'avaient \emph{pas} le fonctionnement qu'un esprit humain attendait instinctivement d'un objet volant.

Un balai volant assisté d'une fusée ne se déplaçait donc \emph{pas} comme les balais magiques sur lesquels Harry était si doué.

Rien de tout ceci ne traversa alors l'esprit de Harry.

Déjà, le bruit le plus fort qu'il avait entendu de sa vie l'empêchait de s'entendre penser.

Ensuite, accélérer vers le haut à quatre gravités signifiait qu'il avait environ deux secondes et demie en tout pour aller des fondations d'Azkaban jusqu'à son sommet.

Et même si elles furent les deux secondes et demi les plus \emph{longues} de l'histoire du Temps, cela ne lui laissa pas vraiment assez de place pour penser.

Il y avait seulement le temps de voir la lumière des malédictions Aurors fondre droit sur lui, de légèrement réorienter le balai pour les éviter, de se rendre compte que le balai continuait simplement sur son élan au lieu d'aller dans la direction vers laquelle il pointait, d'activer les concepts dépourvus de symboles~:

\emph{*merde*}

et

\emph{*Newton*}

ce sur quoi Harry inclina le balai bien plus fort et là ils commencèrent à s'approcher très vite du mur alors il l'inclina dans l'autre sens et il y avait d'autres lumières qui descendaient et les Détraqueurs glissaient en douceur vers eux accompagnés d'une espèce de créature géante ailée faite de flammes blanc-or alors Harry tira brusquement le balai vers le ciel mais maintenant il dérivait encore vers un autre mur alors il inclina légèrement le balai et cessa de s'approcher, mais il était déjà trop proche alors il l'inclina de nouveau et là les distants Aurors et leurs balais ne furent plus distants du tout et il allait percuter cette femme alors il dévia son balai loin d'elle et l'instant d'après il réalisa que sa fusée était un lance-flammes extrêmement puissant et que dans une fraction de seconde il serait directement braqué sur l'Auror alors il inclina le balai de côté et continua de monter et il ne pouvait pas se rappeler si le lance-flammes était braqué sur d'autres Aurors mais au moins il n'était pas pointé vers \emph{elle.}

Harry manqua un autre Auror à un mètre près et lui passa sous le nez, juché sur son lance-flammes de travers qui grimpait à, Harry estimerait plus tard, environ 300 kilomètres par heure.

S'il y eut des cris d'Aurors grillés il ne les entendit pas mais cela ne constituait une preuve de rien du tout parce que pour l'instant tout ce qu'il entendait c'était un son extrêmement puissant.

Deux secondes \emph{plus calmes si ce n'est plus silencieuses} plus tard, il ne sembla plus y avoir ni Auror ni Détraqueur ni géante créature de flammes ailée et le vaste et terrible édifice d'Azkaban semblait étonnamment petit vu de cette hauteur.

Harry dirigea le balai volant vers le Soleil, à peine visible à travers les nuages, il n'était pas haut dans le ciel à cette heure du jour en ce mois d'hiver et le balai accéléra deux secondes de plus dans la même direction et amassa très rapidement une vitesse extraordinaire avant que le combustible solide ne s'épuise.

Après cela Harry put de nouveau s'entendre penser, il n'y avait que le vent hurlant provoqué par leur absurde vitesse, et bien qu'assistés par de la magie ses doigts agrippés au manche résistaient à peine à la traînée de décélération due au fait qu'ils étaient bien au-delà de leur vitesse terminale, et c'est \emph{alors} que Harry pensa enfin à tous ces trucs sur la mécanique de Newton et la physique Aristotélicienne et les balais et les fusées et l'importance de la curiosité et qu'il n'allait jamais refaire quoi que ce soit d'aussi Gryffondor de sa vie ou au moins pas avant d'avoir appris le secret de l'immortalité du Seigneur des Ténèbres et \emph{pourquoi} avait-il écouté le professeur Quirinus «~\parsel{Je t'asssure, garççon, je n'esssaierai pas çcela ssi je n'antiçcipais pas ma propre ssurvie}~» Quirrell au lieu du Professeur Michael «~Mon fils, si tu essaies quelque chose avec des fusées sans supervision, je dis bien \emph{quoi que ce soit} sans qu'un professionnel entraîné te regarde, tu mourras et Maman sera triste~» Verres-Evans.

\later

«~\scream{Quoi~?}~» glapit Amelia à l'intention du miroir.

\later

Le vent s'était assourdi jusqu'à atteindre un niveau supportable à mesure que la résistance de l'air les avait ralentis, donnant ainsi à Harry la chance d'écouter le son à mi-chemin entre le carillon et le bourdonnement qui semblait emplir son cerveau.

Le professeur Quirrell était censé avoir lancé un sortilège de mutisme sur le pot d'échappement de la fusée… apparemment les capacités des sortilèges de mutismes avaient des limites… rétrospectivement, Harry aurait dû métamorphoser une paire de bouchons d'oreille au lieu de simplement faire confiance au sortilège, même si cela n'aurait probablement pas non plus suffi…

Eh bien, les soins magiques avaient probablement quelque chose qui soignait les dommages auditifs permanents.

Non, vraiment, les soins magiques avaient probablement quelque chose pour ça. Il avait vu des élèves aller voir madame Pomfresh avec des blessures qui lui avaient semblé être bien pires…

\emph{Existe-t-il un moyen de transplanter une personnalité imaginaire dans la tête de quelqu'un d'autre~?} demanda Poufsouffle. \emph{Je ne veux plus vivre dans la tienne}.

Harry repoussa le tout au fond de son esprit car il n'y avait vraiment rien qu'il puisse faire à ce sujet pour l'instant. Y avait-il quoi que ce soit dont il \emph{devrait} s'inquiéter…

Puis Harry regarda derrière lui, pensant pour la première fois à vérifier si Bellatrix et le professeur Quirrell avaient été soufflés loin du balai.

Mais le serpent vert était toujours dans son harnais, la femme émaciée s'accrochait toujours au balai, son visage était toujours pris par cette couleur malsaine et ses yeux toujours brillants et dangereux. Ses épaules s'agitaient comme si elle riait de façon hystérique et ses lèvres bougeaient comme si elle criait mais aucun son ne s'échappait…

Ah, c'est vrai.

Harry abaissa la capuche de sa Cape et tapota ses oreilles afin de lui faire savoir qu'il ne pouvait entendre.

Ce sur quoi Bellatrix se saisit de sa baguette et la pointa vers Harry. Le bourdonnement dans ses oreilles diminua soudain et il put l'entendre.

Il le regretta un instant plus tard~; les imprécations qu'elle hurlait à l'intention d'Azkaban, des Détraqueurs, des Aurors, de Dumbledore, de Lucius, de Barty Croupton, de quelque chose appelé l'Ordre du Phénix, de tout ce qui se tenait sur le chemin de son Seigneur des Ténèbres, etc. n'étaient pas faites pour une audience plus jeune et plus sensible~; et son rire faisait mal à ses oreilles récemment soignées.

«~Assez, Bella~», dit enfin Harry, et la voix de celle-ci se tut instantanément.

Il y eut un moment de silence. Harry retira la Cape de par-dessus sa tête, par principe~; et se rendit compte au même moment qu'ils avaient peut-être des télescopes en bas, ou quelque chose du genre, et rétrospectivement, retirer sa capuche ne serait-ce que pour un moment avait été une décision incroyablement stupide. Il espérait que la mission n'allait pas échouer à cause de cette unique erreur…

\emph{On n'est vraiment pas faits pour ça, hein~?} remarqua Serpentard.

\emph{Hé}, dit Poufsouffle par pur réflexe, \emph{on ne peut pas s'attendre à tout faire parfaitement la première fois, on a probablement juste besoin de plus de pratique OUBLIE QUE J'AI DIT ÇA.}

Harry regarda de nouveau en arrière et vit que Bellatrix regardait autour d'elle avec un air perplexe et émerveillé. Elle n'arrêtait pas de pivoter la tête.

Et elle dit enfin, sa voix maintenant plus basse~: «~Seigneur, où sommes-nous~?~»

\emph{Que veux-tu dire~?} aurait voulu dire Harry, mais le Seigneur des Ténèbres n'admettrait jamais qu'il ne comprenait pas quelque chose, et Harry répondit donc d'un ton sec~: «~Nous sommes sur un balai volant.~»

\emph{Est-ce qu'elle pense qu'elle est morte et que c'est le paradis~?}

Les mains de Bellatrix étaient toujours enchaînées au balai, si bien que seul un doigt s'éleva lorsqu'elle dit~: «~Qu'est-ce que c'est que \emph{ça}~?~»

Harry suivit la direction indiquée par le doigt et vit… rien de spécial, à vrai dire…

Puis il comprit. Il n'y avait plus de nuages pour l'obscurcir maintenant qu'ils s'étaient suffisamment élevés.

«~C'est le soleil, chère Bella.~»

Le ton fut remarquablement maîtrisé, un Seigneur des Ténèbres parfaitement calme et peut-être un peu agacé, alors même que les larmes commençaient à couler le long des joues de Harry.

Dans le froid sans fin, dans les ténèbres absolues, le soleil avait sûrement été…

Un souvenir heureux…

Bellatrix continua de se tordre le cou.

«~Et les choses moelleuses~? dit-elle.

--- Des nuages.~»

Il y eut une pause, puis Bellatrix dit~: «~Mais que \emph{sont}-ils~?~»

Harry ne répondit pas car il aurait été incapable de garder une voix égale, tout ce qu'il pouvait faire était de respirer de façon parfaitement régulière tout en pleurant.

Après un moment, Bellatrix souffla, d'une voix si basse que Harry faillit ne pas l'entendre~: «~Joli…~»

Son visage se détendit doucement, la couleur quittant son pâle visage presque aussi vite qu'elle l'avait gagné.

Son corps squelettique s'affaissa sur le balai volant.

La baguette empruntée pendait sans vie depuis la bride attachée à sa main immobile.

\emph{C'EST UNE BLAGUE…}

L'esprit de Harry se souvint alors que la potion de Pimentine avait un prix~; Bellatrix \emph{ss'assoupirait pour un temps conssidérable} avait dit le professeur Quirrell.

Et au même instant, une autre partie de Harry devint profondément convaincue en regardant la femme émaciée d'une pâleur de craie à l'air plus morte sous le soleil radieux que toutes les choses vivantes que Harry avait jamais vues, qu'elle \emph{était} morte, qu'elle venait de prononcer ses derniers mots, que le professeur Quirrell avait mal jugé le dosage…

… ou qu'il avait délibérément sacrifié Bellatrix pour protéger leur évasion…

\emph{Est-ce qu'elle respire~?}

Harry n'arrivait pas à voir si c'était le cas.

Tant qu'il était sur le balai volant il ne pouvait pas se pencher en arrière et prendre son pouls.

Harry regarda devant lui pour vérifier qu'ils n'étaient pas sur le point de percuter des rochers volants et continua d'orienter le balai vers le soleil, le garçon invisible et la femme peut-être morte partants vers l'après-midi, les doigts de Harry serrés si fort autour du balai qu'ils étaient devenus blancs.

Il ne pouvait pas l'atteindre et pratiquer une ventilation artificielle.

Rien de son kit de soins n'aurait été utile.

\emph{Croire que le professeur Quirrell ne l'aurait pas mise en danger~?}

Étrange, c'était étrange de vraiment croire que le professeur Quirrell n'avait pas eu l'intention de tuer l'Auror (car cela \emph{aurait} été stupide) et de ne pourtant plus être rassuré par les paroles rassurantes du professeur de Défense.

Puis l'idée vint à Harry qu'il lui fallait encore vérifier…

Harry regarda derrière lui et siffla~: «~\parsel{Professseur~?}~»

Dans son harnais, le serpent ne remua pas et ne prononça pas un mot.

… peut-être que le serpent, n'étant pas à proprement parler un passager du balai, n'avait pas été protégé de l'accélération. Ou peut-être que s'approcher autant des Détraqueurs sans bouclier et même sous forme Animagus l'avait assommé.

La situation n'était pas bonne.

Ça aurait dû être au professeur Quirrell de dire à Harry quand il pouvait utiliser le Portoloin en toute sécurité.

Harry dirigea le balai volant de ses doigts blanchis et réfléchit, il réfléchit très fort pendant une durée courte et incommensurable pendant laquelle Bellatrix pouvait respirer ou ne pas respirer, pendant laquelle le professeur Quirrell lui-même avait peut-être déjà cessé de respirer depuis un moment.

Et Harry décida que tandis qu'il était possible de rattraper l'erreur consistant à gâcher le Portoloin en sa possession, il n'était pas possible que de rattraper l'erreur consistant à laisser un cerveau sans oxygène pendant trop longtemps.

Alors il tira de sa bourse le prochain Portoloin de la série et, alors qu'il arrêtait le balai volant au milieu du ciel bleu clair (maintenant qu'il y pensait, il ne savait pas si la capacité d'un Portoloin à s'ajuster à la rotation de la Terre incluait aussi la capacité de s'adapter de façon générale à la vélocité de son nouvel environnement), il plaça le Portoloin en contact avec le balai et…

Harry marqua une pause, la brindille toujours dans sa main, sœur de celle qu'il avait brisée voilà ce qui lui semblait être deux semaines. Il sentit une réticence soudaine~; son cerveau semblait avoir appris la règle par le biais d'un processus de renforcement négatif purement neuronal que Briser des Brindilles Est Une Mauvaise Idée.

Mais ce n'était pas vraiment logique alors Harry brisa quand même la brindille.

\later

Il y eut une explosion orageuse derrière la porte de métal située non loin d'Amelia, ce qui lui fit tomber le miroir qu'elle tenait et pivoter baguette en main, et alors la porte s'ouvrit grand, révélant Albus Dumbledore, debout face à un grand trou fumant dans le mur de la prison.

«~Amelia~», dit le vieux sorcier. Il n'y avait aucune trace de sa légèreté habituelle, et derrières ses lunettes en demi-lune ses yeux avaient la dureté du saphir. «~Je dois quitter Azkaban et je dois le faire \emph{maintenant}. Y a-t-il un moyen de sortir de l'enceinte qui soit plus rapide que le balai~?

--- Non…

--- Alors j'ai besoin de ton balai volant le plus rapide, immédiatement~!~»

Là où Amelia \emph{voulait} être, c'était avec l'Auror qui avait été blessé par ce Feudeymon, ou par quoi que cela ait bien pu être.

Ce qu'elle \emph{devait} faire, c'était découvrir ce que Dumbledore savait.

«~Vous~! aboya la vieille sorcière à l'intention de l'équipe qui l'entourait. Continuez de sécuriser les couloirs jusqu'à avoir atteint le fond, ils ne se sont peut-être pas tous encore échappés~!~» Puis au vieux sorcier~: «~Deux balais. Tu pourras me mettre au courant quand on sera dans les airs.~»

Il y eut un duel de regards, mais il ne fut pas long.

\later

Une saccade écœurante considérablement plus forte que celle qui l'avait amené à Azkaban attrapa Harry à l'abdomen, et cette fois la distance traversée fut si grande qu'il put écouter un instant de silence et observer l'espace entre les espaces, la crevasse entre un lieu et l'autre.

\later

Alors qu'ils se projetaient loin d'Azkaban dans la direction du vent et plus rapidement que lui, le soleil qui avait brièvement brillé sur eux deux fut prestement occlus par un nuage de pluie.

«~Qui est derrière ça~?~» cria Amelia au balai qui volait un pas derrière elle.

«~L'une des deux personnes, répondit Dumbledore, j'ignore encore laquelle. Si c'est la première, alors nous avons des ennuis. Si c'est la seconde, alors nous avons de bien plus graves ennuis.~»

Amelia ne s'octroya pas un soupir. «~Quand sauras-tu~?~»

La voix du vieux sorcier, sombre et basse, parvenait pourtant à s'élever au-dessus du vent.

«~Trois choses dont ils ont besoin pour atteindre la perfection, s'il s'agit de cela~: la chair du plus fidèle des serviteurs du Seigneur des Ténèbres, le sang de son plus grand ennemi, et accès à une certaine tombe. Je pensais Harry Potter à l'abri, avec leur tentative à Azkaban quasiment échouée -- même si je l'ai quand même mis sous bonne garde -- mais à présent j'ai bel et bien peur. Ils ont accès au Temps, quelqu'un avec un retourneur de temps envoie des messages pour eux~; et je soupçonne que la tentative de kidnapping sur Harry Potter a déjà eu lieu il y a quelques heures. C'est pourquoi \emph{nous} n'en avons pas entendu parler, puisque nous sommes à Azkaban, où le temps ne peut pas se nouer lui-même. Le passé est venu après notre propre futur, vois-tu.

--- Et si c'est l'autre~?~» cria Amelia. Ce qu'elle avait entendu était suffisamment inquiétant~; cela semblait être le plus noir des rituels noirs, centré sur feu le Seigneur des Ténèbres lui-même.

Le vieux sorcier, son visage à présent encore plus sombre, ne répondit pas et se contenta de secouer la tête.

\later

Lorsque la saccade du Portoloin se fut calmée, le soleil pointait juste au-dessus de l'horizon, plus proche d'une aurore que d'un crépuscule, et leur balai flottait bas au-dessus d'une courte étendue de roche noir-orange et de sable arrangé en collines bosselées comme si quelqu'un avait pétri la pâte du sol plusieurs fois puis avait oublié de l'aplatir. Non loin, des vagues défilaient le long d'un panorama d'eau sans fin, mais le sol au-dessus duquel le balai flottait surplombait le niveau de la mer par plusieurs mètres au moins.

Harry cligna des yeux face aux couleurs de l'aube et comprit que le Portoloin avait été international.

Un vif cri féminin «~Hé~!~» retenti de derrière lui, et il pivota le balai volant pour regarder. Une femme d'âge mûr à l'air très affairé avait une main portée à sa bouche et l'appelait visiblement. Ses traits sympathiques, ses yeux étroits et sa peau terre d'ombre appartenaient à une ethnie peu familière à Harry~; elle était habillée d'une robe d'un violet vif d'un genre que Harry n'avait jamais vu auparavant~; et lorsque ses lèvres s'ouvrirent de nouveau elle parla avec un accent que Harry n'aurait su situer, n'étant pas un grand voyageur. «~Où étais-tu~? Tu as deux heures de retard~! J'ai failli renoncer à vous voir arriver… bonjour~?~»

Il y eut une brève pause. Les pensées de Harry semblaient se mouvoir étrangement, trop lentes, tout semblait distant, comme s'il y avait eu une épaisse paroi de verre entre lui et le monde et une autre entre lui et ses émotions, si bien qu'il pouvait voir mais il ne pouvait pas toucher. Cela lui était venu lorsqu'il avait vu la lumière aurorale et la sorcière sympathique, lorsqu'il avait pensé que tout cela ressemblait à une fin adéquate pour cette aventure.

Puis la sorcière se précipita et tira sa baguette~; un marmonnement sectionna les attaches qui liaient la femme émaciée au balai volant et Bellatrix fut lévitée jusqu'à un rocher sableux, ses bras squelettiques et ses jambes pâles pendantes comme des choses sans vie. «~Oh, Merlin, murmura la sorcière, Merlin, Merlin, Merlin…~»

\emph{Elle semble préoccupée}, pensa une chose abstraite et distance entre les deux parois de verre. \emph{Est-ce ce qu'une véritable guérisseuse dirait ou est-ce ce que quelqu'un à qui l'on a dit de jouer un rôle dirait~?}

Comme si ce n'était pas Harry qui avait parlé mais une autre partie de lui située de l'autre côté d'une énième paroi de verre, un murmure vint de ses lèvres~: «~Le serpent vert sur son dos est un Animagus.~» Pas aigu ni froid, le murmure, seulement bas. «~Il est inconscient.~»

La tête de la sorcière se releva brusquement pour regarder vers l'endroit vide d'où cette voix semblait être venue, puis elle rabaissa les yeux vers Bellatrix.

«~Vous n'êtes pas M. Jaffe.

--- Vous parlez de l'Animagus,~» murmurèrent les lèvres de Harry. \emph{Oh}, pensa le Harry derrière le verre en entendant le son de ses propres lèvres, \emph{c'est sensé, le professeur Quirrell doit avoir utilisé un autre nom.}

«~Depuis quand est-\emph{il} -- bah, quelle importance.~» La sorcière plaça sa baguette sur le nez du serpent pendant un moment puis secoua vivement la tête. «~Rien qu'une journée de repos ne saura guérir. \emph{Elle…}

--- Pouvez-vous le réveiller maintenant~?~» murmurèrent les lèvres de Harry. \emph{Est-ce une bonne idée~?} pensa-t-il, mais ses lèvres semblaient clairement penser que c'était le cas.

Encore le vif secouement de tête. «~Si un Innerver n'a pas fonctionné…~» commença-t-elle.

«~Je ne m'y suis pas essayé, murmurèrent les lèvres de Harry.

--- Quoi~? Pourquoi -- oh, pas grave. \emph{Innerver}.~»

Il y eut une pause puis un serpent rampa lentement hors de son harnais. La tête verte s'éleva lentement, regarda autour d'elle.

L'espace se brouilla et le professeur Quirrell se tint là, et l'instant d'après il tombait à genoux.

«~Allonge-toi, dit la sorcière sans lever les yeux de Bellatrix. C'est toi qu'es là, Jérémie~?

--- Oui~», dit le professeur de Défense d'une voix plutôt rauque avant de s'allonger précautionneusement sur une portion relativement plate du rocher sableux et orangé. Il n'était pas aussi pâle que Bellatrix mais son visage était comme exsangue sous la faible lueur de l'aurore. «~Salutations, mademoiselle Chomblechette.

--- Je t'ai dit, fit la sorcière avec une dureté dans sa voix et un léger sourire sur son visage, de m'appeler Crystal, on n'est pas en Angleterre, pas de ces formalités ici. \emph{Et} ce sera Docteur maintenant, pas mademoiselle.

--- Mes excuses, docteur Chomblechette~», accompagné d'un gloussement sec.

Le sourire de la sorcière s'élargit quelque peu mais sa voix se durcit d'autant plus.

«~Qui est ton ami~?

--- Tu n'as pas besoin de le savoir.~» Le professeur de Défense était allongé sur le sol, les yeux fermés.

«~À quel point ça a mal tourné~?~»

Une sécheresse certaine~:

«~Tu pourras le lire demain dans n'importe quel journal doté d'une rubrique internationale.~»

La baguette de la sorcière tapait ici ou là, appuyait et sondait toute la surface du corps de Bellatrix.

«~Tu m'as manqué, Jérémie.

--- Vraiment~? dit le professeur de Défense d'un ton légèrement surpris.

--- Même pas un tout petit peu. Si je ne t'en devais pas une…~»

Le professeur de Défense entama un rire qui se rapprocha ensuite de la quinte de toux.

\emph{T'en penses quoi~?} dit Serpentard au Critique Interne tandis que Harry écoutait de derrière les murs de verre. \emph{C'est un numéro ou c'est pour de vrai~?}

\emph{Saurais pas dire}, répondit le Critique Interne de Harry. \emph{Je suis pas en grande forme critique pour l'instant.}

\emph{Est-ce que quelqu'un a une idée d'un bon moyen d'obtenir plus d'information~?} dit Serdaigle.

Encore le souffle venu du vide au-dessus du balai~:

«~Quelles sont les chances qu'on puisse annuler tout ce qui lui a été fait~?

--- Oh, voyons. Légilimancie et rituels noirs inconnus, dix ans pour que ça se mette en place, puis dix ans d'exposition aux Détraqueurs~? Défaire \emph{ça}~? Vous êtes complètement toqué, M. Qui-Que-Vous-Soyez. La question est de savoir s'il \emph{reste} quoi que ce soit, et je dirais peut-être une chance sur trois…~» La sorcière s'interrompit soudain. Lorsqu'elle parla de nouveau, sa voix était plus basse. «~Si vous étiez son ami avant… alors non, vous ne la retrouverez jamais. Mieux vaut que vous le compreniez maintenant.~»

\emph{Je vote numéro} dit le Critique Interne. \emph{Elle ne déblatérerait pas tout ça pour répondre à une seule question à moins d'avoir été en train d'attendre une opportunité de le faire.}

\emph{Noté, mais j'y accorde une confiance réduite}, dit Serdaigle. \emph{Il est très difficile de ne pas laisser ses soupçons contrôler ses perceptions lorsqu'on essaie de mesurer des preuves aussi subtiles.}

«~Quelle potion lui as-tu donnée~?~» dit la sorcière après avoir ouvert la bouche de Bellatrix et y avoir jeté un coup d'œil alors que sa baguette l'illuminait de multiples flashs colorés.

L'homme allongé au sol dit calmement~:

«~Pimentine…

--- \emph{Tu as perdu la tête~?}~»

Encore la quinte de rire.

«~Elle va dormir pendant une semaine, avec de la chance, dit la sorcière avant de claquer la langue. J'imagine que je t'enverrai une chouette quand elle ouvrira les yeux pour que tu puisses revenir et la convaincre pour ce Serment Inviolable. Est-ce que tu as quelque chose pour l'empêcher de me liquider sur place si jamais elle arrive à bouger dans le mois qui vient~?~»

Les yeux toujours fermés, le professeur de Défense prit une feuille de papier de sa robe~; un moment plus tard, des mots commencèrent à apparaître à sa surface, accompagnés de petites volutes de fumées. Lorsque la fumée eut cessé de s'élever, le papier lévita en direction de la femme.

Elle regarda le papier avec ses sourcils arqués et eut un reniflement sardonique~: «~Il vaudrait mieux que ça marche, Jérémie, ou ma dernière volonté et mon testament diront que tous mes biens constituent une prime pour la personne qui aura ta tête. En parlant de ça…~»

Le professeur de Défense chercha de nouveau dans sa robe et jeta à la sorcière un sac qui émit un cliquetis. La sorcière l'attrapa, le soupesa, et eut un bruit de contentement.

Puis elle se leva et la pâle femme squelettique lévita à côté d'elle.

«~Je rentre, dit la sorcière. Je ne peux pas commencer à travailler ici.

--- Attends~», dit le professeur de Défense, et d'un geste il récupéra sa baguette depuis la main et l'étui de Bellatrix. Puis il orienta la baguette vers Bellatrix et fit un petit geste circulaire accompagné d'un discret «~\emph{Oubliettes}~».

«~\emph{Ça suffit}, lâcha la sorcière, je la sors d'ici avant que quelqu'un ne fasse encore plus de dégâts…~». Un bras entoura la forme osseuse de Bellatrix Black et la serra contre la sorcière, puis elles disparurent toutes deux au puissant “POP~!” de Transplanage.

Et ce lieu bosselé devint silencieux, hormis le discret flot des vagues qui passaient et un léger souffle de vent.

\emph{Je pense que le numéro est terminé}, dit le Critique Interne. \emph{Je lui mets deux sur cinq. Ce n'est probablement pas une actrice très expérimentée.}

\emph{Je me demande si une vraie guérisseuse semblerait plus fausse qu'une actrice à qui l'on aurait dit d'en jouer une~?} médita Serdaigle.

Comme de regarder une émission de télévision, c'était ça, comme de regarder une émission de télévision dont les personnages ne suscitait aucune empathie particulière, c'était là tout ce qui pouvait être vu et ressenti depuis l'autre côté des murs de verre.

Harry parvint sans savoir comment à déplacer ses lèvres lui-même, à envoyer sa propre voix dans l'air toujours auroral, puis il fut surpris d'entendre sa question~: «~Et sinon, vous êtes combien de personnes différentes~?~»

L'homme pâle allongé au sol ne rit pas mais depuis le balai Harry put voir le coin des lèvres du professeur Quirrell qui se recourbaient, la lisière du sourire sardonique familier. «~Je ne peux pas prétendre m'être embêté à en garder le compte. Et vous, combien en êtes-vous~?~»

Entendre cette réponse n'aurait pas dû autant secouer le Harry intérieur et pourtant il se sentait -- il se sentait -- instable, comme si son centre avait été soustrait…

Oh.

«~Excusez-moi~», dit la voix de Harry. Le caractère distant et détaché de celle-ci reflétait maintenant l'effacement que Harry ressentait. «~Je pense que je vais m'évanouir dans quelques secondes.

--- Utilisez le quatrième Portoloin que je vous ai donné, celui dont j'ai dit que c'était notre refuge de secours, dit calmement mais rapidement l'homme allongé au sol. Ce sera plus sûr là-bas. Et continuez de porter votre cape.~»

La main libre de Harry récupéra une autre brindille depuis sa bourse et la brisa.

Il y eut une autre saccade de Portoloin, une saccade internationalement lointaine, et il fut dans un endroit noir.

«~\emph{Lumos}~», dirent les lèvres de Harry alors qu'une partie de lui évaluait la sûreté de l'ensemble.

Il était à l'intérieur de ce qui ressemblait à un entrepôt Moldu, un entrepôt désert.

Ses jambes descendirent du balai et s'allongèrent au sol. Ses yeux se fermèrent et une petite fraction de son être intima à sa lumière de s'éteindre avant que les ténèbres ne le recouvrent.

\later

«~Où iras-tu~?~» cria Amelia. Ils étaient presque à la limite de l'enceinte.

«~En arrière dans le temps pour protéger Harry Potter~», dit le vieux sorcier, et avant qu'Amelia ne puisse ne serait-ce qu'ouvrir ses lèvres pour demander s'il voulait de l'aide, elle ressentit le passage d'un bord à l'autre de la lisière de l'enceinte.

Il y eut un pop de Transplanage et le sorcier et le phénix disparurent, laissant derrière eux le balai emprunté.
%  LocalWords:  TSPE roomsticks Celestria Relevo Oy Camblebunker
