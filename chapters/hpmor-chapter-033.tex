\partchapter{Problèmes de coordination}{I}


\lettrine{C}{e} qui était le plus effrayant, c'était la vitesse à laquelle l'affaire avait échappé à tout contrôle.

«~Albus~», dit Minerva lorsqu'ils entrèrent dans la Grande Salle, sans même essayer de maintenir l'inquiétude hors de sa voix, «~il faut faire quelque chose.~»

L'atmosphère à Poudlard avant Yule était généralement radieuse et éclatante.
La Grande Salle avait déjà été décorée en vert et rouge d'après un Gryffondor et une Serpentard dont le mariage Yulien était devenu un symbole d'amitié transcendant les Maisons et les allégeances, une tradition presque aussi ancienne que Poudlard elle-même et qui s'était même répandue jusqu'à certains pays Moldus.

Les élèves dînaient maintenant en jetant des regards nerveux par-dessus leurs épaules, ou en jetant des regards mauvais à d'autres tablées, ou en débattant furieusement avec celles-ci.
Peut-être aurait-on pu décrire l'atmosphère par le terme de \emph{tendue}, mais l'expression qui venait à Minerva était \emph{cinquième degré de prudence}.

Prenez une école, en quatre Maisons divisée…

Maintenant, dans chaque année, trois armées en guerre.

Et l'esprit partisan pour Dragon, Soleil et Chaos s'était répandu au-delà de la première année~; ils étaient devenus les armées de ceux qui n'en avaient pas.
Les élèves portaient des brassards au signe du feu, du sourire ou de la main levée, et ils se jetaient des sortilèges dans les couloirs.
Les trois généraux de première année leur avaient dit d'arrêter -- même Drago avait amèrement acquiescé après avoir écouté jusqu'au bout -- mais leurs partisans n'avaient rien voulu entendre.

Dumbledore parcourut les tables du regard, l'air distant.

«~Il y a longtemps, cita doucement le vieux sorcier, que les Habitants de chaque ville sont divisés en deux factions, les Bleus et les Verts… ils se battent sans savoir la raison pour laquelle ils se querellent et se mettent en danger… ils conçoivent sans raison une haine implacable pour leurs proches et ils la conservent toute leur vie sans la faire céder ni aux règles de l'honneur, de la parenté ou de l'amitié, et il en est de même si ceux qui ne diffèrent que par ces couleurs sont frères ou liés de quelque autre façon.
Ce que je ne puis attribuer qu'à je ne sais quelle maladie d'esprit…

--- Je suis navrée, dit Minerva, je ne…

--- Procope de Césarée\footnotemark{}, dit Dumbledore.
Ils prenaient les courses de chariot très au sérieux dans l'Empire romain.
Oui Minerva, je suis d'accord, il faut faire quelque chose.~»
\translatorsnotetext{La traduction officielle \emph{française} de l'extrait de Procope de Césarée cité par Dumbledore dans ce chapitre est si différente de la version officielle \emph{anglaise} que j'ai modifié la traduction \emph{française} pour ne pas trop m'éloigner du sens précis des paroles de Dumbledore dans le contexte de cette histoire.
Par souci de transparence, voici~\href{http://books.google.com/books?id=LaSDKWqZZ2oC\&pg=PA79\&lpg=PA79\&dq=procopec\%C3\%A9sar\%C3\%A9e\%22deuxpartis\%22\&source=bl\&ots=KRyS9lQT2Y\&sig=T9whMnEOaskHpkcIxOpvyrdSj1o\&hl=en\&ei=UhZUTqb5GsbNsgaHnuUX\&sa=X\&oi=book_result\&ct=result\&resnum=1\&sqi=2\&ved=0CBUQ6AEwAA\#v=onepage\&q=deux\%20partis\&f=false}{la source où j'ai trouvé la version française}.
J'en ai fait une retranscription partielle en français moderne.}
«~Bientôt~», dit Minerva, sa voix s'assourdissant encore.
«~Albus, je pense qu'il faut que ce soit avant samedi.~»

Le dimanche, la plupart des élèves quitteraient Poudlard pour passer les vacances avec leur famille~; le samedi serait donc la bataille finale entre les trois armées de première année, celle qui déterminerait le récipiendaire du vœu de Noël trois fois maudit du professeur Quirrell.

Dumbledore lui jeta un regard, l'étudiant d'un air grave.
«~Tu crains que l'explosion ne survienne à ce moment et que quelqu'un soit blessé.~»

Minerva hocha la tête.

«~Et que le professeur Quirrell soit tenu pour responsable.~»

Minerva hocha de nouveau la tête, le visage pincé.
Il y avait longtemps qu'elle avait tiré une leçon de la façon dont les professeurs de Défense étaient renvoyés.

«~Albus, dit Minerva, nous ne pouvons pas perdre le professeur Quirrell maintenant, \emph{nous ne pouvons pas}~!
Mais s'il reste jusqu'à janvier, alors nos cinquième année passeront leur BUSE, s'il reste jusqu'à mars, nos septième année passeront leur ASPIC, il rattrape des années de négligence en l'espace de quelques mois, toute une génération grandira capable de se défendre en dépit de la malédiction du Seigneur des Ténèbres -- vous devez arrêter la bataille, Albus~!
Interdisez les armées maintenant~!

--- Je ne suis pas certain que le professeur de Défense accepterait avec le sourire,~» dit Dumbledore, jetant un œil en direction de la grande table où le professeur Quirrell bavait dans sa soupe.
«~Il m'a semblé des plus attaché à ses armées, même si lorsque j'ai accepté, je croyais qu'il y en aurait quatre par année.~»
Le vieux sorcier soupira.
«~Un homme intelligent, probablement avec les meilleures des intentions~; mais peut-être pas assez intelligent, j'en ai peur.
Et interdire les armées pourrait aussi déclencher l'explosion.

--- Mais alors Albus, qu'allez-vous \emph{faire}~?~»

Le vieux sorcier lui fit don d'un sourire bienveillant.
«~Allons, je vais conspirer, bien sûr.
C'est la nouvelle mode à Poudlard.~»

Et ils étaient maintenant trop proches de la grande table pour que Minerva puisse ajouter quoi que ce soit.

\later

Ce qui était le plus effrayant, c'était la vitesse à laquelle l'affaire avait échappé à tout contrôle.

La première bataille de décembre avait été…
\emph{brouillon}, c'est du moins ce que Drago avait entendu dire.

La deuxième avait été \emph{dérangée}.

Et la prochaine serait \emph{pire} à moins qu'ils ne réussissent tous les trois dans leur dernière tentative désespérée de l'arrêter.

«~Professeur Quirrell, c'est de la folie, dit Drago sur le ton de l'évidence.
Ce n'est plus Serpentard, c'est juste…~»
Drago manquait de mots.
Il agita les mains avec impuissance.
«~On ne peut pas intriguer quand ce genre de choses se passe.
À la dernière bataille, l'un de mes soldats a feint son suicide.
On a des \emph{Poufsouffle} qui essaient de conspirer et ils pensent qu'ils peuvent, mais ils ne \emph{peuvent pas}.
Les choses se produisent au hasard maintenant, ça n'a rien à voir avec qui est le plus malin, ou avec quelle armée combat le mieux, c'est…~»
Il ne pouvait même pas le décrire.

«~Je suis d'accord avec M. Malfoy~», dit Granger d'un ton qui laissait penser qu'elle ne s'était jamais attendue à s'entendre dire une chose pareille.
«~Autoriser les traîtres ne fonctionne pas, professeur Quirrell.~»

Drago avait essayé d'interdire à quiconque dans son armée de comploter mis à part lui, et cela avait juste fait continuer les complots de façon souterraine, personne ne voulait être hors du coup, pas quand les soldats des \emph{autres} armées avaient la chance de comploter.
Après avoir misérablement perdu leur dernière bataille, il avait enfin cédé et il avait révoqué son décret~; mais ses soldats avaient déjà mis leurs propres plans en mouvement sans la moindre forme de coordination centrale.

Après avoir entendu tous les plans, ou ce que ses soldats disaient être les plans, Drago avait essayé d'en esquisser un qui permette de remporter la bataille finale.
Cela avait nécessité que bien plus de trois choses se déroulent comme prévu et Drago avait utilisé \emph{Incendio} suivi de \emph{Everto} pour faire disparaître les cendres, car si Père l'avait vu, Drago aurait été déshérité.

Les yeux du professeur Quirrell étaient mis-clos, son menton reposait sur ses mains et il était penché sur son bureau.

«~Et vous M. Potter~? dit le professeur de Défense.
Êtes-vous d'accord avec eux~?

--- Il n'y aurait qu'à descendre Franz Ferdinand et on pourrait faire démarrer la première guerre mondiale, dit Harry.
C'est le chaos complet.
Je suis carrément pour.

--- \emph{Harry~!}~» dit Drago, profondément choqué.

Il lui fallut une seconde pour se rendre compte qu'il l'avait dit exactement au même moment et avec le même ton d'indignation choquée que Granger.

Granger lui jeta un regard surpris et Drago conserva un visage prudemment neutre.
Oups.

«~Eh ouais~! dit Harry.
Je vous trahis~!
Tous les deux~!
Encore~!
Ha ha~!~»

Le professeur Quirrell avait un fin sourire, même si ses yeux étaient toujours mi-clos.

«~Et pourquoi donc, M. Potter~?

--- Parce que je pense que je peux supporter le chaos mieux que Mlle Granger et M. Malfoy, dit le traître.
Notre guerre est un jeu à somme nulle, et que ce soit facile ou difficile dans l'absolu n'a aucune importance, ce qui compte, c'est qui s'en sort mieux et qui s'en sort moins bien.~»

Harry Potter apprenait trop vite.

Les yeux du professeur Quirrell se déplacèrent sous leur paupière en direction de Drago, puis de Granger.

«~En vérité, M. Malfoy, Mlle Granger, je ne pourrais tout simplement pas me regarder en face si j'arrêtais la grande débâcle avant son paroxysme.
L'un de vos soldats est même devenu un agent quadruple.

--- \emph{Quadruple}~! dit Granger.
Mais il n'y a que trois camps dans la guerre~!

--- Oui, dit le professeur Quirrell, c'est ce que vous auriez pensé, n'est-ce pas.
Je ne suis pas certain qu'il y ait jamais eu un agent quadruple dans l'Histoire, ou une armée avec une fraction si élevée de vrais et de faux traîtres.
Nous explorons de nouveaux territoires, Mlle Granger, et nous ne pouvons plus rebrousser chemin.~»

Drago quitta le bureau du professeur de Défense avec ses dents grinçant avec force les unes contre les autres, et Granger à côté, avait l'air encore plus irritée.

«~Je n'arrive pas à croire que tu aies fait ça, Harry~! dit Granger.

--- Désolé~», dit Harry, l'air pas désolé du tout, ses lèvres recourbées en un joyeux sourire maléfique.
«~Souviens-toi Hermione, c'\emph{est} juste un jeu, et pourquoi les généraux comme nous devraient-ils être les seuls à avoir le droit de comploter~?
Et puis, qu'est-ce que vous allez y faire, tous les deux~?
Vous allier contre moi~?~»

Drago échangea des regards avec Granger, sachant que son propre visage était aussi pincé que le sien.
Harry s'était appuyé, de plus en plus ouvertement et avec de plus en plus de jubilation, sur le refus de Drago de faire cause commune avec une fille Sang-de-Bourbe~; et Drago commençait à être \emph{malade} qu'on utilise ça contre lui.
Si ça continuait un peu plus longtemps il \emph{allait} s'allier avec Granger juste pour écraser Harry Potter, et on verrait bien si le fils de Sang-de-Bourbe aimerait \emph{ça}.

\later

Ce qui était le plus effrayant, c'était la vitesse à laquelle l'affaire avait échappé à tout contrôle.

Hermione fixa le parchemin que Zabini lui avait donné et elle se sentit complètement, absolument impuissante.

Il y avait des noms, et des lignes reliant les noms à d'autres noms, et certaines des lignes avaient d'autres couleurs et…

«~Dites-moi, dit le général Granger, y a-t-il quelqu'un dans mon armée qui \emph{n'est pas} un espion~?~»

Ils n'étaient pas dans le bureau mais dans une autre salle, déserte, et ils étaient seuls~; parce que le colonel Zabini avait dit qu'il était maintenant presque certain qu'au moins un des capitaines était un traître.
Probablement le capitaine Goldstein, mais Zabini n'était pas sûr.

Sa question avait fait naître un sourire ironique sur le jeune visage du Serpentard.
Blaise Zabini avait toujours semblé légèrement dédaigneux envers elle, mais il ne semblait pas éprouver une aversion franche à son égard~; rien de semblable à la dérision dans laquelle il tenait Drago Malfoy ou le ressentiment qu'il avait développé contre Harry Potter.
Elle s'était d'abord inquiétée que Zabini la trahisse, mais le garçon semblait désespérément vouloir montrer que les deux autres généraux n'étaient pas meilleurs que lui~; et Hermione pensait que même si Zabini serait probablement heureux de la vendre à n'importe qui d'\emph{autre}, il ne laisserait jamais Malfoy ou Harry gagner.

«~La plupart de vos soldats vous \emph{sont} toujours loyaux, j'en suis assez certain, dit Zabini.
C'est juste que personne ne veut passer à côté de l'amusement que la trahison procure.~»
Le regard dédaigneux de Zabini laissa clairement entendre son opinion des gens qui ne prenaient pas les complots au sérieux.
«~Alors ils pensent pouvoir être des agents doubles et travailler secrètement pour notre camp tout en prétendant nous trahir.

--- Et cela vaudrait aussi pour toute personne des \emph{autres} armées disant qu'elles veulent être \emph{nos} espions~», dit prudemment Hermione.

Le jeune Serpentard haussa les épaules.
«~Je pense que j'ai bien réussi à déterminer ceux qui veulent vraiment vendre Malfoy, je ne suis pas certain que \emph{quiconque} veuille vraiment vous vendre Potter.
Mais Nott va sûrement vouloir trahir Potter pour Malfoy et puisque j'ai demandé à Soufflebranche de l'approcher, prétendument de la part de Malfoy, et que Soufflebranche présente en fait ses rapports à notre camp, c'est presque aussi bien que…~»

Hermione ferma ses yeux un moment.

«~On va perdre, c'est ça~?

--- Écoutez, dit Zabini avec patience, vous êtes en tête en points Quirrell pour l'instant.
Il nous suffit de ne pas \emph{complètement} perdre cette bataille et vous aurez assez de point pour gagner le vœu de Noël.~»

Le professeur Quirrell avait annoncé que la bataille finale suivrait un système de points formalisé qu'on lui avait demandé de créer afin d'éviter toute récrimination après la bataille.
Chaque fois que vous abattiez quelqu'un, le général de votre armée gagnait deux points Quirrell.
Un gong retentirait à travers l'arène (ils ne savaient pas où ils se battraient, même si Hermione espérait que ce serait encore la forêt, où Soleil se débrouillait bien) et la note jouée indiquerait quelle armée avait gagné les points.
Et si quelqu'un faisait semblant d'être touché, le gong retentirait quand même, et un double gong retentirait plus tard, après une durée non déterminée, pour annoncer la rétraction.
Et si vous acclamiez le nom d'une armée, si vous criiez «~Pour Soleil~!~»
ou «~Pour Chaos~!~»
ou «~Pour Dragon~!~»
 cela plaçait votre allégeance sur cette armée…

Même Hermione avait pu voir le défaut de \emph{cet} ensemble de règles.
Mais le professeur Quirrell avait alors annoncé que si vous aviez été originellement assigné à Soleil, personne ne pourrait vous abattre au nom de Soleil -- ou plutôt, ils le pourraient, mais Soleil perdrait alors un seul point Quirrell, symbolisé par un triple gong.
Cela vous empêchait d'abattre vos propres soldats pour deux points et décourageait les suicides avant que l'ennemi ait pu vous atteindre, mais vous pouviez toujours abattre les espions si nécessaire.

Pour l'instant, Hermione avait deux-cent-quarante-quatre points Quirrell, et Malfoy en avait deux-cent-dix-neuf, et Harry en avait deux-cent-vingt-et-un~; et il y avait vingt-quatre soldats dans chaque armée.

«~Donc nous nous battons avec précaution, dit Hermione, et nous essayons de ne pas perdre trop salement.

--- Non~», dit Zabini.
Le visage du jeune Serpentard était à présent sérieux.
«~Le problème, c'est que Malfoy et Potter savent maintenant que leur seule façon de gagner est de s'allier et de nous écraser, puis de combattre seuls.
Alors voilà ce que je pense que nous devrions faire…~»

Hermione quitta la salle relativement hébétée.
Le plan de Zabini n'avait pas été évident, il avait été étrange et compliqué et intriqué et le genre de chose qu'elle aurait vu Harry inventer plutôt que Zabini.
Il lui semblait anormal de pouvoir \emph{comprendre} un plan comme celui-ci.
Les jeunes filles n'auraient pas dû être capables de comprendre des plans comme celui-ci.
Le Choixpeau l'aurait répartie à Serpentard s'il avait vu qu'elle pouvait comprendre des plans comme celui-ci…

\later

Ce qui était le plus génial, c'était la vitesse à laquelle il était parvenu à faire monter le niveau de chaos une fois qu'il avait commencé à le faire délibérément.

Harry était assis dans son bureau~; il avait obtenu l'autorité de commander du mobilier chez les elfes de maisons, alors il avait commandé un trône et des rideaux imprimés d'un motif noir et cramoisi.
Mêlée à l'ombre, une lumière sanguine écarlate se déversait au sol.

Quelque chose à l'intérieur de Harry lui disait qu'il était enfin à la maison.

Devant lui se tenaient ses lieutenants du Chaos, ses sbires de confiance, dont l'un d'eux était un traître.

Comme ça.
C'est comme ça que la vie aurait dû être.

«~Nous sommes rassemblés, dit Harry.

--- Que le Chaos règne, dirent les quatre lieutenants en chœur.

--- Mon aéroglisseur est plein d'anguilles, dit Harry.

--- Je n'achèterai pas ce disque, il est rayé, dirent les quatre lieutenants en chœur.

--- Tout smouales étaient les Borogoves.

--- Les vergons fourgus bourniflaient~!~»

Cela conclut les formalités.

«~Comment va la confusion~? dit Harry dans un souffle sec semblable à celui de l'empereur Palpatine.

--- Elle va bien, général Chaos~», dit Neville du ton qu'il utilisait toujours pour les affaires militaires, un ton si profond que le garçon devait souvent s'arrêter et tousser.
Le lieutenant Chaotique était bien habillé dans sa robe scolaire noire, brodée du jaune de la maison Poufsouffle, et ses cheveux étaient coiffés à la façon des jeunes garçons sérieux.
Harry avait aimé cette incongruité plus qu'aucune des capes qu'ils avaient essayées.
«~Nos légionnaires ont commencé cinq nouveaux complots depuis hier soir.~»

Harry eut un sourire maléfique.

«~L'un d'entre eux a-t-il une chance de fonctionner~?

--- Je ne pense pas, dit Neville du Chaos.
Voici le rapport.

--- Excellent~», dit Harry, et il eut un sourire glacé tout en prenant le parchemin de mains de Neville, faisant de son mieux pour faire comme s'il s'étouffait sur de la poussière.
Cela faisait un total de soixante.

Que Drago \emph{essaie} de gérer ça.
Qu'il \emph{essaie}.

Et pour Blaise Zabini…

Harry rit de nouveau, et cette fois il n'eut même pas à faire l'effort d'avoir l'air maléfique.
Il fallait vraiment qu'il emprunte le fléreur de compagnie de quelqu'un, pour avoir un chat à caresser en même temps.

«~La Légion peut-elle arrêter de faire des complots maintenant~? dit Finnigan du Chaos.
Je veux dire, n'avons-nous pas déjà assez de…

--- Non, dit catégoriquement Harry.
On ne peut \emph{jamais} avoir assez de complots.~»

Le professeur Quirrell l'avait parfaitement formulé.
Ils repoussaient les limites très loin, peut-être plus loin qu'elles n'avaient jamais été repoussées~; et Harry n'aurait pas pu se regarder en face s'il avait rebroussé chemin maintenant.

On entendit un coup contre la porte.

«~Ce sera le général Dragon~», dit Harry, souriant de sa prescience maléfique.
«~Il arrive, précisément au moment où je m'y attendais.
Invitez-le donc à entrer, et vous-même à sortir.~»

Et les quatre lieutenants du Chaos s'éparpillèrent, jetant des regards noirs à Drago tandis que le général ennemi entrait dans le repère secret de Harry.

S'il n'avait pas le droit de faire ça une fois devenu adulte, Harry allait juste avoir onze ans pour toujours.

\later

Le soleil dégoulinait à travers les rideaux rouges, envoyant des rayons de sang danser à la surface du sol situé derrière la chaise rembourrée à taille adulte de Harry Potter, chaise qu'il avait recouverte de paillettes or et argent et qu'il s'évertuait à appeler son trône.

(Drago commençait à être de plus en plus persuadé qu'il avait eu raison de décider de renverser Harry Potter avant que celui-ci ne puisse conquérir le monde.
Drago n'arrivait même pas à \emph{imaginer} ce que ça pourrait être que de vivre sous son règne).

«~Bonsoir, général Dragon, dit Harry Potter d'un murmure brutal.
Vous voilà, comme je m'y attendais.~»

Ce n'était pas surprenant puisque Drago et Harry s'étaient mis d'accord sur l'heure de leur rendez-vous.

Et aussi, on n'était pas le soir, mais à ce stade, Drago avait assez d'expérience pour savoir qu'il valait mieux ne rien dire.

«~Général Potter, dit Drago avec autant de dignité qu'il en était capable, vous savez que nos deux armées doivent travailler ensemble pour qu'\emph{un} d'entre nous ait une chance de remporter le vœu du professeur Quirrell~?

--- Sc'est scela~», siffla Harry, comme si le garçon avait cru être un Fourchelangue.
«~Nous devons coopérer pour détruire Soleil, puis en découdre entre nous.
Mais si l'un de nous trahit l'autre en début de combat, il pourrait avoir un avantage plus tard.
Et le général Soleil, qui sait tout cela, essaiera de nous tromper en nous faisant penser que l'autre l'a trahi.
Et vous et moi, qui savons cela, serons tentés de trahir l'autre et de prétendre que c'est une tromperie de Granger.
Et Granger sait aussi \emph{cela}.~»

Drago hocha la tête.
Évident jusque-là.

«~Et… nous voulons tous deux gagner et \emph{rien d'autre}, et il n'y a personne pour nous punir si nous trahissons l'autre…

--- Exactement,~» dit Harry Potter, son visage devenant maintenant sérieux.
«~Nous faisons face à un \emph{vrai} dilemme du prisonnier.~»

Selon les enseignements de Harry, le dilemme du prisonnier fonctionnait ainsi~: deux prisonniers avaient été enfermés dans des cellules séparées.
Il y avait des preuves contre chaque prisonnier, mais seulement des preuves mineures, assez pour une peine de prison de deux ans.
Chaque prisonnier pouvait décider de \emph{trahir} l'autre, c'est-à-dire de témoigner contre lui au tribunal~; et cela diminuerait sa peine d'un an mais ajouterait deux ans à celle de l'autre.
Ou alors, un prisonnier pouvait décider de \emph{coopérer}, de garder le silence.
Donc si les deux prisonniers trahissaient, chacun témoignant contre l'autre, ils purgeraient chacun une peine de trois ans~; s'ils coopéraient, c'est-à-dire s'ils restaient silencieux, ils purgeraient deux ans chacun~; mais si l'un trahissait et que l'autre coopérait, le traître purgerait une seule année et le coopérant en purgerait quatre.

Et les deux prisonniers devaient prendre leur décision sans connaître le choix de l'autre, et aucun n'aurait la chance de revenir sur sa décision.

Drago avait fait remarquer que si les deux prisonniers avaient été des Mangemorts pendant la guerre des sorciers, le Seigneur des Ténèbres aurait tué tout traître.

Harry avait acquiescé et avait dit que c'était \emph{une} façon de résoudre le dilemme du prisonnier -- et que de fait, c'était pour cette exacte raison que les deux Mangemorts \emph{auraient voulu} qu'il y ait un Seigneur des Ténèbres.

(Drago avait demandé à Harry de s'interrompre et de le laisser y réfléchir un moment avant qu'ils ne continuent.
Cela avait expliqué \emph{beaucoup} quant à la raison pour laquelle Père et ses amis avaient accepté de servir un Seigneur des Ténèbres qui était bien souvent peu amène envers eux…)

En fait, avait dit Harry, c'était plus ou moins la raison pour laquelle les gens avaient des gouvernements -- \emph{vous} seriez probablement mieux loti si vous voliez quelque chose à quelqu'un, tout comme chaque prisonnier serait individuellement mieux loti s'il trahissait dans un dilemme du prisonnier.
Mais si \emph{tout le monde} pensait comme ça, le pays sombrerait dans le chaos et tout le monde serait dans une situation bien pire, exactement comme ce qui se passait quand les deux prisonniers trahissaient.
Les gens se laissaient donc être dirigés par des gouvernements, tout comme les Mangemorts s'étaient laissés diriger par le Seigneur des Ténèbres.

(Drago avait demandé à Harry de s'interrompre une fois de plus.
Drago avait toujours tenu pour acquis que les sorciers ambitieux s'emparaient du pouvoir parce qu'ils voulaient diriger et que les gens se laissaient être dirigés parce qu'ils étaient de petits Poufsouffle effrayés.
Et, à la réflexion, cela semblait toujours être vrai~; mais le point de vue de Harry était fascinant, tout erroné qu'il soit).

Mais, avait continué Harry, la peur d'un tiers vous punissant n'était pas la \emph{seule} raison possible de coopérer au dilemme du prisonnier.

Imagine, avait dit Harry, que tu joues à un jeu contre une copie identique de toi-même créée par magie.

Drago avait dit que s'il y avait deux Drago, bien sûr qu'aucun Drago ne voudrait que quelque chose de mal arrive à l'autre, sans parler du fait qu'aucun Malfoy n'accepterait jamais de passer pour un traître.

Harry avait encore acquiescé, et il avait dit que c'était \emph{encore} une autre solution au dilemme du prisonnier -- que les gens pouvaient coopérer parce qu'ils se souciaient les uns des autres, ou parce qu'ils avaient un sens de l'honneur, ou parce qu'ils voulaient préserver leur réputation.
De fait, avait dit Harry, il était assez difficile de créer un \emph{véritable} dilemme du prisonnier -- dans la vraie vie, les gens se souciaient en général des autres, ou de leur honneur ou de leur réputation ou de la punition d'un Seigneur des Ténèbres ou \emph{d'autre chose} à part la peine de prison.
Mais en imaginant que la copie soit celle de quelqu'un \emph{d'entièrement} égoïste…

(Ils avaient pris Pansy Parkinson en exemple)

… alors chaque Pansy se soucierait uniquement de ce qui lui arrivait à \emph{elle} et pas à l'autre Pansy.

\emph{Étant donné} que c'était tout ce dont Pansy se souciait… et qu'il n'y avait pas de Seigneur des Ténèbres… et que Pansy ne s'inquiétait pas de sa réputation… et que Pansy n'avait soit pas de sens de l'honneur soit pas de sentiment d'obligation envers l'autre prisonnier…
\emph{alors} le choix rationnel pour Pansy serait-il de coopérer ou de trahir~?

Harry avait dit que certains avaient prétendu que le choix rationnel aurait été que Pansy trahisse sa copie, mais Harry, ainsi que quelqu'un d'autre nommé Douglas Hofstadter, pensait que ces gens avaient tort.
Parce que, avait continué Harry, si Pansy trahissait -- non pas au hasard, mais pour ce qui lui semblait être des raisons \emph{rationnelles} -- alors l'autre Pansy penserait exactement de même.
Deux copies identiques ne feraient pas des choix différents.
Pansy devait donc choisir entre un monde où toutes les Pansies coopéraient et un monde où toutes les Pansies trahissaient, et elle serait mieux lotie si les deux copies coopéraient.
Et si Harry avait pensé que les gens “rationnels” \emph{trahissaient} au dilemme du prisonnier, il n'aurait alors rien fait pour répandre ce genre de “rationalité”, parce qu'un pays ou une conspiration pleins de gens “rationnels” se seraient dissolus dans le chaos.
Vous parleriez de la “rationalité” à vos \emph{ennemis}.

Sur le moment, tout cela avait \emph{semblé} raisonnable à Drago, mais \emph{maintenant} la pensée lui venait que…

«~\emph{Tu} as dit, dit Drago, que la solution rationnelle au dilemme du prisonnier est de coopérer.
Mais bien sûr que \emph{tu} voudrais que je croie ça.~»
Et si Drago se faisait duper et coopérait, Harry dirait juste, \emph{ha ha, je t'ai encore trahi}~! et il se moquerait ensuite de lui.

«~Je ne me permettrai pas de falsifier les leçons, dit Harry avec sérieux.
Mais Drago, je dois te rappeler que je n'ai pas dit que tu devrais juste automatiquement coopérer.
Pas dans un \emph{vrai} dilemme du prisonnier tel que celui-ci.
Ce que j'ai dit, c'est que quand tu choisis, tu ne devrais pas penser comme si tu choisissais seulement pour toi-même \emph{ni} comme si tu choisissais pour tout le monde.
Tu dois penser comme si tu choisissais pour tous les gens \emph{suffisamment similaires} à toi pour avoir de fortes chances de faire la même chose que toi et pour les mêmes raisons.
Et aussi de choisir en fonction les prédictions faites par quiconque te connaissant suffisamment bien pour faire des prédictions correctes te concernant, pour que tu n'aies jamais à regretter d'avoir été rationnel à cause des prédictions correctes que les autres auraient faites à ton sujet -- rappelle-moi de te parler un jour du problème de Newcomb.
La question que nous devons donc tous les deux poser, Drago, est celle-ci~: sommes-nous suffisamment similaires pour avoir de fortes chances de faire \emph{la même chose}, quelle qu'elle soit, et pour les mêmes raisons générales~?
Ou~: nous connaissons-nous assez pour nous prédire l'un l'autre, pour que \emph{je} puisse prédire si tu coopéreras ou si tu trahiras, et que \emph{tu} puisses prédire si j'ai décidé de faire la même chose que j'ai prédit que tu feras, parce que \emph{je} sais que tu peux prédire que je le déciderai~?~»

… et Drago ne pouvait pas s'empêcher de penser que, vu qu'il avait besoin de faire un effort pour comprendre \emph{la moitié} de ce que Harry venait de dire, la réponse était évidemment “Non”.

«~Oui~», dit Drago.

Il y eut une pause.

«~Je vois, dit Harry l'air déçu.
Eh bien.
Dans ce cas, j'imagine que nous allons devoir trouver un autre moyen.~»

Drago n'avait pas pensé que cela allait fonctionner.

Drago et Harry en discutèrent encore et encore.
Bien plus tôt, ils s'étaient mis d'accord sur le fait que leurs actes sur le champ de bataille ne \emph{compteraient pas} comme des promesses rompues dans la vraie vie -- même si Drago était un peu en colère au sujet de ce que Harry avait fait dans le bureau du professeur Quirrell, et il le dit à Harry.

Mais si aucun d'eux ne pouvait compter sur l'honneur ou l'amitié, cela \emph{laissait} \emph{en suspens} la question de savoir comment faire coopérer leurs armées pour vaincre Soleil, en dépit de tout ce que Granger pourrait essayer de faire pour les diviser.
Les règles du professeur Quirrell ne créaient pas de tentation de laisser Soleil tuer les soldats de l'autre armée -- cela ne ferait qu'augmenter le niveau de difficulté pour l'armée restante -- mais elles créaient la tentation de se voler des victimes au lieu d'agir comme une armée unie le ferait, ou d'abattre quelques soldats du camp d'en face dans la confusion de la bataille…

\later

Hermione marchait en direction de Serdaigle sans vraiment regarder où elle allait, son esprit préoccupé par la guerre, par la duplicité, et par d'autres concepts impropres à son âge, et elle passa un angle et alla se cogner directement contre un adulte.

«~Pardon~», dit-elle, puis, sans vraiment y songer~: «~\emph{Aaaaaaaaaah~!}

--- Ne vous en faites pas, mademoiselle Granger~», dit le sourire joyeux placé sous les yeux pétillants et au-dessus de la barbe argentée du DIRECTEUR DE POUDLARD.
«~Vous êtes tout à fait pardonnée.~»

Son regard était inextricablement bloqué sur le doux visage du sorcier le plus puissant du monde, qui se trouvait aussi être l'Enchanteur-en-chef et le Manitou suprême, qui était devenu fou il y a plusieurs années en raison du stress occasionné par son combat contre le Seigneur des Ténèbres, et de nombreux autres faits apparaissaient dans son esprit les uns après les autres tandis que sa gorge émettait d'embarrassants petits couinements.

«~À vrai dire, dit Albus Percival Wulfric Brian Dumbledore, c'est un heureux hasard que nous nous soyons ainsi rentrés dedans.
Allons, je me demandais justement, et avec curiosité, ce que vous trois comptiez demander pour votre vœu…~»

\later

L'aube du samedi vint, lumineuse et claire, les élèves parlant de voix étouffées comme si le premier cri aurait risqué de déclencher l'explosion.

\later

Drago avait espéré qu'ils se battraient de nouveau dans les étages supérieurs de Poudlard.
Le professeur Quirrell avait dit que les véritables combats risquaient plus d'avoir lieu dans des villes que dans des forêts, et se battre dans les couloirs et les salles de classes était censé le simuler, avec des rubans pour délimiter les zones autorisées.
L'armée Dragon s'en était bien sortie lors de ces combats.

Au lieu de cela, et comme Drago l'avait craint, le professeur Quirrell avait eu une idée \emph{spéciale} pour ce combat.

Le champ de bataille serait le lac de Poudlard.

Et pas dans des bateaux.

Ils se battraient \emph{sous l'eau}.

Le poulpe géant avait été temporairement paralysé~; des sorts avaient été mis en place pour paralyser les strangulots~; le professeur Quirrell avait été parler à l'ondin~; et tous les soldats avaient reçu des \emph{potions d'action sous-marine} qui leur permettait de respirer, de voir clairement, de se parler et de nager pas tout à fait aussi vite qu'ils n'auraient marché.

Une immense sphère d'argent était suspendue au centre du champ de bataille, étincelant comme une petite lune sous-marine.
Cela aiderait à s'orienter -- au début.
La lune s'éclipserait lentement au cours de la bataille, et lorsqu'elle serait devenue entièrement noire, la bataille prendrait fin, si elle ne l'avait pas déjà fait.

La guerre sous l'eau.
Impossible de défendre un périmètre, les attaquants pouvaient venir de n'importe quelle direction, et même avec la potion on ne pouvait pas voir très loin dans les ténèbres du lac.

Et si vous nagiez trop loin des combats, vous commenceriez à briller passé un laps de temps, et vous seriez simple à pourchasser -- généralement, si une armée se dispersait et courait au lieu de se battre, le professeur Quirrell les déclarait simplement vaincus~; mais aujourd'hui, ils utilisaient un système de points.
Bien sûr, vous aviez un peu de temps \emph{avant} de commencer à briller, si vous vouliez jouer à l'assassin.

Pour le début du jeu, l'armée du Dragon avait été placée en profondeur~; au-dessus et loin de là, la distante Lune sous-marine brillait.
L'eau trouble était principalement éclairée par des sorts de \emph{Lumos}, bien que les soldats éteindraient les lumières dès qu'ils commenceraient les manœuvres.
Il n'y avait aucun avantage à laisser l'ennemi vous voir avant que vous ne l'ayez vous-même aperçu.

Drago battit des jambes plusieurs fois, se propulsant jusqu'à une position plus haute d'où il put observer l'endroit où ses soldats se tenaient, suspendus dans l'eau.

Les conversations moururent presque instantanément sous le regard de glace de Drago, ses soldats levant les yeux vers lui avec de gratifiantes expressions de peur et d'inquiétude.

«~Écoutez-moi très attentivement~», dit le général Malfoy.
Sa voix était un peu plus grave, un peu bulleuse, \emph{éboutez boi dlès attentibement}, mais le son se propageait avec clarté.
«~Il n'y a qu'une seule façon de gagner ceci.
Nous devons avancer sur Soleil avec Chaos et vaincre Soleil.
\emph{Ensuite}, nous nous battons contre Potter et nous gagnons.
Ça \emph{doit} se passer comme ça, compris~?
Peu importe ce qui se passe par ailleurs, cette partie \emph{doit} avoir lieu ainsi…~»

Et Drago expliqua le plan que lui et Harry avait concocté.

Des regards effarés furent échangés parmi les soldats.

«~… et si l'un de \emph{vos} complots fait obstacle à ceci, conclut Drago, je vous immolerai après que nous serons sortis de l'eau.~»

Il y eut un chorus nerveux de \emph{ouimsieur}.

«~Et tous ceux avec des ordres secrets, assurez-vous de les exécuter \emph{à la lettre}~», dit Drago.

Environ la moitié de ses soldats \emph{hochèrent ouvertement la tête}, et Drago les marqua comme devant être tués après son ascension au pouvoir.

Bien sûr, tous les ordres privés étaient faux, comme celui où un Dragon devait offrir une fausse commission de traître à un autre Dragon, le second Dragon ayant reçu l'ordre lors d'une confidence murmurée de faire part de tout ce que le premier Dragon dirait.
Drago avait dit à chaque Dragon que toute la guerre pouvait dépendre de cet ordre secret, et qu'il espérait qu'ils comprenaient que c'était plus important que les plans qu'ils avaient jusqu'alors eus en tête.
Avec de la chance, cela maintiendrait tous ces idiots dans un état de bonne humeur et permettrait peut-être d'évacuer quelques espions si les rapports ne correspondaient pas aux instructions.

Les vrais plans de Drago pour vaincre Chaos… et bien, c'était plus simple que celui qu'il avait brûlé, mais Père ne l'aurait toujours pas apprécié.
Même en essayant, Drago n'avait pas réussi à avoir une meilleure idée.
C'était un plan qui n'aurait \emph{jamais} pu fonctionner contre quelqu'un d'autre que Harry Potter.
À vrai dire, cela \emph{avait} été le plan original de Harry, à ce qu'en avait dit le traître, même si Drago l'avait deviné sans avoir besoin qu'on le lui dise.
Drago et le traître l'avaient juste modifié quelque peu…

\later

Harry prit une profonde inspiration, sentant l'eau gargouiller sans causer de dommages à travers ses poumons.

Ils s'étaient battus dans la forêt, et il n'avait pas pu le dire.

Ils s'étaient battus dans les couloirs de Poudlard, et il n'avait pas pu le dire.

Ils s'étaient battus dans les airs, des balais donnés à chaque soldat, et ça n'avait toujours eu aucun sens de le dire.

Harry s'était dit qu'il ne pourrait jamais prononcer ces mots, pas alors qu'il était encore assez jeune pour que cela veuille dire quelque chose…

Les légionnaires du Chaos regardaient Harry avec perplexité tandis que leur général nageait, ses pieds pointant en l'air, vers la lointaine lumière de la surface, la tête dirigée vers le bas, vers les troubles profondeurs.

«~\emph{Pourquoi êtes-vous à l'envers} \emph{?}~» cria le jeune commandeur à l'attention de son armée, et il commença à expliquer comment se battre après avoir abandonné l'orientation privilégiée de la gravité.

\later

Une cloche creuse et tonnante fit écho à travers l'eau, et à cet instant, Zabini, Anthony et cinq autres soldats foncèrent vers le bas, dans les troubles profondeurs du lac.
Parvati Patil, la seule Gryffondor du groupe, regarda derrière elle un instant et leur fit à tous un joyeux signe d'au revoir~; et un moment plus tard Scott et Matt firent de même.
Les autres se contentèrent de couler et de disparaître.

Le général Granger avala une boule coincée dans sa gorge en les regardant partir.
Elle risquait tout sur cela, diviser ainsi son armée au lieu de se limiter à abattre le plus de soldats possible.

La chose dont il fallait se rendre compte, lui avait dit Zabini, c'était qu'aucune armée ne bougerait avant d'avoir un plan qui lui permette de s'attendre à une victoire.
Soleil ne pouvait pas simplement gagner, il leur fallait faire \emph{croire} aux deux autres armées qu'elles gagneraient, et ce jusqu'à ce qu'il soit trop tard pour elles.

Ernie et Ron semblaient être encore en état de choc.
Susan regardait les soldats disparaissant avec un regard calculateur.
Son armée, ou ce qui en restait, avait juste l'air abasourdie, de fins réseaux de lumières tachetant leurs uniformes tandis qu'ils dérivaient juste en dessous de la surface ensoleillée du lac.

«~Et maintenant \emph{quoi}~? dit Ron.

--- Maintenant, on attend~», dit Hermione, assez fort pour que tous ses soldats l'entendent.
C'était étrange de parler la bouche pleine d'eau, elle avait constamment l'impression de commettre une horrible impolitesse à la table du dîner et d'être sur le point de se baver dessus.
«~Nous tous ici allons nous faire dézinguer, mais ça allait de toute façon avoir lieu avec Dragon et Chaos s'alliant contre nous.
Nous devons juste essayer d'en emporter le plus possible avec nous.

--- J'ai un plan~», dit l'un de ses soldats Soleil…
Hannah, sa voix avait été un peu difficile à reconnaître au début.
«~C'est plutôt compliqué, mais je sais comment nous pouvons pousser Dragon et Chaos à se battre les uns contre les autres…

--- Moi aussi~! dit Fay.
J'ai un plan aussi~!
Vous voyez, Neville Londubat est secrètement de notre côté…

--- \emph{Tu} parlais à Neville~? dit Ernie.
C'est pas normal, c'était \emph{moi} qui…~»

Daphné Greengrass et deux autres Serpentard qui n'étaient pas partis avec Zabini gloussaient à en perdre haleine alors que retentissait des «~Non, attends, c'est \emph{moi} qui ai converti Londubat~» venant d'un soldat après l'autre.

Hermione les regarda d'un air fatigué.

«~D'accord, dit Hermione lorsque le tumulte se fut tari, vous avez tous compris~?
Tous vos plans étaient fabriqués par la légion du Chaos, et certains peut-être par Dragon.
Toute personne voulant \emph{vraiment} trahir Harry ou Malfoy allait me voir moi ou Zabini, pas vous.
Allez-y, comparez vos notes et tous vos plans secrets et vous verrez par vous-mêmes.~»
Elle n'était peut-être pas aussi bonne en complot que Zabini, mais elle pouvait toujours comprendre ce que ses officiers lui disaient, c'était pour ça que le professeur Quirrell l'avait faite général.
«~Alors ne vous fatiguez pas à essayer d'autres complots quand les armées arriveront.
Battez-vous, d'accord~?
S'il vous plaît~?

--- Mais~», dit Ernie, un air choqué sur le visage, «~Neville est \emph{Poufsouffle}~!
Tu veux dire qu'il nous a \emph{menti}~?~»

Daphné riait si bruyamment et avec tant de force que ses exhalaisons sous-marines l'avaient mise tête en bas.

«~Je ne suis pas certain de ce qu'\emph{est} Londubat, dit Ron d'un air sombre, mais je ne pense pas qu'il soit toujours un Poufsouffle.
Pas maintenant que \emph{Harry Potter} a fait main basse sur lui.

--- Sais-tu, dit Susan, que je lui ai \emph{posé} cette question, et qu'il m'a dit qu'il était devenu un Poufsouffle du Chaos~?

--- \emph{Quoi qu'il en soit}, dit Hermione d'une voix forte.
Zabini est parti avec tous ceux que nous pensons être des espions, donc j'espère que dans \emph{notre} armée nous pouvons arrêter de nous surveiller autant.

--- \emph{Anthony} était un espion~? cria Ron.

--- \emph{Parvati} était une espionne~? s'étrangla Hannah.

--- Parvati était carrément une espionne, dit Daphné.
Elle faisait ses courses au marché des espions et portait du rouge à lèvre d'espion, et un jour elle va se marier avec un gentil mari espion et avoir plein de petits espions.~»

Et le son d'un gong fit alors écho à travers l'eau, indiquant que Soleil venait de marquer deux points.

Ce qui fut rapidement suivi par le son de Dragon perdant un seul point.

Les traîtres n'avaient pas le droit de tuer les généraux, pas après le désastre de la première bataille en décembre où les trois généraux avaient été abattus durant la première minute.
Mais avec de la chance…

«~Oh, dit Hermione.
On dirait que M. Crabbe fait une petite sieste.~»

\later

Comme deux bancs de poissons, les armées nageaient côte à côte.

Neville Londubat battait des pieds de mouvements lents et mesurés.
Plonger, toujours plonger dans la direction dans laquelle vous vous trouviez vous diriger.
Vous vouliez montrer à l'ennemi la plus petite silhouette, leur montrer votre tête ou vos pieds.
Alors vous plongiez toujours, tête la première, et l'ennemi était toujours \emph{en bas}.

Comme tous les légionnaires du Chaos, la tête de Neville tournait continuellement tandis qu'il nageait, regardant en haut, en bas, autour, de tous les côtés.
Ne cherchant pas seulement les soldats du Soleil, mais surveillant tout signe qu'un légionnaire du Chaos avait tiré sa baguette et était sur le point de les trahir.
Les traîtres attendaient généralement jusqu'à la confusion de la bataille pour agir, mais ce gong prématuré les avait tous mis sur leurs gardes.

… en vérité, cela attristait Neville.
En novembre, il avait été dans une armée unie, tous rassemblés, s'entraidant, et maintenant ils se surveillaient constamment à la recherche du premier signe de trahison.
Cela était peut-être plus amusant pour général Chaos, mais c'était loin de l'être autant pour Neville.

La direction anciennement connue sous le nom de “haut” devenait de plus en plus claire tandis qu'ils s'approchaient de la surface et de Soleil.

«~Baguettes dehors~», dit le général Chaos.

L'escouade de Neville sortit ses baguettes, les pointant droit vers l'ennemi tandis que leur tête scannait encore plus vite.
S'il y avait des traîtres Soleil, leur moment d'action approchait.

L'autre banc de poissons, l'armée du Dragon, faisait de même.

«~\emph{Maintenant~!} hurla la distante voix du général Dragon.

--- \emph{Maintenant}~! hurla le général Chaos.

--- \emph{Pour Soleil~!}~» hurlèrent tous les soldats des deux armées, et ils chargèrent vers le bas.

\later

«~\emph{Quoi~?}~» dit involontairement Minerva en regardant les écrans situés à côté du lac, un cri qui provoqua des échos en d'autres lieux~; tout Poudlard regardait cette bataille comme ils avaient regardé la première.

Le professeur Quirrell riait d'un rire acerbe.
«~Je vous ai prévenu, Directeur.
Il est impossible d'avoir des règles sans que M. Potter ne les exploite.~»

\later

Pendant de longues et précieuses secondes, alors que les quarante-sept soldats chargeaient les dix-sept de ses rangs, l'esprit de Hermione se vida.

Pourquoi…

Puis tout se mit en place.

Chaque fois qu'un soldat à l'origine Soleil se ferait abattre par quelqu'un s'étant déclaré Soleil, elle perdrait un point Quirrell.
Quand deux soldats Soleil seraient abattus par n'importe laquelle des armées, les \emph{deux} armées seraient deux points plus près de renverser la situation, c'était le même gain, mais \emph{partagé}.
Et si quiconque abattait un autre soldat mais \emph{pas} au nom de Soleil, ce gong ne \emph{serait pas} perdu dans la confusion…

Hermione était soudain très heureuse que Zabini n'ait pas suivi le plan évident consistant à semer le trouble parmi les deux armées tandis qu'elles attaquaient Soleil.

C'était quand même démoralisant, cette impression de voir ses chances diminuer, de voir son espoir ravi.

La plupart des soldats de Hermione semblaient encore confus, mais une expression d'horreur naissante se dessinait sur le visage de ceux qui commençaient à comprendre.

«~Tout va bien~», dit Susan Bones d'une voix forte.
Des têtes se tournèrent pour regarder le capitaine Soleil.
«~Notre tâche reste la même, en abattre autant que nous le pouvons.
Et souvenez-vous, Zabini a emporté tous les espions~!
Nous n'avons pas à être vigilants comme \emph{eux} le sont~!~»
La fille souriait avec défiance, provoquant des sourires chez de nombreux autres soldats et même chez Hermione.
«~Ça peut être comme en novembre.
Nous devons juste garder la tête bien haute, nous battre de notre mieux et nous faire confiance…~»

Daphné l'abattit.

\later

«~\emph{Du sang pour le dieu du sang}~!~»
glapit Neville du Chaos, bien que, comme il était sous l'eau, cela sonna plutôt comme “Bu ban bou le bieu bu ban~!”
Le capitaine Weasley pointa sa baguette vers Neville et fit feu.
Mais Neville nageait \emph{vers le bas}, vers lui, baguette dirigée vers lui, et cela voulait dire que le bouclier simple pouvait couvrir l'intégralité de sa silhouette~; si quelqu'un devait l'abattre maintenant, ça n'allait pas être Ron Soleil.

Un air sinistre et déterminé apparut sur le visage du capitaine Weasley, et il jaillit droit vers Neville, prononçant le mot \emph{Contego}, mais le bouclier n'était pas visible sous l'eau.

Les deux champions ennemis foncèrent l'un vers l'autre tels des flèches échappées d'arcs, chacune destinée à fendre l'autre en deux.
Ils s'étaient battus en duel maintes fois auparavant, mais cette fois paierait pour toutes les autres.

(Loin, sur les berges du lac, cent respirations étaient retenues.)

«~\emph{Arcs-en-ciels et licornes}~! rugit le capitaine Soleil.

--- \emph{Le bouc noir aux mille chevreaux~!}

--- \emph{Fais tes devoirs}~!~»

De plus en plus près, les deux champions chargeaient, aucun prêt à se détourner, car le premier à le faire présenterait un flanc vulnérable et serait abattu, mais si aucun d'eux ne perdait son sang-froid ils s'écraseraient l'un dans l'autre…

Tombant en piquée tandis que l'ennemi s'élevait à sa rencontre, la marteau s'abattant pour rencontrer l'enclume le long d'une trajectoire qu'aucun n'était prêt à quitter…

«~\emph{Attaque spéciale, vrille chaotique~!}~»

Neville vit l'air horrifié sur le visage du capitaine Weasley lorsque le sort de lévitation l'atteint.
Ils l'avaient testé avant le début de la bataille~; et comme Harry s'en était douté, \emph{Wingardium Leviosa} devenait un tout nouveau genre d'arme lorsque tout le monde nageait sous l'eau.

«~\emph{Sois maudit, Londubat~!} piailla Ron Weasley, \emph{ne peux-tu jamais te battre sans tes attaques spéciales idiotes…}~»

et lorsque le capitaine Soleil eut fini d'être incliné sur le côté, Neville tira sur sa jambe.

«~Je ne me bats pas en juste, dit Neville à la forme endormie, je me bats comme Harry Potter.~»

\emph{Granger~: 237 / Malfoy~: 217 / Potter~: 220}

\later

Ça faisait encore mal à chaque fois qu'il devait tirer sur Hermione.
Harry pouvait à peine supporter de voir l'air apaisé qui s'était formé sur son visage endormi, les bras dérivant sans but, les courbes de la lumière solaire se déplaçant sur son uniforme de camouflage, le nuage de ses cheveux châtain.

Et si Harry avait essayé de se dérober, de ne pas être celui qui l'abattrait… non seulement Drago aurait compris ce que cela voulait dire mais \emph{Hermione} aurait été offensée.

\emph{Elle n'est pas morte}, dit Harry à son cerveau, ses pieds battant pour le repousser loin d'elle, \emph{elle se repose.
IMBÉCILE}.

\emph{Tu es sûr~?} dit son cerveau.
\emph{Et si c'était une ex-Hermione~?
On peut y retourner et vérifier~?}

Harry jeta un bref regard en arrière.

\emph{Tu vois, elle va bien, il y a des bulles qui s'échappent de sa bouche.}

\emph{Ça aurait pu être son dernier souffle qui s'échappait.}

\emph{Oh, tais-toi.
Et puis pourquoi est-ce que tu es autant paranoïaque-protecteur~?}

\emph{Euh, la première vraie amitié de ta vie~?
Dis, tu te souviens de ce qui est arrivé à ton rocher de compagnie~?}

\emph{Vas-tu la FERMER sur ce bout de gravats sans valeur, ça n'était même pas vivant et encore moins sentient, c'est, genre, le trauma d'enfance le plus pathétique jamais…}

Les deux armées se séparèrent vivement l'une de l'autre, devenant à nouveau deux bancs de poissons.

Le général Granger avait perdu dix-sept points et prit trois chaotiques et deux dragons avec elle~; et un chaotique et deux dragons avaient été abattus pour trahison.
Elle avait perdu sept points en tout, Harry en avait perdu un, Drago en avait perdu deux~; cela donnait à Soleil un avantage de vingt points sur Drago et de dix-sept points sur Chaos.
Chaos pourrait facilement gagner s'ils exterminaient les vingt Dragons restant.
Le joker, bien sûr, était les sept soldats Soleil restants…

… si on pouvait les appeler ainsi.

Les deux bancs nagèrent l'un à côté de l'autre, mal à l'aise, les soldats de chaque armée attendant un ordre pour proclamer leur véritable allégeance et attaquer…

«~Tous ceux qui les ont reçus, dit Harry d'une voix forte, souvenez-vous des Ordres Spéciaux Un à Trois.
Et n'oubliez pas que c'est Merlin a dit pour le Trois.
Ne confirmez pas.~»

Les deux tiers de l'armée auxquels il pouvait faire confiance ne hochèrent pas la tête, et l'autre tiers eut simplement l'air confus.

\emph{Ordre Spécial Un}~: \emph{Ne pas se fatiguer à crier des phrases codées pendant cette bataille, ne pas dépenser le moindre effort sur des plans non approuvés par le commandeur~; se contenter de nager, de se protéger et de tirer.}

Hermione et Drago avaient tous deux combattus leurs soldats, essayés de les empêcher de comploter par eux-mêmes pendant tout décembre.
Harry avait encouragé les siens et avait soutenu leurs manigances pendant les deux dernières batailles… tout en leur disant qu'à un moment \emph{futur} il leur demanderait peut-être d'interrompre un complot ou deux, ce qu'ils avaient promptement accepté.
Et maintenant, lors de cette bataille critique, ils étaient heureux d'obéir.

Harry était certain que ni Hermione ni Drago n'auraient pu donner cet ordre avec succès.
C'était la différence entre des soldats qui vous voyaient comme un allié dans leur conspiration et des soldats qui vous voyaient comme un vieux schnock rabat-joie qui ne voulait pas qu'ils s'amusent.
L'imposition de l'ordre était à hauteur de l'intensification du chaos, et cela marchait aussi à l'envers…

«~Les voilà~!~»
cria quelqu'un, et il pointa du doigt.

Depuis les profondeurs du lac montèrent les oubliés, ceux qui avaient délaissé la dernière bataille, les sept soldats Soleil manquants, brillant de la vive aura des pleutres qui s'estompait maintenant qu'ils revenaient vers la bataille.

Les deux bancs de poissons ondulèrent, pointant leurs baguettes avec difficulté.

«~Ne tirez pas~!~»
cria Harry, et un cri similaire vint du général Malfoy.

Il y eut un moment de respiration suspendue.

Puis les sept soldats Soleil rejoignirent l'armée Dragon.

Il y eut une acclamation triomphante venant de Dragon.

Il y eut des cris de consternation venant d'un tiers de la légion du Chaos.

Certains des deux autres tiers sourirent, mais ils n'auraient pas dû.

Harry ne souriait pas.

\emph{Oh, ça va tellement rater…}

Mais Harry n'avait rien trouvé de mieux.

«~Les Ordres Spéciaux Deux et Trois s'appliquent toujours~! hurla Harry.
Battez-vous~!

--- \emph{Pour la légion du Chaos~!} rugirent vingt légionnaire Chaotiques.

--- \emph{Pour l'armée du Dragon~!}~» rugirent vingt guerriers Dragon et sept soldats Soleil.

Et les chaotiques piquèrent, et tous les traîtres se préparèrent à frapper.

Granger~: 237 / Malfoy~: 220 / Potter~: 226

\later

La tête de Drago changeait frénétiquement de direction, essayant de mesurer ce qui se passait~; étrangement, malgré ses forces supérieures, il avait \emph{perdu l'initiative}.
Quatre petites forces chaotiques étaient poursuivies par quatre forces dragons plus importantes, mais le fait que les forces de Drago soient celles qui essayaient de provoquer un engagement signifiait qu'il leur fallait \emph{suivre} la \emph{fuite} de Chaos, et d'une façon ou d'une autre cela créait des concentrations de forces chaotiques qui tiraient dans les flancs exposés de Dragon…

Ça se produisait \emph{à nouveau}~!

«~\emph{Prismatis~!}~» cria Drago en levant sa baguette, et apparut ce bouclier qu'on pouvait voir même à travers l'eau, un mur plat et étincelant de toutes les couleurs, assez large pour protéger Drago et les cinq autres dragons qui l'accompagnaient de la force chaotique qui venait de commencer à leur tirer dessus en profitant de leur approche latérale, et \emph{cela} laissa aux cinq autres dragons le loisir de rediriger \emph{leur} attention sur les forces chaotiques qu'ils avaient pourchassées…

Il y eut un moment de tension tandis que les sorts de sommeil s'écrasaient les uns après les autres dans le mur prismatique de Drago, et Drago priait Merlin qu'aucun de ces quatre chaotiques n'ait appris le sort de bris de bouclier…

Puis il y eut le son de cloche d'une victoire Dragon, et la force chaotique inversa ses têtes et ses pieds et commença à s'éloigner~; et Drago, ses mains tremblant à présent légèrement, relâcha le mur prismatique et abaissa sa baguette.

Combattre sous l'eau était plus épuisant que de combattre sur des balais.

«~\emph{Ne les poursuivez pas~!}~» cria Drago à l'intention de ses soldats, qui commençaient à suivre, puis~: «~\emph{Sonorus~!
EN FORMATION AUTOUR DE MOI~!}~»

Les forces dragons convergèrent vers Drago et les forces chaotiques pivotèrent et commencèrent à \emph{poursuivre} les dragons à cet instant même -- Drago jura à voix haute quand il entendit le son de cloche d'une victoire chaotique, quelqu'un avait mal orienté son bouclier simple -- puis les forces Dragon furent à bonne distance pour se protéger l'une l'autre et les chaotiques repartirent vers les remous lointains.

En dépit de leur supériorité numérique, les dragons avaient marqué trois points contre les chaotiques et les chaotiques en avaient marqué quatre en retour, et il avait entendu un espion Dragon se faire exécuter.
Soit Harry Potter avait eu beaucoup de bonnes idées très rapidement, ou pour une raison inimaginable il avait déjà passé beaucoup de temps à réfléchir à la meilleure façon de combattre sous l'eau.
Cela ne fonctionnait pas, et Drago avait besoin de réfléchir à nouveau.

Il semblait aussi que tout le monde avait du mal à viser en nageant, la bataille durerait peut-être assez longtemps pour que le temps maximal expire… la lointaine lune sous-marine était à moitié pleine maintenant, ça n'était pas bon… il lui fallait penser \emph{vite}…

«~Qu'y a-t-il~?~»
dit Padma Patil, et elle et sa force nagèrent en direction de Drago.

Padma était sa commandante en second~; elle était maline et puissante, et encore mieux, elle haïssait Granger et voyait Harry comme un rival, ce qui la rendait \emph{digne de confiance}.
Travailler avec Padma lui avait permis de se rendre compte de la véracité du vieil adage selon lequel Serdaigle était sœur de Serpentard~; Drago avait été surpris lorsque son père lui avait dit que c'était une maison acceptable pour sa future femme, mais il comprenait maintenant pourquoi.

«~Attendez que nous soyons tous là~», dit Drago.
À vrai dire, il avait besoin de reprendre son souffle.
C'était le problème quand on était le général \emph{et} le sorcier le plus puissant, il fallait sans cesse utiliser sa magie.

Zabini vint ensuite, commandant une force de deux Soleil et quatre Dragons, dont l'un était Gregory et qui gardait un œil sur Zabini.
Drago ne faisait pas confiance à Zabini.
Et ni Drago ni Zabini ne faisaient assez confiance aux Soleil pour leur donner la majorité d'une escouade~; ils étaient \emph{censés} être loyaux, soit directement à Drago, soit à Granger qui avait été trompée par la promesse que les dragons seraient trahis à la fin après que les deux forces eurent été diminuées, tout comme les chaotiques en lesquels Harry avait le plus confiance auraient dû être fourbement convaincus de ne pas tirer sur les soleils en échange de la promesse que ceux-ci tireraient de faux sorts de sommeil et passeraient dans le camp de Chaos plus tard~; mais il était possible que certains soleils \emph{soient} loyaux à Chaos et qu'ils ne tirent \emph{pas} de véritables sorts de sommeil et que ce soit pour cela que l'armée Dragon ne gagnait pas comme son avantage numérique aurait dû le faire advenir…

L'unité suivante était réduite, trois soldats tenant leur baguette braquée sur deux autres soldats qui nageaient les mains vides.

Drago grinça des dents.
Encore des problèmes de trahison.
Il fallait qu'il parle au professeur Quirrell d'un moyen permettant au moins de \emph{punir} les traîtres, des conditions comme celles-ci n'étaient pas \emph{réalistes}, dans la vraie vie on torturait les traîtres à mort.

«~Général Malfoy~!~»
cria le commandant de l'unité à problème tout en s'élevant, un garçon Serdaigle prénommé Terry.
«~Nous ne savons pas quoi faire, Cesi a abattu Bogdan, mais Cesi dit que Kellah lui a dit que Bogdan a abattu Specter…

--- Je ne l'ai \emph{pas} abattu~! dit Kellah.

--- \emph{Si}~! hurla Cesi.
Général Malfoy, c'est \emph{elle} l'espion, j'aurais dû m'en ren-

--- \emph{Somnium}~», dit Drago.

Il y eut le triple son de cloche d'une perte d'un point pour Dragon, et le corps mou de Kellah commença alors à dériver loin d'eux.

À ce stade, Drago \emph{avait} entendu le mot “récursion”, et il savait reconnaître un complot de Harry Potter quand il en avait un sous les yeux.

(Malheureusement, Drago n'avait \emph{pas} entendu parler des maladies auto-immunes, et la pensée ne lui vint \emph{pas} naturellement qu'un virus malin commencerait son attaque en créant les symptômes d'une maladie auto-immune pour pousser le corps à ne plus faire confiance à son système immunitaire…)

«~\emph{Ordre général}~!~»
dit Drago, élevant sa voix.
«~Personne ne peut abattre d'espions à part moi, Gregory, Padma et Terry.
Si quelqu'un voit quelque chose de suspect, qu'il vienne \emph{nous} voir.~»

Et alors…

Il y eut la cloche de Soleil gagnant deux points.

«~\emph{Quoi}~?~»
dirent Drago et Zabini presque au même instant~; leurs têtes pivotèrent.
Personne ne semblait avoir été touché et tous les soldats de Soleil étaient présents (à part Parvati qui s'était fait abattre par un traître encore inconnu dans l'équipe de Padma~; et bien sûr Padma avait de nouveau tiré sur Parvati au cas où elle aurait fait semblant, donc ce n'était pas elle…)

«~Un Soleil traître chez Chaos~? dit Zabini d'un ton perplexe.
Mais tous ceux que je connaissais étaient censés frapper pendant l'attaque de Chaos sur Soleil…

--- Non~! dit Padma du ton de la prise de conscience soudaine.
C'était \emph{Chaos} exécutant un espion~!

--- \emph{Quoi~!} dit Zabini.
Mais pourquoi…~»

Et Drago comprit.
\emph{Bon sang~!} «~Parce que Potter croit qu'il est tranquille pour ce qui est de sa marge vis-à-vis de Soleil, mais pas vis-à-vis de \emph{nous}~!
Alors il ne veut pas perdre un seul point en exécutant un traître~!
\emph{Ordre général}~!
Si vous avez un traître dans vos rangs, prêtez d'abord allégeance à Soleil~!
Et n'oubliez pas de redevenir Dragon après…~»

\later

Granger~: 253 / Malfoy~: 252 / Potter~: 252

Le corps de Londubat dérivait chaotiquement dans l'eau, bras et jambes désordonnés.
Après que Drago fut enfin parvenu à le toucher ils lui avaient tous tiré dessus \emph{à nouveau}, juste pour être sûr.

Non loin se trouvait Harry Potter, maintenant protégé par une sphère prismatique, les regardant tous d'un air sinistre tandis que le dernier éclat du croissant de lune diminuait lentement, quelque part, très loin.
Si Londubat était parvenu à abattre un soldat de plus (Drago savait que Harry songeait), si les deux chaotiques étaient parvenus à tenir juste un peu plus longtemps, ils auraient pu \emph{gagner}…

Après que Drago eut rassemblé ses forces et ait frappé de nouveau, la bataille qui avait suivi et l'exécution des espions au nom de Soleil avaient laissé Soleil exactement un point devant Dragon et Chaos.
Une fois que Harry avait commencé à le faire, Drago n'avait pas eu d'autre choix que de l'imiter.

Mais à présent, le général Chaos était surpassé à trois contre un, les survivants de l'armée Dragon et le dernier traître Soleil encore debout~: Drago, Padma et Zabini.

Et Drago, qui n'était pas un idiot, avait ordonné à Padma de prendre la baguette de Zabini après que Londubat eut abattu Gregory et soit ensuite tombé par la main de Drago.
Le garçon lui avait jeté un regard insulté, lui avait dit qu'il lui en devrait une, et lui avait donné sa baguette.

Cela laissait Drago et Padma pour descendre le général Chaos.

«~J'imagine que tu ne voudrais pas te rendre~?~»
dit Drago, souriant du sourire le plus démoniaque qu'il ait jamais adressé à Harry Potter.

«~Le sommeil plutôt que de se rendre~!~»
tonna le général Chaos.

«~Juste pour que tu saches, dit Drago, Zabini n'a pas \emph{vraiment} une grande sœur que tu pourrais sauver des griffes de brutes de Gryffondor.
Mais Zabini \emph{a} une mère qui ne voit pas d'un bon œil les nés-Moldus tels que Granger, et je lui ai écrit quelques mots, et j'ai offert quelques faveurs à Zabini -- rien qui implique mon père, seulement des choses que \emph{je} peux faire à l'école.
Et au fait, la mère de Zabini ne voit pas le Survivant d'un bon œil non plus.
Juste au cas où tu pensais encore que Zabini était vraiment de ton côté.~»

Le visage de Harry devint encore plus sinistre.

Drago leva sa baguette et commença à respirer en rythme, accumulant de la force pour le sort de bris de bouclier.
La sphère prismatique de Granger était maintenant presque aussi forte que celle de Drago, et celle de Harry n'était pas beaucoup plus faible, où ces deux-là trouvaient-ils le \emph{temps}~?

«~\emph{Lagann~!}~» dit Drago, y mettant tout ce qu'il avait encore en lui, et la spirale verte jaillit et le bouclier de Harry vola en éclats, et presque au même moment…

«~\emph{Somnium}~!~»
dit Padma.

\later

Granger~: 253 / Malfoy~: 252 / Potter~: 254

Harry expira longuement sous l'effet du soulagement, et pas seulement parce qu'il n'avait plus besoin de maintenir sa sphère prismatique.
Lorsqu'il abaissa sa baguette, sa main tremblait.

«~Tu sais, dit Harry, j'ai été plutôt inquiet l'espace d'un instant.~»

\emph{Ordre Spécial Deux~: Si un traître Soleil n'a pas l'air de vraiment vous tirer dessus, faire occasionnellement semblant d'être touché.
Préférer prendre des dragons pour cible plutôt que des soleils mais ne pas hésiter à abattre des soleils s'il est impossible d'abattre des dragons.}

\emph{Ordre Spécial Trois~: Merlin a dit ne pas abattre ni Blaise Zabini ni aucune des jumelles Patil.}

Avec un large sourire, Parvati Patil arracha la partie métamorphosée du blason de son uniforme et le laissa flotter dans l'eau.

«~Gryffondor pour Chaos~», dit-elle, et elle rendit sa baguette à Zabini.

«~Merci \emph{beaucoup}~», dit Harry, et il s'inclina prestement en direction de la fille Gryffondor.
«~Et merci à \emph{toi} aussi~», s'inclinant vers Zabini.
«~Tu sais, quand tu es venu me voir avec ce plan, je me suis demandé si tu étais génial ou fou, et j'ai décidé que tu étais les deux.
Au fait~», dit Harry, se tournant maintenant comme pour s'adresser au corps de Drago, «~Zabini \emph{a} un cousin…

--- \emph{Somnium}~», dit la voix de Zabini.

\later

Granger~: 255 / Malfoy~: 252 / Potter~: 254

Et le corps de Harry Potter dériva, l'expression de choc et d'horreur se relaxant rapidement sous l'effet du sommeil.

«~Réflexion faite, dit Parvati d'un ton joyeux, disons plutôt Gryffondor pour Soleil.~»

Elle commença à rire, plus euphorique qu'elle ne l'avait jamais été de sa vie, elle avait \emph{enfin} eu l'occasion d'assassiner et de remplacer sa sœur jumelle et elle avait voulu le faire depuis \emph{toujours}, et ça avait été \emph{parfait}, tout avait été \emph{parfait}…

… et sa baguette se retourna à la vitesse de l'éclair juste quand la baguette de Zabini se tourna vers elle.

«~Attends~! dit Zabini.
Ne tire pas, ne résiste pas.
C'est un ordre.

--- \emph{Quoi}~? dit Parvati.

--- Désolé, dit Zabini d'un air désolé pas-vraiment-sincère, mais je ne peux pas être \emph{entièrement} certain que tu es pour Soleil.
Donc je t'ordonne de me laisser t'abattre.

--- \emph{Attends~!} dit Parvati.
On est seulement devant Chaos par un point~!
Si tu m'abats maintenant…

--- Je t'abats au nom de Dragon, \emph{évidemment}, dit Zabini d'un ton à présent légèrement supérieur.
Ce n'est pas parce que nous sommes parvenus à \emph{les} pousser à le faire que ça ne marchera pas pour nous.~»

Parvati le fixa, ses yeux s'amincissant.

«~Le général Malfoy a dit que ta mère n'aime pas Hermione.

--- J'imagine~», dit Zabini, toujours avec ce rictus supérieur.
«~Mais certains d'entre nous sont plus disposés à agacer un parent que ne l'est Drago Malfoy.

--- Et Harry Potter a dit que tu as un cousin…

--- Nan~», dit Zabini.

Parvati le fixa, essayant de réfléchir, mais elle n'était vraiment pas douée en complot~; Zabini avait dit que le plan était de secrètement maintenir les scores de Chaos et de Dragon aussi proches l'un de l'autre que possible afin qu'ils utilisent le nom de Soleil pour exécuter les traîtres au lieu de perdre un seul point, et ça avait \emph{marché}… mais… elle avait l'impression qu'elle ratait quelque chose, elle n'était pas Serpentard…

«~Pourquoi est-ce que \emph{je} ne t'abats pas \emph{toi} au nom de Dragon~? dit Parvati.

--- Parce que je suis ton supérieur~», dit Zabini.

Parvati avait un mauvais pressentiment.

Elle le fixa pendant un long moment.

Et puis…

«~\emph{Somni}-~» commença-t-elle à dire, puis elle se rendit compte qu'elle n'avait pas dit \emph{pour Dragon}, et elle se coupa avec frénésie…

\later

Granger~: 255 / Malfoy~: 254 / Potter~: 254

«~Bonjour tout le monde~», dit le visage de Blaise Zabini depuis les écrans, affichant air plutôt amusé, «~j'ai l'impression qu'il ne reste que moi.~»

Sur les berges du lac, tout le monde retenait son souffle.

Soleil devançait Dragon et Chaos d'un point exactement.

Blaise Zabini pouvait s'abattre au nom de Dragon ou Chaos ou juste laisser les choses en l'état.

Une série de carillons indiquait que la dernière minute de la bataille était en train de s'écouler.

Et le Serpentard eut un étrange sourire tordu, et il joua négligemment avec sa baguette, le bois sombre à peine visible dans l'eau trouble.

«~Vous savez~», dit la voix de Blaise Zabini du ton de quelqu'un qui avait préparé ces mots depuis longtemps, «~c'est juste un jeu, en fait.
Et les jeux sont censés être \emph{amusants}.
Alors pourquoi est-ce que je ne ferais pas ce qui me chante~?~»
%  LocalWords:  mimsy borogroves mome raths outgrabe Kneazle Yesss Eeeeek
%  LocalWords:  burbly libsten vebwy carebfully yessirs Aw Blubbled glub
%  LocalWords:  blubbled Cesi Bogdan Kellah Specter rea Kellah’s Somni
