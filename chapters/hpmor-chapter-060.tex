\partchapter{L'Expérience de Prison de Stanford}{X}

\lettrinepara[ante=«~]{D}{ebout}.~»

\hplettrineextrapara
Les yeux de Harry s'ouvrirent grand alors qu'il s'éveillait dans un halètement étranglé accompagné d'une convulsion de son corps allongé.
Il ne pouvait se souvenir d'aucun rêve~; peut-être que son cerveau avait été trop épuisé pour rêver, car il lui semblait n'avoir fermé les yeux qu'un instant avant d'avoir entendu ce mot.

«~Vous devez vous éveiller, dit la voix de Quirinus Quirrell.
Je vous ai donné autant de temps que possible mais il serait sage de garder au moins une utilisation du retourneur de temps en réserve.
Nous allons bientôt devoir revenir quatre heures en arrière, chez \emph{Marie}, et donner sous tous rapports l'impression que nous n'avons rien fait d'intéressant aujourd'hui.
Je souhaitais vous parler avant cela.~»

Harry s'assit lentement au milieu des ténèbres.
Son corps lui faisait mal, et pas seulement aux endroits où il avait reposé sur du béton.
Des images se chevauchèrent les unes les autres dans sa mémoire, tout ce dont son cerveau inconscient, trop fatigué pour cela, n'avait pu se décharger sous la forme d'un bon cauchemar.

Douze terribles néants lévitant le long d'un couloir, corrodant le métal autour d'eux, les lumières tamisées, la température chutant à mesure que les vides essaient de drainer toute vie hors du monde…

Peau d'une blancheur de craie, étirée à la surface des os qui demeurent une fois la graisse et les muscles disparus…

Une porte de métal…

La voix d'une femme…

\emph{Non, je ne voulais pas, ne meurs pas s'il te plaît…}

\emph{Je ne sais plus comment mes enfants s'appellent…}

\emph{Ne pars pas, ne l'emmène pas, non non non…}

«~Quel était cet endroit~?~»
dit Harry d'une voix rauque, d'une voix qui sortait de sa gorge comme de l'eau qu'on aurait forcée à traverser un tuyau trop étroit, d'une voix qui dans ces ténèbres semblait aussi brisée que l'avait été celle de Bellatrix Black.
«~\emph{Quel était cet endroit~?
Ce n'était pas une prison, c'était L'ENFER~!}

--- L'enfer~? dit la voix calme du professeur de Défense.
Vous voulez dire le fantasme de punition Chrétien~?
J'imagine qu'il existe une ressemblance.

--- Comment…~»
la voix de Harry était bloquée comme si quelque chose d'énorme s'était logé dans sa gorge.
«~Comment -- comment ont-ils pu…~»
Des \emph{gens} ont construit cet endroit, quelqu'un a \emph{fait} Azkaban, ils l'ont fait \emph{sciemment}, ils l'ont fait \emph{délibérément}, cette femme, elle avait des enfants, des enfants dont elle ne pouvait plus se souvenir, un juge a \emph{décidé} que cela lui soit infligé, quelqu'un a dû la \emph{traîner} jusqu'à cette cellule et la fermer à clé alors qu'elle hurlait, quelqu'un l'a nourrie tous les jours et est reparti \emph{sans la laisser sortir…}

«~\shout{Comment des gens peuvent-ils faire ça~?}

--- Pourquoi ne le feraient-ils pas~?~»
dit le professeur de Défense.
Une pâle lumière bleue éclaira alors l'entrepôt, révélant un haut toit caverneux et un sol poussiéreux tous deux faits de béton~; ainsi que le professeur Quirrell, assis à une certaine distance de Harry, le dos appuyé contre un mur peint~; la pâle lumière bleue transformait les murs en parois de glaciers, la poussière au sol en mouchetures de neige, et l'homme lui-même était devenu une sculpture de glace enveloppée de ténèbres là où sa robe noire le recouvrait.
«~En quoi les prisonniers d'Azkaban leur sont-ils utiles~?~»

La bouche de Harry s'ouvrit et émit un croassement.
Aucun mot ne sortit.

Un léger sourire fit trembler les lèvres du professeur de Défense.
«~Vous savez, M. Potter, si Celui-Dont-On-Ne-Doit-Pas-Prononcer-Le-Nom en était venu à régner sur l'Angleterre Magique et avait construit un lieu tel qu'Azkaban, il l'aurait fait pour le plaisir de voir ses ennemis souffrir.
Et si au lieu de cela il avait commencé à trouver leur souffrance déplaisante, eh bien, il aurait ordonné qu'Azkaban soit démolie le lendemain même.
Quant à ceux qui ont réellement créé Azkaban, ceux qui ne l'ont pas démolie tout en prêchant de nobles sermons et en s'imaginant ne \emph{pas} être méchants… eh bien M. Potter, je pense que si j'avais à choisir entre prendre le thé avec eux ou avec Vous-Savez-Qui, je me découvrirais être moins heurté dans mes sensibilités par le Seigneur des Ténèbres.

--- Je ne comprends pas~», dit Harry, sa voix tremblait, il avait lu l'expérience classique au sujet de la psychologie des prisons, les étudiants ordinaires qui étaient devenus sadiques à l'instant où ils s'étaient vus assigner le rôle de garde de prison~; seulement Harry comprit à cet instant que l'expérience ne s'était pas penchée sur la bonne question, sur la question la plus importante, ils n'avaient pas examiné les personnes-clés, pas les gardiens de prison mais \emph{le reste de la population}, «~je ne comprends vraiment pas, professeur Quirrell, comment des gens peuvent-ils simplement se tenir là et laisser ces choses se produire, \emph{pourquoi} l'Angleterre magique \emph{fait-elle ça}…~»
la voix de Harry se tut.

Les yeux du professeur de Défense avaient la même teinte qu'à l'accoutumée car cette pâle lumière bleue était de la même couleur que ses iris, que ces fragments de glace à jamais saisis.
«~M. Potter, bienvenue~: ceci est votre première rencontre avec les réalités de la politique.
Que ces misérables créatures à Azkaban ont-elles à offrir à une faction ou à une autre~?
Qui gagnerait à les aider~?
En se mettant officiellement de leur côté, un politicien s'associerait à des criminels, à de la faiblesse, à des choses dégoûtantes auxquelles les gens préfèrent ne pas penser.
En revanche, ce politicien pourrait démontrer sa force et sa cruauté en demandant des peines plus longues~; après tout, démontrer sa force requiert une victime à écraser.
Et la populace applaudit car il est dans ses instincts de soutenir le vainqueur.~»
Un rire froidement amusé.
«~Vous voyez M. Potter, personne ne croit tout à fait qu'\emph{il} ira à Azkaban, et les gens n'y voient donc aucune conséquence néfaste pour eux.
Quant à ce qu'ils infligent aux autres… j'imagine que l'on vous a un jour dit qu'ils se préoccupent de ce genre de choses~?
C'est un mensonge, M. Potter, la plupart des gens ne s'en préoccupent pas le moins du monde, et si vous n'aviez pas mené une enfance en grande partie cloîtrée, vous l'auriez remarqué il y a longtemps.
Consolez-vous ainsi~: ceux aujourd'hui prisonniers à Azkaban ont voté pour ces ministres de la Magie qui s'étaient engagés à rapprocher leur cellule des Détraqueurs.
J'admets, M. Potter, avoir peu d'espoir pour la démocratie en tant que forme de gouvernement efficace, mais j'admire la poésie qu'elle dégage lorsqu'elle rend ses victimes complices de leur propre destruction.~»

Récemment redevenu cohérent, le moi de Harry menaçait de se fracasser à nouveau~; les mots tombaient comme des coups de marteau sur sa conscience, ils le ramenaient pas à pas au-dessus du précipice où se tapissait un vaste abysse~; il essaya de trouver quelque chose qui le sauverait, une répartie maline qui réfuterait les mots, mais elle ne vint pas.

Le regard du professeur Quirrell sur Harry reflétait plus de curiosité que d'autorité.
«~M. Potter, il est très facile de comprendre comment Azkaban a été construite et comment elle continue d'exister.
Les hommes se soucient de leurs perspectives de perte et de gain propres~; leur cruauté et leur négligence ne connaît aucune limite tant qu'ils ne s'attendent pas à ce que cela ait des répercussions sur eux.
Tous les autres sorciers de ce pays ne sont au fond pas différents de celui qui souhaitait régner sur eux, Vous-Savez-Qui~; il ne leur manque que son pouvoir et sa… franchise.~»

Les mains du garçon étaient serrées si fort que ses ongles pénétraient sa paume, mais il aurait été impossible de voir si ses doigts étaient blancs, ou si son visage était pâle, car la faible lumière bleue transformait tout en ombre et en glace.
«~Vous avez un jour offert de me soutenir si mon ambition était de devenir le prochain Seigneur des Ténèbres.
En est-ce la raison, professeur~?~»

Le professeur de Défense inclina sa tête, un fin sourire sur les lèvres.
«~Apprenez tout ce que j'ai à vous apprendre, M. Potter, et vous finirez par diriger ce pays.
Vous pourrez alors abattre la prison que la démocratie a créée si vous trouvez alors qu'Azkaban vous rebute encore.
Que cela vous plaise ou non, M. Potter, vous savez aujourd'hui que votre volonté entre en conflit avec celle de la population de ce pays et que lorsque cela se produit, vous ne vous inclinez pas, vous ne vous soumettez pas à leur décision.
Et qu'ils le sachent ou pas, que vous le reconnaissiez ou pas, vous êtes par conséquent leur prochain Seigneur des Ténèbres.~»

Sous l'inébranlable lumière monochromatique le garçon et le professeur de Défense ressemblaient tous deux à des statues de glace immobiles, les iris de leurs yeux réduits, sous cette lumière, à des couleurs en apparence très semblables.

Harry regarda droit dans ces yeux pâles.
Toutes les questions longtemps refoulées, celles dont il s'était dit qu'il les laissait en suspens jusqu'aux calendes grecques.
Harry savait maintenant que cela avait été un mensonge, qu'il s'était aveuglé, qu'il était resté silencieux par peur de ce qu'il pourrait entendre.
Et à présent tout parvenait à ses lèvres en même temps.

«~Lors de notre premier jour de cours, vous avez essayé de convaincre mes camarades que j'étais un tueur.

--- Vous l'êtes.~»
Avec amusement.
«~Mais si votre question est de savoir pourquoi je le leur ai \emph{dit}, M. Potter, la réponse est que vous découvrirez que l'ambiguïté est un grand allié sur la route vers le pouvoir.
Donnez des signes de Serpentard un jour, contredisez-les le lendemain avec des signes de Gryffondor~; et les Serpentard pourront croire ce qu'ils souhaitent croire tandis que les Gryffondor se persuaderont de vous soutenir.
Tant qu'il y a incertitude, les gens peuvent croire ce qui semble être à leur avantage.
Et tant que vous semblez fort, tant que vous semblez gagner, leurs instincts leur disent qu'il est à leur avantage d'être de votre côté.
Si vous marchez toujours dans l'ombre, la lumière et les ténèbres vous suivront toutes deux.

--- Et, dit le garçon d'une voix égale, que voulez-\emph{vous} tirer de tout cela~?~»

Le professeur Quirrell s'était incliné en arrière, dos contre le mur, plaçant son visage dans l'ombre, ses yeux passant d'une glace pâle à des crevasses ténébreuses semblables à celles de sa forme animale.

«~Je souhaite que l'Angleterre devienne forte sous l'égide d'un meneur fort~; cela \emph{est} mon souhait.
Quant à mes raisons,~» dit le professeur Quirrell avec un sourire sans joie, «~je pense qu'elles resteront miennes.

--- Cette sensation funeste que je ressens autour de vous.~»
Les mots devenaient de plus en plus difficiles à prononcer à mesure que le propos dansait de plus en plus près d'une chose terrible et interdite.
«~Vous avez toujours su ce qu'elle signifie.

--- J'ai eu plusieurs idées~», dit le professeur Quirrell, et son visage était inscrutable.
«~Et je ne dirais pas encore tout ce que j'ai deviné.
Mais je vous dirai cela~: c'est \emph{votre} perte qui flamboie lorsque nous sommes proches, pas la mienne.~»

Pour une fois, le cerveau de Harry parvint à marquer cette affirmation comme étant douteuse, comme un mensonge possible, au lieu de croire tout ce qu'il entendait.

«~Pourquoi vous transformez-vous parfois en zombie~?

--- Raisons personnelles, dit le professeur Quirrell sans le moindre humour dans la voix.

--- Quel était votre véritable but en sauvant Bellatrix~?~»

Il y eut un bref silence pendant lequel Harry essaya de toutes ses forces de contrôler sa respiration et de maintenir le rythme de celle-ci.

Enfin le professeur de Défense haussa les épaules comme si cela n'avait aucune importance.
«~Je vous l'ai pratiquement expliqué en détail, M. Potter.
Je vous ai dit tout ce que vous auriez eu besoin de savoir pour déduire la réponse si vous aviez été assez mûr pour considérer la question évidente.
Bellatrix Black était la plus puissante des serviteurs du Seigneur des Ténèbres, sa loyauté la plus assurée de toutes~; elle était la personne avec le plus de chances de recevoir en confidence une partie de celles des connaissances de Serpentard qui auraient dû être vôtres.~»

La colère monta lentement en Harry, la rage lente, quelque chose de terrible qui commençait à bouillir dans son sang, dans quelques instants il dirait quelque chose qu'il ne devrait vraiment pas dire alors qu'ils étaient seuls tous les deux dans un entrepôt désert…

«~Mais elle \emph{était} innocente~», dit le professeur de Défense.
Il ne souriait pas.
«~Et le degré auquel tous ses choix lui furent enlevés, tant et si bien qu'elle n'eut jamais la chance de souffrir de ses \emph{propres} erreurs… cela m'a semblé \emph{excessif}, M. Potter.
Si elle ne vous révèle rien d'utile…~»
Le professeur de Défense eut un autre petit haussement d'épaules.
«~Je ne considérerai pas que cette journée de travail aura été gâchée.

--- Que c'est altruiste de votre part, dit Harry d'une voix froide.
Donc si tous les sorciers sont comme Vous-Savez-Qui en leur for intérieur, seriez-vous donc une exception~?~»

Les yeux du professeur de Défense étaient toujours dans l'ombre, des abîmes noirs dont le regard ne pouvait être croisé.
«~Disons que c'est un caprice, M. Potter.
Il m'amuse parfois de jouer le rôle du héros.
Qui sait si Vous-Savez-Qui ne dirait pas de même.~»

Harry ouvrit la bouche une dernière fois…

Et découvrit qu'il ne pouvait pas le dire, il ne pouvait pas poser la dernière question, la dernière et la plus importante, il ne pouvait pas faire sortir les mots.
Même si un tel refus était interdit à un rationaliste, en dépit de toutes les fois où il avait récité la litanie de Tarski et la litanie de Gendlin, en dépit de celles où il avait fait le serment que ce qui pouvait être détruit par la vérité devait l'être, il ne put alors se pousser à prononcer sa dernière question à voix haute.
Même s'il savait qu'il ne pensait pas correctement, même s'il savait qu'il était censé valoir mieux que cela, il ne parvint pas à le dire.

«~Maintenant c'est à mon tour de vous interroger.~»
Le dos du professeur Quirrell se raidit contre le mur glacé de béton peint.
«~Je me demandais, M. Potter, si vous aviez le moindre commentaire sur le fait que vous avez failli me tuer et faire échouer notre entreprise commune.
Je crois comprendre que dans de tels cas, une excuse constitue un signe de respect.
Mais vous ne m'en avez pas offert.
Est-ce seulement que vous n'avez pas encore trouvé le temps de le faire, M. Potter~?~»

Le ton était calme, le tranchant si discret et fin qu'il vous aurait traversé de part en part avant que vous ayez pu vous rendre compte que l'on était en train de vous tuer.

Et Harry se contenta de regarder le professeur de Défense avec ses yeux froids, ses yeux qui jamais plus ne cilleraient devant quoi que ce soit~; plus même devant la mort.
Il n'était plus à Azkaban, il n'avait plus peur de la partie de lui-même qui ignorait la peur~; et la gemme qu'il était avait pivoté pour faire face à la tension, elle était passée en douceur d'une facette à l'autre, de la lumière aux ténèbres, de la chaleur au froid.

\emph{Un stratagème voulu, destiné à me faire me sentir coupable, à me pousser à me soumettre~?}

\emph{Une émotion réelle~?}

«~Je vois, dit le professeur Quirrell.
J'imagine que cela répond…

--- Non, dit le garçon d'une voix calme et froide, vous ne réussirez pas à orienter la conversation aussi facilement, professeur.
J'ai fait des efforts considérables dans le but de vous protéger et de vous faire sortir d'Azkaban en sécurité \emph{après} que j'ai cru avoir découvert que vous aviez tenté de tuer un policier.
Y compris faire face à douze Détraqueurs sans Patronus.
Je me demande si vous m'auriez remercié si je m'étais excusé lorsque vous me l'avez demandé.
Ou ai-je raison de penser que c'était ma soumission que vous exigiez ici, et non pas seulement mon respect~?~»

Il y eut un moment de silence puis la voix du professeur Quirrell vint en réponse, ouvertement glaciale, dangereuse, débarrassée de toute volonté de masquer ce danger.
«~Il semble que vous ne parveniez toujours pas à vous laisser perdre, M. Potter.~»

Les ténèbres du regard imperturbable de Harry englobaient le professeur de Défense et le réduisaient à une simple créature mortelle.
«~Oh, et hésitez-\emph{vous} maintenant à faire semblant de perdre contre moi, à prétendre vous soumettre à ma colère afin de préserver vos plans~?
L'idée de fausses excuses vous a-t-elle seulement \emph{traversé l'esprit}~?
Moi non plus, professeur Quirrell.~»

Le professeur de Défense eut un rire bas et sans humour, un rire plus vide que le néant entre les étoiles, aussi dangereuse qu'un espace empli de radiations dures.

«~Non M. Potter, vous n'avez pas du tout appris votre leçon.

--- J'ai songé à perdre de nombreuses fois à Azkaban, dit le garçon d'une voix maîtrisée.
J'ai songé que je devrais simplement abandonner et me rendre aux Aurors.
Il aurait été sensé de perdre.
J'ai entendu votre voix me le dire dans mon esprit, et je l'aurais \emph{fait} si j'avais été seul.
Mais je ne suis pas parvenu à me permettre de \emph{vous} perdre.~»

Ils furent alors silencieux pendant un moment~; comme si le professeur de Défense ne savait pas tout à fait quoi répondre à cela.

«~Je suis curieux, dit enfin ce dernier.
Pour quoi exactement pensez-vous que je devrais m'excuser~?
Je vous ai donné des instructions précises en cas de combat.
Vous deviez rester au sol, hors de mon chemin, et ne pratiquer aucune magie.
Vous avez violé ces instructions et avez fait échouer la mission.

--- Je n'ai pris aucune décision, dit le garçon d'une voix calme, il n'y a eu aucun choix, seulement le souhait qu'un Auror ne meure pas, et mon Patronus était là.
Pour que ce souhait n'ait jamais eu lieu, vous auriez dû me prévenir de la possibilité que vous bluffiez en utilisant un sort de Mort.
Par défaut, je pars du principe que si vous pointez votre baguette vers quelqu'un et que vous dites Avada Kedavra, c'est parce que vous souhaitez voir cette personne mourir.
Cela ne devrait-il pas être la première règle de sécurité des Sortilèges Impardonnables~?

--- Les règles sont bonnes pour les duels~», dit le professeur de Défense.
Une partie du froid était revenu dans sa voix.
«~Et l'art du duel est un sport, pas une branche de la magie de combat.
Dans un véritable combat, un sortilège qui ne peut être arrêté et \emph{doit} être évité constitue une tactique indispensable.
Je pensais que cela vous serait apparu comme évident, mais il semble que j'ai mal jugé de votre intellect.

--- Il me semble aussi imprudent~», dit le garçon, continuant comme si l'autre n'avait pas parlé, «~de ne pas \emph{me dire} que tout sortilège que je lancerai sur vous risquerai de nous tuer tous deux.
Et s'il y avait eu un incident, et si j'avais dû essayer un Innerver ou un sortilège de lévitation~?
Cette ignorance, que vous avez permise pour un but que je ne peux deviner, a aussi joué un rôle dans cette catastrophe.~»

Il y eut un autre silence.
Les yeux du professeur s'étaient plissés et il avait l'air légèrement perplexe, comme s'il venait de rencontrer une situation totalement inconnue~; mais il ne parla pourtant pas.

«~Eh bien~», dit le garçon.
Ses yeux ne s'étaient pas détournés de ceux du professeur de Défense.
«~Je regrette certainement de vous avoir fait du mal, professeur.
Mais je ne pense pas que cette situation mérite que je me soumette.
Je n'ai jamais vraiment bien compris le concept d'excuse, et encore moins lorsqu'il s'agit d'une situation comme celle-ci~; vous donner mes regrets mais pas ma soumission, est-ce comme de dire que je suis désolé~?~»

Encore ce rire froid, froid et plus sombre que le vide entre les étoiles.

«~Je ne pourrais le savoir, dit le professeur de Défense, je n'ai moi non plus jamais compris le concept d'excuse.
Il semble que ce stratagème serait inutile entre nous, puisque nous le savons tous deux être un mensonge.
Alors n'en parlons plus.
Les dettes que nous devons l'un à l'autre finiront un jour par être payées.~»

Un silence, pendant un moment.

«~Au fait, dit le garçon.
Hermione Granger n'aurait jamais construit Azkaban, quelles que soient les personnes destinées à y être placées.
Et elle mourrait plutôt que de faire du mal à un innocent.
Je mentionne ça parce que vous avez dit plus tôt qu'au fond, tous les sorciers étaient comme Vous-Savez-Qui, et c'est une affirmation factuellement erronée.
M'en serais rendu compte plus tôt si je n'avais pas été,~» le garçon eut un bref sourire lugubre, «~stressé~».

Les yeux du professeur de Défense étaient mi-clos, son visage distant.

«~L'intérieur des gens ne ressemble pas toujours à leur apparence, M. Potter.
Peut-être souhaite-t-elle simplement que les autres la voie comme une bonne fille.
Elle ne peut pas utiliser le Patronus…

--- Hah~», dit le garçon~; son sourire semblait maintenant plus réel, plus chaud.
«~Elle a des difficultés exactement pour les mêmes raisons que moi.
Il y a assez de lumière en elle pour détruire des Détraqueurs, j'en suis sûr.
Elle ne pourrait pas \emph{s'empêcher} de les détruire, même au prix de sa vie…~»
le garçon laissa sa phrase en suspens puis sa voix reprit~: «~\emph{je} ne suis pas quelqu'un d'assez bien pour le faire, peut-être pas~; mais ces gens existent, et elle en fait partie.~»

Sèchement.
«~Elle est jeune, et faire montre de gentillesse lui coûte peu.~»

Un silence répondit à cela.
Puis le garçon dit~: «~Professeur, je dois vous poser une question~: quand vous voyez quelque chose qui est sombre et lugubre, cela ne vous vient-il jamais à l'esprit d'essayer de l'\emph{améliorer} d'une façon ou d'une autre~?
Par exemple~: oui, quelque chose tourne horriblement mal dans l'esprit des gens et les voilà qui pensent que c'est génial de torturer des criminels, mais cela ne veut pas dire qu'au fond ils sont réellement maléfiques~; et peut-être qu'en leur prodiguant les bons enseignements, en leur montrant que ce qu'ils font est mal, on pourrait changer…~»

Le professeur Quirrell rit alors, sans le néant de tout à l'heure.

«~Ah, M. Potter, j'oublie parfois à quel point vous être jeune.
Vous auriez plus de facilité à changer la couleur du ciel.~»
Un autre gloussement, plus froid.
«~Et la raison pour laquelle il vous est facile de pardonner de tels idiots et d'avoir une haute opinion d'eux, M. Potter, est que vous n'avez pas été douloureusement atteint.
Vous aurez une opinion moins favorable des idiots ordinaires après que leur folie vous aura coûté quelque chose de cher.
Qui sait, peut-être comme une centaine de vos propres Gallions plutôt que la mort atroce de cent inconnus.~»
Le professeur de Défense avait un fin sourire.
Il sortit une montre de poche de sa robe et l'observa.
«~Partons maintenant, si nous n'avons rien de plus à nous dire.

--- Vous n'avez aucune question au sujet des choses impossibles que j'ai accomplies à Azkaban~?

--- Non, dit le professeur de Défense.
Je crois en avoir déjà compris la majeure partie.
Quant au reste, il m'est trop rare de trouver une personne qui ne m'est pas immédiatement et entièrement transparente, qu'elle soit amie ou alliée.
Je finirai de démêler ces puzzles vous concernant par moi-même, en temps et heure.~»

Le professeur de Défense se releva en poussant sur le mur de ses mains, avec une élégance cependant trop lente.
Le garçon l'imita avec moins de grâce.

Puis ce dernier laissa échapper la dernière question, la plus terrible, celle qu'il n'avait pas pu poser plus tôt~; comme si la prononcer à voix haute l'aurait rendue réelle, et comme si elle n'était pas déjà amplement évidente.

«~Pourquoi ne suis-je pas comme les autres enfants de mon âge~?~»

\later

Dans une petite contre-allée déserte du Chemin de Traverse, là où des restes d'ordures qu'on avait pas fait disparaître pouvaient être vus, coincés dans la rainure entre la rue de brique et le flanc des immeubles qui l'entourait, au côté de poussières éparpillées et d'autres signes de négligence, un vieux sorcier et son phénix Transplanèrent.

Le sorcier fouillait déjà dans sa robe à la recherche de son sablier lorsque par habitude, ses yeux se fixèrent sur un point au hasard situé entre la route et le mur afin de le mémoriser…

Et le vieux sorcier cligna les yeux de surprise~; un morceau de parchemin se trouvait là.

Un froncement de sourcils traversa le visage d'Albus Dumbledore alors qu'il faisait un pas en avant, se saisissait du fragment et le dépliait.

Sur celui-ci se trouvait un seul mot~: «~NON~».
Rien de plus.

Le sorcier le laissa lentement s'échapper de ses doigts.
Il se pencha d'un air absent au-dessus du pavé et ramassa le morceau de parchemin le plus proche, un morceau remarquablement similaire à celui qu'il venait de prendre~; il le toucha de sa baguette, et un moment plus tard s'y trouvait le même mot~: «~NON~», de la même écriture, la sienne.

Le vieux sorcier avait escompté revenir trois heures en arrière, au moment où Harry Potter était arrivé au Chemin de Traverse.
Il avait déjà observé le garçon quitter Poudlard grâce à ses appareils, et cela ne pouvait être défait (sa seule tentative de tromper ses propres appareils, et donc de contrôler le Temps sans altérer sa propre perspective sur celui-ci, s'était achevée par un désastre suffisamment conséquent pour le dissuader de jamais retenter de telles ruses).
Il avait espéré retrouver le garçon à l'instant de son arrivée et le mener dans un endroit sûr autre que Poudlard (car ses instruments n'avaient pas fait voir le retour du garçon).
Mais à présent…

«~Un paradoxe si je le récupère immédiatement après son arrivée au Chemin de Traverse~? murmura le vieux sorcier dans sa barbe.
Peut-être n'ont-ils pas mis en route leur plan de dévaliser Azkaban avant d'avoir confirmé son arrivée ici… ou alors… peut-être…~»

\later

Béton peint, sol dur, toit élevé, deux silhouettes face à face.
Une entité portait la forme d'un homme approchant la quarantaine et qui devenait déjà chauve, et un autre esprit portait la forme d'un garçon de onze ans avec une cicatrice sur le front.
Glace et ombre, pâle lumière bleue.

«~Je ne sais pas~», dit l'homme.

Le garçon se contenta de le regarder.
Puis il dit~:

«~Oh, vraiment~?

--- Vraiment, dit l'homme.
Je ne sais rien, et je ne dirai rien de ce dont je me doute.
Mais je dirai ceci…~»
%  LocalWords:  ake Gendlin Hah un
