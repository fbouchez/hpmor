% vim:spell:spelllang=fr

\chapter{Poser les mauvaises questions}

\lettrine{D}{ès} que Harry ouvrit les yeux, dans le dortoir des garçons de première année de Serdaigle, le matin de sa première vraie journée à Poudlard, il sut que quelque chose n'allait pas.

Tout était calme.

\emph{Bien trop} calme.

Ah, mais oui… il y avait un enchantement de Sourdinam placé sur sa tête de lit, contrôlable par un petit curseur glissant, qui était la seule raison pour laquelle quiconque pouvait jamais espérer s'endormir à Serdaigle.

Harry se redressa et regarda autour de lui, s'attendant à voir les autres se lever…

Le dortoir, vide.

Les lits, draps froissés et défaits.

Le soleil, entrant dans la chambre depuis un angle plutôt élevé.

Et son réveil mécanique toujours en marche, mais dont l'alarme était désactivée.

On l'avait laissé à dormir jusqu'à 9h52, apparemment.
En dépit de ses meilleurs efforts pour synchroniser son rythme de sommeil de 26 heures avec l'arrivée à Poudlard, il n'avait réussi à s'endormir qu'aux alentours de 1h du matin.
Il avait prévu de se lever à 7h avec les autres élèves~; il pouvait supporter un léger manque de sommeil le premier jour, du moment qu'on trouvait une solution magique quelconque avant le lendemain.
Mais là il avait raté le petit déjeuner.
Et son tout premier cours à Poudlard, Herbologie, avait commencé une heure et vingt-deux minutes plus tôt.

La colère se réveillait en lui, lentement, très lentement.
Oh, le bon canular.
Désactiver son alarme.
Augmenter le Sourdinam.
Et laisser M. Gros Bonnet Harry Potter rater son premier cours et se faire réprimander pour son sommeil lourd.

Quand Harry découvrirait qui avait fait ça…

Non, cela n'avait pu être fait qu'avec la coopération de tous les douze autres garçons du dortoir de Serdaigle.
Ils avaient tous vu sa forme endormie sous les draps.
Ils l'avaient tous laissé dormir pendant le petit déjeuner.

La colère refoula hors de lui, remplacée la confusion et par un horrible sentiment de blessure.
Ils l'avaient \emph{aimé}. Il lui semblait.
La nuit dernière, il avait pensé qu'ils l'aimaient. \emph{Pourquoi…}

Alors que Harry se levait de son lit, il aperçut un bout de papier fixé à sa tête de lit, face vers l'extérieur.

Le papier disait~:
\begin{writtenNote}
\letterAddress{Mes chers camarades Serdaigle,}

Ce fut une journée particulièrement longue. Merci de me laisser dormir et ne vous en faites pas pour mon petit déjeuner. Je n'ai pas oublié le premier cours.

\letterClosing[Bien à vous,]{Harry Potter.}
\end{writtenNote}
% \vspace{-1em}
Harry regardait le papier, figé, tandis que de l'eau glacée commençait couler dans ses veines.

C'était sa propre écriture, des lettres tracées avec son critérium.

Et il ne se souvenait pas l'avoir écrit.

Et… Harry plissa les yeux en observant le papier.
Et à moins que se ne soit son imagination, les mots «~Je n'ai pas oublié~» étaient écrits d'une façon différente, comme s'il essayait de se dire quelque chose…~?

Avait-il \emph{su} qu'il allait être Oublietté~?
S'était-il couché tard, après avoir commis un crime quelconque ou exercé une activité secrète, et puis il avait… mais il ne \emph{connaissait} pas le sort d'Oubliettes… est-ce que quelqu'un d'autre aurait… que…

Une pensée lui vint à l'esprit.
S'il \emph{avait} su qu'il allait être Oublietté…

Toujours en pyjama, Harry couru autour du lit vers sa malle, appuya sur le loquet avec son pouce, récupéra sa bourse, y plongea sa main et dit~: «~Note à moi-même.~»

Un autre morceau de papier apparut dans sa main.

Harry le sortit pour l'observer. Lui aussi portait son écriture.

La note disait~:

\begin{writtenNote}
\letterAddress{Cher Moi,}

S'il te plaît, joue au jeu.
On ne peut y jouer qu'une seule fois dans une vie.
C'est une opportunité irremplaçable.

Code de reconnaissance 927, je suis une pomme de terre.

\letterClosing[Bien à toi,]{Toi.}
\end{writtenNote}

Harry hocha lentement la tête. «~Code de reconnaissance 927, je suis une pomme de terre~» était en effet le message qu'il avait mis au point à l'avance -- quelques années plus tôt, alors qu'il regardait la télévision -- et que lui seul connaissait.
Au cas où il devrait déterminer si une copie de lui-même était vraiment \emph{lui}, ou quelque chose du genre.
Juste au cas où. Soyez Prêts.

Harry ne pouvait pas faire \emph{confiance} au message, il pouvait y avoir d'autres sorts impliqués.
Mais ça éliminait la possibilité d'un simple canular.
Il avait définitivement écrit cela~; et il ne se rappelait définitivement pas l'avoir écrit.

En observant le papier, Harry se rendit compte qu'on pouvait voir de l'encre visible par transparence.

Il retourna la feuille et lut l'autre côté:
\medskip
\begin{theGame}
\textsc{\underline{Instructions pour le Jeu~:}}
\\

tu ne connais pas les règles du jeu\\
tu n'en connais pas les enjeux\\
tu ne connais pas le but du jeu\\
tu ne sais pas qui contrôle le jeu\\
tu ne sais pas comment se termine le jeu
\\

Tu démarres avec 100 points.\\
La partie commence maintenant.\\
\end{theGame}

Harry regarda longuement les “instructions.”
Ce côté n'était pas manuscrit, l'écriture était parfaitement régulière, et donc artificielle.
On aurait dit que le message avait été écrit par une Plume à Papote, comme celle qu'il avait achetée pour dicter du texte.

Il n'avait \emph{absolument aucune idée} de ce qui était en train de se passer.

Bon… l'étape numéro un était de s'habiller et de manger.
Peut-être inverser l'ordre de ces actions.
Son estomac lui semblait plutôt vide.

Il avait raté le petit déjeuner, bien sûr, mais il s'était Préparé à cette éventualité, car il l'avait visualisée à l'avance.
Harry mit sa main dans sa bourse et dit «~barres énergétiques~», s'attendant à obtenir la boîte de barres de céréales qu'il avait achetées avant de partir à Poudlard.

Ce qui se matérialisa dans sa main ne ressemblait pas à une boîte de barres de céréales.

Lorsque Harry ramena sa main dans son champ de vision, il vit deux petits bonbons en forme de barre -- loin d'être suffisants pour un repas -- attachés à une note, et la note portait la même écriture que les instructions du jeu.

La note disait~:
\begin{theGameResults}
  \begin{align*}
  \gameResult{Tentative échouée}{-1 point}
  \gameResult{Points actuels}{99}
  \gameResult{État physique}{toujours faim}
  \gameResultnnl{État mental}{confus}
  \end{align*}
\end{theGameResults}

«~Bluuhubluu~» fit la bouche de Harry sans aucune forme d'intervention ou de décision consciente de sa part.

Il resta immobile environ une minute.

Une minute plus tard, ça n'avait \emph{toujours} aucun sens et il n'avait \emph{toujours} aucune idée de ce qui se passait et son cerveau n'avait même pas \emph{commencé} à s'accrocher à la moindre \emph{hypothèse}, comme si ses mains mentales étaient encastrées dans des balles en caoutchouc et qu'il était incapable de se saisir de quoi que ce soit.

Son estomac, qui avait ses propres priorités, suggéra un petit test expérimental.

«~Ah… dit Harry à la pièce vide. J'imagine que je ne pourrais pas dépenser un point pour récupérer ma boîte de barres de céréales~?~»

Silence.

Harry mit sa main dans la bourse et dit~: «~boîtes de barres de céréales.~»

Une boîte qui semblait avoir la bonne forme se matérialisa dans sa main… mais elle était trop légère, et elle était ouverte, et elle était vide, et la note qui y était attachée disait~:
\begin{theGameResults}%
  \begin{align*}
  \gameResult{Points dépensés}{1}
  \gameResult{Points actuels}{98}
  \gameResultnnl{Tu as obtenu}{une boite de barres de céréales}
  \end{align*}
\end{theGameResults}

«~J'aimerais dépenser un point et obtenir les \emph{vraies barres de céréales}~», dit Harry.

Silence à nouveau.

Harry plaça sa main dans la bourse et dit «~barres de céréales.~»

Rien.

Harry haussa les épaules, abattu, et se rendit à l'armoire qu'on lui avait attribuée, située près de son lit, pour prendre sa robe de sorcier pour la journée.

Sur le plancher de l'armoire, sous ses vêtements, il trouva les barres, et une note~:

\begin{theGameResults}
  \begin{align*}
  \gameResult{Points dépensés}{1}
  \gameResult{Points actuels}{97}
  \gameResult{Tu as obtenu}{6 barres de céréales}
  \gameResultnnl{Tu portes encore}{un pyjama}
  \end{align*}
  \gameInstruction{Ne mange pas en pyjama}\\
  \gameInstruction{Tu aurais une pénalité pyjama}
\end{theGameResults}

\emph{Et maintenant je sais que celui ou celle qui contrôle ce jeu est dingue.}

«~Je pense que ce jeu est contrôlé par Dumbledore~», dit Harry à voix haute.
Peut-être que \emph{cette fois} il pouvait établir un nouveau record de vitesse en sagacité.

Silence.

Harry commençait à comprendre le schéma~; la note se trouverait au prochain endroit où il regarderait.
Alors Harry regarda sous son lit.

\begin{theGameResults}
\textsc{%
\noindent%
Ha~! Ha ha ha ha ha~!\\
Ha ha ha ha ha ha~!\\
Ha~! Ha~! Ha~! Ha~! Ha~! Ha~!\\
Dumbledore ne contrôle pas le jeu\\
Mauvaise supposition\\
Très mauvaise supposition\\
−20 points\\
Et tu es toujours en pyjama\\
C’est ton quatrième coup\\
Et tu es toujours en pyjama
}
\begin{align*}
\gameResult{Pénalité pyjama}{−2 points}
\gameResultnnl{Points actuels}{75}
\end{align*}
\end{theGameResults}

Bon, c'était un taré, voilà tout.
C'était seulement son premier jour d'école et si on éliminait Dumbledore, il ne connaissait personne d'autre qui pourrait être aussi fou.

Son corps plus ou moins en pilote automatique, Harry récupéra un ensemble de robe et de sous-vêtements, ouvrit le niveau caverne de sa malle (il était très pudique et quelqu'un aurait pu entrer dans le dortoir), s'habilla, et remonta les escaliers pour ranger son pyjama.

Harry marqua une pause avant d'ouvrir le tiroir de l'armoire qui contenait ses pyjamas.
Si le schéma continuait…

«~Comment puis-je gagner plus de points~?~» dit Harry à voix haute.

Puis il ouvrit le tiroir.

\begin{theGameResults}
\textsc{%
Les opportunités de faire le bien sont partout\\
Mais l'obscurité est là où la lumière doit être
\begin{align*}
  \gameResult{Coût de la question}{1~point}[-1ex]
\gameResultnnl{Points actuels}{74}%\\[1.5EX]
\end{align*}
Jolis sous-vêtements\\
C'est ta mère qui les a choisis~?
}
\end{theGameResults}

Harry broya la note dans sa main, le visage écarlate.
L'injure de Drago lui revint. \emph{Fils de sang-de-bourbe}…

Il eut la présence d'esprit pour ne pas le dire à voix haute.
Il recevrait probablement une pénalité injures.

Harry s'équipa de sa bourse en peau de Moke et de sa baguette.
Il retira l'emballage d'une de ses barres de céréales et le jeta dans la poubelle de la chambre, par-dessus une Chocogrenouille à peine entamée, une enveloppe froissée et du papier d'emballage rouge et vert.
Il mit les autres barres dans sa bourse.

Il balaya les alentours du regard dans une tentative ultime, désespérée, et en définitive futile, de trouver des indices.

Puis Harry quitta le dortoir, mangeant pendant qu'il marchait, à la recherche des donjons de Serpentard.
Du moins il \emph{pensait} que c'est ce à quoi la note faisait allusion.

Essayer de naviguer dans les couloirs de Poudlard était un peu comme… probablement \emph{pas} aussi terrible que de se promener dans une peinture d'Escher, c'est le genre de choses qu'on dit pour l'effet rhétorique plutôt que par véracité.

Peu de temps après, Harry se disait qu'en fait, une peinture d'Escher aurait des avantages et des inconvénients comparé à Poudlard.
Inconvénients~: pas d'orientation cohérente de la gravitation.
Avantages~: au moins les escaliers ne bougeaient pas \emph{PENDANT QU'ON ÉTAIT ENCORE DESSUS}.

Harry avait initialement grimpé quatre escaliers pour atteindre son dortoir.
Après avoir descendu pas moins de douze escaliers sans arriver en vue des donjons, Harry avait conclu que (1) une peinture d'Escher serait \emph{du gâteau} en comparaison, (2) il était étrangement \emph{plus haut} dans le château qu'à son départ, et (3) il était si \emph{complètement} paumé qu'il n'aurait pas été surpris si, en regardant par une fenêtre, il voyait deux lunes dans le ciel.


Le plan de secours A avait été de s'arrêter et de demander son chemin, mais il semblait y avoir une pénurie aiguë de promeneurs, comme si ces pauvres types étaient tous en cours comme ils étaient censés le faire, ou quelque chose du genre.

Plan de secours B…

«~Je suis perdu, dit Harry à voix haute. Le, euh, l'esprit de Poudlard pourrait-il m'aider ou quelqu'un d'autre~?

--- Je ne pense pas que ce château ait un esprit, remarqua une vieille femme desséchée depuis l'un des tableaux accrochés au mur. Une vie, peut-être, mais pas un esprit.~»

Il y eut une brève pause.

«~Êtes-vous…~» commença Harry, puis il se tut.
À bien y réfléchir, non, il n'allait PAS demander au portrait si elle était pleinement consciente, au sens d'être consciente de sa propre conscience.

«~Je suis Harry Potter~», dit sa bouche, plus ou moins en pilote automatique.
Et plus ou moins automatiquement, Harry tendit sa main vers le tableau.

La femme dans le tableau baissa les yeux vers la main de Harry et leva les sourcils.

Lentement, Harry redescendit sa main jusqu'à son flanc.

«~Désolé, dit Harry, je suis un peu nouveau ici.

--- J'avais remarqué, jeune aigle. Où essayez-vous d'aller~?~»

Harry hésita. «~Je ne suis pas vraiment sûr, dit-il.

--- Alors peut-être y êtes-vous déjà.

--- Eh bien, quel que soit l'endroit où \emph{j'essaie} d'aller, je ne pense pas que ce soit \emph{ici}…~» Harry referma la bouche, se rendant soudain compte qu'il devrait vraiment avoir l'air d'un idiot.
«~Laissez-moi réessayer. Je joue à un jeu, seulement je ne connais pas les règles…~»
Ça ne marchait pas vraiment non plus, non?
«~Bon, troisième essai.
Je recherche des opportunités de faire le bien pour gagner des points, et tout ce que j'ai c'est cet indice cryptique à propos de ténèbres se trouvant là où la lumière devrait être, alors j'essayais de descendre, mais il me semble qu'au lieu de ça je ne fais que monter plus haut…~»

La vieille dame dans le portrait le regardait d'un air plutôt sceptique.

Harry soupira.
«~Ma vie tend à être un peu singulière.

--- Serait-il juste de dire que vous ne savez pas où vous essayez d'aller ni même pourquoi vous essayez d'y aller~?

--- \emph{Parfaitement} juste.~»

La vieille femme hocha la tête.
«~Je ne suis pas sûre qu'être perdu soit le plus important de vos problèmes, jeune homme.

--- Vrai, mais à la différence de mes problèmes les plus importants, c'est un problème que je peux apprendre à résoudre et \emph{wouah} est-ce que cette conversation est en train de devenir une métaphore de l'existence humaine, je ne m'en étais pas rendu compte jusqu'à présent.~»

La dame observa Harry avec appréciation. «~Voilà un \emph{splendide} petit aigle. Pendant un instant j'ai commencé à douter. Eh bien, dans ce cas, en règle générale, si vous ne faites que tourner à gauche, vous finirez forcément par descendre.~»

Ça avait l'air familier, mais Harry n'arrivait pas à se souvenir où il avait entendu ça auparavant. «~Euh… vous semblez être une personne très intelligente.
Ou l'image d'une personne très intelligente… quoi qu'il en soit, avez-vous entendu parler d'un mystérieux jeu auquel on ne peut jouer qu'une fois et dont on ne vous donne pas les règles~?

--- La vie, répondit immédiatement la dame.
C'est une des devinettes les plus simples que j'ai jamais entendues.~»

Harry cligna des yeux. «~Non, dit-il lentement. Je veux dire que j'ai eu une vraie note et tout le reste, disant que je devais jouer au jeu, mais qu'on ne me dirait pas les règles, et quelqu'un me laisse ces petits bouts de papier me disant combien de points j'ai perdu pour avoir enfreint les règles, comme une pénalité de moins deux points pour port de pyjama par exemple.
Connaissez-vous qui que ce soit à Poudlard qui soit assez fou et puissant pour faire quelque chose comme ça~?
À part Dumbledore, bien sûr~?~»

L'image de la dame soupira.
«~Je ne suis qu'une image, jeune homme.
Je me souviens de Poudlard telle qu'elle était -- pas telle qu'elle est maintenant.
Tout ce que je puis vous dire, c'est que si c'était une énigme, la réponse serait que le jeu est la vie, et que bien que nous ne décidions pas de toutes les règles nous-mêmes, c'est toujours nous-mêmes qui nous décernons ou nous ôtons des points.
Si ce n'est pas une énigme, mais la réalité -- alors je ne sais pas.~»

Harry s'inclina profondément devant l'image. «~Merci, ma dame.~»

La dame lui fit une révérence. «~J'aimerais pouvoir dire que je me souviendrai de vous avec affection, dit-elle, mais je ne me souviendrai probablement pas du tout de vous. Adieu, Harry Potter.~»

Il s'inclina à nouveau en guise de réponse, et commença à descendre les escaliers les plus proches.

Quatre virages à gauche plus tard il se retrouva face à un corridor qui s'arrêtait brusquement en un large monticule de rochers -- comme s'il y avait eu un éboulement, sauf que les murs et le plafond étaient intacts et faits de pierres de château assez normales.

«~Très bien, dit Harry au vide qui l'entourait, j'abandonne.
Je demande un autre indice.
Comment puis-je aller là où je dois aller~?

--- Un indice~! Un indice, dites-vous~?~»

La voix exaltée venait d'un tableau non loin, le portrait d'un homme d'âge mûr portant la robe rose la plus voyante que Harry avait jamais vue ou même imaginée.
Dans le tableau, il portait un vieux chapeau pointu affaissé surmonté d'un poisson (pas un dessin de poisson, notez, mais un poisson).

«~Oui~! dit Harry. Un indice~! Un indice, c'est ça~! Mais pas \emph{n'importe quel} indice, je recherche un indice \emph{spécifique}, c'est pour un jeu auquel je joue…

--- Oui, oui~! Un indice pour le jeu~! Vous êtes Harry Potter, n'est-ce pas~? Je suis Cornelion Gommalse~!
Erin le Consort me l'a dit, c’est Lord Labelette qui lui avait dit, et j’ai complètement oublié qui le lui avait dit.
Mais c'était un message que \emph{je} devais vous donner~!
\emph{Moi}~! Personne ne s'est préoccupé de moi depuis, je ne sais pas depuis combien de temps, peut-être depuis toujours, je suis coincé ici dans ce satané couloir inutile… un indice~! J'ai votre indice~!
Ça ne vous coûtera que trois points~! Le voulez-vous~?

--- Oh oui~! Je le veux~!~» Harry se rendait compte qu'il ferait mieux d'atténuer son sarcasme mais il ne semblait pas capable de s'en empêcher.

«~Les ténèbres se trouvent entre la salle d'étude verte et le cours de métamorphose de McGonagall~!
C'est l'indice~! Et bouge-toi les fesses, tu es plus lent qu'un sac d'escargot~!
Moins dix points pour lenteur~!
Tu as maintenant 61 points~!
C'était le reste du message~!

--- Merci~», dit Harry. Il commençait à prendre un sacré retard sur le jeu.
«~Euh… j'imagine que vous ne savez pas d'où le message provenait \emph{initialement}~?

--- Il a été prononcé par une voix caverneuse qui émanait d'un trou dans l'air lui-même, un trou qui s'ouvrait sur un abysse flamboyant~! C'est ce qu'ils m'ont dit~!~»

Au point où il en était, Harry ne parvenait plus à savoir s'il fallait être sceptique de ce genre de chose ou juste accepter l'information sans sourciller.
«~Et comment puis-je trouver l'espace entre la salle d'étude verte et le cours de Métamorphose~?

--- Faites juste demi-tour et allez à gauche, à droite, en bas, en bas, à droite, à gauche, à droite, en haut et à nouveau à gauche, vous serez face à la salle d'étude verte et si vous la traversez et sortez de l'autre côté, vous verrez un grand couloir courbe qui va à une intersection et sur le côté droit de cette intersection vous trouverez un grand couloir tout droit qui va à la salle de classe de Métamorphose~!~»
L'image de l'homme d'âge mûr s'interrompit.
«~Du moins, c'est comme çà que c'était quand \emph{j'étais} à Poudlard.
Nous \emph{sommes} un lundi d'une année impaire, n'est-ce pas~?

--- Papium et feuille de critérié, dit Harry à sa bourse. Euh, annule ça, critérium et feuille de papier. Il leva les yeux. Vous pourriez répéter ça~?~»

Après s'être perdu deux fois de plus, Harry sentit qu'il commençait à comprendre les règles de bases pour naviguer dans le labyrinthe en perpétuelle transformation qu'était Poudlard, c'est-à-dire~: \emph{demande ton chemin aux tableaux}.
Si cela renvoyait à une leçon de vie incroyablement profonde il n'avait pas la moindre idée de ce qu'elle pouvait être.

La salle d'étude verte était un espace étonnamment plaisant.
La lumière du soleil traversait des fenêtres aux vitrages verts décorés de dragons dans des paysages calmes et pastoraux.
Elle était équipée de fauteuils qui avaient l'air extrêmement confortables et de tables qui semblaient parfaitement adaptées à l'étude en compagnie de quelques amis.

Harry était \emph{incapable} de la traverser en marchant pour ressortir par la porte de l'autre côté.
Il y avait des \emph{étagères pleines de livres} le long du mur, et il \emph{devait} y aller et lire quelques-uns des titres sous peine de perdre son droit au nom de famille Verres.
Mais il le fit rapidement, gardant à l'esprit qu'on lui avait reproché sa lenteur, puis ressortit de l'autre côté.

Il suivait le “grand corridor courbe” lorsqu'il entendit un cri, celui d'une voix de jeune garçon.

En de telles circonstances, Harry avait une excuse pour sprinter à fond sans se préoccuper de conserver son énergie ou de faire les exercices d'échauffement adaptés ou de s'inquiéter de percuter quelque chose, une course frénétique et soudaine qui se transforma presque en un arrêt tout aussi soudain quand il déboula sur un groupe de six Poufsouffle de première année…

… qui se tenaient blottis les uns contre les autres, l'air plutôt effrayé et qui semblaient vouloir désespérément faire quelque chose, mais sans savoir exactement quoi, ce qui avait probablement quelque chose à voir avec le groupe de cinq Serpentard plus âgés qui se tenaient en cercle autour d'un autre jeune garçon.

Harry sentit la colère monter en lui.

«~\emph{Excusez-moi~!}~» cria Harry à pleins poumons.

Cela n'avait peut-être pas été forcément nécessaire.
Les gens le regardaient déjà. Mais cela avait sûrement contribué à figer la situation.

Harry dépassa le groupe de Poufsouffle et se dirigea vers les Serpentard.

Ils le regardèrent avec des expressions allant de la colère à l'amusement à la délectation.

Une partie du cerveau de Harry hurlait, paniquée, que c'étaient des garçons bien plus grands et bien plus âgés, qui pourraient le réduire en bouillie.

Une autre partie répondait sèchement que quiconque se ferait attraper en train de réduire le Survivant en bouillie se récupérerait une \emph{montagne} d'ennuis, en particulier s'il faisait partie d'une bande de Serpentard plus âgés et devant les yeux de sept Poufsouffle, et que le risque qu'ils lui causent des dommages permanents devant des témoins était quasi nul.
La seule véritable arme que ces garçons plus grands que lui avaient était sa propre peur, s'il la leur laissait.

Puis Harry vit que le garçon auquel ils s'en prenaient était Neville Londubat.

Bien sûr.

La question était réglée.
Harry avait décidé de s'excuser humblement auprès de Neville, ce qui signifiait que Neville était \emph{sien}, comment \emph{osaient}-ils~?

Harry tendit le bras, attrapa Neville par le poignet et le \emph{tira brusquement} hors du cercle formé par les Serpentard~; le garçon, choqué, trébucha tandis que Harry le tirait et presque dans le même mouvement se propulsa lui-même par le passage ainsi ouvert.

Et Harry se tenait au centre des Serpentard, là où Neville s'était tenu, le regard levé vers les garçons bien plus âgés, bien plus grands et bien plus forts.

«~Bonjour, dit Harry. Je suis le Survivant.~»

Il y eut un silence assez gênant.
Personne ne semblait savoir comment la conversation était censée évoluer.

Les yeux de Harry glissèrent jusqu'au sol où il vit quelques livres et papiers éparpillés.
Oh, ce vieux jeu où l'on laisse l'enfant essayer de ramasser ses livres pour les lui arracher des main et les faire tomber de nouveau.
Harry ne se souvenait pas avoir été lui-même victime de ce jeu, mais il avait une bonne imagination et son imagination le rendait furieux.
Eh bien, lorsque la situation serait réglée, Neville aurait tout le loisir de revenir et de ramasser ses affaires, du moment que les Serpentard restaient suffisamment concentrés sur Harry pour penser à faire quoi que ce soit aux livres.

Malheureusement, ceux-ci avaient suivi le regard de Harry.
«~Ooh, dit le plus grand des garçons, on veut ses p'tits livres…

--- La ferme~», dit Harry froidement.
\emph{Continue à les déstabiliser.
Ne fais pas ce à quoi ils s'attendent.
Ne rentre pas dans un schéma qui les incite à te malmener.}
«~Cela fait-il partie d'un plan incroyablement intelligent qui vous donnera des avantages futurs, ou est-ce juste une disgrâce au nom de Salazar Serpentard aussi inutile que…~»

Le plus grand des garçons bouscula Harry Potter avec force, ce qui l'envoya s'étaler hors du cercle de Serpentard sur le sol en pierre de Poudlard.

Les Serpentard éclatèrent de rire.

Harry se releva dans ce qui lui sembla être un mouvement terriblement lent.
Il ne savait pas encore comment utiliser sa baguette, mais il n'y avait aucune raison de laisser cela l'arrêter, vu les circonstances.

«~Je voudrais payer \emph{autant de points que nécessaire} pour me débarrasser de cette personne~», dit Harry, pointant du doigt le plus grand des Serpentard.

Puis Harry leva son autre main, dit «~Abracadabra~», et claqua des doigts.

Au son d'\emph{Abracadabra}, deux des Poufsouffle hurlèrent, y compris Neville, trois autres Serpentard se jetèrent désespérément hors de la trajectoire pointée par le doigt de Harry, et le plus grand des Serpentard recula en titubant, l'air choqué, le visage soudain décoré d'une grande éclaboussure rouge qui s'étalait également sur son cou et sa poitrine.

Harry ne s'attendait \emph{pas} à ça.

Lentement, le Serpentard porta la main à la tête et décolla le plat de tarte à la cerise qui venait de s'écraser sur le côté de son visage.
Il regarda le plat un moment, avant de le laisser tomber au sol.

Ce n'était probablement pas le meilleur moment possible pour qu'un des Poufsouffle commence à rire, mais c'est exactement ce qu'un des Poufsouffle était en train de faire.

Puis Harry remarqua la note collé au dessous du plat.

«~Une seconde~», dit Harry, qui s'élança pour ramasser la note.
«~Cette note est pour moi je pense…

--- \emph{Toi,} grogna le Serpentard, \emph{toi, tu, vas…}

--- \emph{Regarde}-moi ça~!~» cria Harry, fourrant la note sous le nez du Serpentard.
«~Franchement, \emph{regarde}~!
On me facture 30 points pour la livraison et la manutention d'une pauvre tarte, tu y crois~?
30 points~!
J'y perds, même après avoir secouru un innocent en détresse~!
Et des frais de stockage~?
Des coûts de transport~?
Du frêt~?
Comment peut-on avoir du \emph{frêt} pour une \emph{tarte}~?~»

À nouveau, silence gênant.
Harry eut des pensées meurtrières envers un des Poufsouffle qui ne pouvait s'empêcher de glousser, cet idiot allait le mettre dans le pétrin.

Harry fit un pas en arrière et lança aux Serpentard son plus beau regard mortel.
«~Maintenant dégagez ou je continuerai à rendre votre existence de plus en plus surréaliste.
Laissez-moi vous prévenir… se frotter à \emph{ma} vie a tendance à rendre \emph{vôtre} vie… \emph{un poil dangereuse}. Compris~?~»

D'un seul mouvement terrifiant, le grand Serpentard fit jaillir sa baguette et la pointa vers Harry, et reçu au même moment une nouvelle tarte sur le coin de la figure, de l'autre côté, celle-ci à la myrtille.

La note sur cette tarte était assez grande et clairement lisible.
«~Tu devrais lire la note sur la tarte, remarqua Harry. Je pense que c'est pour toi cette fois.~»

Le Serpentard leva lentement la main pour attraper le plat à tarte, le retourna dans un bruit de succion qui fit tomber encore plus de myrtilles par terre, et lut la note qui disait~:

\begin{theGameResults}
\centering
\textsc{\underline{AVERTISSEMENT}}

\textsc{
\underline{Aucune} magie n'est autorisée\\ sur le participant\\
tant que le jeu est en cours\\
Toute interférence durant le jeu\\
\underline{sera} reportée aux autorités du jeu\\}
\end{theGameResults}

La pure perplexité qui se lisait sur le visage du Serpentard était un chef-d'œuvre.
Harry songea qu'il commençait peut-être à aimer le contrôleur du jeu.

«~Écoute, dit Harry, et si on en restait là~?
On dirait que les choses commencent à devenir incontrôlables par ici.
Que dirais-tu de retourner à Serpentard pendant que je retourne à Serdaigle et que tout le monde se calme un peu, d'accord~?

--- J'ai une meilleure idée, siffla le Serpentard. Et si tu te cassais accidentellement tous les doigts~?

--- Comment, au nom de Merlin, penses-tu mettre en scène un accident crédible après avoir fait cette menace devant une douzaine de personnes, espèce \emph{d'idiot}…~»

Le grand Serpentard tendit sa main vers celle de Harry, lentement, délibérément, et Harry se figea, tandis que la partie de son cerveau qui se rendait compte de l'âge et de la force du garçon parvenait enfin à se faire entendre, criant~: \scream{qu'est-ce que je suis en train de faire~?}

«~Attends~!~» dit l'un des autres Serpentard, la panique dans la voix.
«~Arrête, tu ne devrais pas le faire vraiment~!~»

Le grand Serpentard l'ignora, attrapa fermement la main droite de Harry dans sa main gauche, et prit l'index de Harry dans sa main droite.

Harry regarda le Serpentard droit dans les yeux.
Une partie de Harry hurlait, ce n'était pas censé arriver, ce n'était pas \emph{autorisé} à arriver, les adultes ne laisseraient jamais une chose pareille arriver \emph{pour de vrai}…

Lentement, le Serpentard commença à tordre son doigt en arrière.

\emph{Il n'a pas vraiment cassé mon doigt et ce serait indigne de moi de ne serait-ce que tressaillir avant qu'il le fasse.
Jusque-là, ce n'est qu'une autre tentative de provoquer la peur.}

«~Arrête~! dit le Serpentard qui avait auparavant protesté. Arrête, c'est une très mauvaise idée~!

--- Je suis plutôt d'accord~», dit une voix de glaciale. La voix d'une femme plus âgée.

Le Serpentard relâcha la main de Harry et fit un bond en arrière comme s'il s'était brûlé.

«~Professeure Chourave~!~» s'écria l'un des Poufsouffle, d'un ton plus heureux que Harry n'en avait jamais entendu de sa vie.

Alors qu'il pivotait, une petite femme rondouillette débarqua dans son champ de vision.
Elle avait des cheveux gris bouclés et hirsutes, et ses vêtements étaient couverts de poussière.
Elle pointa un doigt accusateur en direction des Serpentard.
«~Expliquez-vous, dit-elle.
Que faites-vous avec mes Poufsouffle et…, elle le regarda, mon superbe étudiant, Harry Potter.~»

\emph{Oh oh. C'est vrai, c'était SA classe que j'ai ratée ce matin.}

«~Il a menacé de nous tuer~!~» lâcha l'un des Serpentard, celui qui avait demandé qu'ils arrêtent.

«~Quoi~? dit Harry, neutre.
Certainement \emph{pas}~!
Si je voulais vous tuer je ne commencerais pas par faire des menaces publiques~!~»

Un troisième Serpentard ne put s'empêcher de rire mais s'arrêta brusquement lorsque les autres lui jetèrent des regards noirs.

La professeure Chourave semblait plutôt sceptique.
«~Et de quelle menace de mort s'agirait-il exactement~?

--- Le sortilège de la Mort~! Il a fait semblant de nous lancer le sortilège de la Mort~!~»

Chourave se retourna pour regarder Harry.
«~Oui, une menace vraiment terrible venant d'un garçon de onze ans.
Mais tout de même vous ne devriez jamais prétendre faire cela, pas même en rêve, Harry Potter.

--- Je ne connais même pas les \emph{mots} qui composent le sortilège de la Mort, dit promptement Harry. Et je n'avais même pas sorti ma baguette.~»

C'était maintenant à Harry que Chourave jetait un regard sceptique.

«~J'imagine que ce garçon s'est jeté deux tartes à la figure \emph{lui-même} dans ce cas.

--- Il n'a \emph{pas} utilisé sa baguette~! lâcha l'un des jeunes Poufsouffle.
Je ne sais pas non plus comment il a fait, il a jusque claqué des doigts, et il y avait de la tarte~!

--- Vraiment~», dit Chourave après une pause.
Elle tira sa propre baguette.
«~Je ne l'exigerai pas puisque vous semblez être la victime, mais pourrais-je examiner votre baguette pour vérifier cela~?~»

Harry sortit sa baguette.
«~Qu'est-ce que je…

--- \emph{Priori Incantatum}~», dit Chourave.
Elle fronça les sourcils.
«~C'est étrange, votre baguette semble n'avoir jamais été utilisée.~»

Harry haussa les épaules. «~C'est le cas, à vrai dire, je n'ai eu ma baguette et mes manuels scolaire qu'il y a quelques jours.~»

Chourave hocha la tête.
«~Alors c'est clairement un cas de magie accidentelle de la part d'un garçon qui se sentait menacé.
Et les règles sont formelles, vous ne pouvez pas être tenu pour responsable.
En ce qui \emph{vous} concerne…~» elle se tourna vers les Serpentard.
Elle montra délibérément du regard les livres de Neville, éparpillés au sol.

S'en suivit un long silence pendant lequel elle jaugea les cinq Serpentard.

«~Moins trois points pour Serpentard, \emph{chacun}, dit-elle enfin.
Et six pour \emph{lui}~», pointant le garçon couvert de tarte.
«~Ne touchez plus \emph{jamais} à mes Poufsouffle, ni à mon élève Harry Potter.
Maintenant \emph{partez}.~»

Elle n'eut pas à se répéter~; les Serpentard firent demi-tour et détalèrent.

Neville alla ramasser ses livres.
Il semblait pleurer, mais un petit peu seulement.
C'était peut-être l'effet différé du choc, ou peut-être parce que les autres garçons l'aidaient.

«~Merci \emph{beaucoup}, Harry Potter, lui dit la professeure Chourave.
Sept points pour Serdaigle, un pour chaque Poufsouffle que vous avez aidé à protéger.
C'est tout ce que j'ai à dire.~»

Harry cligna des yeux.
Il s'était attendu à quelque chose ressemblant à un sermon sur l'importance de se tenir à l'écart des ennuis, et une réprimande plutôt sévère pour avoir manqué son tout premier cours.

Peut-être qu'il aurait \emph{dû} aller à Poufsouffle.
Chourave était cool.

«~\emph{Récurvite}~», dit Chourave à la bouillie de tartes qui s'étalait au sol, qui disparut promptement.

Et elle partit, marchant le long du couloir qui menait à la salle d'étude verte.

«~Comment as-tu \emph{fait} ça~?~» souffla l'un des garçons de Poufsouffle après son départ.

Harry sourit avec suffisance.
«~Je peux faire survenir tout ce que je veux juste en claquant des doigts.~»

Les yeux du garçon s'agrandirent.

«~\emph{Vraiment~?}

--- Non, dit Harry. Mais quand vous raconterez cette histoire à tout le monde, assurez-vous de la partager avec Hermione Granger, en première année à Serdaigle, elle a une anecdote que vous trouverez amusante.~»
Il n'avait aucune idée de ce qui venait d'arriver, mais il n'allait pas laisser passer cette opportunité de renforcer sa légende grandissante.
«~Oh, et c'était quoi cette histoire autour du sortilège de la Mort~?~»

Le garçon lui jeta un étrange regard.
«~Tu ne sais vraiment pas~?

--- Si je le savais je ne poserais pas la question.

--- Les mots du sortilège de la Mort sont~», le garçon avala sa salive, sa voix devint un murmure, et il tint ses mains loin de ses flancs comme pour rendre très clair le fait qu'il ne tenait pas de baguette, «~\emph{Avada Kedavra}~».

\emph{Évidemment.}

Harry ajouta cela à sa liste croissante de choses-à-ne-jamais-raconter-à-papa, le professeur Michel Verres-Evans.
C'était déjà suffisamment difficile d'expliquer que vous étiez la seule personne à avoir survécu au terrible sortilège de la Mort.
S'il fallait en plus admettre que le sortilège de la Mort était «~Abracadabra~»…

«~Je vois, dit Harry après une pause.
Eh bien c'est la dernière fois que je dis \emph{ça} avant de claquer des doigts.~»
En même temps, cela \emph{avait} produit un effet qui pourrait être tactiquement utile.

«~\emph{Pourquoi} as-tu…

--- Élevé par des Moldus.
Les Moldus pensent que c'est une blague et que c'est drôle.
Je suis sérieux, c'est ce qui s'est passé.
Désolé, mais pourrais-tu me rappeler ton nom~?

--- Je suis Ernie Macmillan~», dit le Poufsouffle.
Il tendit sa main, et Harry la serra.
«~Honoré de te rencontrer.~»

Harry s'inclina légèrement.
«~Ravi de te rencontrer, tu peux laisser tomber le “honoré.”~»

Puis les autres garçons firent foule autour de lui et il y eut un déluge de présentations.

Une fois terminé, Harry avala sa salive.
Cela allait être très difficile.
«~Euh… si vous voulez bien m'excuser… j'ai quelque chose à dire à Neville…~»

Tous les yeux se tournèrent vers Neville, qui fit un pas en arrière, appréhensif.

«~J'imagine, dit Neville d'une toute petite voix, que tu vas dire que j'aurais dû être plus brave…

--- Oh, non, rien de ce genre~! dit hâtivement Harry.
Rien à voir avec \emph{ça}.
C'est juste, euh, quelque chose que le Choixpeau magique m'a dit…~»

Soudain les autres garçons eurent l'air \emph{très} intéressés, sauf Neville, qui avait l'air encore \emph{plus} appréhensif.

Quelque chose semblait bloquer la gorge de Harry.
Il savait qu'il devrait juste le sortir, mais c'était comme s'il avait avalé une grosse brique qui s'était coincée dans le passage.

Harry dût presque prendre manuellement le contrôle de ses lèvres pour produire chaque syllabe individuellement, mais il parvint à le dire.
«~Je, suis, dés, olé.~»
Il expira et prit une profonde inspiration.
«~Pour ce que j'ai fait, euh, l'autre jour.
Tu… tu n'as pas à être courtois ou quoi que ce soit.
Je comprendrai si tu me détestes.
Ce n'est pas que j'essaye d'avoir l'air cool en présentant mes excuses ou de te forcer à les accepter.
Ce que j'ai fait était mal.~»

Il y eut une pause.

Neville serra ses livres plus fort encore contre sa poitrine.
«~Pourquoi est-ce que tu as fait ça~?~» demanda-t-il d'une voix fluette et tremblante.
Il cligna des yeux comme pour retenir ses larmes.
«~Pourquoi est-ce que \emph{tout le monde} me fait ça, même le Survivant~?~»

Harry se sentit soudain plus petit qu'il ne l'avait jamais été de sa vie.
«~Je suis désolé~», répéta Harry, la voix rauque.
«~C'est juste que… tu avais l'air si effrayé, c'était comme si tu tenais un écriteau au-dessus de ta tête où serait inscrit “victime,” et je voulais te montrer que les choses ne tournent pas \emph{toujours} mal, que parfois les monstres donnent du chocolat… je me suis dit que si je te montrais ça, tu te rendrais compte qu'il y a pas de quoi avoir si peur…

--- Mais il y \emph{a} de quoi, chuchota Neville, tu l'as vu aujourd'hui, \emph{il y a} de quoi~!

--- Ils n'auraient rien fait de vraiment grave devant des témoins.
Leur arme principale est la peur.
C'est pour ça qu'ils \emph{te} prennent pour cible, parce qu'ils voient ta peur.
Je voulais que tu aies moins peur… te montrer que la peur est pire que l'objet de la peur… du moins c'est ce que je m'étais dit, mais le Choixpeau m'a montré que je me mentais à moi-même et que je l'avais fait pour m'amuser.
Donc c'est pour ça que je te demande pardon…

--- Tu m'as fait mal, dit Neville. Juste là.
Quand tu m'as attrapé pour m'éloigner d'eux.~»
Neville montra son bras, indiquant l'endroit où Harry l'avait saisi.
«~J'aurai peut-être un bleu ici, tellement tu as tiré fort.
En fait, tu m'as fait plus mal qu'aucun des Serpentard ne l'avait fait en me bousculant.

--- \emph{Neville~!} siffla Ernie. Il essayait de te \emph{sauver}~!

--- Je suis désolé, murmura Harry. Quand j'ai vu ça je me suis juste… vraiment mis en colère…~»

Neville le regarda sans sourciller.
«~Alors tu m'as éjecté avec force, tu as pris ma place et tu as dit “Bonjour, je suis le Survivant.”~»

Harry acquiesça.

«~Je pense qu'un jour tu seras vraiment cool, dit Neville. Mais pour l'instant tu ne l'es pas.~»

Harry avala le nœud soudain apparu au fond de sa gorge et s'en alla.
Il continua le long du couloir jusqu'à l'intersection, puis tourna à gauche dans un corridor et continua à marcher, à l'aveuglette.

Qu'était-il \emph{censé} faire~? Ne jamais se mettre en colère~?
Il n'était pas certain qu'il aurait pu faire quoi que ce soit sans se mettre en colère, et qui sait alors ce qui serait arrivé à Neville et à ses livres.
De plus, Harry avait lu assez de livres de fantasy pour savoir comment \emph{cela} se terminait.
Il essaierait de réprimer sa colère, il échouerait, et elle continuerait de se manifester encore et encore.
Et après un long voyage à la recherche de lui-même, il apprendrait au final que sa colère était une partie de lui et que c'était seulement en l'acceptant qu'il pourrait apprendre à l'utiliser avec sagesse.
\emph{Star Wars} était le seul univers dans lequel la réponse \emph{était} que vous étiez vraiment censé vous séparer de toute émotion négative, et quelque chose chez Yoda avait toujours poussé Harry à haïr ce petit crétin vert.

Donc le plan évident pour gagner du temps était de zapper le voyage à la recherche de soi-même et d'aller directement au moment où il se rendait compte qu'accepter sa colère comme une partie de lui était la seule manière de pouvoir la contrôler.

Le problème, c'était qu'il n'avait pas \emph{l'impression} de perdre le contrôle quand il était en colère.
Sa rage froide lui donnait le sentiment de se contrôler parfaitement.
Ce n'est que lorsqu'il repensait aux événements \emph{dans leur ensemble} qu'ils semblaient avoir… explosé hors de son contrôle, de façon incompréhensible.

Il se demanda l'importance que le Contrôleur du Jeu attachait à ce genre de choses, et si cela lui avait fait perdre ou gagner des points.
Harry avait le sentiment d'avoir perdu pas mal de points, et il était certain que la vieille dame dans le tableau lui aurait dit que la seule opinion qui comptait était la sienne.

Harry se demandait également si le Contrôleur du Jeu avait envoyé la professeure Chourave.
C'était logique~: la note avait menacé d'avertir les autorités du Jeu, et Chourave était arrivée.
Peut-être que Chourave \emph{était} la Contrôleuse du Jeu, la \emph{directrice de la Maison Poufsouffle} était la \emph{dernière} personne que quiconque aurait soupçonné, ce qui devait la mettre presque en haut de la liste de Harry.
Il avait aussi lu un ou deux romans policiers.

«~Alors, comment je m'en sors dans le jeu~?~» demanda Harry à voix haute.

Une feuille de papier vola par dessus sa tête, comme si quelqu'un l'avait jetée dans son dos -- Harry se retourna, mais il n'y avait personne -- et quand Harry se retourna de nouveau, la note se posait au sol.

On pouvait lire~:
\begin{theGameResults}\centering
\begin{align*}
\gameResult{Points pour le style}{10}
\gameResult{Points pour la réflexion}{−3~000~000}
\gameResult{Bonus de points maison serdaigle}{70}
\gameResult{Points actuels}{−2~999~871}
\gameResult{Tours restants}{2}
\end{align*}
\end{theGameResults}

«~\emph{Moins trois millions de points~?} s'indigna Harry auprès du couloir vide.
Ça me semble excessif~! Je fais appel auprès des Autorités du Jeu~!
Et comment je suis censé regagner trois millions de points en seulement deux tours~?~»

Une autre note vola au-dessus de sa tête.

\begin{theGameResults}\centering
\begin{align*}
\gameResult{Appel}{rejeté}
\gameResult{Poser les mauvaises questions}{−1~000~000~000~000~points}
\gameResult{Points actuels}{−1~000~002~999~871}
\gameResult{Tours restants}{1}
\end{align*}
\end{theGameResults}

Harry abandonna. Avec un tour restant, il n'avait pas d'autre choix que de tenter le tout pour le tout, même si cela ne semblait pas très prometteur.
«~Ma réponse est que le jeu représente la vie.~»

Une dernière feuille de papier vola au-dessus de sa tête, il lu:


\begin{center}
\begin{theGameResults}\centering
  \scshape
Tentative échouée\\
Échouée échouée échouée\\
Aiiiiiiiiiieeeeeeeeeeeeee\\
Points actuels~: moins l'infini\\
\underline{TU AS PERDU AU JEU}

Instruction finale~:\\
\end{theGameResults}
\begin{writtenNote}\centering
Va au bureau de la professeure McGonagall.
\end{writtenNote}
\end{center}

La dernière ligne était manuscrite, sa propre écriture.

Harry regarda la dernière ligne un moment, puis haussa les épaules.
Très bien. Le bureau de McGonagall.
Si c'était \emph{elle} la Contrôleuse du Jeu…

Bon, d'accord. Honnêtement, Harry n'avait absolument aucune idée de ce qu'il ressentirait si McGonagall était la Contrôleuse du Jeu.
Son esprit était totalement vide. C'était, littéralement, inimaginable.

Quelques portraits plus tard -- ce n'était pas un long trajet, le bureau de McGonagall n'était pas loin de sa classe de Métamorphose, du moins pas les lundis des années impaires -- Harry se trouvait face à la porte de son bureau.

Il frappa.

«~Entrez~», c'était la voix étouffée de la professeure.

Il entra.

%  LocalWords:  Bigshot Gleehhhhh Welp Weaselnose oo widdle glop sor ry
%  LocalWords:  Aiiiiiiiiiieeeeeeeeeeeeee
