\partchapter{Après-coup, quelque chose à protéger}{0}

\lettrine{A}{u} début, Anna avait été ravie de voir la finale de la coupe de Quidditch durer si longtemps -- en tant que Gryffondor, elle n'était qu'un simple témoin de l'événement, car ce n'était pas comme si Gryffondor l'avait déjà gagnée.
Contrairement à celle-ci, la finale de la coupe du monde de Quidditch, pour laquelle sa famille avait acheté des tickets à prix d'or, s'était finie en \emph{dix minutes}, ce qui avait été \emph{horrible}.
Les jeux de Quidditch modernes ne duraient plus assez longtemps, le Vif se faisait attraper bien trop vite.
Les aficionados du genre parlaient beaucoup de ce problème~: les enchantements pour balai avaient progressé tandis que le Vif s'en était tenu à la vitesse réglementaire, si bien que les matchs avaient été de plus en plus courts.
À un niveau professionnel, le sport s'était réduit à la profondeur des poches de l'équipe qui pourrait offrir le meilleur balai de course expérimental à son Attrapeur.
Les autres joueurs auraient aussi bien pu regarder le match depuis les gradins.

Tout le monde savait qu'il fallait agir, la situation avait empiré pendant des \emph{siècles} et était devenue \emph{intolérable}.
Mais le Comité de Quidditch de la Confédération Internationale des Sorciers baignait dans l'amertume habituelle de la CIS.
: des concours de hurlements entre Allemands et Bulgares.
Personne ne pouvait se mettre d'accord sur la \emph{façon} de réparer les règles.
Pour Anna, la réponse semblait évidente~: rendre le Vif assez rapide pour restaurer les parties de quatre à cinq heures du début du dix-neuvième siècle, l'Âge d'Or du Quidditch.
Sauf que les Belges trouvaient qu'un match professionnel devait durer deux heures, comme à \emph{La Belle Époque}, où la Belgique avait dominé le Quidditch, ces lunatiques d'Italiens voulaient revenir aux parties longues d'une semaine du quatorzième siècle, et les Puristes du Sang anglais, encore plus fous, n'avaient cesse de mentionner les occasionnels matchs de plus de vingt-quatre heures comme preuve que les balais ne pouvaient pas \emph{vraiment} s'être améliorés, puisque tout était mieux avant -- même si \emph{ce n'était pas comme ça que l'Interdit de Merlin fonctionnait}.

Elle était à cent pour cent du côté de Harry Potter~: il était temps que Poudlard laisse tomber ces lambins bredouillants et change simplement les règles, dès aujourd'hui.
Mais pas en \emph{éliminant le Vif}, ç'aurait été revenir au \emph{Kwidditch du onzième siècle}.
Peu importe que la directrice de Poufsouffle ait alors introduit cette innovation parce que l'un de ses élèves qui avait voulu jouer au jeu s'était révélé inadapté aux postes existants.
Un match qui pouvait se terminer n'importe quand était plus palpitant, et les Vifs avaient saisi l'attention du monde entier.

Anna avait s'était époumonée à défendre ce point de vue pendant une demi-heure, et elle avait tout à fait cessé de suivre la partie.
Par chance, elle s'était retrouvée assise à côté du Survivant et de son panneau, et elle était donc parvenue à asseoir sa position dès le début.

Elle savait, pas tout à fait consciemment, que si les règles du Quidditch changeaient vraiment aujourd'hui, ce serait \emph{la chose la plus importante qu'elle aurait jamais faite}.
Elle pouvait presque \emph{sentir} la pression du Temps autour d'elle, comme si le destin du Quidditch serait écrit le jour même et qu'elle se tenait au centre des événements… mais elle n'avait évidemment pas eu une note de Divination assez bonne pour véritablement pouvoir sentir ce genre de choses.

Quand le Survivant se leva pour aller aux toilettes, elle le remarqua à peine.

Mais elle lui jeta un regard lorsqu'il revint en traînant les pieds~: Harry Potter avait l'air un peu fatigué et vacillant.
Pourtant, son uniforme était en parfait état, comme s'il venait d'en enfiler un nouveau.

Elle lui prêta de nouveau attention une demi-heure plus tard, quand il parut tanguer, puis se pencher, et que ses mains couvrirent son front~; il avait l'air d'appuyer sur sa cicatrice.
L'idée l'inquiéta un peu~: tout le monde savait que \emph{quelque chose} clochait chez lui, et si sa cicatrice lui faisait mal, peut-être qu'une horreur scellée était sur le point de jaillir hors de son front et de manger tout le monde.
Elle abandonna cette idée et continua d'expliquer le Quidditch aux ignorants de son Histoire, à pleins poumons.

Elle lui prêta une attention toute particulière quand il se leva, ses mains toujours sur son front, et qu'il les baissa alors pour révéler que sa célèbre cicatrice en forme d'éclair était rouge vif, irritée.
Elle \emph{saignait}, et le sang coulait sur son front.

Elle se tut au beau milieu d'une phrase.
D'autres personnes se retournèrent, suivirent son regard.

«~Professeur McGonagall~?~»
dit Harry d'une voix instable.
Il avait des larmes au coin des yeux, ce qui la surprit~; le Survivant n'avait pas l'air d'être du genre à éclater en sanglots.
Il éleva un peu plus la voix, donnant l'impression qu'il avait du mal à parler.
«~Euh, professeur McGonagall~?~»

Le professeur McGonagall se détourna de l'équipe de Quidditch Poufsouffle avec qui elle était en plein débat.
Sous l'effet de la surprise, la directrice de Gryffondor écarquilla les yeux, et elle se mit alors à fendre la foule en courant presque.
«~Harry~! dit-elle.
Votre \emph{cicatrice}~!~»

Un silence se répandait autour de ce point central.

«~Je crois, dit Harry d'une voix toujours instable mais plus forte, je crois qu'il est revenu.
Je crois que je vois par l'esprit de Voldemort…~»

Lorsqu'elle entendit le nom de Vous-Savez-Qui, Anna fit un pas en arrière et faillit trébucher sur un gradin.
À côté d'elle, un garçon plus âgé laissa échapper un cri effarouché, et soudain le Survivant glapit encore plus fort.

«~\scream{Il les tue~!}~» hurla Harry Potter.

La moitié du stade de Quidditch se tourna vers lui.

«~Le rituel~! s'écria Harry Potter.
Le sang de ses serviteurs~!
Le sang, la vie~!
Il les appelle, il prend leur tête, leur sang, leur vie, pour rebâtir la sienne -- \scream{Le seigneur des ténèbres se relève, Voldemort est revenu}~!~»

Madame Bibine poussa un coup de sifflet strident et les balais de Quidditch qui ne s'étaient pas encore arrêtés commencèrent à ralentir.
Elle n'était pas certaine que ce soit une blague~: si c'était le cas, Survivant ou pas, il s'était mis dans un sacré pétrin.

Le professeur McGonagall leva sa baguette pour lancer un sortilège de silence et Harry Potter lui attrapa la main.

«~Attendez…~»
haleta Harry Potter d'une voix plus basse, mais toujours assez fort pour qu'elle et les gens proches puissent l'entendre clairement.
«~On peut l'arrêter… je vois son esprit, son erreur… on peut l'arrêter \emph{maintenant} …
\shout{Le portail est toujours ouvert~!
Elle le suit~!
Celle qu'il a tuée~!}~» Sa voix monta d'un cran et la bouche d'Anna s'ouvrit grand, dans un signe d'ahurissement soudain.
«~\scream{Reviens~!
Reviens, reviens, ranime-toi et arrête-le~!
Arrête-le, Hermione}~!~»

Puis il se tut.
Il rendit leur regard à ceux qui le dévisageaient.

Elle venait de décider que tout ceci devait être une blague d'un mauvais goût \emph{incroyable} quand un BOUM lointain mais très net retentit.

Harry Potter vacilla et tomba à genoux, et son cœur manqua un battement.
Une explosion de babillages agités les entoura.

Elle pouvait toujours entendre ce que Harry Potter disait, à côté du professeur McGonagall, qui s'était agenouillé.

«~Ça a marché, haleta-t-il, elle l'a eu, il est parti.

--- \emph{Quoi~?}~» s'écria le professeur McGonagall, et elle regarda autour d'elle.
«~\emph{Silence~!
Silence, vous tous~!} Harry, qu'est-ce qui s'est passé~?~»

Harry Potter parlait vite, mais fort.
«~Voldemort… a essayé de revenir… il a fait venir les Mangemorts \emph{et il les a tués}, il leur a volé leur sang et leur vie… le corps de Hermione était là, je ne sais pas pourquoi, peut-être que Voldemort voulait en faire usage…
Voldemort est revenu, il s'est ressuscité, mais Hermione \emph{l'a suivi} et elle l'a \emph{détruit}, il est parti, c'est fini.
Ça s'est passé dans un cimetière près de Poudlard, c'est…~»
il se leva, toujours vacillant, «~je crois que c'est par là.~»
Il indiqua la direction d'où le BOUM était venu, «~je ne sais pas si c'est très loin.
Le son a mis environ vingt secondes à arriver, alors peut-être deux minutes en balai…~»

D'un geste si gracieux qu'il parut inconscient, le professeur McGonagall se mit en position et dit~: «~\emph{Expecto Patronum.}~» Elle s'adressa au chat lumineux dès qu'il apparut.
«~Vas voir Albus, dis-lui de venir immédiatement…

--- Dumbledore est parti~! s'écria Harry Potter.
Le directeur est parti, professeur McGonagall~!
Le Seigneur des Ténèbres l'a piégé, il a inversé une sorte de piège que le directeur avait prévu et Dumbledore est piégé hors du Temps, il est parti~!~»

Les bavardages horrifiés autour d'eux montèrent d'une octave.

«~Vas voir Albus~!~»
dit le professeur McGonagall à son Patronus.

Le chat nimbé de Lune se contenta de regarder tristement le professeur McGonagall, et Anna inspira avec horreur, sentant que quelqu'un venait de lui donner un coup à l'estomac.
C'était réel, tout cela était réel, ce n'était pas une blague.

«~Professeur McGonagall, Hermione est en \emph{vie}~!~»
Harry Potter éleva à nouveau la voix.
«~Elle est vraiment en vie, ce n'est pas un Inferius ni rien, et elle est toujours là-bas, dans le cimetière~!

--- \emph{Un balai~!}~» hurla le professeur McGonagall.
Elle se tourna vers les joueurs qui flottaient, immobiles au-dessus du terrain de Quidditch.
«~J'ai besoin d'un balai.
\shout{Maintenant}~!~»

En dépit de tout, Anna leva une main en signe de protestation muette, puis se ressaisit au moment où les Attrapeurs Serdaigle et Serpentard arrivèrent à toute vitesse (une excellente décision stratégique puisqu'ils ne faisaient effectivement rien).

Harry Potter récupérait déjà un autre balai de sa bourse, muni plusieurs sièges.

Le professeur McGonagall vit cela et hocha fermement la tête.

«~Vous restez ici, M. Potter, à moins que vous n'ayez une excellente raison de venir.
J'y vais tout de suite.

--- Non~!~»
glapit le professeur Flitwick qui s'était frayé un petit chemin dans la foule, parfois en passant entre des jambes.
Il avait les yeux écarquillés et semblait être sur le point défaillir.
«~Vous devez rester à Poudlard, Minerva~!
Vous… vous êtes la…~»
il semblait avoir du mal à parler.

Le professeur McGonagall pivota pour lui faire face et s'arrêta soudain.
Son sang avait quitté son visage.

Puis elle prit le balai des mains de Harry Potter et le tendit au petit demi-gobelin.
«~Filius~», dit-elle vivement.
Tout début de panique avait quitté sa voix, et elle parlait maintenant de son sec accent Écossais, comme si elle avait donné cours.
«~Cherchez le cimetière dont a parlé M. Potter, trouvez Mlle Granger.
Transplanez-la à Sainte Mangouste et restez à ses côtés.

--- Je pense… dit Harry Potter d'une voix rauque.
Je pense que des Métamorphoses ont pu être utilisées pendant la bataille… le professeur Quirrell a essayé de combattre Voldemort… faites attentions…~»

Filius Flitwick hocha la tête sans s'interrompre et partit sur le balai.

«~Le professeur Quirrell est mort~!~»
vagit Harry Potter.
La souffrance dans sa voix était clairement audible.
«~Il est mort~!
Le Seigneur des Ténèbres l'a tué~!
Son corps…~»
Harry Potter s'étrangla.
«~Il est là, dans le cimetière.~»

Elle flancha à nouveau, comme un autre coup porté à l'estomac.
Le professeur Quirrell avait été… l'un de ses professeurs préférés, de loin.
Il lui avait fait revoir tout ce qu'elle avait pensé de Serpentard~; elle avait su, sans le ressentir, qu'il allait probablement mourir très bientôt, mais d'entendre qu'il était vraiment, véritablement mort…

Le Survivant s'assit sur le banc comme si ses jambes ne pouvaient plus le soutenir.

Le professeur McGonagall se tourna vers la foule et toucha sa gorge de sa baguette.

«~\shout{Le Quidditch est terminé}, tonna sa voix amplifiée.
\shout{Retournez à vos dortoirs…}

--- \emph{Non~!}~» s'écria Harry Potter.

Le professeur McGonagall se tourna vers lui.

Des larmes coulaient le long des joues du Survivant et on aurait dit que l'interruption l'avait surpris autant qu'elle avait surpris les autres.
«~C'était le dernier plan du professeur Quirrell~», dit-il d'une voix brisée.
Le survivant regarda les joueurs de Quidditch qui s'étaient rapprochés, comme pour s'adresser directement à eux.
«~Son dernier plan.~»

Harry Potter fut lévité jusqu'à l'infirmerie par le professeur McGonagall.
Les autres professeurs coururent gérer on-ne-sait-quoi, ne laissant que les professeurs Sinistra et Bibine derrière.
Dans le stade, les rumeurs couraient follement~; Anna faisait de son mieux pour répéter tout ce qu'elle avait entendu.
Quelque chose était arrivé à Dumbledore, des Mangemorts avaient été appelés puis tués (non, Harry Potter n'avait pas précisé lesquels), le professeur Quirrell était allé faire face au Seigneur des Ténèbres et y avait perdu la vie, Vous-Savez-Qui était revenu pour mourir une fois de plus, le professeur Quirrell était mort, il était mort.

La plupart des élèves finirent par errer jusqu'à leurs dortoirs pour tenter d'y dormir.

Anna resta dans le stade et regarda la fin du match, faisant fi de son besoin de sommeil et de ses yeux qui s'humectaient souvent de larmes.

L'équipe Serdaigle combattit vaillamment.

Mais ce jour-là, aucune équipe de Quidditch au monde n'aurait pu vaincre les Serpentard.

L'aube illuminait le ciel quand les Serpentard remportèrent leur dernier match, la coupe de Quidditch, et la coupe des Maisons.
%  LocalWords:  Confédération Internationale des Comités Magiciens Époque
%  LocalWords:  Kwidditch
