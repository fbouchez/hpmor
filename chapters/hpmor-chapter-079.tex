\partchapter{Compromis Tabous}{I}

\lettrine{L}{es} Les mots chutèrent dans l'esprit de Harry et brisèrent ses pensées en un milliers d'éclats d'incrédulité. Le choc causé par l'adrénaline causa tant de confusion que…

«~Elle… dit-il. Elle… Elle ne… QUOI~?~»

Les Aurors ne faisaient pas attention à lui. Komodo parla de nouveau, toujours de cette voix fade.

«~M. Malfoy a repris conscience à Sainte Mangouste et vous a nommé, vous, Hermione Granger, comme son agresseur. Il a répété ces accusations sous l'effet de deux gouttes de Veritaserum. Le sortilège de refroidissement sanguin que vous lancez sur M. Malfoy l'aurait tué s'il n'avait pas été trouvé et soigné, et il nous faut supposer que vous saviez que ce sortilège est fatal. Vous êtes donc inculpée d'un crime sérieux~: je vous arrête pour tentative de meurtre. Vous serez mise sous garde du ministère et interrogée sous l'effet de trois gouttes de Veritaserum…

--- \emph{Vous êtes dingues~?}~» les mots jaillirent de la bouche de Harry alors qu'il se levait violemment de la table Serdaigle, un instant avant que la main de l'Auror Butnaru ne se pose avec force sur son épaule. Harry l'ignora. «~Vous essayez d'arrêter \emph{Hermione Granger}, la plus gentille fille de Serdaigle, elle aide les Poufsouffle à faire leurs devoirs, elle \emph{mourrait} plutôt que d'essayer de tuer \emph{qui que ce soit…}~»

Le visage de Hermione s'était effondré. «~Je l'ai fait, chuchota-t-elle. C'était moi.~»

Un autre rocher immense tomba sur les pensées de Harry, broya leur fragile structure, fit tomber des fragments de compréhension en poussière.

Le visage de Dumbledore semblait avoir vieilli de plusieurs décennies en quelques secondes.

«~Pourquoi, Mlle Granger~?~», dit-il d'une voix qui dépassait à peine le niveau du murmure. «~Pourquoi feriez-vous une chose pareille~?

--- Je, dit Hermione, je, je suis… Désolée… Je ne sais pas pourquoi j'ai…~» Elle sembla s'effondrer sur elle-même, sa voix n'était plus que des sanglots, et les seuls mots qu'on put comprendre furent~: «~Je pensais… L'ai tué… Désolée…~»

Et Harry aurait dû dire quelque chose, faire quelque chose, bondir de son siège, étourdir les trois Aurors, trouver quelque chose d'incroyablement intelligent à faire ensuite, mais les fragments par deux fois brisés de ses pensées ne pouvaient plus rien produire. La main de Butnaru ramena gentiment mais fermement Harry sur son banc et Harry se retrouva \emph{coincé} là comme s'il y avait été collé. Il essaya de se saisir de sa baguette pour lancer un \emph{Finite} mais elle ne voulait pas sortir de sa poche. Les trois Aurors et Dumbledore escortèrent Hermione hors de la grande salle au milieu d'une tempête naissante de cris de protestations et les portes commencèrent à se refermer derrière eux -- plus rien n'avait de sens, l'irréalité avait atteint son apogée, comme s'il avait été transporté dans un monde parallèle, l'esprit de Harry revint alors en un éclair à un autre jour de confusion, et dans un acte d'inspiration désespérée il comprit enfin ce que les jumeaux Weasley avaient fait à Rita Skeeter. Sa voix s'éleva, devenue un cri~: «~\scream{Hermione ce n'était pas toi on t’a lancé un sort de faux souvenirs}~!~»

Mais les portes s'étaient déjà refermées.

\later

Minerva n'aurait pas pu rester immobile. Elle marchait de long en large dans le bureau du directeur, s'attendant à demi consciemment à ce que Severus ou Harry lui disent d'arrêter et de s'asseoir, mais ni le maître des potions ni le Survivant ne semblaient très préoccupés par elle et leurs regards à tous les deux étaient concentrés sur Albus Dumbledore, qui venait d'émerger de la cheminée. Il y avait des sons environnants mais personne ne les entendait. Severus semblait plus détaché que jamais, assis dans une petite chaise rembourrée située à côté du bureau du directeur. Le vieux sorcier s'élevait, terrible et droit, à côté du feu encore allumé, vêtu d'une robe de nuit sans lune, irradiant de pouvoir et de désarroi. Toutes les pensées de Minerva étaient faites de confusion et d'horreur pure. Harry Potter était assis sur un tabouret de bois, les doigts rivés au siège, et ses yeux étaient faits de furie et de glace.

À six heures et trente-trois minutes du matin, Quirinus Quirrell avait appelé Sainte Mangouste par cheminette depuis son propre bureau pour y faire immédiatement récupérer Drago Malfoy. Le professeur Quirrell avait trouvé M. Malfoy dans la salle des trophées de Poudlard, à l'article de la mort, victime des effets prolongés d'un sortilège de refroidissement sanguin qui avait lentement abaissé la température de son corps. Le professeur Quirrell avait immédiatement dissipé le charme, avait lancé des sortilèges stabilisants sur M. Malfoy et l'avait fait léviter jusqu'à son bureau afin de l'envoyer à Sainte Mangouste par cheminette pour que son traitement s'y poursuive~; Après cela, le professeur Quirrell avait informé le directeur, mentionnant brièvement les faits avant de disparaître dans la cheminette~: les Aurors, avisés par Sainte Mangouste, avaient requis sa présence afin qu'il soit questionné.

Il était clair que l'intention derrière le sortilège de refroidissement sanguin avait été de tuer Drago Malfoy si lentement que les systèmes de sécurité de Poudlard, réglés pour détecter les blessures, ne se déclenchent pas. Interrogé par les Aurors, le professeur Quirrell leur avait dit qu'il avait lancé plusieurs sortilèges de surveillance sur Drago Malfoy après son retour des vacances de Yule. Il avait lancé ces sortilèges de surveillance, car il avait appris l'existence d'une personne qui possédait un motif de faire du mal à M. Malfoy. Le professeur Quirrell avait refusé d'identifier cette personne. Les sortilèges de surveillances lancés par le professeur avaient été déclenchés lorsque la santé de M. Malfoy était tombée en dessous d'un certain niveau plutôt qu'à cause de changements soudains et avait donc alerté le professeur Quirrell avant que M. Malfoy ne meure.

Deux gouttes de Veritaserum, suffisantes pour empêcher M. Malfoy d'omettre tout qualificatif de ses déclarations, qu'il soit mélioratif ou péjoratif, avaient révélées que ce dernier avait -- légalement selon les lois des maisons Nobles, illégalement selon les règles de Poudlard -- provoqué Hermione Granger en duel. M. Malfoy avait gagné le duel mais avait alors, en partant, été attaqué dans le dos par Mlle Granger qui lui avait lancé un sortilège d'étourdissement. M. Malfoy ignorait tout de ce qui s'était passé ensuite.

Trois gouttes de Veritaserum, qui l'avaient forcée à spontanément fournir toute information en rapport avec l'affaire, avaient permises de voir Hermione Granger confesser qu'elle avait étourdi M. Malfoy par-derrière et avait alors, sur un coup de colère, lancé le sortilège de refroidissement sanguin à son encontre avec l'intention de le tuer assez lentement pour échapper à toute détection par les systèmes de sécurité de Poudlard dont elle avait appris le fonctionnement en lisant \emph{Poudlard~: Une Histoire}. Elle avait été horrifiée par son acte en se réveillant le lendemain matin mais n'avait dit à personne ce qu'elle avait fait, croyant que M. Malfoy était déjà mort -- comme il l'aurait certainement été au bout de sept heures si la magie de son propre corps n'avait pas résisté aux effets du sortilège de refroidissement sanguin.

«~Son procès, dit Albus Dumbledore, est prévu pour demain, midi.

--- \emph{Quoi~?}~» les mots jaillirent de Harry Potter. Le Survivant ne se leva pas de sa chaise mais Minerva vit les doigts de ce dernier blanchir lorsqu'ils agrippèrent le tabouret de bois sur lequel il était assis. «~C'est de la folie~! On ne peut pas faire une enquête de police en vingt-quatre heures…~»

Le maître des potions éleva la voix.

«~Nous ne sommes pas en Angleterre \emph{moldue}, M. Potter~!~» Le visage de Severus était aussi inexpressif que d'habitude mais le mordant de ses mots était acéré. «~Les Aurors ont une accusation sous Veritaserum et une confession sous Veritaserum. De leur point de vue, cette enquête est \emph{terminée}.

--- Pas tout à fait~», dit Dumbledore au moment où Harry sembla être sur le point d'exploser. «~J'ai insisté auprès d'Amelia afin que cette affaire soit l'objet d'un examen des plus approfondis. Malheureusement, comme le funeste duel s'est déroulé à minuit…

--- Duel \emph{supposé}, dit Harry d'un ton brusque.

--- Comme le duel \emph{supposé} était à minuit -- oui Harry, tu as tout à fait raison -- il est hors de portée de tout retourneur de temps…

--- \emph{Supposément} aussi, dit le Survivant avec froideur. Et assez \emph{étrangement}, puisque la présumée coupable de meurtre ignore l'existence des Retourneurs de Temps. J'espère qu'un Auror invisible a été envoyé dans le passé aussi loin que possible afin d'observer…~»

Dumbledore inclina la tête. «~J'y ai été \emph{moi-même}, Harry, au moment où je l'ai appris. Mais lorsque j'ai atteint la salle des trophées, M. Malfoy était déjà inconscient et Mlle Granger était partie…

--- Non, dit Harry Potter. Vous avez atteint la salle des trophées et avez vu Drago inconscient. C'est tout ce que vous avez observé, M. le directeur. Vous n'avez pas \emph{vu} Hermione dans la salle, pas plus que vous ne l'avez vue partir. Distinguons les observations des inférences.~» Le visage du garçon se tourna vers elle. «~Imperius, Oubliettes, Sortilège de faux souvenirs, Légilimancie. Professeur McGonagall, ai-je omis un sortilège affectant l'esprit qui aurait pu pousser Hermione à faire ça ou à lui faire croire qu'elle l'a fait~?

--- Le sortilège de confusion~», dit-elle. Et elle n'avait jamais étudié les arts noirs, mais elle savait… «~ainsi que certains rituels noirs. Mais aucun d'entre eux ne pourrait être réalisé à Poudlard sans provoquer d'alarme.~»

Le garçon hocha la tête. Ses yeux s'adressaient toujours directement à elle.

«~Lesquels de ces sortilèges peuvent être détectés~? Lesquels les Aurors essaieraient-ils de détecter~?

--- Le sortilège de confusion s'estomperait au bout de quelques heures, dit-elle après avoir rassemblé ses pensées pendant quelques instants. Mlle Granger se souviendrait de l'Imperius. Oubliettes ne peut être détecté par aucun moyen connu, mais seul un professeur aurait pu lancer ce sortilège sur un élève sans provoquer d'alerte dans les systèmes de sécurité de Poudlard. La Légilimancie… ne peut être détectée que par un autre Legilimens, il me semble…

--- J'ai demandé à ce que Mlle Granger soit examinée par le Legilimens de la cour, dit Dumbledore. L'examen a montré…

--- Faisons-nous confiance à cet homme~? dit Harry.

--- À cette femme, dit Dumbledore. Sophie McJorgenson, dont je me souviens comme d'une honnête élève de Serdaigle, et elle liée par un Serment Inviolable à révéler sincèrement ce qu'elle a vu…

--- Quelqu'un d'autre aurait-il pu se polynectarer en elle~?~» dit Harry Potter, interrompant de nouveau. «~Qu'avez-vous \emph{observé}, M. le directeur~?~»

Albus dit d'un ton grave, «~une personne qui ressemblait à Mme McJorgenson nous a dit qu'un seul Legilimens avait légèrement touché l'esprit de Mlle Granger il y a plusieurs mois. Cela date de janvier, Harry, lorsque j'ai communiqué avec Mlle Granger au sujet d'un certain Détraqueur. Cela était attendu, mais ce à quoi je ne m'attendais pas, c'est le reste de ce que Sophie a découvert.~» Le vieux sorcier pivota et regarda dans le feu de cheminette, laissant les flammes orange se refléter sur son visage. «~Comme tu l'as dit, Harry, un sortilège de faux souvenirs constitue une possibilité~; lorsqu'ils sont parfaitement lancés, ils sont indistinguables de véritables souvenirs…

--- Cela ne me surprend pas, interrompit Harry. Les études ont montré que les souvenirs humains sont plus ou moins réécris à chaque fois que l'on se les remémore…

--- Harry~», dit doucement Minerva, et la bouche de Harry se referma d'un coup sec.

Le vieux sorcier continua~:

«~… mais un sortilège de faux souvenirs d'une telle qualité demande autant de temps qu'il en faut pour créer un véritable souvenir. Créer un souvenir détaillé de dix minutes exige dix minutes d'efforts. Et selon le Legilimens de la cour,~» le visage d'Albus semblait maintenant encore plus fatigué et ridé qu'avant, «~Mlle Granger est obsédée par M. Malfoy depuis le jour où Severus… lui a crié dessus. Elle a songé à comment M. Malfoy pourrait être en lice avec le professeur Rogue, comment il pourrait se préparer à faire du mal à elle et à Harry, elle l'a imaginé pendant des heures, tous les jours… il serait impossible de créer de faux souvenirs d'une durée aussi longue.

--- L'apparence de la folie…~» murmura doucement Severus, comme s'il se parlait à lui-même. «~Cela \emph{pourrait-il} être naturel~? Non, c'est trop désastreux pour être un pur accident~; trop utile à \emph{quelqu'un}, je n'en doute pas. Une drogue moldue, peut-être~? Mais cela ne suffirait pas… la folie de Mlle Granger devrait être \emph{guidée}…

--- Ah~! dit soudain Harry. J'ai compris. Le \emph{premier} sortilège de faux souvenirs a été lancé sur Hermione après que le professeur Rogue lui eut crié dessus et lui montrait, disons, que Drago et le professeur Rogue complotaient dans le but de la tuer. La nuit dernière, le faux souvenir a été \emph{enlevé} par un sortilège d'Oubliettes, laissant derrière les souvenirs de sa fixation sans raison apparente sur Drago, et au même moment, lui et Drago ont reçu de faux souvenirs du duel.~»

Minerva cilla de surprise. Elle n'aurait pas pensé à cette possibilité en mille ans de réflexion.

Le maître de des potions fronçait pensivement les yeux, comme absorbé.

La \emph{réaction} à un sortilège de faux souvenirs est difficile à prédire, M. Potter, sans Légilimancie. Le sujet ne réagit pas toujours comme prévu lorsqu'il se souvient pour la première fois du faux souvenir. Cela aurait été un stratagème risqué. Mais j'imagine que c'est l'un des moyens que le professeur Quirrell aurait pu employer.

«~\emph{Le professeur Quirrell~?} dit Harry. Quel motif pourrait-il avoir pour…~»

Le maître des potions dit sèchement~: «~Le professeur de Défense est toujours un suspect, M. Potter. Vous remarquerez cette tendance, avec le temps.~»

Albus leva une main, demandant le silence, et leurs têtes se tournèrent pour le regarder. «~Mais dans ce cas, il y a un autre suspect, dit doucement Albus. Voldemort.~»

Le plus mortel des mots imprononçables sembla faire écho à travers la pièce et oblitérer la chaleur des flammes orange du foyer.

«~J'ignore, dit lentement le vieux sorcier, je ne sais que trop peu des méthodes par lesquelles Voldemort compte atteindre l'immortalité. Je crois qu'il s'est mis en quête de ces livres avant que je ne le fasse. Tout ce que j'ai pu trouver, ce sont d'anciennes légendes, éparpillés entre trop de volumes pour qu'il puisse tous les subtiliser. Mais trouver la vérité entre plusieurs histoires constitue aussi l'un des arts sorciers, et je me suis efforcé d'y parvenir. Il y a un sacrifice humain, un meurtre~; de cela, je suis certain. Commis du sang le plus froid, la victime morte d'horreur. Et de vieilles, de très vieilles légendes de sorciers possédés, commettant des actes fous, se clamant être des Seigneur des Ténèbres qu'on croyait vaincus~; et il y a souvent un appareil possédé par ce Seigneur des Ténèbres que le sorcier manie…~» Albus regarda Harry, les yeux anciens scrutant les pupilles plus jeunes. «~Harry, je pense -- mais tu diras que ce n'est qu'une inférence -- que commettre un meurtre sépare l'âme en deux. Que par un rituel de l'horreur la plus sombre, le fragment d'âme arraché est enchaîné à ce monde. À un objet matériel appartenant à ce monde. Et qui doit être ou qui devient alors un artefact chargé de puissance.~»

\emph{Horcruxe}. Le terrible mot fit écho dans l'esprit de Minerva, même s'il semblait que -- pour une raison qu'elle ignorait -- Albus se refusait à prononcer ce mot devant Harry.

«~Et par conséquent, conclut doucement le vieux sorcier, le reste de l'âme est liée à cette partie enchaînée et demeure ici alors que son corps a été détruit. Je pense que ce serait une existence triste et douloureuse, plus vile que celle d'un esprit, plus vile que celle du pire des fantômes…~» Les yeux du vieux sorcier étaient fixés sur ceux de Harry, qui le regardait en retour en plissant les yeux. «~Il faudrait du temps pour que cette âme mutilée récupère un semblant de vie. Je crois que c'est pour cela que nous avons eu ce sursis de dix ans, que c'est pour cela que Voldemort n'est pas encore revenu. Mais à force… ce revenant pourrait devenir capable de s'élever à nouveau.~» Le vieux sorcier s'exprimait avec une précision sinistre. «~Il est clair, selon les histoires, que les Seigneur des Ténèbres qui revinrent en possédant le corps d'un autre étaient maîtres d'une magie inférieure à celles qu'ils avaient jadis connue. Je ne pense pas que Voldemort se satisfera de cela. Il choisirait un autre moyen de revivre. Mais Voldemort était plus Serpentard que Salazar, il se saisissait de la moindre opportunité. Il \emph{utiliserait} cet état piteux, il \emph{utiliserait} sa capacité à prendre possession d'un autre s'il avait une raison de le faire. S'il pouvait bénéficier de… l'explicable furie d'un autre.~» La voix d'Albus n'était presque plus qu'un murmure. «~Je soupçonne que c'est cela qui est arrivé à Mlle Granger.~»

La gorge de Minerva devint très sèche. «~Il est \emph{ici}, hoqueta-t-elle. \emph{Ici}, à \emph{Poudlard}…~»

Puis elle se tut, car la \emph{raison} pour laquelle Voldemort était venu à Poudlard…

Le vieux sorcier ne la regarda que brièvement et dit, toujours de ce murmure~: «~Je suis navré Minerva, tu avais raison.~»

La voix de Harry était tranchante.

«~Raison à quel sujet~?

--- Son meilleur moyen de revivre, dit Dumbledore avec gravité. La voie qu'il désire le plus, par laquelle il s'élèverait à nouveau, plus fort et plus terrible que jamais auparavant. Elle est gardée ici, dans ce château…

--- Excusez-moi, dit poliment Harry. Est-ce que vous êtes stupide~?

--- Harry~», dit-elle, mais sa voix était dénuée de force.

«~Je veux dire, peut-être que vous n'avez pas remarqué, M. le directeur, mais ce château est plein D'ENFANTS…

--- \emph{Je n'avais pas le choix~!}~» mugit Dumbledore. Les yeux bleus étincelaient à présent sous les verres en demi-lune. «~Je ne la \emph{possède} pas, cette chose que Voldemort désire. Elle appartient à un autre et elle est maintenue ici avec \emph{son} consentement~! J'ai \emph{demandé} si elle pouvait être gardée au département des mystères. Mais \emph{il} ne l'a pas permis -- il a dit qu'elle devait être dans l'enceinte de Poudlard, sous la protection de ses Fondateurs…~» Dumbledore se passa une main se le front. «~Non~», continua le vieux sorcier d'une voix plus basse. «~Je ne peux pas lui transmettre le blâme. Il a raison. Cette chose contient trop de pouvoir, trop de choses désirée par les hommes. J'ai reconnu que le piège devait être tendu derrière les murs de Poudlard, là où mon pouvoir réside.~» Le vieux sorcier inclina la tête. «~Je savais que Voldemort parviendrait à se faufiler jusqu'ici et je comptais le prendre au piège. Je ne pensais pas -- je ne rêvais pas -- qu'il s'éterniserait dans une forteresse ennemie un instant de plus que nécessaire.

--- Mais, dit Severus, assez perplexe, que le Seigneur des Ténèbres pourrait-il avoir à gagner en tuant le seul descendant de Lucius~?

--- Remarque d'ordre pratique~», dit Harry Potter, un tranchant acéré dans la voix. «~Les motifs de la personne qui est derrière tout cela, quelle qu'elle soit, ne sont pas le problème principal. Notre principale priorité pour l'instant est qu'un élève innocent de Poudlard a des \emph{problèmes}~!~»

Les yeux verts se braquèrent sur les bleus lorsque Albus Dumbledore rendit son regard au Survivant…

«~Tout à fait, M. Potter~», dit Minerva, elle n'y avait même pas réfléchi, les mots semblaient juste sauter hors de ses lèvres. «~Albus, qui surveille Mlle Granger en ce moment~?

--- Le professeur Flitwick est allé la voir, dit le directeur.

--- Elle a besoin d'un \emph{avocat}, dit Harry. Quelqu'un qui se contente de dire “c'était moi” à la police…

--- Malheureusement~», dit Minerva en se rapprochant sans s'en rendre compte du ton sévère du professeur McGonagall, «~je doute qu'un avocat serait d'une quelconque utilité à Mlle Granger à ce stade, M. Potter. Elle fera face au jugement du Magenmagot et il est hautement improbable qu'il la libèrent à cause d'un vice de procédure.~»

Harry la regardait avec une expression des plus incrédules, comme si la suggestion qu'Hermione puisse se passer d'un avocat était analogue à celle qu'il faille l'immoler par le feu.

«~Elle a raison, M. Potter, dit doucement Severus. Peu de procédures judiciaires dans ce pays font appel à des avocats.~»

Harry souleva ses lunettes et se frotta les yeux.

«~Très bien. Et comment exactement tire-t-on Hermione d'affaire~? J'imagine que ce serait trop d'espérer qu'une fois tous les avocats partis, les juges comprennent les concepts de “sens commun” et de “probabilité à priori” assez bien pour se rendre compte que les filles de douze ans ne commettent essentiellement jamais de meurtre~?

--- C'est au Magenmagot qu'elle fait face, dit Severus. Les plus anciennes des maisons Nobles, et certains autres sorciers influents.~» Le visage de Severus se tordit en quelque chose qui s'approchait de son sarcasme habituel. «~Quant à ce qu'ils fassent preuve de sens commun… vous pourriez aussi bien vous attendre à ce qu'ils vous fassent un sandwich au bacon, Potter.~»

Harry hocha la tête, mâchoire serrée.

«~À quelle peine Hermione fait-elle face, exactement~? Baguette confisquée, expulsion…

--- Non, dit Severus. Rien d'aussi léger. Faites-vous exprès de ne pas comprendre, Potter~? Elle fait face au \emph{Magenmagot}. Il n'y a pas de peine pré-écrite. Il n'y a que le vote.~»

Harry Potter murmura~: «~\emph{L'autorité de la loi, en ces temps complexes, s'est avérée déficiente. Nous aimons mieux l'autorité des hommes, elle est beaucoup plus efficace…} Alors il n'y a aucune contrainte légale~?~»

De la lumière miroita sur les lunettes en demi-lune du vieux sorcier~; il parla avec précaution mais pas sans colère.

«~Légalement, Harry, nous avons affaire à une dette de sang de Hermione Granger envers la maison Malfoy. Le Lord des Malfoy propose un remboursement de cette dette et le Magenmagot vote sur cette proposition. C'est tout.

--- Mais… dit lentement Harry. Lucius a été réparti à Serpentard, il \emph{doit} se rendre compte qu'Hermione n'est qu'un pion. Pas celle à qui il devrait vraiment en vouloir. Pas vrai~?

--- Non, Harry Potter, dit lourdement Albus Dumbledore. C'est ce que tu \emph{souhaites} que Lucius Malfoy pense. Lucius Malfoy lui-même… ne partagera pas ton désir de le voir penser ainsi.~»

Harry regarda le directeur d'un regard qui devint de plus en plus froid tandis qu'au même moment, Minerva dut resserrer sa prise sur ses émotions, arrêter ses déambulations et essayer de respirer. Elle avait essayé de ne pas y penser, de garder ses pensées à l'écart, mais elle le savait. Elle l'avait su depuis l'instant où elle avait entendu la nouvelle. Elle pouvait le voir dans les yeux d'Albus…

«~Risque-t-elle la peine capitale~?~» dit doucement Harry, et des frissons descendirent le long de la colonne vertébrale de Minerva lorsqu'elle entendit les nuances contenues dans la voix qui avait prononcé ces mots.

«~Non, dit Albus. Non, pas le Baiser, pas Azkaban, pas pour une élève de Poudlard en première année. Notre pays n'est pas perdu à ce point, pas encore.

--- Mais Lucius Malfoy, dit Severus d'une voix sans timbre, ne sera certainement pas satisfait de ne voir que sa baguette confisquée.

--- Très bien, dit Harry avec autorité. Comme je vois les choses, nous avons essentiellement deux plans d'action. Premier plan, trouver le véritable coupable. Second plan, pouvoir influencer Lucius. Le professeur Quirrell a sauvé la vie de Drago, cela crée-t-il une dette de sang de la maison Malfoy envers lui qu'il pourrait racheter pour annuler celle de Hermione~?~»

Minerva cilla à nouveau de surprise.

«~Non~», dit Dumbledore. Le vieux sorcier secoua la tête. «~C'était une idée maline, mais non, Harry, j'ai peur que non. Même au cas peu probable où le professeur de Défense s'avérerait appartenir à une maison Noble, il existe une exception lorsque le Magenmagot soupçonne qu'une dette a été créée délibérément et dans ce but précis. Et le professeur de Défense est loin d'être au-dessus de tout soupçon. C'est ce que Lucius soutiendrait.~»

Harry hocha une fois la tête, le visage serré.

«~Un roturier peut donc avoir une dette de sang envers une maison Noble, mais pas l'inverse. Je ne sais pas pourquoi, mais je ne suis pas surpris. Mais la maison Potter \emph{est} une maison Noble, à ce que j'ai compris. M. le directeur, je sais que j'ai dit que je ne le ferai pas -- mais étant donné les circonstances -- cette fois où Drago m'a lancé un sortilège de torture, est-ce une dette suffisante pour…

--- Non~», dit le vieux sorcier (au moment même où elle lâchait un «~Quoi~?~» et où Severus soulevait un sourcil). «~Cela n'aurait pas suffi, et ce n'est plus une dette du tout. Tu es un Occlumens et tu ne peux pas témoigner sous Veritaserum. Drago Malfoy pourrait être purgé de son souvenir de l'événement avant de pouvoir témoigner…~» Albus hésita. «~Harry… quoi que tu aies fait avec Drago, tu dois partir du principe que Lucius Malfoy sera bientôt au courant.~»

La tête de Harry plongea dans ses mains.

«~Il va donner du Veritaserum à Drago.

--- Oui~», dit doucement Albus.

Le Survivant ne dit rien, assit, la tête entre les mains.

Le maître des potions avait l'air sincèrement abasourdi.

«~Drago essayait \emph{vraiment} d'aider Mlle Granger, dit-il. Vous… Potter, vous l'avez \emph{vraiment}…

--- Fait changer de camp~? dit Harry entre ses mains. J'en étais à peu près aux trois quarts. Je lui ai appris le Patronus et tout ça. Mais maintenant, je ne sais pas ce qui va se passer.

--- Voldemort nous a sévèrement atteints aujourd'hui~», dit Albus. La voix du vieux sorcier rappelait l'apparence du garçon dont la tête était entre ses mains. «~Il a pris deux de nos pièces d'un seul… non. J'aurais dû le voir plus tôt. Il a pris deux des pièces de \emph{Harry} d'un seul coup. Voldemort a recommencé son jeu, pas contre moi mais contre \emph{Harry}. Voldemort connaît la prophétie, il sait qui sera son dernier ennemi. Il n'attend pas d'avoir à faire face à Hermione Granger et Drago Malfoy aux côtés de Harry lorsqu'ils auront grandi. Il s'en prend à eux \emph{maintenant}.

--- Peut-être est-ce Vous-Savez-Qui, peut-être pas, dit Harry d'une voix légèrement instable. Ne rétrécissons pas l'espace de nos hypothèses prématurément.~» Harry inspira et abaissa les mains. «~L'autre possibilité est de coincer le véritable coupable avant le procès -- ou au moins de trouver de bonnes preuves que \emph{quelqu'un d'autre} est coupable.

--- M. Potter, dit Minerva. Le professeur Quirrell a parlé aux Aurors d'une personne ayant un motif de faire du mal à Malfoy. Savez-\emph{vous} de qui il parlait~?

--- Oui, dit Harry après avoir hésité. Mais je pense que je poursuivrai cette partie de mon enquête avec le professeur de Défense -- tout comme je n'aurais pas souhaité avoir le professeur Quirrell dans cette pièce lorsque nous discutions de la façon dont nous allions enquêter sur \emph{lui}.

--- Il me soupçonne~?~» dit Severus, puis il eut un rire bref. «~Ah, mais évidemment qu'il me soupçonne.

--- Mon plan, dit Harry, est d'aller voir la salle des trophées où le supposé duel a eu lieu et de voir si je peux découvrir quoi que ce soit d'anormal. Si vous pouviez dire aux Aurors sur l'enquête de me laisser passer…

--- Quels Aurors sur l'enquête~?~» dit Severus d'une voix sans timbre.

Harry Potter prit une profonde inspiration, la laissa lentement s'échapper, puis parla de nouveau.

«~Dans les livres policiers, un crime met généralement plus d'une journée à être résolu, mais vingt-quatre heures représentent -- non, \emph{trente} heures représentent mille-huit-cents minutes. Et je peux songer à au moins un autre endroit important où chercher des indices -- même si ce devra être fait par quelqu'un qui peut se rendre dans le dortoir des filles de Serdaigle. Lorsqu'Hermione se battait contre les brutes, elle trouvait des mots sous son oreiller chaque matin qui lui disaient où aller…

--- \emph{Albus…} gronda Minerva.

--- Je ne les ai pas envoyées~», dit le vieux sorcier. Ses sourcils blancs s'étaient élevés sous le coup de la surprise. «~J'ignorais tout de cela. Pensez-vous qu'on se jouait d'elle, Harry~?

--- C'est une possibilité, dit-il. D'autant plus que vous ne connaissez pas encore une autre pièce du puzzle.~» La voix de Harry baissa d'un ton et devint plus intense. «~M. le directeur, vous savez déjà que j'ai obtenu la cape d'invisibilité de mon père par quelqu'un qui a laissé un mot sous mon oreiller en disant que c'était un cadeau de Noël en avance. Je pense que nous devons supposer que c'est la même personne qui laissait des mots sous l'oreiller de Hermione…

--- Harry~», dit le vieux sorcier, et il hésita l'espace d'un instant. «~Te rendre la cape de ton père ne semble pas être le fait d'un ennemi…

--- \emph{Écoutez}~», dit Harry Potter avec une note d'urgence dans la voix. «~Ce que vous \emph{ignorez} c'est qu'après que Bellatrix Black s'est échappée d'Azkaban, j'ai trouvé un autre mot sous mon oreiller, signé “Père Noël”, qui me disait qu'il avait entendu dire que vous m'enfermiez à Poudlard et qu'il me donnait une issue vers l'institut des sorcières de Salem, aux États-Unis. Le mot état accompagné d'un jeu de carte dont le roi de cœur est censé être un Portoloin…

--- \emph{M. Potter~!}~» s'écria le professeur McGonagall, elle avait parlé sans même y songer. «~Cela pourrait très bien être une \emph{tentative d'enlèvement}~! Vous auriez dû…

--- \emph{Oui}, professeur, j'ai réagi de façon sensée, dit le garçon d'un ton égal. \emph{Au vu des circonstances}, j'ai réagi de façon sensée. J'en ai parlé au professeur Quirrell. Et selon le professeur Quirrell, ce Portoloin va quelque part à Londres -- il n'est définitivement pas assez puissant pour être un Portoloin international. Il demeure \emph{possible} que la personne qui m'a envoyé le mot soit honnête et que cet endroit à Londres ne soit qu'une gare intermédiaire.~» Le garçon fouilla dans sa robe et en sortit un jeu de cartes ainsi qu'un papier plié. «~Je vous fais confiance pour ne \emph{pas} débarquer arme au poing -- je veux dire baguette au poing -- juste au cas où l'émetteur serait un de mes alliés ou un des vôtres. Mais si c'est un piège, je propose que nous le déclenchions \emph{maintenant}. Et qui que soit cette personne, prenez-la \emph{vivante} afin que nous puissions la présenter au Magenmagot. Je ne saurais trop insister sur ce point.~»

Severus se leva de sa chaise, l'air résolu, et s'avança vers Harry.

«~J'aurais besoin de l'un de vos cheveux pour le Polynectar, M. Potter…

--- Ne soyons pas hâtifs~! dit Albus. Nous n'avons pas encore examiné les mots envoyés à Mlle Granger~; peut-être n'y aura-t-il aucune ressemblance. Severus, pourriez-vous vous rendre dans son dortoir et voir si vous pouvez les trouver~?~»

Harry Potter, debout afin d'offrir au maître des potions un meilleur accès au fatras de ses cheveux, venait d'élever les sourcils. «~Vous pensez que deux personne \emph{différentes} se promènent dans Poudlard en laissant des mots sous des oreillers~?~»

Severus eut un bref rire sardonique au moment où sa main s'avança et arracha un cheveu qui fut bientôt précautionneusement enrobé de soie. «~C'est tout à fait possible. S'il y a quelque chose que j'ai appris à mon poste de directeur de Serpentard, c'est l'absurdité des pagailles qui émergent lorsque sont présents plus d'un comploteur et plus d'un complot. Mais M. le directeur… je pense que M. Potter a raison et que je devrais suivre ce Portoloin et voir où il mène.~»

Albus hésita puis hocha la tête avec réticence. «~Dans ce cas, je te parlerai avant que tu y ailles.~»

\later

Alors même que Harry Potter quittait la pièce pour mener sa propre enquête, Severus pivota sur ses talons et avança vers le bocal de poudre de cheminette à une vitesse telle que sa cape s'éleva à sa traîne.

«~Je vais prendre du Polynectar brut, ajouter le cheveu et partir. M. le directeur, pourriez-vous rester ici pour…

--- Albus~», dit Minerva, surprise par la stabilité de la propre voix, «~est-ce vous qui avez laissé ces mots sous l'oreiller de M. Potter~?~»

La main de Severus s'arrêta un instant avant de lancer la poudre dans le feu.

Dumbledore hocha la tête à l'intention de McGonagall mais son sourire était un peu creux.

«~Tu ne me connais que trop bien, ma chère.

--- Et j'imagine que le Portoloin mène à un foyer accueillant où M. Potter aurait été gardé à l'abri jusqu'à ce que vous veniez le récupérer et le ramener à Poudlard~?~» Sa voix était serrée -- c'était raisonnable, elle ne pouvait pas le nier, mais cela lui semblait pourtant un peu cruel.

«~Cela aurait dépendu des circonstances, dit doucement le vieux sorcier. Si Harry en était arrivé là… je l'aurais peut-être laissé profiter de son évasion pendant un moment. Mieux valait savoir où il allait et s'assurer que ce soit un endroit sûr, peuplé d'amis…

--- Et dire, continua le professeur McGonagall, que j'avais pensé à réprimander M. Potter pour ne pas nous avoir fait part de ce fait important~! À lui reprocher de ne pas avoir le bon sens de nous faire confiance~!~» Sa voix avait monté d'un cran. «~J'imagine que je sauterai cette leçon~!~»

Severus regardait fixement le directeur en plissant les yeux.

«~Et les mots à Mlle Granger…

--- Le professeur de Défense, très probablement, dit le vieux sorcier. Mais… ce n'est qu'une supposition.

--- J'irai les chercher, dit Severus. Et je suppose que je commencerai ensuite à chercher Vous-Savez-Qui.~» Il fronça brièvement les sourcils. «~Une tâche pour laquelle je ne sais absolument pas par où commencer. Connaîtriez-vous des magies permettant de trouver une âme, M. le directeur~?~»

\later

La salle de cours divination était éclairée par la lumière rouge tamisée de mille petits feux où brûlaient mille encens différents, si bien que si vous aviez cherché à savoir, en un mot, à quoi cette pièce ressemblait, la réponse aurait été~: à de la fumée (en supposant que vous vous seriez fatigué à regarder quoi que ce soit alors que votre nez menaçait de surchauffer et de mourir). Si votre regard pouvait percer ces vapeurs humides, vous verriez une petite pièce encombrée où quarante fauteuils rembourrés, la plupart d'entre eux inusités, étaient amassés autour d'un petit espace ouvert au centre de la pièce où une trappe circulaire attendait que vous vous évadiez.

«~Le sinistre~!~» dit le professeur Trelawney d'une voix chevrotante en jetant un œil dans la tasse à thé de George Weasley. «~Le sinistre~! C'est un signe de mort~! Une personne que tu connais, George -- quelqu'un que tu connais va mourir~! Et bientôt -- oui, ce sera très bientôt, je pense -- à moins bien sûr que ce ne soit plus tard…~»

Fred et George songèrent que cela aurait été bien plus effrayant si elle n'avait pas dit la même chose à tous les autres élèves de leur cours de Divination. À ce stade, ils écoutaient à peine~; toutes leurs pensées étaient concentrées sur le désastre d'aujourd'hui…

La trappe du plancher s'ouvrit d'un grand coup et produisit un bruit qui fit piailler le professeur Trelawney et surpris tant George qu'il en jeta du thé sur sa robe. Un instant plus tard, Dumbledore émergeait du plancher comme une tornade, un oiseau de feu sur l'épaule.

«~Fred~!~» dit impérieusement le vieux sorcier. Sa robe était du noir d'une nuit sans lune, ses yeux durs comme des diamants bleus. «~George~! Avec moi, maintenant~!~»

Il y eut un hoquet collectif et lorsque Fred et George eurent commencé à descendre l'échelle à la suite du directeur, toute la classe spéculait déjà sur le rôle qu'ils avaient joué dans la tentative de meurtre à l'encontre de Drago Malfoy.

La trappe venait à peine de se refermer au-dessus d'eux que tous les sons environnants diminuèrent, le vieux sorcier pivota, tendit une main et ordonna~:

«~Donnez-moi la carte~!

--- C-carte~?~» dirent Fred et George, totalement abasourdis. Ils n'avaient jamais soupçonné que Dumbledore avait soupçonné. «~Pourquoi, n-nous ne savons pas ce que vous…

--- Hermione Granger a des ennuis, dit le vieux sorcier.

--- La Carte est dans notre dortoir, dirent immédiatement Fred et George. Donnez-nous juste quelques minutes pour la prendre et nous…~»

Les bras du sorcier les soulevèrent comme s'ils étaient d'immenses oreillers, il y eut un cri perçant, un flash de feu, et ils se retrouvèrent tous les trois dans le dortoir des Gryffondor de troisième année.

Quelques instants plus tard, Fred et George tendaient la Carte au directeur en ne grimaçant que légèrement face au sacrilège qu'il y avait à donner leur précieux morceau du système de sécurité de Poudlard à la personne qui en était le véritable propriétaire alors que le vieux sorcier fronçait lui-même les sourcils devant l'apparence vierge du parchemin.

«~Il faut dire, expliquèrent-ils, \emph{je jure solennellement que mes intentions sont mauvaises…}

--- Je me refuse à mentir~», dit le vieux sorcier. Il leva la Carte bien haut et mugit~: «~Entends-moi, Poudlard~! \emph{Deligitor prodi~!}~» Un instant plus tard le directeur était coiffé du Choixpeau, et ce dernier lui allait \emph{effroyablement} \emph{bien}, comme si Dumbledore avait toujours attendu qu'un chapeau pointu rapiécé ne vienne compléter son existence.

(Fred et George mémorisèrent immédiatement la phrase juste au cas où elle fonctionnerait pour quelqu'un d'autre que le directeur et commencèrent à imaginer des farces qui feraient usage du Choixpeau).

Le vieux sorcier ne perdit pas un instant, ôta le Choixpeau de sa tête, le retourna -- c'était difficile à voir à l'envers, mais le Choixpeau avait l'air un peu contrit par le traitement qu'on lui faisait subir -- puis il y plongea sa main et en tira une tige de cristal. De cet instrument il commença à tracer des motifs runiques sur la Carte tout en marmonnant d'étranges incantations qui ne ressemblaient pas tout à fait au Latin et faisaient échos dans les oreilles de Fred et George d'une façon particulièrement glaçante. Au beau milieu d'une rune, il leva les yeux et les regarda tous les deux d'un regard sévère.

«~Je vous la rendrai plus tard, fils des Weasley. Retournez en cours.

--- Oui, M. le directeur, dirent-ils, et ils hésitèrent. Ah… à propos de Hermione Granger, va-t-elle vraiment devoir servir Drago pour toujours comme sa…

--- \emph{Partez}~», dit le vieux sorcier.

Ils s'en furent.

Une fois seul dans la pièce, le vieux sorcier baissa les yeux vers la carte sur laquelle se trouvait maintenant un fin dessin au trait représentant les dortoirs Gryffondor dans lesquels il se trouvait, le nom manuscrit \emph{Albus P.W.B Dumbledore} seul à l'occuper.

Le vieux sorcier défroissa la carte, se pencha au-dessus d'elle et murmura~: «~Trouve Tom Jedusor.~»

\later

La salle d'interrogatoire du département de justice magique était généralement éclairé d'une petite lumière orange afin que l'Auror qui vous interrogeait se penche vers votre chaise de métal inconfortable en gardant la majeure partie de son visage dans le noir, vous empêchant ainsi de voir son expression alors même qu'il voyait la vôtre.

Dès que M. Quirrell était entré dans la pièce, la petite lumière orange s'était tamisée et avait commencé à vaciller comme une bougie sur le point d'être soufflée par le vent. La pièce était maintenant illuminée d'une lueur sans source identifiable et couleur de glace qui illuminait la peau pâle comme de l'albâtre du professeur Quirrell, mis à part ses yeux qui, sans que l'on sache comment, demeuraient dans les ténèbres.

L'Auror de garde avait subrepticement essayé de dissiper cet effet quatre fois de suite sans le moindre succès en dépit du fait que M. Quirrell avait poliment rendu sa baguette au début de sa détention pour interrogatoire et qu'il n'avait semblé prononcer aucune incantation ni exercer aucun autre pouvoir.

«~Quirinus… Quirrell~», dit d'une voix traînante l'homme qui était maintenant assis face au professeur de Défense qui avait courtoisement attendu. L'interrogateur avait des cheveux fauves tirés en arrière comme la crinière d'un lion et des yeux jaunâtres plantés dans le visage sévèrement ridé d'un homme sur la fin de sa dixième décennie. L'homme feuilletait pour le moment un grand dossier de parchemins qu'il avait sorti d'une mallette noire à l'apparence très solide après être entré dans la pièce et s'être assis sans sembler regarder le visage de l'homme qu'il devait interroger. Il ne s'était pas présenté.

Après avoir continué de feuilleter quelques parchemins supplémentaires en silence, l'Auror parla de nouveau.

«~Né le 26 septembre 1955, né de Quondia Quirrell suite à un rendez-vous galant reconnu avec lirinus Lumblung… entonna l'Auror. Réparti à Serdaigle… pas mauvaises BUSE… ASPIC en charmes, métamorphose… une mention d'excellence en études moldues, impressionnant… anciennes runes, et, ah oui, Défense. Mention d'excellence là aussi. Devient ensuite un sacré touriste, visite toutes sortes d'endroits. Visas Portoloins pour la Transylvanie, l'Empire Interdit, la Ville de la Nuit Sans Fin… eh bien eh bien, le \emph{Texas}.~» L'homme releva les yeux des documents, yeux plissés. «~Que faisiez-vous \emph{là-bas}, M. Quirrell~?

--- Du tourisme, surtout dans les zones moldues, répondit le professeur de Défense à l'aise. Comme vous l'avez dit, un sacré touriste.~»

L'homme écouta cela avec un froncement de sourcils, rabaissa les yeux puis les releva. «~Je vois aussi que vous avez visité Fuyuki City en 1983.~»

Le professeur de Défense leva un sourcil, comme modérément perplexe.

«~Eh bien~?

--- Que faisiez-vous à Fuyuki City~?~» la question avait été jetée comme une lame de rasoir.

Le professeur de Défense fronça légèrement les sourcils.

«~Rien de mémorable. J'ai visité les attractions touristiques principales, d'autres moins connues, et à part ça je me suis occupé de mes affaires.

--- Vraiment~? dit l'Auror d'une voix douce. Je trouve cette réponse très intéressante.

--- Comment cela~? dit le professeur de Défense.

--- Parce qu'il n'y avait pas de visa pour Fuyuki City dans la liste.~» L'homme referma le dossier d'un bruit sec. «~Vous n'êtes pas Quirinus Quirrell. Alors qui \emph{diable} êtes-vous~?~»

\later

Le maître des potions entra silencieusement dans le dortoir des filles de Serdaigle, la chambre des première année, un lieu festif où le bronze et le bleu se battaient pour être la couleur principale d'animaux en peluche, d'écharpes, de robes, de petits bijoux de pacotille et de posters de gens célèbres. Le lit de Hermione Granger était facile à identifier~: c'était celui qui avait été attaqué par un monstrolivre.

Personne ne semblait être dans les parages à cette heure et un certain nombre de sortilèges permit de s'en assurer.

Le maître des potions chercha sous l'oreiller de Hermione Granger, puis sous son lit, puis il commença à chercher dans son coffre, fouillant à travers des objets quotidiens et d'autres inavouables sans changer d'expression avant de finir par réussir à en extraire un lot de mots indiquant des lieux et des heures où trouver des brutes, tous signés d'un unique “S” très travaillé.

Un bref jaillissement de feu plus tard, les papiers n'étaient plus et le maître des potions partit pour rendre état de l'échec de sa mission.

\later

Le professeur de Défense était calmement assis, ses mains toujours croisées sur ses genoux. «~Si vous consultez le directeur, M. Dumbledore, dit le professeur de Défense, vous découvrirez qu'il est parfaitement au courant de l'affaire et que j'ai accepté d'enseigner à ce cours de Défense à la condition explicite qu'aucune question ne soit posée quant à mon…~»

Vif comme l'éclair, l'interrogateur brandit sa baguette et cracha «~\emph{Polyfluis Reverso~!}~» au moment même où le professeur de Défense éternuait, ce qui brisa mystérieusement le rayon argenté en une douche d'étincelles blanches.

«~Excusez-moi~», dit poliment le professeur de Défense.

Le sourire que lui envoya l'Auror ne comportait aucune joie.

«~Alors où est le véritable Quirinus Quirrell, hein~? Victime d'un Imperius quelque part au fond d'un coffre, où vous lui prenez un cheveu de temps à autre pour préparer illégalement votre Polynectar~?

--- Vous faites des suppositions hautement douteuses, dit le professeur de Défense d'une voix qui n'était plus dénuée de tranchant. Qu'est-ce qui vous fait croire que je n'ai pas entièrement volé son corps à l'aide d'une magie incroyablement Noire~?~»

Ce qui fut suivi par un certain silence.

«~Je suggère, dit l'Auror, que vous preniez ceci au sérieux, M. Qui-Que-Vous-Soyez.

--- Je suis navré, dit le professeur de Défense en s'inclinant dans sa chaise, mais je vois pas de raison particulière de me rabaisser à cela en pareille occasion. Qu'est-ce que vous allez faire, me tuer~?

--- Je n'apprécie pas votre humour, dit l'Auror d'une voix douce.

--- Que c'est dommage pour vous, Rufus Scrimgeour, dit le professeur de Défense, vous avez ma plus profonde compassion.~» Il inclina la tête, semblant étudier l'interrogateur~; et depuis l'ombre même de la lumière de glace, ses yeux étincelèrent.

\later

Padma regardait son assiette.

«~Hermione ne ferait pas ça \emph{sans raison}~!~» hurla Mandy Brocklehurst, quasiment en larmes, en fait elle \emph{était} en larmes, sa voix aurait été assez forte pour réduire la grande salle au silence si tous les autres élèves n'avaient pas aussi été en train de se crier dessus.

«~Je… je parie que Malfoy et essayé de… de lui \emph{faire} des choses…

--- Notre général ne ferait \emph{jamais} ça~! hurla Kevin Sifflebranche encore plus fort que Mandy.

--- Bien sûr que si~! s'écria Anthony Goldstein. Malfoy est le fils d'un \emph{Mangemort}~!~»

\later

Padma regardait son assiette.

Drago était le général de son armée.

Hermione était la fondatrice de la SPEHS.

Drago lui avait fait confiance, l'avait nommée commandante en second.

Hermione était Serdaigle, comme elle.

Ils étaient tous les deux ses amis, peut-être les deux meilleurs amis qu'elle ait.

Padma regardait son assiette. Elle était heureuse que le Choixpeau ne lui ait pas offert Poufsouffle. S'il avait été réparti à Poufsouffle, essayer de choisir où allait sa loyauté divisée aurait été probablement beaucoup plus douloureux…

Elle cligna les yeux et se rendit compte que sa vision s'était de nouveau brouillée, alors elle leva une main tremblante et essuya encore ses yeux.

Morag MacDougal grogna si fort que ce fut audible au milieu du pandémonium de ce déjeuner et dit d'une voix forte~:

«~Je parie que Granger a \emph{triché} lors de sa bataille d'hier, je parie que c'est pour ça que Malfoy l'a défiée…

--- \shout{Fermez-la} tous~!~» rugit Harry Potter en frappant des poings sur la table avec tellement de force que les assiettes alentours s'entrechoquèrent.

À n'importe quel autre moment, cela lui aurait valu une réprimande des professeurs, mais cette fois, cela n'attira le regard que de quelques élèves proches de lui.

«~Je voulais déjeuner, dit Harry Potter, et retourner à mon enquête, donc je ne comptais pas parler. Mais vous vous comportez tous comme des \emph{idiots} et quand la vérité éclatera vous regretterez ce que vous avez dit au sujet de gens innocents. Drago n'a rien fait, Hermione n'a rien fait, ils ont tous les deux été victimes d'un sortilège de faux souvenirs~!~» La voix de Harry Potter s'était élevée sur ces derniers mots.

«~\emph{Comment est-ce que ça peut ne pas \shout{Crever les yeux}~?}

--- Tu penses qu'on va croire \emph{ÇA}~?~» lui répondit Kevin Sifflebranche du tac au tac en hurlant. «~C'est ce que tout le monde dit~! “Je ne l'ai pas fait, c'était juste un sortilège de faux souvenirs~!” Tu crois qu'on est \emph{stupides}~?~»

Et Morag hocha la tête d'approbation et prit un air condescendant.

L'apparence que prit le visage de Harry fit tressaillir Padma.

«~Je vois~», dit Harry Potter, et il parla sans crier, si bien que Padma dut faire un effort pour l'entendre. «~Le professeur Quirrell n'est pas là pour m'expliquer à quel point les gens sont bêtes, mais je parie que cette fois-ci je peux y arriver tout seul. Des gens font quelque chose de stupide, se font attraper, et on leur donne du Veritaserum. Pas des grands criminels audacieux, parce que \emph{eux} ne se feraient pas attraper, \emph{eux} auraient appris l'Occlumancie. Non, des criminels triste, pathétiques et incompétents se font attraper et confessent sous Veritaserum, et ils veulent éviter Azkaban à tout prix alors ils disent qu'on leur a lancé un sortilège de faux souvenirs. C'est ça, hein~? Alors votre cerveau, par pure association pavlovienne, relie l'idée du sortilège de faux souvenirs à celle de criminels pathétiques aux excuses invraisemblables. Vous n'avez pas à prendre en compte les détails, votre cerveau \emph{complète} juste le \emph{motif} de l'hypothèse, le range dans le bac des choses auxquelles vous ne croyez pas, et le travail est fait. Exactement comme mon père pensait qu'on ne pouvait pas croire aux hypothèses magiques parce qu'il avait entendu tellement de gens stupides parler de magie. Croire à une hypothèse qui suppose des sortilèges de faux souvenirs est un signe \emph{d'infériorité}.

--- Qu'est-ce que tu \emph{baragouines}, Potter~? dit Morag en regardant le Survivant avec mépris.

--- Tu crois qu'on va croire tout ce que \emph{tu} dis~? hurla une Serdaigle à l'air légèrement plus âgée que Padma ne reconnut pas. Alors que \emph{tu} a tiré Granger vers les ténèbres~?

--- Et je ne vais même pas me plaindre, dit Harry d'une voix étrangement calme, en disant que les sorciers ne sont pas logiques et croient aux choses les plus folles. Parce que j'ai déjà dit ça au professeur Quirrell et il m'a juste lancé ce \emph{regard} et il a dit que si je n'avais pas été aveuglé par mon éducation je pourrais trouver des centaines de choses encore plus ridicules que beaucoup de Moldus croient. Tout ce que vous faites est très humain, très normal et ne fait pas de vous des gens \emph{exceptionnellement} mauvais, alors je ne vais pas me plaindre.~» Le Survivant se leva de son banc. «~Je vous verrai tous plus tard.~»

Et Harry Potter s'éloigna d'eux, les quitta tous autant qu'ils étaient.

«~Tu ne penses pas qu'il a \emph{raison}, quand même~?~» dit Su Li, assise à côté d'elle, d'un ton qui rendait clair ce \emph{qu'elle} pensait.

«~Je…~» dit Padma. Ses mots semblaient être coincés dans sa gorge, ses pensées bloquées dans sa tête. «~Enfin… je… je…~»

\later

En réfléchissant assez fort, vous pouvez accomplir l'impossible.

(Cela avait toujours été une profession de foi chez Harry. Il y avait eu un temps où il avait accepté les lois de la physique comme limite ultime, et maintenant il soupçonnait qu'il n'y ait pas de limites du tout).

Si vous pensez assez \emph{vite} vous pouvez parfois accomplir l'impossible \emph{rapidement}…

… parfois.

Seulement parfois.

Pas toujours.

Pas de façon \emph{fiable}.

Le survivant regardait autour de lui dans la salle des trophées, entouré par des récompenses, des plats, des boucliers, des statues et des médailles gardés derrière des milliers, peut-être des dizaines de milliers de présentoirs en cristal. Pendant tous les siècles d'existence de Poudlard, cette pièce avait accumulé des détails. Une semaine, un mois, peut-être même une année n'aurait pas suffi à sélectionner l'option “examiner” de tous les objets de la pièce. Une fois que le professeur Flitwick était parti, Harry avait demandé au professeur Vector s'il existait un moyen de détecter les dommages causés aux alentours des boîtes en cristal ou de vérifier l'existence de résidus qu'un véritable duel aurait dû laisser derrière lui. Harry avait couru à travers la bibliothèque de Poudlard à la recherche de sortilèges permettant de faire la différence entre des vieilles empreintes et des empreintes fraîches ou de détecter des restes d'expirations dans une pièce. Toutes ces tentatives de jouer au détective avaient échoué.

Il n'y avait aucun indice, aucun qu'il soit assez intelligent pour trouver.

Le professeur Rogue avait dit que le Portoloin menait dans une maison vide de Londres et qu'il ne semblait y avoir ni personne ni rien d'intéressant.

Le professeur Rogue n'avait trouvé aucun mot dans le dortoir de Hermione.

Le directeur avait suggéré que l'esprit de Voldemort se cachait probablement dans la Chambre des Secrets, là où le système de sécurité de Poudlard ne pouvait pas le trouver. Harry s'était faufilé dans les donjons Serpentard sous la Cape d'Invisibilité et avait passé le reste de l'après-midi à chercher tous les lieux évidents, mais il n'avait rien trouvé de reptilien qui réponde quand il lui parlait. Il semblait que l'entrée de la Chambre des Secrets n'était pas censée être trouvée en une seule journée.

Harry avait parlé à tous les amis de Hermione qui acceptaient encore de lui parler et aucun d'eux ne se souvenait avoir entendu Hermione dire quoi que ce soit de précis quant à la raison pour laquelle elle croyait que Drago complotait contre elle.

Le professeur Quirrell n'était pas revenu du ministère à l'heure du dîner. Les élèves plus âgés semblaient penser que le professeur de Défense de cette année finirait probablement par être tenu responsable de cet incident et qu'il serait renvoyé pour avoir enseigné aux élèves de Poudlard à être trop violents. Ils avaient parlé du professeur de Défense comme s'il était déjà parti.

Harry avait utilisé les six heures de son retourneur de temps, il n'y avait toujours pas d'indices et il lui fallait maintenant aller se coucher s'il voulait être en état de marche le lendemain, pour le procès de Hermione.

Le Garçon-Qui-Avait-Détruit-Un-Détraqueur se tenait au milieu de la salle des trophées de Poudlard, sa baguette tombée au sol.

Il pleurait.

Parfois, on appelle son cerveau et il ne répond pas.

Le procès de Hermione commença le lendemain à l'heure prévue.
%  LocalWords:  McJorgenson Quondia Lirinus Lumblung Fuyuki
