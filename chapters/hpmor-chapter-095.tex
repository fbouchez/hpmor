\partchapter{Rôles}{VI}

\section{La troisième rencontre\\
(Le 17 avril 1992 à 10h31)}

\lettrine{L}{e} printemps avait commencé, l'air de cette fin d'après-midi encore frais des restes de l'hiver.
Des jonquilles aux doux pétales avaient éclos parmi les pousses herbeuses de la forêt, et leurs cœurs d'or pendaient, flasques, de leurs tiges mortes, blessées ou tuées par l'une des soudaines gelées que l'on voyait souvent en avril.
Dans la Forêt Interdite on pourrait trouver des formes de vie plus étranges, au moins des centaures et des licornes, et Harry avait entendu des allégations de loups-garous.
Bien que, étant donné ses lectures concernant les vrais loups-garous, cela n'avait strictement aucun sens.

Harry ne s'aventurait pas près du bord de la Forêt Interdite puisque prendre ce risque aurait été inutile.
Il marchait, invisible, parmi les formes de vie plus ordinaires de la partie autorisée des bois, baguette à la main, un balai accroché dans le dos afin de pouvoir y accéder plus facilement.
Il n'avait pas vraiment peur~; il trouvait étrange de ne pas avoir peur.
L'état de vigilance permanente, d'être prêt à combattre ou à fuir, cela n'était pour lui un fardeau ni quelque chose de surprenant.

Au bord des bois autorisés, Harry marchait sans jamais que ses pieds ne s'éloignent du sentier battu où l'on pourrait le trouver plus facilement, sans jamais perdre de vue les fenêtres de Poudlard.
Il avait programmé une alarme sur sa montre pour le prévenir de l'heure du déjeuner puisqu'il ne pouvait pas vraiment regarder son poignet, vu qu'il était invisible.
Cela soulevait la question de savoir comment ses lunettes fonctionnaient puisqu'il portait la Cape.
D'ailleurs, le principe du tiers exclu semblait avoir pour conséquence que soit les complexes de rhodopsine dans sa rétine absorbaient les photons et les transformaient en pics d'activité neuronale, soit ces photons traversaient son corps et sortaient de l'autre côté, mais pas les deux.
Il semblait vraiment de plus en plus probable que ces capes d'invisibilité vous laissaient voir l'extérieur tout en étant invisible parce que, à un niveau fondamental, c'était ainsi que le lanceur du sortilège avait -- non pas \emph{voulu} -- mais \emph{implicitement cru} que l'invisibilité se devait de fonctionner.

Ce sur quoi il fallait s'interroger quant à la raison pour laquelle personne n'avait essayé de lancer un sortilège de Confusion ou de Légilimancie sur quelqu'un afin de lui faire croire comme à une évidence que \emph{Resolus Totalus} était certainement un sortilège de première année facile, puis de lui faire essayer d'inventer ce sort.

Ou peut-être fallait-il trouver quelque né-Moldu méritant dans un pays qui n'identifiait pas les enfants nés-Moldus puis de lui dire quelque mensonge élaboré, de construire une histoire et des preuves adéquates, afin que, depuis le début, l'individu ait une notion très différente de ce dont la magie était capable.
Bien qu'apparemment l'individu devrait encore apprendre un certain nombre de sortilèges avant de devenir capable d'inventer les siens…

Cela ne fonctionnerait peut-être pas.
Il y avait certainement des sorciers profondément fous qui avaient réellement cru qu'ils pouvaient être des dieux et avaient pourtant échoué à le devenir.
Mais même les fous avaient probablement cru que le sortilège d'ascension se devait d'avoir quelque rituel grandiose et théâtral, pas un bref mais minutieux mouvement de baguette accompagné de l'incantation \emph{Devenus Deus}.

Harry était déjà plutôt certain que ça ne pouvait pas être si facile que ça.
Mais la question était alors~: \emph{pourquoi pas~?} Quel motif son cerveau avait-il appris~?
Pouvait-il prévoir l'explication à l'avance~?

Un léger frémissement d'appréhension monta alors en Harry, une touche d'inquiétude, tandis qu'il contemplait cette question.
La préoccupation sans nom s'affûta, grandit…

\emph{Professeur Quirrell~?}

«~M. Potter~», dit une voix douce derrière lui.

Harry pivota, sa main alla à son retourneur de temps situé sous sa Cape~; à nouveau, l'idée d'être prêt à fuir à n'importe quel instant lui sembla ordinaire.

Lentement, paumes ouvertes et tournées vers le ciel, le professeur Quirrell avança vers lui, à la bordure de la forêt, approchant depuis le château de Poudlard.

«~M. Potter, répéta le professeur Quirrell.
Je sais que vous êtes ici.
Vous savez que je sais que vous êtes ici.
Je dois vous parler.~»

Harry continua de se taire.
Le professeur Quirrell n'avait toujours pas dit de quoi il s'agissait, et sa balade de fin de matinée, à la frontière de la forêt, l'avait rendu d'humeur silencieuse.

Le professeur Quirrell fit un petit pas à gauche, un pas vers l'avant, puis un autre vers la droite.
Il pencha la tête avec un air calculateur puis marcha presque directement quasiment jusqu'à l'endroit où Harry se tenait, et s'arrêta à quelques pas~; la sensation funeste brûlait à la limite du supportable.

«~Êtes-vous toujours décidé à poursuivre dans cette voie~? dit le professeur Quirrell.
La voie dont vous m'avez parlé hier~?~»

Une fois de plus, Harry ne répondit pas.

Le professeur Quirrell soupira.
«~J'ai fait beaucoup pour vous, dit l'homme.
Quoi que puissent être vos interrogations à mon sujet, vous ne pouvez le nier.
Je réclame paiement d'une partie de la dette.
Parlez-moi, M. Potter.~»

\emph{Je n'ai pas envie de faire ça maintenant}, songea Harry, puis~: \emph{Oh, mais bien sûr.}

\later

Deux heures plus tard, après que Harry eut fait pivoter le retourneur de temps une fois, eut noté l'heure exacte, eut mémorisé son emplacement exact, eut passé une heure à marcher, fut rentré, eut dit au professeur McGonagall qu'il parlait en ce moment au professeur Quirrell dans les bois devant Poudlard (juste au cas où quelque chose lui arriverait), eut marché une heure de plus puis fut revenu à son emplacement initial exactement une heure après être parti et avoir à nouveau fait pivoter le retourneur de temps…

\later

«~Qu'est-ce que c'était~?~»
dit le professeur Quirrell en clignant des yeux.

«~Est-ce que vous venez…

--- Rien d'important~», dit Harry sans soulever sa Cape d'Invisibilité ni retirer sa main de son retourneur de temps.
«~Oui, je suis toujours décidé.
Pour être honnête, je pense que je n'aurais rien dû en dire.~»

Le professeur Quirrell inclina la tête.

«~Une pensée qui vous servira beaucoup durant votre vie.
Y a-t-il quoi que ce soit qui puisse vous faire changer d'avis~?

--- Professeur, si je connaissais \emph{déjà} l'existence d'un argument qui pourrait modifier ma décision…

--- C'est vrai pour nos semblables.
Mais vous seriez surpris d'entendre comme il est fréquent que quelqu'un sache ce qu'il attend qu'on lui dise et doive pourtant attendre que ce soit dit.~»
Le professeur Quirrell secoua la tête.
«~Pour utiliser vos termes… il existe un fait, vrai, que je sais et que vous ignorez, et de la véracité duquel j'aimerais vous convaincre, M. Potter.~»

Harry leva un sourcil mais comprit l'instant d'après que le professeur Quirrell ne pouvait pas le voir.

«~C'est effectivement en mes termes.
Allez-y.

--- L'intention qui vous habite maintenant est bien plus dangereuse que vous ne le croyez.~»

Répondre à cette affirmation étonnante ne demanda pas beaucoup de réflexion à Harry.

«~Définissez dangereux, et dites-moi ce que vous savez et comment vous pensez le savoir.

--- Parfois, dit le professeur Quirrell, parler d'un danger à quelqu'un peut les pousser à aller droit vers celui-ci.
Cette fois-ci, je compte bien éviter que cela se produise.
Vous attendez-vous à ce que je vous dise exactement ce que vous ne devez pas faire~?
La raison exacte de ma peur~?~»
L'homme secoua la tête.
«~Si vous étiez né sorcier, M. Potter, et qu'un puissant sorcier vous disait seulement de faire attention, vous sauriez le prendre au sérieux.~»

Il était allé mentir que de dire que Harry n'était pas agacé, mais il n'était pas un idiot non plus, et il dit donc seulement~: «~Y a-t-il quoi que ce soit que vous \emph{puissiez} me dire~?~»

Avec précaution, le professeur Quirrell s'assit sur l'herbe et sortit sa baguette qu'il mit dans une position que Harry reconnut.
Il inspira soudainement.

«~C'est la dernière fois que je pourrais faire cela pour vous~», dit doucement le professeur Quirrell.
Puis l'homme commença à prononcer des mots étranges, d'aucun langage que Harry pouvait reconnaître, d'une intonation qui ne semblait pas tout à fait humaine, des mots qui semblèrent s'échapper de la mémoire de Harry alors même qu'il essayait de les saisir, ils quittaient son esprit aussi vite qu'ils y étaient entrés.

Le sortilège prit forme plus lentement, cette fois.
Les arbres semblèrent s'obscurcir, les branches et les feuilles disparaître par taches, comme si elles avaient été vues à travers des lunettes de soleil parfaites qui auraient estompé, atténué la lumière sans la déformer.
Le bol bleu de ciel recula, l'horizon auquel le cerveau de Harry assignait par erreur une distance finie s'éloigna, devint gris, puis gris plus sombre.
Les nuages devinrent translucides, transparent, et se défirent en volutes pour laisser les ténèbres briller.

La forêt s'assombrit, s'effaça, plongea dans la noirceur.

La grande rivière céleste qui les entourait devenait à nouveau visible à mesure que les yeux de Harry s'ajustaient, devenaient capables de voir le plus grand objet que des yeux humains pourraient jamais voir comme autre chose qu'un point~: la Voie Lactée.

Et les étoiles, d'une lumière perçante, et pourtant lointaines, dans leurs grandes profondeurs.

Le professeur Quirrell prit une profonde inspiration.
Puis il éleva sa baguette de nouveau (à peine visible dans la lumière stellaire sans soleil ni lune) et frappa sur sa propre tête au son d'un œuf que l'on brisait.

Le professeur de Défense s'estompa lui aussi, devint aussi invisible.

Un petit disque d'herbe, à peine éclairé, dérivait, vide, dans l'espace.

Aucun d'eux ne parla pendant un moment.
Harry était heureux de regarder les étoiles sans même son corps pour le distraire.
Ce que le professeur Quirrell avait voulu lui dire en le faisant venir ici serait dit en temps et heure.

En temps et heure, une voix parla.

«~Il n'y a pas de guerre ici, dit une douce voix émanant du vide.
Pas de conflit ni de bataille, pas de politique ni de trahison, pas de mort ni de vie.
Tout cela appartient à la folie des hommes.
Les étoiles surplombent ces sottises et ne s'y frottent pas.
Ici se trouvent la paix et un silence éternel.
C'est du moins ce que je pensais.~»

Harry se retourna pour voir d'où venait la voix et ne vit que les étoiles.

«~Ce que vous pensiez~? dit Harry quand il sembla qu'aucune autre parole n'allait venir.

--- Il n'y a rien qui surplombe la folie des hommes, chuchota la voix du vide.
Il n'y a rien au-delà des pouvoirs destructeurs d'une idiotie suffisamment intelligente, pas mêmes les étoiles elles-mêmes.
J'ai travaillé très dur afin qu'une certaine plaque en or dure toujours.
Je n'aimerais pas la voir détruite par la folie humaine.~»

De nouveau, les yeux de Harry se braquèrent par réflexe vers là où la voix aurait dû être, et de nouveau ils ne virent que du vide.

«~Je pense pouvoir vous rassurer à ce sujet, professeur.
Les armes nucléaires n'ont pas une portée capable d'atteindre… à quelle distance se trouve Pioneer 11~?
Environ un milliard de kilomètres, peut-être~?
Les Moldus disent que les armes nucléaires peuvent détruire le monde, mais ce qu'ils veulent vraiment dire, c'est qu'elles peuvent légèrement réchauffer une partie de la surface de la Terre.
Le \emph{soleil} est une réaction de fusion géante et \emph{il} ne vaporise pas les sondes spatiales.
Même dans le pire des cas possibles, une guerre nucléaire n'aurait pas la moindre de chance de détruire le système solaire, non pas que ce soit une consolation.

--- Vrai pour ce qui est des Moldus, dit la douce voix entre les étoiles.
Mais que les Moldus savent-ils du véritable pouvoir~?
Ce ne sont pas eux qui me font peur.
C'est vous.

--- Professeur, dit Harry avec précaution, bien que je doive admettre avoir tiré quelques échecs critiques dans ma vie, il y a une certaine distance entre ça et rater un jet de sauvegarde d'une façon si catastrophique que la sonde Pioneer 11 se retrouve dans le rayon de l'onde de choc.
Je ne vois pas de moyen réaliste d'accomplir ça sans faire exploser le soleil.
Et avant que vous me posiez la question, le soleil est une étoile de type G V, il ne \emph{peut pas} exploser.
Toute entrée d'énergie ne ferait qu'augmenter le volume de plasma d'hydrogène, le soleil n'a pas un cœur dégénéré que l'on pourrait faire exploser.
Le soleil n'a pas assez de masse pour devenir une supernova, même à la fin de sa vie.

--- Les Moldus ont appris tant de choses incroyables, murmura l'autre voix.
Comment les étoiles vivent, comment elles se préservent de la mort, comment elles meurent.
Et ils ne se demandent jamais si un tel savoir pourrait être dangereux.

--- Pour être entièrement franc, professeur, cette pensée en particulier ne m'était jamais venue non plus.

--- Vous êtes un né-Moldu.
Je ne parle pas de votre sang, je parle de la façon dont vous avez vécu vos années d'enfance.
Il y a là une certaine liberté de pensée, oui.
Mais il y a aussi de la sagesse dans la prudence des sorciers.
Cela fait trois-cent-vingt-deux ans que le pays d'Italie magique a été détruit par la folie d'un seul homme.
De tels incidents étaient courant à l'époque où Poudlard fut créée.
Encore plus à l'époque qui suivit Merlin.
D'avant Merlin, il ne reste que peu à étudier.

--- Il y a environ trente ordres de grandeur entre ça et faire exploser le soleil~», remarqua Harry, puis il se reprit.
«~Mais j'ergote, pardon, faire exploser un pays aussi serait mal, je suis d'accord.
Quoi qu'il en soit, professeur, je ne compte pas faire quoi que ce soit qui y ressemble.

--- Nul besoin que vous le vouliez, M. Potter.
Si vous aviez lu plus de romans pour sorciers et moins de romans pour Moldus, vous le sauriez.
Dans la littérature sérieuse, le sorcier dont la sottise menace de relâcher les Hommes-Os Traînants ne serait pas consciemment résolu à atteindre ce but, cela, c'est bon pour les livres d'enfants.
Ce sorcier réellement dangereux sera peut-être résolu à accomplir quelque autre projet dont il comptera tirer grand renom, et la certitude de perdre ce renom et de vivre sa vie en inconnu lui semblera bien plus réelle que la possibilité ignorée de détruire son pays.
Ou il aura promis le succès à quelqu'un qu'il ne peut se permettre de décevoir.
Peut-être ses enfants sont-ils endettés.
Il y a dans ces histoires une grande sagesse littéraire, née d'expériences difficiles et de villes de cendres.
Le plus sûr chemin vers le désastre est un puissant sorcier qui, quelle qu'en soit la raison, ne sait pas s'arrêter une fois apparus les signes d'avertissements.
Bien qu'il parle haut et fort de prudence, il ne peut pas se contraindre à vraiment s'arrêter.
Je me demande, M. Potter si vous avez songé à essayer quelque chose qu'Hermione Granger elle-même vous aurait dit de ne pas faire~?

--- Très \emph{bien}, j'ai compris, dit Harry.
Professeur, je suis bien conscient du fait que si je sauve Hermione au prix de deux autres vies, j'aurai perdu des points en termes utilitaristes.
Je suis \emph{extrêmement} conscient du fait Hermione ne voudrait pas que je prenne le risque de détruire tout un pays juste pour la sauver.
C'est du bon sens.

--- Enfant qui détruit les Détraqueurs, dit la voix douce, s'il n'y avait qu'un seul pays dont je craignais que vous provoquassiez la destruction, je serais moins inquiet.
Je ne croyais pas, au début, que votre connaissance de la science moldue et de ses pratiques serait une source de grand pouvoir.
J'y crois maintenant plus.
Je suis, pour être parfaitement sincère, inquiet pour cette plaque en or.

--- Eh bien, si la science-fiction m'a appris quoi que ce soit, dit Harry, c'est que détruire un système solaire n'est pas moralement acceptable, surtout si on le fait avant que l'humanité n'ait colonisé un autre système stellaire.

--- Alors vous abandonnerez cette…

--- Non~», dit Harry sans même réfléchir avant d'ouvrir la bouche.
Après un moment, il ajouta~: «~Mais je comprends ce que vous essayez de me dire.~»

Silence.
Les étoiles ne s'étaient pas déplacées, pas même d'autant qu'elles ne l'auraient fait dans le ciel terrestre.

Un très léger bruissement, comme si quelqu'un réajustait sa position.
Harry se rendit compte qu'il était resté debout un moment sans bouger et s'assit sur le cercle quasiment invisible qui demeurait sous lui en faisant attention de ne pas toucher aux bords du sortilège.

«~Dites-moi, dit la voix douce.
Pourquoi cette fille a-t-elle autant d'importance pour vous~?

--- Parce qu'elle est mon amie.

--- Dans l'anglais tel qu'il est usité, M. Potter, le mot “ami” n'est pas associé à la notion d'un effort désespéré visant à réanimer les morts.
Avez-vous l'impression qu'elle est votre véritable amour, ou quelque chose comme cela~?

--- Oh, pas vous aussi, dit Harry d'un ton las.
Entre tous, pas vous, professeur.
D'accord, nous sommes meilleurs amis, mais c'est \emph{tout}, d'accord~?
Cela suffit.
Les amis ne laissent pas leurs amis rester morts.

--- Les gens normaux n'en font pas tant pour ceux qu'ils disent être leurs amis.~»
La voix semblait à présent plus distante, plus absente.
«~Pas même pour ceux qu'ils disent aimer.
Leurs compagnons meurent et ils ne partent pas à la recherche d'un pouvoir permettant de les ressusciter.~»

Harry ne put s'en empêcher.
Il regarda à nouveau, sachant que ce serait futile, et ne vit que des étoiles.
«~Laissez-moi deviner, vous avez déduit de cette observation que… les gens n'accordent pas autant d'importance à leurs amis qu'ils le prétendent.~»

Un bref rire.

«~Il leur serait difficile de prétendre leur en accorder \emph{moins}.

--- Ils leur en accordent, professeur, et pas seulement à leur véritable amour.
Des soldats se jettent sur des grenades pour sauver leurs amis, des mères se précipitent dans des maisons en flammes pour sauver leur enfant.
Mais si vous êtes un Moldu, vous ne savez pas que la magie existe et qu'elle peut ramener les gens à la vie.
Et les sorciers normaux ne…
\emph{sortent pas des sentiers battus} comme ça.
Après tout, la plupart des sorciers ne recherchent pas comment devenir assez puissants pour se rendre \emph{eux-mêmes} immortels.
Est-ce que ça prouve qu'ils n'accordent aucune importance à leur vie~?

--- Comme vous le dites, M. Potter.
Je jugerais certainement que leur vie n'a aucun sens et aucune valeur.
Peut-être, en quelque lieu caché de leur cœur, croient-ils eux aussi que l'opinion que j'ai d'eux est correcte.~»

Harry secoua la tête, puis, agacé, releva la capuche de la Cape et secoua de nouveau la tête.
«~Cela me semble être un point de vue \emph{choisi} sur le monde, Professeur~», dit le visage mal éclairé d'un garçon suspendu au-dessus d'un cercle d'herbe sombre au milieu des étoiles.
«~Quelqu'un de normal n'essaierait tout simplement pas d'inventer un sortilège de résurrection, si bien que vous ne pouvez rien déduire du fait qu'ils ne le font pas.~»

Un instant plus tard, la silhouette mal éclairée d'un homme assis sur un cercle d'herbe devint visible.

«~S'ils se préoccupent \emph{vraiment} de ceux qu'ils disent aimer, dit doucement le professeur Quirrell, pourquoi n'y pensent-ils pas~?

--- Les cerveaux ne fonctionnent pas comme ça.
Ils ne mettent pas le turbo quand l'enjeu augmente -- ou s'ils le font, c'est dans le cadre de contraintes strictes.
Je ne pourrais pas calculer la millième décimale de pi, même si la vie de quelqu'un en dépendait.~»

Le visage mal éclairé s'inclina.
«~Mais il y a une autre explication possible, M. Potter.
C'est que les gens jouent le \emph{rôle} de l'amitié.
Ils en font autant que le rôle exige d'eux et rien de plus.
L'idée me vient que la différence entre vous et eux n'est peut-être pas que vous vous souciez plus des choses qu'eux.
Pourquoi seriez-vous né doté de sentiments d'amitié anormalement forts, si bien que vous seul entre tous les sorciers auriez la motivation nécessaire à ressusciter Hermione Granger après sa mort~?
Non, la différence la plus probable n'est pas que c'est plus important pour vous.
C'est que, étant un être plus logique qu'eux, vous avez été seul à penser que le rôle d'un Ami exigeait un tel acte de votre part.~»

Harry regarda les étoiles.
Il aurait menti s'il avait prétendu ne pas avoir été secoué.

«~Ça… ça ne peut pas être vrai, professeur.
Je pourrais citer dix exemples de romans Moldus où les gens vont jusqu'à ressusciter leurs amis morts.
Les auteurs de ces histoires comprenaient clairement ce que je ressens pour Hermione.
Même si vous ne les avez pas lues, j'imagine que… peut-être Orphée et Eurydice~?
Je n'ai à vrai dire pas lu celle-ci, mais je sais ce qui s'y passe.

--- De telles histoires sont aussi racontées parmi les sorciers.
Il y a l'histoire des frères Elric.
L'histoire de Dora Kent, qui fut protégée par son fils Saul.
Il y a Ronald Mallett et son défi au Temps perdu d'avance.
En Italie, avant sa chute, le drame de Precia Testarossa.
Au Japon, on parle de Akema Homura et de son amour perdu.
Ce que ces histoires ont en commun, M. Potter, c'est qu'elles sont des \emph{fictions}.
Les vrais sorciers ne s'y essaient pas, alors que l'idée ne dépasse clairement \emph{pas} leurs capacités d'imagination.

--- Parce qu'ils ne pensent pas que c'est \emph{possible~!}.~»
La voix de Harry monta.

«~Devrions-nous aller voir le bon professeur McGonagall et lui parler de votre intention de trouver un moyen de ressusciter Mlle Granger afin de découvrir ce qu'elle en pense~?
Peut-être ne lui est-il jamais venu à l'esprit d'envisager cette option… ah, mais vous hésitez.
Vous connaissez déjà la réponse, M. Potter.
Savez-vous pourquoi vous la connaissez~?~»
On pouvait entendre le froid sourire derrière la voix.
«~Une délicieuse technique que voilà.
Merci de me l'avoir enseignée.~»

Harry était conscient de la tension de son visage et ses mots semblèrent avoir été hachés.
«~Le professeur McGonagall n'a pas grandi avec le concept moldu d'un pouvoir scientifique en expansion et personne ne lui a jamais dit que quand la vie d'un ami est en jeu, le moment est venu de \emph{réfléchir de façon très rationnelle…}~»

La voix de professeur de Défense monta aussi d'un ton.

«~Le professeur de métamorphose \emph{lit un script}, M. Potter~!
Ce script lui demande de pleurer et faire son deuil afin que tous sachent à quel point elle se souciait de Hermione.
Les gens ordinaires réagissent mal si on leur suggère de dévier du script.
Comme vous le saviez déjà~!

--- C'est drôle, j'aurais pu jurer avoir vu le professeur McGonagall dévier du script au dîner d'hier.
Si je la voyais en dévier dix autres fois j'irais peut-être bien lui parler de ressusciter Hermione, mais pour l'instant tout cela est nouveau pour elle et elle a besoin de pratique.
En définitive, professeur, ce que vous essayez de balayer en expliquant que l'amour, l'amitié et tout le reste ne sont qu'un mensonge, c'est juste que \emph{les êtres humains sont limités.}~»

La voix du professeur de Défense monta encore d'un ton.

«~Si c'était vous qui aviez été tué par le troll, il ne serait même pas \emph{venu à l'esprit} de Hermione Granger de faire pour vous ce que vous faites pour elle~!
Ça ne viendrait ni à Drago Malfoy, ni à Neville Londubat non plus, ni à McGonagall ni à aucun de vos précieux amis~!
Il n'y a pas une seule personne au monde qui vous rendrait le soin que vous prenez d'eux~!
Alors \emph{pourquoi}~?
Pourquoi le faire, M. Potter~?~»
Il y avait un étrange désespoir fou dans sa voix.
«~Pourquoi être le seul au monde à faire autant d'effort pour maintenir les apparences quand aucun d'eux ne fera jamais de même pour vous~?

--- Je pense que vous vous trompez sur une question de fait, professeur, répondit Harry d'une voix neutre.
Sur plusieurs, en fait.
À minima, votre modèle de mes émotions est erroné.
Parce que vous ne me comprenez pas le moins de monde si vous pensez que la véracité de ce que vous venez de dire m'arrêterait.
Toute chose en ce monde doit commencer quelque part, tout événement doit se produire une première fois.
La vie sur Terre a dû commencer par quelque petite molécule capable d'auto-réplication.
Et si j'étais la première personne au monde, non…~»

La main de Harry fit un geste, comme pour indiquer les points lumineux incroyablement lointains.

«~… si j'étais la première personne de \emph{l'univers} à vraiment me soucier d'un autre, ce qui est \emph{faux}, au fait, alors ce serait pour moi un honneur d'être cette personne et j'essaierai de m'en rendre digne.~»

Il y eut un long silence.

«~Vous vous souciez réellement de cette fille, dit doucement l'obscure silhouette de l'homme.
Vous vous souciez d'elle comme \emph{aucun} d'eux n'est capable de se soucier de sa propre vie et encore moins de celle de leur prochain.~»
La voix du professeur de Défense devint étrange, comme emplie d'une émotion indéchiffrable.
«~Je ne le comprends pas, mais je sais jusqu'où vous irez.
Vous défierez la mort pour elle.
Rien ne vous fera dévier.

--- Je m'en soucie assez pour vraiment faire un effort, dit doucement Harry.
Oui, c'est vrai.~»

La lumière céleste commença à se fracturer lentement, le monde commença à être visible à travers les fissures~; des balafres dans la nuit révélèrent des arbres et des feuilles illuminées par le soleil.
Harry leva une main, cligna plusieurs fois des yeux, alors que la lumière revenait s'écraser dans ses yeux ajustés au noir~; et ceux-ci se posèrent automatiquement sur le professeur de Défense juste au cas où une attaque se produirait pendant qu'il était aveuglé.

Lorsque toutes les étoiles furent parties, que seul le jour demeura, le professeur Quirrell était toujours assis sur l'herbe.
«~Eh bien, M. Potter, dit-il de sa voix normale, si c'est ainsi, alors je vous aiderai de mon mieux tant que je le pourrai.

--- Vous ferez \emph{quoi}~? dit involontairement Harry.

--- L'offre que je vous ai faite hier tient toujours.
Interrogez-moi et je répondrai.
Montrez-moi les livres scientifiques qui vous semblaient convenir à M. Malfoy, je les lirai et vous dirai ce qui me vient à l'esprit.
N'ayez pas l'air aussi surpris, M. Potter, je n'allais certainement pas vous laisser vous débrouiller tout seul.~»

Harry le regarda~; ses canaux lacrymaux toujours actifs en réaction à la lumière soudaine.

Le professeur Quirrell le regarda en retour.
Quelque chose d'étrange scintilla dans les yeux pâles.
«~J'ai fait ce que j'ai pu, et j'ai maintenant peur de devoir prendre congé.
A…~» et le professeur de Défense hésita.
«~À bientôt, M. Potter.

--- À…~» commença Harry.

L'homme assis sur l'herbe tomba et sa tête frappa le sol avec un petit bruit sourd.
Au même moment, la sensation funeste diminua tant que Harry bondit sur pied et que son cœur se mit à battre la chamade.

Mais la silhouette au sol se redressa assez pour pouvoir ramper.
Puis se tourna pour regarder Harry, les yeux vides, la mâchoire pendante.
Essaya de se lever, retomba au sol.

Harry fit un pas en avant, son instinct lui dit de tendre la main, bien que ce fut incorrect~; l'appréhension qui monta en lui, aussi faible qu'elle soit, lui dit qu'un danger était toujours présent.

Mais la silhouette tombée recula devant Harry et commença à lentement ramper à l'écart, en direction du lointain château.

Debout dans la forêt, le garçon le regarda partir.
%  LocalWords:  pring Fixus Everythingus Becomus Goddus Elric Mallett Akemi
%  LocalWords:  Precia Testarossa Homura
