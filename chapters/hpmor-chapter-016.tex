% vim:spell:spelllang=fr

\chapter{Pensée latérale}

\lettrine{D}{ès} l'instant où il posa le pied dans la salle du cours de Défense, le mercredi, Harry sut que \emph{cette} matière allait être \emph{différente}.

Pour commencer, c'était la plus grande salle de classe qu'il ait vue à Poudlard, semblable à un amphithéâtre de cours à l'université, avec des étages de pupitres disposés face à une plateforme immense en marbre blanc.
La salle était haute dans le château -- au cinquième étage -- et Harry savait qu'on ne lui donnerait pas plus d'explication sur comment une pareille salle pouvait bien rentrer ici.
Il était de plus en plus clair que Poudlard n'avait tout simplement \emph{pas} de géométrie, Euclidienne ou autre~; il avait des connexions, et non pas des directions.

Contrairement à une salle d'université, il n'y avait pas de rangées de strapontins rabattables~; au lieu de cela, il y avait les très ordinaires pupitres en bois de Poudlard, agencés en courbe à chaque niveau de la salle de cours.
Sauf que sur chaque pupitre était installé un objet blanc, plat, rectangulaire et mystérieux.

Au centre de la plateforme gigantesque, sur une petite estrade surélevée en marbre plus sombre, se trouvait un unique bureau d'enseignant.
Où se trouvait assis Quirrell, affalé sur sa chaise, la tête en arrière, bavant légèrement sur sa robe.

Ça \emph{me rappelle quelque chose, mais quoi~?}

Harry était arrivé en cours si tôt qu'aucun autre élève n'était encore présent.
(La langue anglaise était déficiente lorsqu'il s'agissait de décrire le voyage dans le temps~;
en particulier, il manquait à l'anglais les mots capables d'exprimer à quel point c'était pratique.)
Quirrell ne semblait pas être… en état de marche… pour le moment, et Harry n'avait de toute façon pas particulièrement envie de s'approcher de lui.

Harry choisit un bureau, grimpa jusqu'à celui-ci, s'assit, et récupéra son manuel de Défense.
Il en était à peu près aux sept huitièmes -- il avait prévu de finir le livre avant le début du cours, mais il était en retard sur sa planification et il avait déjà utilisé le retourneur de temps deux fois aujourd'hui.

Bientôt, le bruit se fit présent au fur et à mesure que la salle commençait à se remplir. Harry l'ignora.

«~Potter~? Qu'est-ce que \emph{tu} fais ici~?~»

\emph{Cette} voix n'aurait pas dû être ici.
Harry leva les yeux. «~Drago~? Qu'est-ce que \emph{tu} fais dans ma oh mon dieu tu as des \emph{sbires}.~»

L'un des garçons qui se tenaient derrière Drago semblait être vraiment très musclé pour un enfant de onze ans, et l'autre maintenait une posture suspicieusement équilibrée.

Drago sourit avec suffisance et fit un geste vers l'arrière.
«~Potter, je te présente M. Crabbe~», sa main passa de Muscles à Équilibre, «~et M. Goyle.
Vincent, Gregory, voici Harry Potter.~»

M. Goyle pencha la tête et jeta à Harry un regard qui était probablement censé signifier quelque chose mais qui lui donna juste l'air bigleux.
M. Crabbe dit «~Ravi d'fair vot'connaissance~», d'un ton qui laissait penser qu'il essayait de forcer sa voix vers des octaves aussi graves que possibles.

Une expression consternée passa de façon fugace sur le visage de Drago, mais fut rapidement remplacée par un rictus supérieur.

«~Tu as des \emph{sbires}~! répéta Harry. Où est-ce que \emph{je} peux avoir des sbires~?~»

Drago élargit son sourire suffisant.

«~Potter, j'ai bien peur que pour ça, la première étape ne soit d'être Réparti à Serpentard…

--- Quoi~? Ce n'est pas juste~!

--- … et ensuite que vos familles aient passé un accord avant même votre naissance.~»

Harry regarda M. Crabbe et M. Goyle.
Tous deux essayaient très fort d'avoir l'air menaçants.
C'est-à-dire qu'ils se penchaient en avant, épaules carrés, cou tendu, et regard fixé sur lui.

«~Euh… attends voir, dit Harry. Ça a été organisé il y a des \emph{années}~?

--- Exactement, Potter. Pas de chance.~»

M. Goyle fit apparaître un cure-dent et commença à se nettoyer les dents, l'air toujours menaçant.

«~Et, dit Harry, Lucius a insisté pour que tu grandisses \emph{sans} jamais connaître tes gardes du corps, et que tu ne devais les rencontrer qu'au premier jour d'école.~»

Cela fit disparaître le sourire narquois du visage de Drago.
«~Oui, Potter, on sait tous que tu es brillant, toute l'école est au courant maintenant, tu peux arrêter de frimer…

--- Donc on leur a répété \emph{toute leur vie} qu'ils seraient tes sbires et ils ont passé des \emph{années} à s'imaginer à quoi des sbires sont censés ressembler…~»

Drago grimaça.

«~… et le pire, c'est qu'ils \emph{se connaissent} et qu'ils se sont \emph{entraînés} à ça…

--- Le boss t'a dit d'la fermer~», gronda M. Crabbe.
M. Goyle mordit son cure-dent, le maintenant entre ses incisives, et utilisa une main pour faire craquer les articulations de l'autre.

«~\emph{Je vous ai dit de ne pas faire ça devant Harry Potter~!}~»

Les deux étaient un peu penauds et M. Goyle se hâta de ranger le cure-dent dans une poche.

Mais à l'instant où Drago se détourna d'eux pour faire à nouveau face à Harry, ils reprirent leur air menaçant.

«~Je te demande pardon, dit Drago avec raideur, pour l'insulte que ces \emph{imbéciles} t'ont fait subir.~»

Harry jeta un regard lourd de sens à M. Crabbe et à M. Goyle.
«~Je dirais que tu es un peu rude avec eux, Drago.
\emph{Je} pense que c'est exactement le comportement que j'attendrais de \emph{mes} sbires.
Enfin, si j'avais des sbires.~»

La mâchoire de Drago se décrocha.

«~Eh, Gregory, tu crois pas qu'y essaie d'nou zappâter loin du boss~?

--- Je suis sûr que M. Potter ne serait pas assez stupide pour ça.

--- Oh, même pas en rêve, dit Harry d'une voix onctueuse.
C'est juste quelque chose à garder à l'esprit si votre employeur actuel vous semble ingrat.
Et puis, ça ne peut pas faire de mal d'avoir d'autres offres pour négocier ses conditions de travail, n'est-ce pas~?

--- Pourquoi qu'\emph{il} est à Serdaigle~?

--- Je n'en ai pas la moindre idée, M. Crabbe.

--- \emph{Taisez vous} tous les deux, dit Drago la mâchoire serrée. C'est un \emph{ordre}.~»
Il parvient à transférer à nouveau son attention sur Harry au prix d'un effort visible.
«~Bon, qu'est-ce que tu fais au cours de Défense des Serpentard~?~»

Harry fronça les sourcils. «~Attends.~» Il plongea la main dans sa bourse.
«~Emploi du temps.~» Il regarda le parchemin.
«~Défense, 14h30, et il est maintenant…~» Harry regarda sa montre, qui affichait 11h23.
«~14h23, à moins que j'aie perdu la notion du temps~?~»
Si c'était le cas, eh bien, Harry savait comment se rendre au cours où il était \emph{censé} être.
Dieu qu'il aimait son retourneur de temps, et un jour, quand il serait en âge, ils se marieraient.

«~Non, c'est la bonne heure~», dit Drago, perplexe.
Son regard parcourut le reste de l'auditorium, qui se remplissait de robes à bordures vertes et de…

«~\emph{Gryffidiots~!} cracha Drago. Qu'est ce \emph{qu'ils} font ici~?

--- Hmm, dit Harry. Le professeur Quirrell a dit… j'ai oublié ses mots exacts… qu'il allait ignorer certaines des conventions éducatives de Poudlard.
Peut-être qu'il a juste regroupé toutes ses classes.

--- Euh, dit Drago. Tu es le premier Serdaigle ici.

--- Ouaip. Je suis arrivé tôt.

--- Qu'est-ce que tu fais au dernier rang alors~?~»

Harry cligna des yeux. «~Je sais pas, ça avait l'air d'être un bon endroit pour s'asseoir~?~»

Drago railla. «~Tu aurais voulu être le plus loin possible du professeur tu n'aurais pas fait mieux.~»
Puis il se pencha légèrement en avant, se rapprochant imperceptiblement.
«~À part ça, Potter, c'est vrai ce qu'on raconte sur ce que tu as dit à Derrick et sa bande~?

--- Qui est Derrick~?

--- Tu l'as frappé avec une tarte~?

--- Deux tartes, à vrai dire. Je suis censé lui avoir dit quoi~?

--- Que ce qu'il faisait n'était ni rusé ni ambitieux, et qu'il était une disgrâce à la mémoire de Salazar Serpentard.~»
Drago regardait Harry avec une grande intensité.

«~C'était… à peu près ça, dit Harry. Je pense que c'était plus proche de~: “Cela fait-il partie d'un plan incroyablement intelligent qui vous donnera des avantages futurs, ou est-ce juste une disgrâce au nom de Salazar Serpentard aussi grande qu'elle en a l'air”
ou quelque chose comme ça. Je ne me souviens pas des mots exacts.~»

--- Tu embrouilles tout le monde, tu sais, dit Drago.

--- Hein~?~» dit Harry, sincèrement confus.

«~Warrington dit que rester longtemps sous le Choixpeau est l'un des signes annonciateurs d'un Seigneur des Ténèbres majeur.
Tout le monde en parle et se demande s'il ne faudrait pas commencer à te faire de la lèche, juste au cas où.
Puis toi tu vas protéger une grappe de Poufsouffle, nom de Merlin~!
Et \emph{ensuite} tu dit à Derrick qu'il est une disgrâce à la mémoire de Salazar Serpentard~!
On est \emph{censé} penser quoi~?

--- Que le Choixpeau a décidé de me mettre dans la maison “Serpentard~! Non j'déconne~! Serdaigle~!” et que je me suis comporté en conséquence.~»

M. Crabbe et M. Goyle gloussèrent tous les deux. Immédiatement, M. Goyle se plaqua une main sur la bouche.

«~On ferait mieux de s'asseoir~», dit Drago.
Il hésita, se redressa un peu et prit un ton plus formel.
«~Je voulais te dire que je souhaite continuer notre dernière conversation et que j'accepte tes conditions.~»

Harry hocha la tête.
«~Ça t'embêterait beaucoup si on attendait jusqu'à samedi après-midi~?
Je suis en pleine compétition en ce moment.

--- Une compétition~?

--- Voir si je peux lire tous mes manuels aussi vite qu'Hermione Granger.

--- Granger~», répondit Drago en écho.
Ses yeux se rétrécirent.
«~La sang-de-Bourbe qui croit qu'elle est Merlin~?
Si tu essaies de \emph{la} remettre à sa place, alors tout Serpentard te souhaite la meilleure des chances, Potter, et je ne te dérangerait pas avant samedi.~»
Drago inclina la tête respectueusement, et s'en alla, suivi par ses sbires.

\emph{Oh, ça va être} tellement \emph{amusant de jongler entre les deux, je le sens déjà venir.}

La salle se remplissait maintenant rapidement avec les quatre couleurs d'ourlets~: vert, rouge, jaune et bleu.
Drago et ses deux amis semblaient tenter de récupérer trois places au premier rang -- déjà occupés bien sûr.
M. Crabbe et M. Goyle prenaient vigoureusement leur air menaçant, mais cela ne semblait pas faire beaucoup d'effet.

Harry se pencha sur son manuel de Défense et continua à lire.

\later

À 14h35, une fois que la plupart des sièges soit occupée et que plus personne ne semble vouloir rentrer dans la salle, le professeur Quirrell eut une soudaine convulsion et se redressa sur sa chaise, puis son visage apparut sur tous les objets blancs, plats et rectangulaires qui avaient été installés sur les pupitres des élèves.

Harry fut pris par surprise, autant par l'apparition soudaine du visage de Quirrell que par la ressemblance à la télévision moldue.
Il y avait là quelque chose à la fois triste et nostalgique, cela ressemblait tant à une partie de sa maison sans pour autant l'être réellement…

«~Bonjour, mes jeunes apprentis~», commença Quirrell.
Sa voix semblait venir de l'écran du pupitre et s'adresser directement à Harry.
«~Bienvenue dans votre premier cours de Magie de Combat, comme les fondateurs de Poudlard l'auraient nommé~; ou, puisque qu'il est ainsi appelé en cette fin de vingtième siècle, Défense contre les forces du Mal.~»

On entendit le bruit d'une agitation frénétique tandis que les élèves, pris par surprise, attrapaient leurs parchemins ou carnets de notes.

«~Non, continua Quirrell, franchement, ne vous embêtez pas à noter la façon dont on appelait autrefois ce cours.
Aucune de ces questions futiles ne comptera dans vos notes durant mon cours. C'est une promesse.~»

Plusieurs élèves se redressèrent dans leur siège en entendant cela, plutôt choqués.

Quirrell eut un léger sourire.
«~Ceux d'entre vous qui ont perdu leur temps en lisant à l'avance votre livre inutile de Défense de première année…~»

Quelqu'un sembla s'étrangler. Harry se demanda si c'était Hermione.

«~… ont peut-être l'impression que, bien que ce cours soit nommé Défense contre les forces du Mal, il concerne en fait la défense contre les papillons de cauchemar, qui provoquent des rêves vaguement mauvais, ou les limaces acides, qui peuvent dissoudre et traverser toute l'épaisseur d'une poutre de bois de cinq centimètres si on leur laisse la journée.~»

Quirrell se leva, repoussant sa chaise loin du bureau.
L'écran sur le pupitre de Harry suivait chacun de ses mouvements.
Quirrell s'avança pour faire face à la salle et parla d'une voix puissante~:

«~Le Magyar à pointes est plus grand que douze hommes~!
Il crache du feu si rapidement et si précisément qu'il peut faire fondre un Vif en plein vol~!
Un sortilège de la Mort l'abattra~!~»

On entendit quelques cris de surprise.

«~Le troll des montagnes est plus dangereux que le Magyar à pointes~!
Il est assez fort pour mordre et trancher de l'acier~!
Son cuir est si résistant qu'il dévie les sorts d'étourdissement et sortilèges de découpe~!
Son odorat est si développé qu'il peut savoir à distance si sa proie fait partie d'un groupe ou si elle est seule et vulnérable~!
Bien plus effroyable que tout cela~: le troll est l'unique créature magique qui maintient en permanence une sorte de métamorphose sur lui-même -- il se transforme en permanence en son propre corps.
Si vous parveniez on ne sait comment à lui arracher un bras, un autre lui pousserait en quelques secondes~!
Le feu et l'acide produiront du tissu cicatriciel qui déboussolera \emph{temporairement} les pouvoirs de régénération d'un troll -- pour une heure ou deux~!
Ils sont assez intelligents pour utiliser des outils tels que des gourdins~!
Le troll des montagnes arrive troisième sur la liste des machines à tuer les plus parfaites de la Nature~!
Un sortilège de la Mort l'abattra.~»

Les élèves écoutaient, l'air stupéfaits.

Quirrell sourit sinistrement.
«~Ce qu'on ose appeler un manuel de Défense de troisième année suggère d'exposer le troll à la lumière du jour, ce qui le figera sur place.
Ceci, mes jeunes apprentis, est le genre de savoir inutile que vous ne trouverez dans aucune de mes leçons.
On ne rencontre jamais de troll en plein jour et à découvert~!
La suggestion selon laquelle vous devriez utiliser la lumière du jour pour les arrêter est le fruit d'auteurs de manuels ineptes essayant de frimer avec leur maîtrise des trivialités, au détriment du sens pratique.
Ce n'est pas parce qu'il existe un moyen ridiculement occulte de se débarrasser des trolls des montagnes que vous devez l'utiliser~!
Le sortilège de la Mort est imparable, inarrêtable, et fonctionne à tous les coups tant que la cible a un cerveau.
Si, une fois devenu un sorcier adulte, vous ne parvenez pas à utiliser le sortilège de la Mort, alors vous pouvez simplement transplaner~!
De même si vous faites face à la deuxième des machines à tuer les plus parfaites, le Détraqueur.
Transplanez-vous ailleurs~!

--- À moins bien sûr~», continua Quirrell, sa voix plus grave et plus dure, «~que vous ne soyez sous l'influence d'une malédiction anti-transplanage.
Non, il y a exactement un seul monstre qui sera capable de vous menacer une fois que vous aurez fini votre croissance.
Le monstre le plus dangereux au monde, si dangereux que rien ne lui arrive à la cheville.
Le Mage Noir.
C'est la seule chose qui pourra encore vous menacer.~»

Les lèvres Quirrell formaient une ligne très fine.
«~C'est à contrecœur que je vous enseignerai assez de trivialités pour que vous ayez une note passable à la partie de vos examens de fin de première année exigée par le Ministère.
Puisque votre note exacte à cette partie n'aura aucune incidence sur votre vie future, toute personne désirant une note meilleure que passable est invitée à perdre son temps en étudiant ce livre pathétique qu'on ose appeler manuel.
Le nom de cette matière n'est pas Défense contre les Nuisibles Mineurs.
Vous êtes ici pour apprendre comment vous défendre contre les forces du Mal.
Ce qui signifie, soyons très clairs à ce sujet, vous défendre contre les Mages Noirs.
Des personnes armés de baguettes qui vous veulent du mal et qui y parviendront probablement à moins que vous ne \emph{leur} fassiez du mal en premier~!
Il n'y a pas de défense sans attaque~!
Il n'y a pas de défense sans combat~!
Cette réalité est jugée trop rude pour des enfants de onze ans par les politiciens gras, surpayés et protégés par des Aurors qui ont décidé de votre curriculum.
Puissent ces idiots tomber dans un abysse~!
Vous êtes ici pour la matière qui a été enseignée à Poudlard pendant huit cents ans~!
Bienvenue dans votre première année de Magie de Combat~!~»

Harry commença à applaudir.
Il ne pouvait pas s'en empêcher, c'était trop inspirant.

Une fois que Harry eut commencé à frapper des mains, il y eut quelques reprises éparses venant de Gryffondor, et d'autres, plus nombreuses, de Serpentard, mais la plupart des élèves étaient tout simplement trop abasourdis pour réagir.

Quirrell fit un geste d'arrêt, et les applaudissements moururent instantanément.
«~Merci beaucoup, dit Quirrell. Maintenant, passons aux détails pratiques.
J'ai combiné toutes mes classes de Combat de première année en une seule, ce qui me permet de vous offrir deux fois plus de temps de classe dans des sessions doubles…~»

On entendit des hoquets d'horreur.

«~… un temps de classe augmenté que je compenserai en ne vous donnant aucun devoir.~»

Les hoquets d'horreur s'arrêtèrent brusquement.

«~Oui, vous m'avez bien entendu.
Je vous enseignerai comment vous battre, pas comment écrire deux rouleaux de parchemin sur la notion de combat pour lundi.~»

Harry souhaitait désespérément s'être assis à côté d'Hermione pour pouvoir voir l'expression qu'elle avait en ce moment sur son visage, mais en même temps il était assez certain d'en imaginer une reproduction fidèle.

Aussi, Harry était amoureux. Ce serait un mariage à trois~: lui, le retourneur de temps, et le professeur Quirrell.

«~Pour ceux d'entre vous qui le désirent, j'ai mis en place quelques activités du soir que vous trouverez, je pense, plutôt intéressantes en plus d'être éducatives.
Vous souhaitez montrer au monde vos \emph{propres} capacités au lieu de regarder quatorze personnes jouer au Quidditch~?
Une armée peut combattre avec plus de sept personnes.~»

Chaud \emph{bouillant}.

«~Ces activités ainsi que d'autres vous permettront aussi d'obtenir des points Quirrell.
Que sont des points Quirrell, me direz-vous~?
Le système de points des maisons de Poudlard ne correspond pas à mes besoins, car ils sont rares.
Je préfère faire savoir à mes élèves comment ils se débrouillent plus fréquemment.
Et dans les rares occasions où je vous proposerai un contrôle écrit, il se notera lui-même au fur et à mesure, et si vous vous trompez sur trop de questions de la même catégorie, votre copie vous montrera le nom des élèves qui ont correctement répondu à ces questions, et ces élèves pourront gagner des points Quirrell en vous aidant.~»

… waouh. Pourquoi les autres professeurs n'utilisaient-ils pas un système comme celui-ci~?

«~À quoi servent les points Quirrell, vous demandez-vous~?
Pour commencer, dix points Quirrell valent un point normal pour votre maison.
Mais ils vous permettront aussi d'obtenir d'autres faveurs.
Vous préfèrez passer votre examen à un horaire particulier~?
Y a-t-il un cours où vous aimeriez vraiment pouvoir être absent~?
Vous découvrirez que je peux être très accommodant envers les étudiants qui ont accumulé assez de points Quirrell.
Les points Quirrell décideront des futurs généraux des armées.
Et pour Noël -- juste avant les vacances de Noël -- j'accorderai un vœu à quelqu'un.
Toute action en rapport avec l'école et accessible à mon pouvoir, mon influence, et par-dessus tout, mon ingéniosité.
Oui, j'étais à Serpentard, et je vous offre la mise en place d'un complot sournois en votre bénéfice, si c'est ce que vous souhaitez.
Ce vœu sera accordé à celui ou celle qui, parmi les élèves des sept années de Poudlard, aura obtenu le plus de points Quirrell.~»

Cela sera Harry.

«~Laissez maintenant vos manuels et objets divers à vos pupitres -- ils seront en sécurité, les écrans les surveilleront pour vous -- et descendez sur cette plateforme.
Nous allons jouer à un jeu appelé \emph{Qui est l'élève le plus dangereux de la classe}.~»

\later

La baguette dans sa main droite, Harry fit un mouvement du poignet et dit «~\emph{Ma-ha-su~!}~»

Il y eut un autre «~bing~» aigu émis par la sphère bleue flottante, la cible que Quirrell avait assignée à Harry.
Ce son particulier signifiait un coup dans le mille, ce que Harry avait réussi neuf fois sur ses dix précédentes tentatives.

Quirrell avait déniché d'on ne sait où un sort qui était incroyablement facile à prononcer, \emph{et} avait un mouvement de baguette ridiculement simple, \emph{et} avait tendance à toucher l'endroit où vous braquiez votre regard.
Quirrell avait proclamé avec dédain que la vraie magie de combat était bien plus difficile que cela.
Que le sort était totalement inutile en vrai combat.
Que c'était une explosion de magie à peine contrôlé, dont le seul vrai intérêt était de viser avec précision, et qui produirait, à l'impact, une douleur brièvement équivalente à recevoir un coup de poing sur le nez.
Que le seul but de ce test était de voir qui apprenait vite, puisque Quirrell était certain que personne n'aurait jamais rencontré ce sort auparavant ni quoi que ce soit l'approchant.

harry se fichait complètement de tout cela.

«~\emph{Ma-ha-su~!}~»

Un \emph{éclair d'énergie rouge} jaillit de sa baguette et frappa la cible, et la sphère bleue fit à nouveau le bing, ce qui voulait dire que le sort \emph{avait vraiment fonctionné pour lui}.

Pour la première fois depuis son arrivée à Poudlard, Harry se sentait être un vrai sorcier.
Il aurait aimé que la cible esquive, comme les petites sphères que Ben Kenobi avait utilisées pour entraîner Luke, mais pour quelque raison, Quirrell avait préféré aligner tous les élèves et les cibles de façon bien ordonnée, ce qui leur assurait qu'ils ne se tiraient pas les uns sur les autres.

Alors Harry abaissa sa baguette, se mit de profil, et d'un mouvement brusque pointa la baguette, fit le geste du poignet et cria~: «~\emph{Ma-ha-su}~!~»

On entendit un «~dong~» plus grave, ce qui voulait dire qu'il avait presque réussi.

Harry mit sa baguette dans sa poche, changea de profil, dégaina et projeta un autre éclair d'énergie rouge.

Le bing aigu qui en résulta était de loin l'un des sons les plus satisfaisants jamais entendus de sa vie.
Harry voulait crier de triomphe à s'en faire éclater les poumons.
\shout{Je fais de la magie~! Craignez-moi, lois de la physique, j'arrive et je vais vous transgresser~!}

«~\emph{Ma-ha-su}~!~» la voix de Harry était forte, mais à peine discernable dans le brouhaha de cris similaires sur tout l'espace de la plateforme.

«~Assez~», dit la voix amplifiée de Quirrell.
(Le son n'était pas fort. Il semblait avoir un volume normal et venir de juste derrière votre épaule gauche, quelle que soit votre position rapport au professeur.)
«~Je vois que tout le monde a réussi au moins une fois.~»
Les sphères cibles devinrent rouges et commencèrent à dériver vers le plafond.

Quirrell se tenait sur l'estrade au centre de la plateforme, s'appuyant légèrement d'une main sur le bureau.

% STOP HERE
«~Je vous ai dit, dit le professeur Quirrell, que nous allions jouer à un jeu nommé \emph{Qui est l'élève le plus dangereux de la classe}. Il y a un élève dans cette classe qui a maîtrisé le Sort d'Attaque Simple Sumérien plus vite que quiconque…~»

Oh blah blah blah.

«~… et a ensuite aidé sept autres étudiants. Ce pour quoi elle a gagné les sept premiers points Quirrell de votre année. Hermione Granger, merci de vous avancer. Le moment est venu de passer à l'étape suivante du jeu.~»

Hermione Granger marcha à grands pas, avec sur son visage un air de triomphe et d'appréhension mélangés. Les Serdaigle la regardaient fièrement, les Serpentard avec mépris, et Harry avec un franc agacement. Harry avait bien réussi cette fois. Il était même probablement dans la moitié supérieure du cours, maintenant que tout le monde avait fait face à un sort uniformément peu familier et que Harry avait lu l'intégralité de \emph{Théorie Magique} d'Adalbert Lasornette. Et pourtant \emph{Hermione était encore meilleure}.

Quelque part, au fond de ses pensées, se cachait la peur qu'Hermione soit tout simplement plus intelligente que lui.

Mais pour le moment, Harry allait ancrer ses espoirs sur les deux faits suivants~: (a) Hermione avait lu bien plus que les manuels standards, et (b) Adalbert Lasornette était un couillon peu inspiré qui avait écrit \emph{Théorie Magique} juste pour plaire à une commission scolaire qui n'avait pas une très bonne opinion des enfants de onze ans.

Hermione parvint à l'estrade centrale et grimpa sur la marche.

«~Hermione Granger a maîtrisé un sort totalement inconnu en deux minutes, presque une minute plus vite que le deuxième plus rapide.~» Le professeur Quirrell pivota lentement pour regarder tous les élèves qui les observaient. «~L'intelligence de Mademoiselle Granger pourrait-elle faire d'elle l'élève la plus dangereuse de cette salle~? Eh bien~? Qu'en pensez-vous~?~»

Pour le moment, personne n'avait l'air de penser quoi que ce soit. Même Harry ne savait pas quoi dire.

«~Et si on le découvrait ensemble~?~» dit le professeur Quirrell. Il se retourna vers Hermione, et fit un geste en direction de la salle. «~Choisissez n'importe quel étudiant et jetez-lui le Sort d'Attaque Simple.~»

Hermione se pétrifia.

«~Allons, dit doucement le professeur Quirrell. Vous avez jeté ce sort à la perfection plus de cinquante fois. Il ne fait aucun dommage permanent et n'est pas si douloureux que ça. Il fait à peu près aussi mal qu'un bon coup de poing dans le nez et ne dure que quelques secondes.~» La voix du professeur Quirrell devint plus dure. «~C'est un ordre direct de votre professeur, Mademoiselle Granger. Choisissez une cible et jetez un Sort d'Attaque Simple.~»

Le visage de Hermione était tordu d'horreur et sa baguette tremblait. Les doigts de Harry se serraient dans son poing, par pure empathie. Même s'il comprenait ce que le professeur Quirrell essayait de faire. Même s'il voyait bien ce que le professeur Quirrell essayait de démontrer.

«~Si vous ne levez \emph{pas} votre baguette et ne tirez \emph{pas}, Mademoiselle Granger, vous perdrez un point Quirrell.~»

Harry regarda Hermione, espérant qu'elle regarde dans sa direction. Sa main droite tapotait doucement sur sa poitrine. \emph{Choisis-moi, je n'ai pas peur…}

La baguette de Hermione pivota dans sa main, puis son visage se détendit, et elle abaissa sa baguette contre son flanc.

«~Non,~» dit Hermione Granger.

Sa voix était calme, et même si elle n'était pas forte, le silence était tel que tout le monde l'entendit.

«~Alors je dois vous ôter un point, dit le professeur Quirrell. C'était un test, et vous l'avez échoué.~»

Hermione fut touchée par ces paroles. Harry le voyait bien. Mais elle garda ses épaules droites.

La voix du professeur Quirrell était compatissante et semblait emplir la salle entière. «~Savoir des choses ne suffit pas toujours, Mademoiselle Granger. Si vous ne pouvez donner et recevoir des coups aussi intenses qu'un choc contre le petit doigt de pied, alors vous ne pouvez pas vous défendre et vous ne réussirez pas mon cours de Défense. Rejoignez vos camarades, s'il vous plaît.~»

Hermione marcha jusqu'au groupe de Serdaigle. Elle semblait être en paix avec elle-même, et, pour une étrange raison, Harry aurait bien voulu applaudir. Même si le professeur Quirrell avait eu \emph{raison}.

«~Donc, dit le professeur Quirrell. Il devient clair qu'Hermione Granger n'est pas l'élève la plus dangereuse de la salle. Qui ici pense être la personne la plus dangereuse~? -- à part moi, bien sûr.~»

Sans même y penser, les yeux de Harry se tournèrent vers le contingent de Serpentard.

«~Drago, de la Noble et Ancienne Maison Malfoy, dit le professeur Quirrell. Il semble que nombreux sont ceux qui regardent dans votre direction. Merci de vous avancer.~»

Drago s'exécuta, et son port comportait un certain orgueil. Il alla jusqu'à l'estrade et regarda le professeur Quirrell en souriant.

«~M. Malfoy, dit le professeur Quirrell. Feu.~»

Harry aurait essayé de l'en empêcher s'il en avait eu le temps, mais d'un mouvement gracieux Drago avait tournoyé vers les Serdaigle, avait levé sa baguette, avait dit «~\emph{Mahasu~!}~» comme si c'était un mot d'une seule syllabe, et Hermione avait dit «~Ouh~!~» et c'était fini.

«~Bien envoyé, dit le professeur Quirrell. Deux points Quirrell pour vous. Mais dites-moi, pourquoi avez-vous visé Mademoiselle Granger~?~»

Il y eut une pause.

Drago dit enfin~: «~Parce que c'était celle qui ressortait le plus.~»

Les lèvres du professeur Quirrell formèrent un fin sourire. «~Et voilà la véritable raison pour laquelle Drago Malfoy est dangereux. S'il avait choisi n'importe qui d'autre, cette personne lui en aurait probablement voulu d'avoir été choisie, et M. Malfoy se serait probablement fait un ennemi. Et bien que M. Malfoy aurait pu donner une autre justification expliquant pourquoi il l'a choisie elle, cela n'aurait servi à rien d'autre qu'à énerver certains d'entre vous, alors que d'autres l'acclament déjà qu'il dise quelque chose ou pas. En bref, M. Malfoy est dangereux parce qu'il sait qui frapper et qui ne pas frapper, comment se faire des alliés, et comment éviter de se faire des ennemis. Deux points Quirrell de plus pour vous, M. Malfoy. Et comme vous venez de démontrer une vertu Serpentard, je pense que la Maison de Salazar a elle aussi gagné un point. Vous pouvez rejoindre vos amis.~»

Drago s'inclina légèrement et retourna au contingent de Serpentard. Quelques applaudissements s'élevèrent des robes à bordures vertes, mais le professeur Quirrell fit un geste cassant et le silence retomba.

«~Il semblerait que notre jeu soit fini, dit le professeur Quirrell. Et pourtant, il reste un étudiant dans cette salle qui est plus dangereux que le descendant de Malfoy.~»

Et \emph{maintenant}, pour une étrange raison, il semblait y avoir vraiment beaucoup de gens regardant en direction de…

«~Harry Potter. Veuillez vous avancer.~»

Ça ne présageait rien de bon.

Avec réticence, Harry marcha jusqu'à l'endroit où le professeur Quirrell se trouvait, sur son estrade surélevée, toujours appuyé contre son bureau.

La nervosité d'être mis sous les projecteurs semblait acérer la sagacité de Harry à mesure qu'il s'approchait de l'estrade, et son cerveau parcourait les possibilités, essayait de deviner ce qui, selon le professeur Quirrell, pourrait démontrer la dangerosité de Harry. Lui demanderait-il de jeter un sort~? De vaincre un Seigneur des Ténèbres~?

De démontrer son immunité au sortilège de la Mort~? Le professeur Quirrell était certainement trop intelligent pour \emph{ça}…

Harry s'arrêta bien avant l'estrade, et le professeur Quirrell ne lui demanda pas de s'approcher plus.

«~Ce qui est ironique, dit le professeur Quirrell, c'est que vous avez tous regardé la bonne personne, mais pour les mauvaises raisons. Vous vous dites~», les lèvres du professeur Quirrell se tordirent, «~que Harry Potter a vaincu le Seigneur des Ténèbres, et qu'il doit donc être très dangereux. Bah. Il avait un an. Quel que soit le caprice du destin qui a tué le Seigneur des Ténèbres, cela avait bien peu à voir avec les capacités de combattant de M. Potter. Mais après avoir entendu des rumeurs parlant d'un Serdaigle faisant face à cinq Serpentard plus âgés, j'ai interrogé plusieurs témoins oculaires et en suis arrivé à la conclusion que Harry Potter serait le plus dangereux de mes étudiants.~»

Un choc d'adrénaline se déversa dans le système sanguin de Harry. Il ne savait pas à quelle conclusion exacte le professeur Quirrell était parvenu, mais ça ne pouvait pas être bon.

«~Ah, professeur Quirrell…~» commença à dire Harry.

Le professeur Quirrell semblait amusé. «~Vous pensez que je suis parvenu à une conclusion erronée, n'est-ce pas, M. Potter~? Vous apprendrez à attendre mieux venant de \emph{moi}.~» Le professeur Quirrell se redressa là où il avait été penché. «~M. Potter, toutes les choses ont des usages courants. Donnez-moi dix usages inhabituels d'objets de cette pièce pouvant être faits en combat~!~»

Pendant un moment, Harry eut le souffle coupé par le choc pur et brut d'avoir été compris.

Et les idées commencèrent à se déverser.

«~Il y a des pupitres assez lourds pour être mortels si jetés d'une hauteur suffisante. Il y a des chaises avec des jambes en métal qui pourraient empaler quelqu'un si on les poussait assez fort. L'air dans la salle serait mortel par son absence, puisque les gens meurent dans le vide, et il peut aussi servir de porteur de gaz empoisonnés.~»

Harry s'arrêta pour reprendre son souffle, et le professeur Quirrell dit au milieu de la pause~:

«~En voilà trois. Il vous en faut dix. Le reste de la classe croit que vous avez épuisé tout le contenu de cette salle.

--- \emph{Ha~!} Le sol peut être enlevé pour créer une fosse de piques où tomber, le plafond peut être écroulé sur quelqu'un, les murs peuvent servir de matériel de base de métamorphose pour toutes sortes de choses mortelles -- des couteaux, par exemple.

--- En voilà six. Mais vous grattez sûrement le fond à présent~?

--- Je n'ai même pas commencé~! Regardez les gens~! Envoyer un Gryffondor attaquer l'ennemi est une utilisation \emph{ordinaire}, bien sûr…

--- Je ne vous aurais pas laissé compter celle-là.

---… mais leur sang peut aussi être utilisé pour noyer quelqu'un. Les Serdaigle sont connus pour leur cerveau, mais leurs organes internes pourraient être revendus au marché noir pour engager un assassin. Les Serpentard ne sont pas seulement utiles en tant qu'assassins, on peut aussi les projeter avec une vélocité suffisante pour écraser le corps d'un ennemi. Et les Poufsouffle, en plus de travailler dur, contiennent aussi certains os qui peuvent être enlevés, aiguisés, et utilisés pour poignarder quelqu'un.~»

Le reste de la salle regardait maintenant Harry avec horreur. Même les Serpentard avaient l'air choqués.

«~En voilà dix, même si je suis généreux en comptant celui de Serdaigle. Maintenant, en bonus, un point supplémentaire pour chaque utilisation d'objet de cette pièce que vous n'avez pas encore nommée.~» Le professeur Quirrell gratifia Harry d'un sourire sympathique.

«~Le reste de la classe pense que vous êtes à présent en difficulté, parce que vous avez tout nommé, mis à part les cibles, et que vous ne savez pas quoi faire d'elles.

--- Bah~! J'ai nommé les gens, mais pas ma robe, qui peut être utilisée pour faire suffoquer un ennemi une fois enroulée plusieurs fois autour de sa tête, ou la robe de Hermione Granger, qui peut être découpée en bandelettes attachées les unes aux autres pour pendre quelqu'un, ou la robe de Drago Malfoy, qui peut être utilisée pour allumer un feu…

--- Trois points, dit le professeur Quirrell, plus de vêtements à présent.

--- Ma baguette peut être poussée dans le cerveau d'un ennemi à travers son globe oculaire~» et quelqu'un fit un bruit d'étranglement horrifié.

«~Quatre points, plus de baguettes.

--- Ma montre pourrait étouffer quelqu'un si je la lui fourrais dans la gorge…

--- Cinq points, c'est assez.

--- Hpmf, dit Harry. Dix points Quirrell pour un point de Maison, c'est ça~? Vous auriez dû me laisser continuer jusqu'à ce que je gagne la coupe des Quatre Maisons, je n'ai même pas commencé à parler des utilisations inhabituelles de ce que j'ai dans mes poches~» ni de la bourse en peau de Moke elle-même, et il ne pouvait pas parler du Retourneur de Temps ou de la Cape d'Invisibilité, mais il devait y avoir \emph{quelque chose} à dire au sujet de ces sphères rouges…

«~\emph{Assez}, M. Potter. Eh bien, pensez-vous tous bien comprendre ce qui fait de M. Potter l'élève le plus dangereux de la classe~?~»

Il y eut un bas murmure d'assentiment.

«~Dites-le haut et fort, s'il vous plaît. Terry Boot, qu'est-ce qui rend votre compagnon de dortoir si dangereux~?

--- Ah… euh… il est créatif~?

--- \emph{Faux}~!~» tonna le professeur Quirrell, et son poing s'abattit avec force sur le bureau dans un bruit amplifié qui fit bondir tout le monde. «~Toutes les idées de M. Potter étaient pires qu'inutiles~!~»

Harry sursauta de surprise.

«~Enlever le sol pour créer un piège à piques~? Ridicule~! En combat, vous n'avez pas le temps de préparation nécessaire à cela, et si vous l'aviez, il y aurait des façons cent fois meilleures de l'utiliser~! Métamorphoser des objets à partir des murs~? M. Potter ne sait pas effectuer une métamorphose~! M. Potter a eu exactement une idée qu'il pourrait utiliser, maintenant, sans une vaste préparation ou la coopération de l'ennemi ou une magie qu'il ne connaît pas. L'idée était de fourrer sa baguette dans l'œil de son ennemi. Ce qui briserait probablement sa baguette plutôt que de tuer son opposant~! En bref, M. Potter, j'ai bien peur que vos suggestions n'aient été uniformément nulles.

--- Quoi~? dit Harry avec indignation. Vous m'avez \emph{demandé} des idées inhabituelles, pas des idées pratiques~! Je sortais des sentiers battus~! Comment utiliseriez-\emph{vous} quelque chose dans cette salle dans le but de tuer quelqu'un~?~»

L'expression du professeur Quirrell était désapprobatrice, mais il y avait des plis rieurs autour de ses yeux. «~M. Potter, je n'ai jamais dit que vous deviez \emph{tuer}. Il y a un temps et un lieu pour prendre un ennemi vivant, et l'intérieur de Poudlard est généralement l'un de ces endroits. Mais pour répondre à votre question, le frapper à l'arrière du cou avec le tranchant d'une chaise.~»

Il y eut quelques rires venant de Serpentard, mais ils riaient avec Harry, pas de lui.

Mis à part eux, tout le monde avait l'air assez horrifié.

«~Mais M. Potter a maintenant démontré pourquoi il est l'étudiant le plus dangereux de la classe. Je lui ai demandé des utilisations inhabituelles d'objets en cas de combat. M. Potter aurait pu suggérer l'utilisation d'un bureau pour bloquer un maléfice, ou celle d'une chaise pour faire trébucher un ennemi approchant, ou d'enrouler du tissu autour de son bras pour créer un bouclier improvisé. Au lieu de ça, chaque utilisation mentionnée par M. Potter était offensive plutôt que défensive, et soit fatale, soit potentiellement fatale.~»

Quoi~? Attendez, ça ne pouvait pas être vrai… Harry eut une sensation de vertige soudaine alors qu'il essayait de se souvenir de ce qu'il avait suggéré, il y avait sûrement un contre-exemple…

«~Et voilà, dit le professeur Quirrell, pourquoi les idées de M. Potter étaient si étranges et si inutiles -- parce qu'il devait aller loin dans l'incommode afin d'atteindre son but~: \emph{tuer l'ennemi}. Pour lui, toute idée ne menant pas à cela ne méritait pas d'être prise en considération. Ceci reflète une qualité que nous pourrions nommer \emph{intention de tuer}. Je l'ai. Harry Potter l'a, et c'est pour cela qu'il a pu regarder cinq Serpentard plus âgés droit dans les yeux. Drago Malfoy ne l'a pas, pas encore. M. Malfoy n'éviterait certainement pas de discuter de meurtre ordinaire, mais même lui a été choqué -- oui, vous l'étiez M. Malfoy, je regardais votre visage -- lorsque M. Potter a décrit comment utiliser les corps de ses camarades comme de la matière première. Il y a des censeurs dans votre esprit qui vous font reculer face à de telles pensées. M. Potter pense \emph{uniquement} à la façon de tuer l'ennemi, il utilisera tout moyen disponible, il ne reculera pas, ses censeurs sont éteints. Bien que son jeune génie soit si indiscipliné et si incommode qu'il en devienne inutile, son \emph{intention de tuer} fait de Harry Potter \emph{le plus dangereux élève de la classe}. Un dernier point Quirrell -- non, disons un point pour Serdaigle -- pour votre possession de cette qualité indispensable à un vrai sorcier de combat.~»

La bouche de Harry était grande ouverte, et, dans un état de choc muet, il cherchait frénétiquement quelque chose à répondre. \emph{Ça n'a tellement aucun rapport avec qui je suis vraiment~!}

Mais il pouvait voir que les autres élèves commençaient à y croire. L'esprit de Harry passait en revue les dénis potentiels et ne trouvait rien qui pourrait tenir contre la voix autoritaire du professeur Quirrell. Le mieux que Harry avait trouvé était «~Je ne suis pas un psychopathe, je suis juste très créatif~», et encore, ça semblait menaçant. Il lui fallait dire quelque chose d'inattendu, quelque chose qui pousserait les gens à s'interrompre dans leurs pensées et à reconsidérer…

«~Et maintenant, dit le professeur Quirrell, M. Potter. Feu.~»

Rien ne se passa, bien sûr.

«~Ah, bon~», dit le professeur Quirrell. Il soupira. «~J'imagine que nous devons tous commencer quelque part. M. Potter, choisissez n'importe quel étudiant et jetez-lui un Sort d'Attaque Simple. Vous le \emph{ferez} avant la fin des cours. Sinon, je vais commencer à déduire des points, et je continuerai jusqu'à ce que vous vous soyez exécuté.~»

Harry leva précautionneusement sa baguette. Il fallait qu'il le fasse, ou le professeur Quirrell risquait de commencer à déduire des points tout de suite.

Doucement, comme s'il avait été sur une plaque chauffante, Harry pivota vers les Serpentard.

Et les yeux de Harry rencontrèrent ceux de Drago.

Drago n'avait pas l'air le moins du monde effrayé. Il ne lui donnait aucun signe visible d'assentiment, tel que celui que Harry avait donné à Hermione, mais on pouvait difficilement s'attendre à ce qu'il le fasse. Les autres Serpentard auraient trouvé cela plutôt étrange.

«~Pourquoi cette hésitation~? dit le professeur Quirrell. Il n'y a qu'un seul choix évident.

--- Oui, dit Harry. Seulement un choix \emph{évident}.~»

Harry fit un mouvement du poignet et dit «~\emph{Ma-ha-su}~!~»

Un silence complet s'abattit dans la salle.

Harry secoua son bras gauche, essayait de se débarrasser de la douleur qui persistait.

Il y eut un peu plus de silence.

Et enfin le professeur Quirrell soupira. «~Oui, oui, très ingénieux, mais il y avait là une leçon à apprendre et vous l'avez esquivée. Un point de moins à Serdaigle pour avoir démontré votre intelligence au détriment du véritable but. Le cours est terminé.~»

Et avant que quiconque puisse dire quoi que ce soit, Harry chanta~:

«~Je rigole~! SERDAIGLE~!~»

Il y eut un silence pendant un bref moment, le bruit de gens réfléchissant, puis les murmures commencèrent et grimpèrent rapidement jusqu'à devenir le grondement d'une conversation.

Harry se tourna vers le professeur Quirrell, ils avaient besoin de parler…

Quirrell s'était affalé sur lui-même et se traînait péniblement jusqu'à sa chaise.

Non. Inacceptable. Ils avaient \emph{vraiment} besoin de parler. Le numéro de zombie pouvait aller se faire voir, le professeur Quirrell se réveillerait probablement si Harry lui flanquait quelques coups de coude. Harry s'avança…

NON

PAS BIEN

MAUVAISE IDÉE

Harry oscilla, arrêté net dans sa marche, se sentant étourdi.

Puis une nuée de Serdaigle s'abattit sur lui et les discussions commencèrent.~
%  LocalWords:  squinty meetcha tryna doin Gryffindorks til su Adalbert
%  LocalWords:  Waffling’s Mahasu Hmph
