% vim:spell:spelllang=fr

\chapter{Être consciencieux}

\lettrinepara[ante=«~]{F}{\emph{rigideiro~!}}~»

\hplettrineextrapara
Harry trempa un doigt dans le verre d'eau posé sur son bureau.
Il aurait dû être froid.
Mais tiède il était, et tiède il était resté.
Une fois de plus.

Harry se sentait trahi. Profondément trahi.

Il y avait des centaines de romans de \emph{fantasy} éparpillés dans la maison Verres.
Harry en avait lu une bonne quantité.
Et il lui semblait de plus en plus qu'il avait un mystérieux côté obscur.
Donc, après que le verre d'eau eut refusé de coopérer les premières fois, Harry avait jeté un regard autour de lui pendant le cours de sortilèges pour vérifier que personne ne le regardait, il avait pris une profonde inspiration et s'était mis volontairement en colère.
Pensé aux Serpentard malmenant Neville et au jeu où on faisait tomber vos livres à chaque fois que vous essayez de les ramasser.
Pensé à ce que Drago Malfoy avait dit sur la fille Lovegood, âgée de dix ans, à la façon dont le Magenmagot fonctionnait vraiment…

Et la fureur s'était répandu dans son sang, il avait levé sa baguette d'une main tremblante de haine et dit d'un ton glacé~«~\emph{Frigideiro~!}~» et absolument rien ne s'était produit.

Harry s'était fait \emph{arnaquer}.
Il voulait porter réclamation et demander un \emph{remboursement} sur son côté obscur, qui aurait clairement \emph{dû} posséder d'irrésistibles pouvoirs magiques mais s'était révélé \emph{défectueux}.

«~\emph{Frigideiro~!}~» dit de nouveau Hermione, depuis le pupitre voisin.
Son eau était un solide bloc de glace et sur le rebord du verre s'étaient formés des petits cristaux blancs.
Elle semblait totalement absorbée par son travail et aucunement consciente que les autres élèves du cours la regardaient les yeux pleins de haine, ce qui était soit (a) dangereusement inconscient de sa part, soit (b) une mise en scène parfaitement maîtrisée qui tenait de la performance artistique.

«~Oh, \emph{très bien}, Mademoiselle Granger~!~» couina Filius Flitwick, leur professeur de sortilèges et directeur de Serdaigle, un homme minuscule qui ne montrait aucun signe d'être un ancien champion de duel.
«~Excellent~! Prodigieux~!~»

Harry s'était attendu, dans le pire des cas, à être deuxième derrière Hermione.
Il aurait bien sûr préféré qu'\emph{elle} rivalise avec \emph{lui} depuis la seconde place, mais il aurait accepté que la situation soit inversée.

En ce lundi, Harry était parti pour être bon dernier de la classe, une place pour laquelle rivalisaient tous les autres enfants élevés par des Moldus mis à part Hermione.
Laquelle était seule et sans rival au sommet, pauvre petite.

Flitwick se tenait au-dessus du pupitre d'une autre née-moldue et ajustait doucement la façon dont elle tenait sa baguette.

Harry jeta un coup d'œil à Hermione.
Il déglutit.
Elle tenait un rôle évident dans l'ordre naturel du monde…
«~Hermione~? dit timidement Harry.
Aurais-tu la moindre idée de ce que je fais de travers~?~»

Les yeux d'Hermione brillèrent d'une effroyable obligeance, et quelque chose au fond du cerveau de Harry hurla de désespoir et d'humiliation.

Cinq minutes plus tard, l'eau de Harry semblait bien être perceptiblement plus froide que la température de la pièce.
Hermione lui avait verbalement tapoté la tête, dit de prononcer avec plus de soin la prochaine fois, et était partie aider quelqu'un d'autre.

Le professeur Flitwick lui avait donné un point pour avoir aidé Harry.

Ce dernier serrait les dents si fort que sa mâchoire était douloureuse, et ça n'aidait pas la prononciation.

\emph{Je me fiche que ce soit de la compétition déloyale.
Je sais exactement ce que je vais faire avec mes deux heures supplémentaires par jour.
Je vais m'asseoir dans ma malle et étudier jusqu'à ce que je sois au niveau d'Hermione.}

\later

«~La métamorphose est l'une des magies les plus complexes et les plus dangereuses qui vous seront données d'apprendre à Poudlard~», dit la professeure McGonagall.
Il n'y avait pas la moindre trace de légèreté sur le visage de la vieille sorcière sévère.
«~Toute personne dissipée dans mon cours devra partir et ne reviendra pas.
Vous avez été prévenus.~»

Ella abaissa sa baguette pour en frapper le bureau, dont la forme changea pour prendre en douceur celle d'un cochon.
Deux élèves nés-moldus émirent de petits glapissements.
Le cochon regarda autour de lui, l'air confus puis redevint un bureau.

La professeure de métamorphose balaya la classe du regard, et ses yeux s'arrêtèrent sur un élève.

«~M. Potter, dit McGonagall.
Vous n'avez reçu vos manuels scolaires qu'il y a quelques jours.
Avez-vous commencé à lire votre livre de métamorphose~?

--- Non, désolé professeure, dit Harry.

--- Vous n'avez pas à vous excuser, M. Potter, si vous deviez le lire en avance, nous vous en aurions fait part.~»
McGonagall frappa le bureau d'un petit coup sec.
«~M. Potter, voudriez-vous bien essayer de deviner si c'est un bureau que j'ai métamorphosé en cochon, ou c'était à la base un cochon et que j'ai brièvement enlevé la métamorphose~?
Vous le sauriez si vous aviez lu le premier chapitre de votre manuel.~»

Harry fronça légèrement les sourcils.
«~J'imagine qu'il serait plus facile de commencer avec un cochon, puisque si ça avait d'abord été un bureau, il ne saurait peut-être pas comment se tenir debout.~»

McGonagall secoua la tête.
«~Ce n'est pas votre faute, M. Potter, mais la réponse correcte est qu'en cours de métamorphose, on n'essaye \emph{pas} de deviner.
Les mauvaises réponses seront notées avec une sévérité extrême, les questions laissées sans réponse seront notées avec une grande indulgence.
Vous devez apprendre à savoir ce que vous ne savez pas.
Si je vous pose n'importe quelle question, peu importe qu'elle soit évidente ou élémentaire, et que vous répondez “Je ne suis pas sûr”, je ne vous en voudrai pas, et celui ou celle qui rira fera perdre des points à sa maison.
Pouvez-vous me dire pourquoi cette règle existe, M. Potter~?~»

\emph{Parce qu'une seule erreur en métamorphose peut être incroyablement dangereuse.}

«~Non.

--- Correct.
La métamorphose est plus dangereuse que le transplanage, qui n'est pas enseigné avant la sixième année.
Malheureusement, la métamorphose doit être apprise et pratiquée jeune afin de maximiser vos capacités une fois adulte.
C'est donc un sujet dangereux, et vous devriez être terrifiés à l'idée de faire la moindre erreur, car aucun de mes élèves n'a jamais eu de séquelles permanentes, et je serais \emph{extrêmement ennuyée} si vous étiez la première classe à \emph{entacher mon dossier}.~»

On entendit plusieurs élèves déglutir.

McGonagall se leva de derrière son bureau pour s'approcher du mur sur lequel se trouvait un tableau en bois poli.
«~Il existe de nombreuses raisons pour lesquelles la métamorphose est dangereuse, mais l'une d'elles s'élève bien au-dessus des autres.~»
Elle sortit une plume courte et large, et dessina en lettres rouge vif qu'elle souligna ensuite en bleu avec la même plume~:

\McGonagallWhiteBoard{Une métamorphose n'est pas permanente~!}

«~Une métamorphose n'est pas permanente~! dit McGonagall.
Une métamorphose n'est pas permanente~!
Une métamorphose n'est pas permanente~!
M. Potter, supposez qu'un élève métamorphose un morceau de bois en un verre d'eau et que vous le buviez.
Que pouvez-vous imaginer qu'il se passera lorsque la métamorphose se dissipera~?~»
Il y eut une pause.
«~Excusez-moi, je n'aurais pas dû vous demander cela, M. Potter, j'oubliais que vous êtes béni d'une imagination exceptionnellement pessimiste…

--- Ça va, dit Harry en déglutissant.
Donc la première réponse est que je ne \emph{sais} pas~», McGonagall approuva de la tête, «~mais j'\emph{imagine} qu'il pourrait y avoir… du bois dans mon estomac, et dans mon système sanguin, et si une partie de cette eau avait été absorbée par mes tissus corporels -- serait-ce de la pulpe de bois ou du bois solide ou…~»
La connaissance limitée qu'avait Harry de la magie le mit en défaut.
Il ne comprenait déjà pas comment les différentes parties du bois se transformeraient en eau,
il ne pouvait donc pas non plus comprendre ce qui se passerait après que les molécules d'eau soient mélangées par les mouvements thermiques habituels, que la magie se dissipe, et que la transformation s'inverse.

Le visage de McGonagall ne laissait passer aucune émotion.
«~Comme M. Potter l'a correctement déduit, il serait extrêmement malade et devrait se faire cheminetter immédiatement à l'hôpital Sainte Mangouste, s'il voulait avoir la moindre chance de survivre.
Merci d'ouvrir vos livres à la page~5.~»

Même sans aucun son pour accompagner l'image mouvante, on pouvait voir que la femme à la peau horriblement décolorée hurlait.

«~Le criminel qui a initialement métamorphosé de l'or en vin et l'a donné à boire à cette femme, “en paiement de la dette,” comme il l'a dit, a reçu une sentence de dix ans à Azkaban.
Merci de tourner la page.
Voici un Détraqueur.
Ce sont les gardiens d'Azkaban.
Ils drainent votre magie, votre vie, et toute pensée heureuse que vous essayez d'avoir.
L'image page 7 est celle du criminel dix ans plus tard, à sa sortie.
Vous remarquerez qu'il est mort -- oui, M. Potter~?

--- Professeure, dit Harry, dans un cas comme celui-ci, si le pire se produit, y a-t-il un moyen de \emph{maintenir} la métamorphose~?

--- Non, répondit catégoriquement McGonagall.
Maintenir une métamorphose est un drain magique permanent qui croît proportionnellement avec la taille de la forme cible.
Et il vous faudrait entrer en contact avec la cible à intervalles réguliers de quelques heures, ce qui, dans ce genre de cas, est impossible.
Les désastres comme celui-ci sont \emph{irrécupérables}~!~»

McGonagall se pencha en avant. Son visage devint très dur.
«~Vous ne métamorphoserez absolument jamais quoi que ce soit en un liquide ou en un gaz, quelles que soient les circonstances.
Pas d'eau, pas d'air.
Rien qui ressemble à de l'eau, rien qui ressemble à de l'air.
Même si ce n'est pas censé être bu.
Les liquides \emph{s'évaporent}, d'infimes petites parties s'échappent dans les airs.
Vous ne métamorphoserez rien qui soit destiné à être brûlé.
Cela ferait de la fumée que quelqu'un pourrait respirer~!
Vous ne métamorphoserez jamais rien qui puisse potentiellement se retrouver dans le corps de quelqu'un par quelque moyen que ce soit.
Pas de nourriture.
Rien qui \emph{ressemble} à de la nourriture.
Même pas une petite blague amusante où vous comptiez les prévenir que s'était une tarte à la boue avant qu'ils ne la mangent pour de vrai.
Vous ne le ferez jamais.
Point.
Dans cette classe ou hors de cette classe ou \emph{où que ce soit}.
Est-ce bien compris par \emph{tous les élèves}~?

--- Oui~», dirent Harry, Hermione, et quelques autres. Le reste de la classe semblait sans voix.

--- \emph{Est-ce bien compris par tous les élèves~?}

--- Oui, dirent-ils, ou murmurèrent-ils, ou chuchotèrent-ils.

--- Si vous enfreignez n'importe laquelle de ces règles, vous n'étudierez plus la métamorphose pendant votre scolarité à Poudlard.
Répétez après moi.
Je ne métamorphoserai jamais rien en liquide ou en gaz.

--- Je ne métamorphoserai jamais rien en liquide ou en gaz, répétèrent les élèves en un chœur désordonné.

--- Encore~! Plus fort~! Je ne métamorphoserai jamais rien en liquide ou en gaz.

--- Je ne métamorphoserai jamais rien en liquide ou en gaz.

--- Je ne métamorphoserai jamais rien qui ressemble à de la nourriture ou toute autre chose qui entre dans le corps humain.

--- Je ne métamorphoserai jamais rien qui puisse être brûlé car cela pourrait faire de la fumée.

--- Vous ne métamorphoserez jamais rien qui ressemble à de l'argent, même de l'argent Moldu, dit McGonagall.
Les gobelins ont leurs méthodes pour retrouver le coupable.
Il est d'ailleurs comme gravé dans le marbre que la nation gobeline est dans un état de \emph{guerre} permanent contre les faussaires magiques.
Ils n'enverront pas d'Aurors.
Ils enverront une armée.

--- Je ne métamorphoserai jamais rien qui ressemble à de l'argent, répétèrent les élèves en chœur.

--- Et \emph{par-dessus tout}, dit McGonagall, vous ne métamorphoserez aucun sujet vivant, et \emph{en particulier vous-même}.
Cela vous rendrait très malade, et peut-être vous tuerait, selon la façon dont vous vous serez métamorphosés et selon la durée pendant laquelle vous aurez maintenu la transformation.~»
McGonagall marqua une pause. «~M. Potter a en ce moment une main levée en l'air parce qu'il a vu une transformation en Animagus -- plus précisément un humain se transformant en chat et à nouveau en humain.
Mais la transformation en Animagus n'est pas une métamorphose \emph{libre}.~»

McGonagall sortit un petit morceau de bois de sa poche.
D'un tapotement de baguette magique, il devint une sphère de verre.
Puis elle dit «~\emph{Crystferrium~!}~» et la sphère de verre devint une sphère d'acier.
Elle la tapota une dernière fois et la sphère d'acier redevint un morceau de bois.
«~\emph{Crystferrium} transforme un sujet en verre massif en un objet d'acier massif de forme similaire.
Il ne peut pas faire l'inverse, pas plus qu'il ne peut transformer un bureau en cochon.
La forme la plus générale de la métamorphose -- la métamorphose libre, que vous allez apprendre ici -- est capable de transformer n'importe quel sujet en n'importe quelle cible, du moins en ce qui concerne la forme physique.
C'est pour cette raison que la métamorphose libre doit se faire sans parole.
Utiliser des enchantements demanderait l'utilisation de mots différents pour chaque transformation entre sujet et cible.~»

McGonagall jeta un regard affûté à ses élèves.
«~\emph{Certains} enseignants commencent par les enchantements de métamorphose et passent ensuite à la métamorphose libre.
Oui, ce serait beaucoup plus simple pour commencer.
Mais vous risqueriez de vous glisser dans un moule dont il est difficile de sortir et qui détériorerait vos capacités ultérieures.
Vous apprendrez ici la métamorphose libre \emph{dès le départ}, ce qui exige que vous jetiez le sort sans prononcer un mot, en maintenant dans votre propre esprit la forme du sujet, la forme cible, et la transformation. %~»

«~Et pour répondre à la question de M. Potter, continua McGonagall, c'est la métamorphose \emph{libre} que vous ne devez jamais opérer sur un sujet vivant.
Il existe des enchantements et des potions qui peuvent transformer sans risque des sujets vivants, de façon \emph{limitée} et réversible.
Un Animagus à qui il manquerait un membre ne retrouvera pas ce membre après s'être transformé, par exemple.
La métamorphose libre n'est \emph{pas} sûre.
Votre corps changera pendant qu'il est métamorphosé -- la respiration par exemple produit une perte constante de matière corporelle dans l'air environnant.
Lorsque la métamorphose s'estompe et que votre corps essaie de revenir à sa forme \emph{originale}, il ne sera pas tout à fait capable de le faire.
Si vous pressez votre baguette contre votre corps et que vous vous imaginez avec des cheveux dorés, vous les perdrez une fois la métamorphose terminée.
Si vous vous visualisez avec une peau plus claire, vous passerez un long séjour à Sainte Mangouste.
Et si vous vous métamorphosez en une forme adulte, alors, quand la métamorphose se dissipera, vous mourrez.~»

Voilà qui expliquait l'existence de garçons trop gros ou de filles qui ne soient pas parfaitement jolies.
Ou simplement de gens âgés.
Cela n'existerait pas si on pouvait juste se métamorphoser tous les matins…
Harry leva la main et essaya de signaler sa présence à McGonagall du regard.

«~\emph{Oui}, M. Potter~?

--- Est-il possible de métamorphoser un sujet vivant en une cible statique, comme une pièce de monnaie -- non, pardon, je suis vraiment désolé, disons juste en une sphère d'acier.~»

McGonagall secoua la tête.
«~M. Potter, même les objets inanimés subissent de petits changements internes au fil du temps.
Il n'y aurait pas de changement visible sur votre corps après la transformation, et vous ne remarqueriez rien d'anormal pendant la première minute.
Mais une heure plus tard, vous seriez très malade, et le lendemain, vous seriez mort.

--- Euh, excusez-moi, mais alors si j'avais lu le premier chapitre, j'aurais pu \emph{deviner} que le bureau était initialement un bureau et non un cochon, dit Harry, mais seulement si j'avais \emph{en plus} émis l'hypothèse que vous ne vouliez pas tuer le cochon, ce qui pourrait \emph{sembler} hautement probable, mais…

--- Je puis prédire que noter vos contrôles sera pour moi une source intarissable d'émerveillement, M. Potter.
Mais si vous avez d'autres questions, puis-je s'il vous plaît vous demander d'attendre la fin du cours~?

--- Pas d'autres questions, professeure.

--- Maintenant répétez après moi, dit McGonagall.
Je n'essaierai jamais de métamorphoser un sujet vivant, et en particulier moi-même, à moins que l'on ne m'ait spécifiquement demandé de le faire à l'aide d'un enchantement spécialisé ou d'une potion.

--- Si je ne suis pas certain que la métamorphose est sûre, je n'essaierai pas avant d'avoir posé la question à la professeure McGonagall ou au professeur Flitwick ou au professeur Rogue ou au directeur, qui sont les seules autorités reconnues en matière de métamorphose à Poudlard.
Demander à un autre élève n'est \emph{pas} une alternative acceptable, même s'ils disent se souvenir avoir posé la même question.

--- Même si le professeur de Défense actuel de Poudlard me dit qu'une métamorphose est sûre, et même si je vois le professeur de Défense la réaliser et que je ne vois rien de néfaste se produire, je ne l'essaierai pas moi-même.

--- J'ai le droit inaliénable de refuser d'opérer toute métamorphose au sujet de laquelle je ressens la moindre nervosité.
Puisque même le directeur de Poudlard ne peut me donner l'ordre de le faire, je n'accepterai absolument aucun ordre de ce genre venant du professeur de Défense, même si le professeur de Défense menace de déduire cent points à ma maison et de me faire exclure.

--- Si je brise une seule de ces règles je n'étudierai plus la métamorphose durant ma scolarité à Poudlard.

--- Nous répéterons ces règles au début de chaque cours pendant un mois, dit McGonagall.
Et maintenant, nous allons commencer avec pour sujet des allumettes et pour cible des aiguilles… posez vos baguettes, merci bien, par “commencer,” je voulais dire que vous allez commencer à prendre des notes.~»

Une demi-heure avant la fin du cours, McGonagall distribua les allumettes.

À la fin du cours, Hermione avait une allumette d'aspect argenté, et le reste de la classe, nés-moldus ou non, avait exactement la même chose que ce qu'on leur avait donné au départ.

McGonagall lui décerna un point de plus pour Serdaigle.

\later

Une fois le cours terminé, Hermione s'approcha du pupitre de Harry tandis qu'il rangeait ses livres dans sa bourse.

«~Tu sais, commença Hermione d'un ton innocent, j'ai gagné deux points pour Serdaigle aujourd'hui.

--- En effet, dit sèchement Harry.

--- Mais ce n'était pas aussi bien que tes \emph{sept} points.
Je suppose que je ne suis pas aussi intelligente que toi.~»

Harry finit de donner ses devoirs à manger à sa bourse et se tourna vers Hermione en plissant les yeux.
Il avait à vrai dire complètement oublié cet épisode.

Elle le regarda en \emph{battant des cils}.
«~Cela dit, nous avons des cours tous les jours.
Je me demande combien de temps cela te prendra de trouver d'autres Poufsouffle à sauver~?
Nous sommes lundi.
Donc cela te donne jusqu'à jeudi.~»

Tous deux se regardèrent dans le blanc des yeux, sans ciller.

Harry parla le premier.
«~Tu te rends bien sûr compte que c'est une déclaration de guerre.

--- Je ne savais pas que nous étions en paix.~»

Tous les autres élèves regardaient à présent la scène avec fascination.
Tous les autres élèves plus McGonagall, malheureusement.

«~Oh, M. Potter, fredonna le professeur McGonagall depuis l'autre extrémité de la pièce, j'ai de bonnes nouvelles pour vous.
Madame Pomfresh a approuvé votre suggestion de prévention des dommages aux médallons bidoules, et nous comptons avoir terminé les modifications d'ici la fin de la semaine prochaine.
Je dirais que cela mérite… disons dix points pour Serdaigle.~»

Le visage d'Hermione montrait son état de choc et son sentiment de trahison.
Harry supposa que son propre visage ne devait pas avoir l'air très différent.

«~\emph{Professeure…} siffla Harry.

--- Il ne fait \emph{aucun doute} que vous méritez ces dix points, M. Potter.
Je ne donnerais pas des points sur un coup de tête.
De votre point de vue, vous avez simplement remarqué quelque chose de fragile et avez suggéré une façon de le protéger, mais les médallons bidoules sont coûteux et le directeur n'était \emph{pas} du tout ravi la dernière fois que quelqu'un en a cassé un.
McGonagall prit un air pensif.
Voyons, je me demande s'il est déjà arrivé qu'un élève gagne dix-sept points dès son premier jour d'école.
Il faudra que je vérifie, mais je pense que nous avons là un nouveau record.
Peut-être devrions-nous faire une annonce pendant le dîner~?

--- \emph{PROFESSEURE}~! s'écria Harry. C'est \emph{notre} guerre~! Arrêtez de vous en mêler~!

--- Vous avez maintenant jusqu'à jeudi de la semaine \emph{prochaine}, M. Potter.
À moins bien sûr que d'ici là vous ne vous prêtiez à quelque espièglerie et ne \emph{perdiez} alors des points.
En vous adressant à un professeur de façon irrespectueuse, par exemple.~»
McGonagall posa un doigt sur sa joue et prit un air songeur.
«~Je m'attends à ce que vous atteigniez les nombres négatifs avant vendredi soir.~»

Harry referma hâtivement la bouche.
Il jeta son meilleur Regard Mortel à McGonagall, qui sembla trouver cela simplement amusant.

«~Oui, une annonce au dîner, certainement, dit McGonagall, songeuse.
Mais il ne faudrait pas offenser les Serpentard, ce sera donc une annonce brève.
Juste le nombre de points et que c'est un record… et si quelqu'un vient vous voir parce qu'il a besoin d'aide pour ses devoirs et se voit déçu d'apprendre que vous n'avez même pas commencé à lire vos manuels scolaires, vous pourrez toujours les rediriger vers Mademoiselle Granger.

--- \emph{Professeure~!} dit Hermione d'une voix assez aiguë.

McGonagall l'ignora.
«~Voyons, je me demande combien de temps Mademoiselle Granger mettra à faire quelque chose digne d'une annonce au dîner~?
Quelle qu'en soit la raison, j'attends ce moment avec impatience.~»

Harry et Hermione, par consentement mutuel et muet, se retournèrent et quittèrent la salle à grands pas.
Ils furent suivis par une colonne de Serdaigle hypnotisés.

«~Euh, dit Harry. C'est toujours d'accord pour après le dîner~?

--- Bien sûr, dit Hermione. Je ne voudrais pas que tu prennes du retard dans tes cours.

--- Eh bien, merci. Et je voulais juste dire que, vu comme tu es déjà brillante, je ne peux m'empêcher de me demander de quoi tu seras capable une fois que tu auras eu une formation élémentaire à la rationalité.

--- Est-ce vraiment si utile~?
Ça n'avait pas l'air de t'aider pour les enchantements ou la métamorphose.~»

Il y eut une courte pause.

«~Eh bien, je n'ai reçu mes manuels qu'il y a quatre jours à peine.
C'est pour ça que j'ai dû gagner ces dix-sept points sans utiliser ma baguette.

--- Il y a quatre jours~?
Tu ne peux peut-être pas lire huit livres en quatre jours, mais tu aurais quand même pu en lire \emph{un}.
Combien de jours cela te prendra-t-il pour les terminer à ce rythme~?
Toi qui connais toutes ces mathématiques, peux-tu me dire combien font huit fois quatre divisé par zéro~?

--- J'ai cours maintenant, ce que tu n'avais pas, mais mes week-ends sont libres, donc… limite de huit fois quatre divisé par epsilon quand epsilon tend vers zéro… dimanche matin à 10h47.

--- J'ai tout lu en \emph{trois} jours, tu sais.

--- Samedi à 14h47 donc. Je suis sûr que je trouverai le temps.~»

Puis ce fut le soir, puis le matin, le premier jour.

%  LocalWords:  rigideiro Flooing Crystferrium Erm
